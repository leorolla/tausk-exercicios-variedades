
% !TeX spellcheck = pt_BR
% !TEX encoding = UTF-8 Unicode

\NeedsTeXFormat{LaTeX2e}

\documentclass[oneside,11pt]{amsart}

\synctex=1

\usepackage[brazil]{babel}
\usepackage[utf8]{inputenc}
\usepackage{lmodern}

\usepackage[all]{xy}
\usepackage{dsfont}
\usepackage{amssymb}
\usepackage{manfnt}

\newcommand{\R}{\mathds R}
\newcommand{\Q}{\mathds Q}
\newcommand{\C}{\mathds C}
\newcommand{\Z}{\mathds Z}
\newcommand{\IH}{\mathds H}
\newcommand{\dd}{\mathrm d}
\newcommand{\Dd}{\mathrm D}
\newcommand{\Id}{\mathrm{Id}}
\newcommand{\I}{\mathrm I}
\newcommand{\transp}{\mathrm t}
%\newcommand{\microspace}{\mskip1mu}

\DeclareMathOperator{\Dom}{Dom}
\DeclareMathOperator{\Img}{Im}
\DeclareMathOperator{\supp}{supp}
\DeclareMathOperator{\Der}{Der}
\DeclareMathOperator{\Gr}{gr}
\DeclareMathOperator{\Lin}{Lin}
\DeclareMathOperator{\posto}{posto}
\DeclareMathOperator{\Dim}{dim}
\DeclareMathOperator{\Ker}{Ker}
\DeclareMathOperator{\sen}{sen}
\DeclareMathOperator{\sgn}{sgn}
\DeclareMathOperator{\mdc}{mdc}
\DeclareMathOperator{\Sym}{Sym}
\DeclareMathOperator{\Alt}{Alt}
\DeclareMathOperator{\Or}{O}
\DeclareMathOperator{\SO}{SO}
\DeclareMathOperator{\Ur}{U}
\DeclareMathOperator{\GL}{GL}
\DeclareMathOperator{\SL}{SL}
\DeclareMathOperator{\SU}{SU}
\DeclareMathOperator{\tr}{tr}

%\title{Exercícios para MAT5799}
\title[Variedades Diferenciáveis via Exercícios Guiados]{Variedades Diferenciáveis \\ via Exercícios Guiados}

\author{Daniel V. Tausk}

\date{10 de agosto de 2010}

\theoremstyle{remark}\newtheorem{exercise}{Exercício}[section]
\swapnumbers
\theoremstyle{plain}\newtheorem{teo}{Teorema}[section]
\theoremstyle{plain}\newtheorem{lem}[teo]{Lema}
\theoremstyle{plain}\newtheorem{prop}[teo]{Proposição}
%\theoremstyle{plain}\newtheorem{cor}[teo]{Corolário}
\theoremstyle{definition}\newtheorem{defin}[teo]{Definição}
\theoremstyle{remark}\newtheorem{rem}[teo]{Observação}
%\theoremstyle{definition}\newtheorem{notation}[teo]{Notação}
%\theoremstyle{definition}\newtheorem{convention}[teo]{Convenção}
\theoremstyle{definition}\newtheorem{example}[teo]{Exemplo}

\numberwithin{equation}{section}

\renewcommand{\datename}{\textit{Data}:}

\begin{document}

\maketitle

\noindent
\copyright\ 2010
Daniel V. Tausk.
\\
Permitido o uso nos termos da licença CC~BY-SA 4.0.


\renewcommand{\contentsline}[3]{\csname novo#1\endcsname{#2}{#3}}
\newcommand{\novochapter}[2]{\bigskip\hbox to \hsize{\vbox{\advance\hsize by -1cm\baselineskip=12pt\parfillskip=0pt\leftskip=3cm\noindent\hskip -2cm #1\leaders\hbox{.}\hfil\hfil\par}$\,$#2\hfil}}
\newcommand{\novosection}[2]{\medskip\hbox to \hsize{\vbox{\advance\hsize by -1cm\baselineskip=12pt\parfillskip=0pt\leftskip=3.5cm\noindent\hskip -2cm #1\leaders\hbox{.}\hfil\hfil\par}$\,$#2\hfil}}
\newcommand{\novosubsection}[2]{\baselineskip=12pt}%{\medskip\hbox to \hsize{\vbox{\advance\hsize by -1cm\baselineskip=12pt\parfillskip=0pt\leftskip=3.5cm\noindent\hskip -2cm #1\leaders\hbox{.}\hfil\hfil\par}$\,$#2\hfil}}
\tableofcontents

\begin{section}{Dia 09/08}

No que segue, $\Dom(f)$ denota o domínio de uma função $f$. Entende-se que, para quaisquer funções $f$ e $g$, a composição $g\circ f$ é a função
com domínio $f^{-1}\big(\!\Dom(g)\big)$ tal que $(g\circ f)(x)=g\big(f(x)\big)$, para todo $x\in f^{-1}\big(\!\Dom(g)\big)$.

\begin{exercise}\label{exe:trescartas}
Sejam $M$ um conjunto e $\varphi$, $\psi$, $\lambda$ sistemas de coordenadas em $M$. Mostre que:
\begin{equation}\label{eq:trescartas}
\varphi\big(\!\Dom(\varphi)\cap\Dom(\psi)\cap\Dom(\lambda)\big)=(\lambda\circ\varphi^{-1})^{-1}\big[\lambda\big(\!\Dom(\lambda)\cap\Dom(\psi)\big)\big]
\end{equation}
e que a restrição de $\psi\circ\varphi^{-1}$ ao conjunto \eqref{eq:trescartas} é igual a:
\[(\psi\circ\lambda^{-1})\circ(\lambda\circ\varphi^{-1}).\]
Conclua que se $\lambda$ é compatível com $\varphi$ e com $\psi$ então o conjunto \eqref{eq:trescartas} é aberto (no espaço $\R^n$ que contém
a imagem de $\varphi$) e a restrição de $\psi\circ\varphi^{-1}$ ao conjunto \eqref{eq:trescartas} é de classe $C^\infty$.
\end{exercise}

\begin{exercise}\label{exe:lambdas}
Sejam $M$ um conjunto, $\varphi$, $\psi$ sistemas de coordenadas em $M$ e $\mathcal A$ uma coleção de sistemas de coordenadas em $M$ cujos domínios
cobrem $\Dom(\varphi)\cap\Dom(\psi)$ (esse é o caso, por exemplo, se $\mathcal A$ é um atlas em $M$). Use o resultado do Exercício~\ref{exe:trescartas}
para mostrar que se qualquer $\lambda\in\mathcal A$ é compatível com $\varphi$ e com $\psi$ então $\varphi$ é compatível com $\psi$.
\end{exercise}

\begin{exercise}\label{exe:lambdas2}
Sejam $M$ um conjunto e $\mathcal A$ um atlas em $M$. Dado um sistema de coordenadas $\varphi$ em $M$ e um subconjunto $\mathcal B$ de $\mathcal A$ tal que:
\[\Dom(\varphi)\subset\bigcup_{\lambda\in\mathcal B}\Dom(\lambda),\]
mostre que se $\varphi$ é compatível com qualquer elemento de $\mathcal B$ então $\varphi$ é compatível com $\mathcal A$ (sugestão:
use o resultado do Exercício~\ref{exe:lambdas}).
\end{exercise}

\begin{exercise}
Sejam $M$ um conjunto e $\mathcal A$ um atlas em $M$.
\begin{itemize}
\item[(a)] Mostre que um sistema de coordenadas $\varphi$ em $M$ é compatível com $\mathcal A$ se e somente se $\mathcal A\cup\{\varphi\}$ é um atlas
em $M$.
\item[(b)] Mostre que um atlas $\mathcal A'$ em $M$ é compatível com $\mathcal A$ se e somente se $\mathcal A\cup\mathcal A'$ é um atlas em $M$.
\item[(c)] Mostre que $\mathcal A$ é um atlas maximal em $M$ se e somente se qualquer sistema de coordenadas
em $M$ compatível com $\mathcal A$ pertence a $\mathcal A$.
\end{itemize}
\end{exercise}

\begin{exercise}
Sejam $M$ um conjunto e $\mathcal A$ um atlas em $M$. Denote por $\mathcal A_{\max}$ o conjunto de todos os sistemas de coordenadas em $M$ que são compatíveis
com $\mathcal A$. Mostre que:
\begin{itemize}
\item[(a)] $\mathcal A_{\max}$ é um atlas em $M$ que contém $\mathcal A$ (sugestão: use o resultado do Exercício~\ref{exe:lambdas});
\item[(b)] se $\mathcal A'$ é um atlas em $M$ que contém $\mathcal A$ então $\mathcal A'\subset\mathcal A_{\max}$ (isto é, $\mathcal A_{\max}$ é o
{\em maior\/} atlas em $M$ que contém $\mathcal A$);
\item[(c)] $\mathcal A_{\max}$ é o único atlas maximal em $M$ que contém $\mathcal A$;
\item[(d)] dois atlas são compatíveis se e somente se estão contidos no mesmo atlas maximal;
\item[(e)] a relação de compatibilidade é uma relação de equivalência no conjunto de todos os atlas em $M$ e os atlas maximais
em $M$ constituem um conjunto escolha para as correspondentes classes de equivalência (i.e., cada classe de equivalência contém exatamente
um atlas maximal).
\end{itemize}
\end{exercise}

\begin{exercise}[topologia definida por um atlas]
Sejam $M$ um conjunto e $\mathcal A$ um atlas em $M$. Mostre que:
\begin{equation}\label{eq:tauA}
\tau_{\mathcal A}=\big\{A\subset M:\text{$\varphi\big(A\cap\Dom(\varphi)\big)$ é aberto, para todo $\varphi\in\mathcal A$}\big\}
\end{equation}
é uma topologia em $M$ (em \eqref{eq:tauA}, ``aberto'' significa aberto no espaço $\R^n$ que contém a imagem de $\varphi$).
Mostre que, se $M$ é munido da topologia $\tau_{\mathcal A}$, então qualquer $\varphi\in\mathcal A$ é um homeomorfismo
com domínio aberto em $M$.
\end{exercise}

\begin{exercise}[caracterização da topologia definida por um atlas]\label{exe:cartopol}
Mostre que:
\begin{itemize}
\item[(a)] se $M$ é um conjunto, $M=\bigcup_{i\in I}U_i$ é uma cobertura de $M$ e $\tau$, $\tau'$ são topologias em $M$ que fazem de cada $U_i$
um aberto de $M$ e que induzem a mesma topologia em cada $U_i$ então $\tau=\tau'$ (sugestão: mostre que a aplicação identidade de $(M,\tau)$
para $(M,\tau')$ é um homeomorfismo);
\item[(b)] se $M$ é um conjunto, $\mathcal A$ é um atlas em $M$ e se $\tau$ é uma topologia em $M$ que faz com que qualquer $\varphi\in\mathcal A$ seja um
homeomorfismo com domínio aberto então $\tau$ coincide com a topologia $\tau_{\mathcal A}$ definida em \eqref{eq:tauA} (sugestão: use o resultado do item~(a)
para a cobertura de $M$ formada pelos domínios dos elementos de $\mathcal A$).
\end{itemize}
\end{exercise}

\begin{exercise}
Sejam $M$ um conjunto e $\mathcal A$, $\mathcal A'$ atlas compatíveis em $M$. Mostre que:
\[\tau_{\mathcal A}=\tau_{\mathcal A'}\]
(sugestão: $\mathcal A\cup\mathcal A'$ é um atlas e a topologia $\tau_{\mathcal A\cup\mathcal A'}$ satisfaz tanto a condição que caracteriza
$\tau_{\mathcal A}$ como a condição que caracteriza $\tau_{\mathcal A'}$).
Conclua que se $\mathcal A_{\max}$ é o atlas maximal em $M$ que contém $\mathcal A$ então $\tau_{\mathcal A}=\tau_{\mathcal A_{\max}}$.
\end{exercise}

\begin{exercise}
Sejam $M$ um conjunto e $\mathcal A$ um atlas maximal em $M$. Se um sistema de coordenadas
$\varphi:U\to\widetilde U$ pertence a $\mathcal A$ e $V\subset U$ é aberto
com respeito a $\tau_{\mathcal A}$, mostre que $\varphi\vert_V:V\to\varphi(V)$ pertence a $\mathcal A$ (sugestão: pelo resultado do Exercício~\ref{exe:lambdas2},
basta mostrar que $\varphi\vert_V$ é compatível com $\varphi$ para concluir que $\varphi\vert_V$ é compatível com $\mathcal A$. E não esqueça
de verificar que $\varphi\vert_V$ é um sistema de coordenadas no conjunto $M$!).
\end{exercise}

\end{section}

\begin{section}{Dia 11/08}

Por uma {\em variedade diferenciável\/} entenderemos um conjunto $M$ munido de um atlas maximal $\mathcal A$ (ou seja,
o par $(M,\mathcal A)$); subentende-se que $M$ está também munido da topologia $\tau_{\mathcal A}$ definida por esse atlas.
Por enquanto, não faremos hipóteses sobre essa topologia. Por ``carta'' ou ``sistema de coordenadas'' numa variedade $M$ entenderemos agora
um elemento $\varphi$ do atlas maximal dado $\mathcal A$. Recorde que, de acordo com nossa definição, dadas variedades diferenciáveis $M$, $N$, então
uma função $f:M\to N$ é dita {\em de classe $C^\infty$\/} se para todo $x\in M$ {\em existem\/} uma carta $\varphi:U\to\widetilde U$ em $M$ e uma carta
$\psi:V\to\widetilde V$ em $N$ tais que $x\in U$, $f(U)\subset V$ e $\psi\circ f\circ\varphi^{-1}:\widetilde U\to\widetilde V$ é de classe $C^\infty$
(no sentido usado em cursos de Cálculo no $\R^n$, i.e., diferenciais de todas as ordens existem).
A aplicação $f:M\to N$ é um {\em difeomorfismo de classe $C^\infty$\/} se $f$ for uma bijeção de classe $C^\infty$ cuja inversa é de classe $C^\infty$.

\begin{exercise}[estrutura diferenciável canônica de um espaço vetorial]\label{exe:varespvet}
Seja $E$ um espaço vetorial real de dimensão $n<+\infty$. Mostre que:
\begin{equation}\label{eq:atlasvec}
\big\{\varphi:E\to\R^n:\text{$\varphi$ é um isomorfismo linear}\big\}
\end{equation}
é um atlas em $E$ e que a topologia induzida por esse atlas coincide com a topologia usual de $E$ (i.e., a topologia
definida por qualquer norma). Mostre também que o atlas maximal que contém \eqref{eq:atlasvec} consiste de todos os difeomorfismos
$\varphi:U\to\widetilde U$ de classe $C^\infty$, sendo $U$ um aberto de $E$ e $\widetilde U$ um aberto de $\R^n$ (aqui ``difeomorfismo de classe $C^\infty$''
deve ser entendido no sentido usado em cursos de Cálculo no $\R^n$, i.e.,
$\varphi$ é bijetora e tanto $\varphi$ como $\varphi^{-1}$ admitem diferenciais de todas as ordens).
\end{exercise}

\begin{exercise}[subvariedades abertas]\label{exe:subvaraberta}
Sejam $(M,\mathcal A)$ uma variedade diferenciável e $U$ um subconjunto aberto de $M$. Mostre que:
\[\mathcal A\vert_U=\big\{\varphi\in\mathcal A:\Dom(\varphi)\subset U\big\}\]
é um atlas maximal em $U$ e que a topologia definida em $U$ pelo atlas $\mathcal A\vert_U$ coincide com a topologia
induzida no subconjunto $U$ pela topologia $\tau_{\mathcal A}$ (sugestão: para a maximalidade de $\mathcal A\vert_U$, use o resultado do Exercício~\ref{exe:lambdas2}
com $\mathcal B=\mathcal A\vert_U$. Para a afirmação sobre as topologias,
mostre que a topologia induzida em $U$ por $\tau_{\mathcal A}$ satisfaz as propriedades que caracterizam
a topologia definida por $\mathcal A\vert_U$).
\end{exercise}

\begin{exercise}\label{exe:smoothcontinuous}
Se $M$, $N$ são variedades diferenciáveis, mostre que toda função $f:M\to N$ de classe $C^\infty$ é contínua.
\end{exercise}

\begin{exercise}
Sejam $M$, $N$ variedades diferenciáveis e $f:M\to N$ uma função. Mostre que:
\begin{itemize}
\item[(a)] se $f$ é de classe $C^\infty$ e $U$ é uma subvariedade aberta de $M$ então a restrição $f\vert_U:U\to N$
também é de classe $C^\infty$;
\item[(b)] se todo ponto de $M$ pertence a uma subvariedade aberta $U$ de $M$ tal que a restrição $f\vert_U:U\to N$
é de classe $C^\infty$ então a função $f$ é de classe $C^\infty$;
\item[(c)] se $U$ é uma subvariedade aberta de $N$ que contém a imagem de $f$ então a função $f:M\to N$ é de classe $C^\infty$
se e somente se a função $f:M\to U$ (que difere de $f$ apenas pelo contra-domínio) é uma função de classe $C^\infty$.
\end{itemize}
\end{exercise}

\begin{exercise}\label{exe:smoothambiguous}
Se $M$, $N$ são abertos de espaços vetoriais reais de dimensão finita então uma função $f:M\to N$ é de classe
$C^\infty$ ``no sentido de variedades'' (i.e., pensamos em $M$, $N$ como subvariedade abertas de espaços vetoriais
reais de dimensão finita munidos do atlas diferenciável descrito no Exercício~\ref{exe:varespvet}) se e somente
se $f:M\to N$ é de classe $C^\infty$ no sentido dos cursos de Cálculo no $\R^n$ (i.e., as diferenciais de $f$ de todas as ordens existem).
\end{exercise}

\begin{exercise}\label{exe:cartadiff}
Seja $(M,\mathcal A)$ uma variedade diferenciável. Mostre que qualquer $\varphi:U\to\widetilde U$ pertencente a $\mathcal A$ é um difeomorfismo
de classe $C^\infty$, sendo $U$ uma subvariedade aberta de $M$ e $\widetilde U$ uma subvariedade aberta do espaço $\R^n$ que contém $\widetilde U$,
onde $\R^n$ é munido da sua estrutura diferenciável canônica (sugestão: a própria $\varphi$ é uma carta em $U$ e a aplicação identidade
é uma carta em $\widetilde U$).
\end{exercise}

\begin{exercise}
Sejam $M$, $N$ variedades diferenciáveis e $f:M\to N$ uma função de classe $C^\infty$. Mostre que se $\varphi:U\to\widetilde U$
é uma carta em $M$ e $\psi:V\to\widetilde V$ é uma carta em $N$ então a função:
\[\psi\circ f\circ\varphi^{-1}:\varphi\big(U\cap f^{-1}(V)\big)\longrightarrow\widetilde V\]
é de classe $C^\infty$ e seu domínio é um aberto do espaço $\R^n$ que contém a imagem de $\varphi$ (sugestão: para mostrar que o domínio é aberto,
use o resultado do Exercício~\ref{exe:smoothcontinuous}. Para mostrar que a função é de classe $C^\infty$, use o resultado do Exercício~\ref{exe:cartadiff}
e o resultado demonstrado em aula de que a composição de funções de classe $C^\infty$ entre variedades é de classe $C^\infty$).
\end{exercise}

\begin{exercise}\label{exe:difeocarta}
Mostre a recíproca do resultado do Exercício~\ref{exe:cartadiff}, i.e., mostre que se $(M,\mathcal A)$ é uma variedade diferenciável
e $\varphi:U\to\widetilde U$ é um difeomorfismo de classe $C^\infty$, sendo $U$ uma subvariedade aberta de $M$ e $\widetilde U$ uma subvariedade aberta
de $\R^n$ então $\varphi\in\mathcal A$ (sugestão: mostre que $\varphi$ é compatível com $\mathcal A$. Para isso, use o resultado do Exercício~\ref{exe:cartadiff}
e o resultado demonstrado em aula de que a composição de funções de classe $C^\infty$ entre variedades é de classe $C^\infty$. Note que você estará
usando o resultado do Exercício~\ref{exe:smoothambiguous} também!).
\end{exercise}

\begin{exercise}\label{exe:Iddifeo}
Sejam $\mathcal A$, $\mathcal A'$ atlas maximais num mesmo conjunto $M$. Mostre que a aplicação identidade de $(M,\mathcal A)$ para $(M,\mathcal A')$ é um difeomorfismo
de classe $C^\infty$ se e somente se $\mathcal A=\mathcal A'$ (sugestão: mostre que se tal aplicação identidade é um difeomorfismo então os atlas $\mathcal A$
e $\mathcal A'$ são compatíveis).
\end{exercise}

\begin{exercise}
Neste curso assumiremos (como parte da definição de atlas) que todos os sistemas de coordenadas pertencentes a um dado atlas tomam valores no mesmo
espaço $\R^n$. Neste exercício, no entanto, investigaremos o que aconteceria se essa condição fosse relaxada, i.e., se permitíssemos que sistemas de coordenadas
diferentes num mesmo atlas tomassem valores em espaços $\R^n$ diferentes.
\begin{itemize}
\item[(a)] Seja $\mathcal A$ um atlas num conjunto $M$. Mostre que, dado $x\in M$, então todos os sistemas de coordenadas $\varphi\in\mathcal A$
cujo domínio contém $x$ tomam valores no mesmo espaço $\R^n$.
\item[(b)] Em vista do item (a), para $x\in M$, denotamos por $n(x)$ o número natural tal que qualquer $\varphi\in\mathcal A$ cujo domínio contém $x$
toma valores em $\R^{n(x)}$. Mostre que a função $x\mapsto n(x)$ é localmente constante. Conclua que ela é constante em cada componente conexa de $M$.
\end{itemize}
\end{exercise}

\begin{exercise}[variedades produto]
Sejam $(M,\mathcal A)$, $(N,\mathcal A')$ variedades diferenciáveis.
\begin{itemize}
\item[(a)] Mostre que:
\begin{equation}\label{eq:atlasprod}
\big\{\varphi\times\psi:\varphi\in\mathcal A,\ \psi\in\mathcal A'\big\}
\end{equation}
é um atlas no conjunto $M\times N$, onde, dadas $\varphi:U\to\widetilde U$, $\psi:V\to\widetilde V$, a função
$\varphi\times\psi:U\times V\to\widetilde U\times\widetilde V$ é definida por:
\[(\varphi\times\psi)(x,y)=\big(\varphi(x),\psi(y)\big),\]
para todos $x\in U$, $y\in V$. O conjunto $M\times N$, munido do atlas maximal que contém \eqref{eq:atlasprod}, chama-se a {\em variedade produto\/}
da variedade $M$ pela variedade $N$.
\item[(b)] Mostre que a topologia definida pelo atlas \eqref{eq:atlasprod} coincide com a topologia produto da topologia
definida por $\mathcal A$ pela topologia definida por $\mathcal A'$ (sugestão: mostre que a topologia produto satisfaz as condições que caracterizam
a topologia definida por \eqref{eq:atlasprod}).
\item[(c)] Mostre que se $U$ é uma subvariedade aberta de $M$ e $V$ é uma subvariedade aberta de $N$ então a variedade produto $U\times V$ é uma subvariedade
aberta da variedade produto $M\times N$ (note que não está se pedindo apenas para se demonstrar que $U\times V$ é aberto
em $M\times N$, mas também que o atlas maximal
que a variedade produto $M\times N$ induz no aberto $U\times V$ coincide com o atlas maximal da variedade produto
$U\times V$).
\end{itemize}
\end{exercise}

\begin{exercise}\label{exe:univproduct}
Sejam $M$, $N$ variedades diferenciáveis e considere a variedade produto $M\times N$. Mostre que vale a seguinte {\em propriedade universal}:
dada uma variedade diferenciável $P$ e uma função $f=(f_1,f_2):P\to M\times N$, então $f$ é de classe $C^\infty$ se e somente se $f_1$ e $f_2$ são de
classe $C^\infty$. Conclua (tomando $P=M\times N$ e $f$ igual à função identidade) que as projeções $M\times N\to M$, $M\times N\to N$ são de classe $C^\infty$.
\end{exercise}

\begin{exercise}[unicidade do produto]
Dadas variedades diferenciáveis $M$, $N$, mostre que se $\mathcal A$, $\mathcal A'$ são atlas maximais em $M\times N$ e se para ambos vale a propriedade
universal que aparece no enunciado do Exercício~\ref{exe:univproduct} então $\mathcal A=\mathcal A'$ (sugestão: mostre que a aplicação identidade de
$(M\times N,\mathcal A)$ para $(M\times N,\mathcal A')$ é um difeomorfismo de classe $C^\infty$).
\end{exercise}

\begin{exercise}[soma de variedades]\label{exe:somavariedades}
Seja $\big((M_i,\mathcal A_i)\big)_{i\in I}$ uma família de variedades diferenciáveis, todas com a mesma dimensão. Considere a união disjunta:
\[M=\bigcup_{i\in I}\big(\{i\}\times M_i\big).\]
Para facilitar a exposição, no que segue identificamos os conjuntos $M_i$ e $\{i\}\times M_i$, de modo que tratamos $M_i$ como um subconjunto de $M$.
\begin{itemize}
\item[(a)] Mostre que a união $\bigcup_{i\in I}\mathcal A_i$ é um atlas em $M$.
\item[(b)] Mostre que se $M$ é munido do atlas maximal que contém $\bigcup_{i\in I}\mathcal A_i$ então $(M_i,\mathcal A_i)$ é uma subvariedade
aberta de $M$, para todo $i\in I$.
\item[(c)] Mostre que o atlas maximal que contém $\bigcup_{i\in I}\mathcal A_i$ é o único atlas maximal em $M$ que faz de $(M_i,\mathcal A_i)$ uma subvariedade
aberta de $M$, para todo $i\in I$.
\end{itemize}
\end{exercise}

Dadas variedades diferenciáveis $M$, $N$ (resp., espaços topológicos $M$, $N$) então uma função $f:M\to N$ é dita um {\em difeomorfismo local\/}
(resp., um {\em homeomorfismo local}) se todo ponto de $M$ pertence a um aberto $U$ de $M$ tal que $f(U)$ é aberto em $N$ e
$f\vert_U:U\to f(U)$ é um difeomorfismo de classe $C^\infty$ (resp., um homeomorfismo). Evidentemente, todo difeomorfismo local é também um homeomorfismo local e todo
homeomorfismo local é uma aplicação aberta. Uma aplicação $f:M\to N$ é um difeomorfismo de classe $C^\infty$ (resp., um homeomorfismo) se e somente se
for um difeomorfismo local bijetor (resp., um homeomorfismo local bijetor).
\begin{exercise}\label{exe:quocdiflocal}
Sejam $M$, $N$, $P$ variedades diferenciáveis, $q:M\to N$ um difeomorfismo local sobrejetor e $f:N\to P$ uma função. Mostre que $f$ é de classe $C^\infty$
se e somente se $f\circ q$ é de classe $C^\infty$. Usando o resultado do Exercício~\ref{exe:Iddifeo}, conclua que se $M$ é uma variedade diferenciável, $N$ é um conjunto,
$q:M\to N$ é uma função sobrejetora e $\mathcal A$, $\mathcal A'$ são atlas maximais em $N$ que fazem de $q$ um difeomorfismo local então $\mathcal A=\mathcal A'$.
\end{exercise}

\begin{exercise}[cartas a valores em variedades]\label{exe:manifoldcharts}
Seja $M$ um conjunto e considere uma família $(\varphi_i:U_i\to N_i)_{i\in I}$, onde, para cada $i\in I$, $U_i$ é um subconjunto de $M$,
$N_i$ é uma variedade diferenciável e $\varphi_i:U_i\to N_i$ é uma bijeção. Assuma que para quaisquer $i,j\in I$, o conjunto $\varphi_i(U_i\cap U_j)$
seja aberto em $N_i$, o conjunto $\varphi_j(U_i\cap U_j)$ seja aberto em $N_j$ e a {\em função de transição\/}
$\varphi_j\circ\varphi_i^{-1}:\varphi_i(U_i\cap U_j)\to\varphi_j(U_i\cap U_j)$ seja um difeomorfismo de classe $C^\infty$. Assuma também
que $M=\bigcup_{i\in I}U_i$ e que todas as variedades $N_i$ tenham a mesma dimensão.
\begin{itemize}
\item[(a)] Mostre que existe um único atlas maximal $\mathcal A$ em $M$ que faz de $U_i$ um aberto em $M$ e de $\varphi_i$ um difeomorfismo de classe $C^\infty$,
para todo $i\in I$ (sugestão: mostre que a coleção formada pelas composições $\psi\circ\varphi_i$, com $i$ percorrendo $I$ e $\psi$ percorrendo o atlas maximal
de $N_i$ é um atlas em $M$ e seja $\mathcal A$ o atlas maximal que contém esse atlas).
\item[(b)] Considere a aplicação sobrejetora:
\[q:\bigcup_{i\in I}\big(\{i\}\times N_i\big)\ni(i,x)\longmapsto\varphi_i^{-1}(x)\in M,\]
sendo o domínio de $q$ munido do atlas maximal considerado no Exercício~\ref{exe:somavariedades}. Mostre que a propriedade exigida para o atlas maximal
$\mathcal A$ no item~(a) é equivalente à condição de que $q$ seja um difeomorfismo local (levando em conta o resultado do item~(b), note que você poderia
mostrar a unicidade do atlas maximal $\mathcal A$ em $M$ satisfazendo a propriedade exigida no item~(a) usando o resultado do Exercício~\ref{exe:quocdiflocal});
\item[(c)] mostre que a topologia $\tau_{\mathcal A}$ definida pelo atlas $\mathcal A$ do item~(a) é a {\em única\/} que faz de cada $U_i$ um subconjunto aberto de $M$
e de cada aplicação $\varphi_i$ um homeomorfismo (sugestão: use o resultado do item~(a) do Exercício~\ref{exe:cartopol}).
\end{itemize}

\end{exercise}

\begin{exercise}\label{exe:empurraatlas}
Se $M$ é um conjunto, $N$ é uma variedade diferenciável e se $\varphi:M\to N$ é uma função bijetora, mostre que existe
um único atlas maximal em $M$ que faz de $\varphi$ um difeomorfismo de classe $C^\infty$ (note que esse resultado
é o caso particular do resultado do Exercício~\ref{exe:manifoldcharts} em que a família de bijeções é unitária).
\end{exercise}

\begin{exercise}\label{exe:manifoldtop}
Seja $M$ uma variedade diferenciável. Mostre que:
\begin{itemize}
\item[(a)] $M$ é um espaço topológico T1 (i.e., os subconjuntos unitários de $M$ são fechados);
\item[(b)] $M$ é um espaço topológico localmente compacto, i.e., todo ponto de $M$ possui um sistema fundamental
de vizinhanças compactas (recorde que um {\em sistema fundamental de vizinhanças\/} de um ponto é um conjunto de vizinhanças
desse ponto tal que qualquer vizinhança do ponto contém uma que pertence a esse conjunto);
\item[(c)] se $M$ é um espaço topológico T2 (i.e., Hausdorff) então $M$ é um espaço topológico T3 (i.e., é T1 e todo ponto
possui um sistema fundamental de vizinhanças fechadas);
\item[(d)] $M$ satisfaz o primeiro axioma da enumerabilidade (i.e., todo ponto de $M$ possui um sistema fundamental
de vizinhanças enumerável);
\item[(e)] $M$ é localmente conexa por arcos (i.e., todo ponto de $M$ possui um sistema fundamental de vizinhanças
conexas por arcos).
\end{itemize}
\end{exercise}

Dados um espaço topológico $X$, um conjunto $Y$ e uma função $q:X\to Y$, recorde que a {\em topologia co-induzida\/}
ou {\em topologia quociente\/} definida por $q$ em $Y$ é a coleção:
\[\big\{U\subset Y:\text{$q^{-1}(U)$ é aberto em $X$}\big\}.\]
A topologia quociente é caracterizada pela seguinte propriedade universal: dado um espaço topológico $Z$ e uma função $f:Y\to Z$ então
$f$ é contínua se e somente se $f\circ q:X\to Z$ é contínua.
\begin{exercise}\label{exe:contopensur}
Sejam $X$, $Y$ espaços topológicos e $q:X\to Y$ uma função. Mostre que se $q$ é contínua, aberta e sobrejetora (esse é o caso,
por exemplo, se $q$ é um homeomorfismo local sobrejetor) então $Y$ está munido da topologia
quociente definida por $q$.
\end{exercise}

\begin{exercise}[reta com duas origens generalizada]\label{exe:duasorigens}
Seja $A$ um subconjunto aberto de $\R$ e considere em $\R\times\{0,1\}$ a relação de equivalência $\sim$ definida por:
\[(x,i)\sim(y,j)\Longleftrightarrow\text{$(x,i)=(y,j)$ ou $x=y\in A$}.\]
Seja $M=\big(\R\times\{0,1\}\big)/{\sim}$ e denote por $q:\R\times\{0,1\}\to M$ a aplicação quociente.
Para $i\in\{0,1\}$, considere a aplicação:
\[\varphi_i:q\big(\R\times\{i\}\big)\ni q(x,i)\longmapsto x\in\R.\]
\begin{itemize}
\item[(a)] Mostre que $\mathcal A=\{\varphi_0,\varphi_1\}$ é um atlas em $M$.
\item[(b)] Mostre que a topologia definida pelo atlas $\mathcal A$ coincide com a topologia quociente, onde $\R\times\{0,1\}$ é munido da topologia
produto da topologia usual de $\R$ pela topologia discreta de $\{0,1\}$ (sugestão:
mostre que se $M$ é munido da topologia $\tau_{\mathcal A}$ então $q$ é um homeomorfismo local e use o resultado do Exercício~\ref{exe:contopensur}).
\item[(c)] Mostre que se $A\ne\emptyset$ e $A\ne\R$ então $M$ não é Hausdorff (sugestão: considere um ponto pertencente ao fecho de $A$ que não está em $A$).
\item[(d)] Mostre que se $A\ne\emptyset$ então $M$ é conexa (sugestão: qual é a imagem inversa por $q$ de um subconjunto
aberto e fechado de $M$?).
\end{itemize}
\end{exercise}

\begin{exercise}
Defina o conjunto $M$ como no Exercício~\ref{exe:duasorigens}, sendo $A$ o intervalo $\left]-\infty,0\right]$ (que não é aberto em $\R$). Considere $M$ munido
da topologia quociente. Mostre que:
\begin{itemize}
\item[(a)] $M$ é homeomorfo ao subconjunto $C$ do plano $\R^2$ obtido pela união de três semi-retas distintas com a mesma origem $O$;
\item[(b)] $M$ não admite um atlas diferenciável que define sua topologia (sugestão: $M$ não é nem mesmo uma variedade topológica! Se fosse, o ponto
$O\in C$ possuiria uma vizinhança aberta $V$ relativa a $C$ homeomorfa a um intervalo aberto. Quantas componentes conexas tem $V\setminus\{O\}$?).
\end{itemize}
\end{exercise}

\begin{exercise}[reconstruindo uma variedade a partir de funções de transição]
Seja $(N_i)_{i\in I}$ uma família de conjuntos e, para cada $i,j\in I$, seja $\alpha_{ij}$ uma função bijetora cujo domínio $\Dom(\alpha_{ij})$
é um subconjunto de $N_i$ e a imagem $\Img(\alpha_{ij})$ é um subconjunto de $N_j$. Defina no conjunto:
\begin{equation}\label{eq:uniaodisjunta}
\bigcup_{i\in I}\big(\{i\}\times N_i\big)
\end{equation}
uma relação binária $\sim$ fazendo:
\[(i,x)\sim(j,y)\Longleftrightarrow\text{$x\in\Dom(\alpha_{ij})$ e $y=\alpha_{ij}(x)$},\quad i,j\in I,\ x\in N_i,\ y\in N_j.\]
Mostre que $\sim$ é uma relação de equivalência se e somente se valem as seguintes condições:
\begin{itemize}
\item[(i)] $\alpha_{ii}$ é a aplicação identidade do conjunto $N_i$, para todo $i\in I$;
\item[(ii)] $\alpha_{ji}$ é a função inversa de $\alpha_{ij}$, para todos $i,j\in I$;
\item[(iii)] para todos $i,j,k\in I$, dado $x$ no domínio de $\alpha_{ij}$ tal que $\alpha_{ij}(x)$ está no domínio de $\alpha_{jk}$ então
$x$ está no domínio de $\alpha_{ik}$ e:
\[\alpha_{ik}(x)=\alpha_{jk}\big(\alpha_{ij}(x)\big).\]
\end{itemize}
Assuma agora que $\sim$ é uma relação de equivalência (ou, equivalentemente, que as condições (i), (ii), (iii) sejam satisfeitas), denote por $M$
o quociente de \eqref{eq:uniaodisjunta} por $\sim$ e denote por:
\[q:\bigcup_{i\in I}\big(\{i\}\times N_i\big)\longrightarrow M\]
a aplicação quociente. Assuma também que cada $N_i$ seja uma variedade diferenciável, todas com a mesma dimensão,
e que as aplicações $\alpha_{ij}$ sejam difeomorfismos de classe $C^\infty$ entre subconjuntos abertos. Para cada $i\in I$, seja
$U_i=q\big(\{i\}\times N_i\big)\subset M$ e considere a aplicação:
\[\varphi_i:U_i\ni q(i,x)\longmapsto x\in N_i.\]
Use o resultado do Exercício~\ref{exe:manifoldcharts} para mostrar que existe um único atlas maximal em $M$ que faz de $U_i$ um aberto em $M$
e de $\varphi_i$ um difeomorfismo de classe $C^\infty$, para todo $i\in I$.
\end{exercise}

\end{section}

\begin{section}{Dia 16/08}

Se $M$ é uma variedade diferenciável e $x\in M$ é um ponto, consideramos o espaço tangente $T_xM$ construído da seguinte forma:
\[T_xM=\big\{\gamma:I\to M\mid\text{$I\subset\R$ aberto, $0\in I$, $\gamma$ de classe $C^\infty$ e $\gamma(0)=x$}\big\}/{\sim}\]
onde a relação de equivalência $\sim$ é definida fazendo $\gamma\sim\mu$ se e somente se para qualquer carta $\varphi:U\to\widetilde U$ (ou,
equivalentemente, se e somente se para {\em alguma\/} carta $\varphi:U\to\widetilde U$) com $x\in U$ temos $(\varphi\circ\gamma)'(0)=(\varphi\circ\mu)'(0)$.
A estrutura de espaço vetorial real em $T_xM$ é a única que torna a bijeção:
\begin{equation}\label{eq:varphiMx}
\varphi^M_x:T_xM\ni[\gamma]\longmapsto(\varphi\circ\gamma)'(0)\in\R^n
\end{equation}
um isomorfismo linear, para toda carta $\varphi:U\to\widetilde U$ com $x\in U$, onde $[\gamma]$ denota a classe de equivalência da curva $\gamma$.
Se $M$, $N$ são variedades diferenciáveis, $x\in M$ é um ponto e $f:M\to N$ é uma função de classe $C^\infty$ então a {\em diferencial\/}
$\dd f_x:T_xM\to T_{f(x)}N$ de $f$ no ponto $x$ (também denotada por $\dd f(x)$) é a aplicação linear definida por:
\begin{equation}\label{eq:defdf}
\dd f_x\big([\gamma]\big)=[f\circ\gamma],\quad[\gamma]\in T_xM.
\end{equation}

\begin{exercise}\label{exe:TxE}
Seja $E$ um espaço vetorial real de dimensão finita. Dado $x\in E$, mostre que a aplicação:
\[\sigma^E_x:T_xE\ni[\gamma]\longmapsto\gamma'(0)\in E\]
está bem definida e é um isomorfismo linear. A partir da próxima aula, o isomorfismo $\sigma^E_x$ será usado para identificar
$T_xE$ com $E$.
\end{exercise}

\begin{exercise}
Sejam $M$ uma variedade diferenciável, $U$ uma subvariedade aberta de $M$ e $x\in U$. Se $i:U\to M$ denota a aplicação inclusão (que é de classe $C^\infty$,
já que é a restrição a $U$ da aplicação identidade), mostre que a diferencial $\dd i_x:T_xU\to T_xM$ é um isomorfismo linear.
A partir da próxima aula, o isomorfismo $\dd i_x$ será usado para identificar $T_xU$ com $T_xM$.
\end{exercise}

\begin{exercise}\label{exe:dusual}
Sejam $E$, $F$ espaços vetoriais reais de dimensão finita, $U$ um aberto de $E$, $f:U\to F$ uma função de classe $C^\infty$ e $x\in U$.
Denote por $\dd f_x:T_xU\to T_{f(x)}F$ a diferencial de $f$ no ponto $x$ ``no sentido de variedades'' (i.e., definida como em \eqref{eq:defdf})
e por $\dd f_x^\diamondsuit:E\to F$ a diferencial de $f$ no ponto $x$ no sentido usado em cursos de Cálculo no $\R^n$. Se $i:U\to E$ denota a aplicação inclusão
e os isomorfismos $\sigma$ são definidos como no Exercício~\ref{exe:TxE}, mostre que o seguinte diagrama é comutativo:
\begin{equation}\label{eq:duasdiffs}
\vcenter{\xymatrix@C+5pt{%
&T_xU\ar[dl]_{\dd i_x}^{\scriptscriptstyle\cong}\ar[r]^-{\dd f_x}&T_{f(x)}F\ar[dd]^{\sigma^F_{f(x)}}_{\scriptscriptstyle\cong}\\
T_xE\ar[dr]_{\sigma^E_x}^{\scriptscriptstyle\cong}\\
&E\ar[r]_{\dd f_x^\diamondsuit}&F}}
\end{equation}
ou seja:
\[\dd f_x^\diamondsuit\circ\sigma^E_x\circ\dd i_x=\sigma^F_{f(x)}\circ\dd f_x.\]
O resultado desse exercício nos diz que, uma vez feitas as identificações:
\[T_xU\cong T_xE\cong E,\quad T_{f(x)}F\cong F,\]
a diferencial $\dd f_x$ coincide com a diferencial $\dd f^\diamondsuit_x$ (de modo que, a partir da próxima aula, a distinção
entre as duas será abandonada).
\end{exercise}

\begin{exercise}
Formule de forma cuidadosa e demonstre as seguintes afirmações:
\begin{itemize}
\item[(a)] se $M$, $N$ são variedades diferenciáveis, $f:M\to N$ é uma aplicação de classe $C^\infty$, $U\subset M$ é uma subvariedade
aberta e $x\in U$ é um ponto então, a menos da identificação $T_xU\cong T_xM$, as diferenciais:
\[\dd f_x:T_xM\longrightarrow T_{f(x)}N,\quad\dd(f\vert_U)_x:T_xU\longrightarrow T_{f(x)}N\]
coincidem (sugestão: $f\vert_U$ é a composição de $f$ com a aplicação inclusão de $U$ em $M$);
\item[(b)] se $M$, $N$ são variedades diferenciáveis, $f:M\to N$ é uma aplicação de classe $C^\infty$, $U\subset N$ é uma subvariedade
aberta que contém a imagem de $f$, $f^0:M\to U$ é a função que difere de $f$ apenas pelo contra-domínio e $x\in M$
é um ponto então, a menos da identificação $T_{f(x)}U\cong T_{f(x)}N$, as diferenciais:
\[\dd f_x:T_xM\longrightarrow T_{f(x)}N,\quad\dd f^0_x:T_xM\longrightarrow T_{f(x)}U\]
coincidem (sugestão: a aplicação $f$ é a composição da aplicação inclusão de $U$ em $N$ com a aplicação $f^0$);
\item[(c)] se $M$ é uma variedade diferenciável, $\varphi:U\subset M\to\widetilde U\subset\R^n$ é uma carta
e $x\in U$ é um ponto então, a menos das identificações:
\[T_xU\cong T_xM,\quad T_{\varphi(x)}\widetilde U\cong T_{\varphi(x)}\R^n\cong\R^n,\]
a diferencial $\dd\varphi_x:T_xU\to T_{\varphi(x)}\widetilde U$ coincide com o isomorfismo linear $\varphi^M_x:T_xM\to\R^n$
(veja \eqref{eq:varphiMx}).
\end{itemize}
\end{exercise}

\begin{exercise}\label{exe:itensdif}
Mostre que:
\begin{itemize}
\item[(a)] se $M$ é uma variedade diferenciável então a diferencial num ponto $x\in M$ da aplicação identidade
de $M$ é a aplicação identidade de $T_xM$;
\item[(b)] se $M$, $N$ são variedades diferenciáveis e se $f:M\to N$ é uma aplicação constante então
$f$ é de classe $C^\infty$ e a diferencial $\dd f_x$ é a aplicação nula, para todo $x\in M$;
\item[(c)] se $M$, $N$ são variedades diferenciáveis e $f:M\to N$ é um difeomorfismo de classe $C^\infty$ então $\dd f_x:T_xM\to T_{f(x)}N$ é um
isomorfismo linear cujo inverso é $\dd(f^{-1})_{f(x)}$;
\item[(d)] se $M$, $N$ são variedades diferenciáveis e $f:M\to N$ é um difeomorfismo local então $\dd f_x:T_xM\to T_{f(x)}N$ é
um isomorfismo linear, para todo $x\in M$.
\end{itemize}
\end{exercise}

\begin{exercise}[vetor tangente]
Se $M$ é uma variedade diferenciável, $I\subset\R$ é um aberto e $\gamma:I\to M$ é uma aplicação de classe $C^\infty$
então, para todo $t\in I$, definimos o {\em vetor tangente\/} $\gamma'(t)\in T_{\gamma(t)}M$ fazendo:
\[\gamma'(t)=\dd\gamma_t(1),\]
onde $\dd\gamma_t:T_tI\to T_{\gamma(t)}M$ é a diferencial de $\gamma$ no ponto $t$ e identificamos $T_tI$ com $T_t\R$ e
com $\R$. Mostre que:
\begin{itemize}
\item[(a)] se $N$ é outra variedade diferenciável e $f:M\to N$ é uma aplicação de classe $C^\infty$ então:
\[(f\circ\gamma)'(t)=\dd f_{\gamma(t)}\big(\gamma'(t)\big),\]
para todo $t\in I$;
\item[(b)] se $0\in I$ então o vetor tangente $\gamma'(0)$ é precisamente a classe de equivalência
$[\gamma]\in T_{\gamma(0)}M$ da curva $\gamma$ (sugestão: você deve verificar que o elemento de $T_0I$ que é identificado
com $1\in\R$ é a classe de equivalência da aplicação identidade de $I$).
\end{itemize}
\end{exercise}

\begin{exercise}[espaço tangente ao produto]\label{exe:esptanprod}
Sejam $M$, $N$ variedades diferenciáveis e considere a variedade produto $M\times N$. Denote por $\pi^1:M\times N\to M$,
$\pi^2:M\times N\to N$ as projeções. Dados $x\in M$, $y\in N$, mostre que a aplicação:
\[(\dd\pi^1_{(x,y)},\dd\pi^2_{(x,y)}):T_{(x,y)}(M\times N)\longrightarrow T_xM\times T_yN\]
é um isomorfismo linear.
\end{exercise}

\begin{exercise}
Sejam $M$, $N$, $P$ variedades diferenciáveis e considere a variedade produto $M\times N$. Seja $f=(f^1,f^2):P\longrightarrow M\times N$
uma aplicação de classe $C^\infty$. Dado $x\in P$, mostre que, usando a identificação:
\[T_{f(x)}(M\times N)\cong T_{f^1(x)}M\times T_{f^2(x)}N\]
dada pelo Exercício~\ref{exe:esptanprod}, então a diferencial de $f$ no ponto $x$ é igual a:
\[\dd f_x=(\dd f^1_x,\dd f^2_x):T_xP\longrightarrow T_{f^1(x)}M\times T_{f^2(x)}N.\]
\end{exercise}

\begin{exercise}\label{exe:difparcial}
Sejam $M$, $N$, $P$ variedades diferenciáveis e considere a variedade produto $M\times N$. Sejam $U$ uma subvariedade aberta de $M\times N$
e $f:U\to P$ uma função de classe $C^\infty$. Dado um ponto $x_0\in M$, podemos definir sobre o aberto:
\[U_{x_0}=\big\{y\in N:(x_0,y)\in U\big\}\]
a função:
\[f(x_0,\cdot):U_{x_0}\ni y\longmapsto f(x_0,y)\in P.\]
Mostre que a função $f(x_0,\cdot)$ é de classe $C^\infty$ e que sua diferencial num ponto $y_0\in U_{x_0}$ é igual à restrição
de $\dd f(x_0,y_0)$ a $T_{y_0}N$, onde identificamos $T_{y_0}U_{x_0}$ com $T_{y_0}N$, $T_{(x_0,y_0)}U$ com $T_{(x_0,y_0)}(M\times N)$, $T_{(x_0,y_0)}(M\times N)$ com
$T_{x_0}M\times T_{y_0}N$ como no Exercício~\ref{exe:esptanprod} e identificamos $T_{y_0}N$ com $\{0\}\times T_{y_0}N$ (sugestão:
$f(x_0,\cdot)$ é a composição de $f$ com a função $y\mapsto(x_0,y)$). A diferencial no ponto $y_0$ da função $f(x_0,\cdot)$ é normalmente
chamada a {\em diferencial parcial de $f$ com respeito à segunda variável\/} no ponto $(x_0,y_0)$ e é denotada por $\partial_2f(x_0,y_0)$
ou por $\frac{\partial f}{\partial y}(x_0,y_0)$.
\end{exercise}

Nos dois exercícios a seguir nós apresentaremos duas ``abordagens a\-xi\-o\-má\-ti\-cas'' para a noção de espaço tangente.
A primeira faz referência direta a cartas e ao espaço $\R^n$. A segunda não faz, mas é mais complicada.

Se $E$, $F$ são espaços vetoriais reais de dimensão finita, $U\subset E$ é um aberto, $f:U\to F$ é uma aplicação
de classe $C^\infty$ e $x\in U$ é um ponto, denotaremos (como no Exercício~\ref{exe:dusual}) nos dois
exercícios que seguem por $\dd f^\diamondsuit_x:E\to F$ a diferencial de $f$ no ponto $x$ no sentido usado em cursos de Cálculo no $\R^n$.
\begin{exercise}\label{exe:axesptan1}
Sejam dadas:
\begin{itemize}
\item uma regra que a cada variedade diferenciável $M$ e a cada ponto $x\in M$ associa um espaço vetorial real
$T_xM$;
\item uma regra que a cada variedade diferenciável $M$, a cada carta
\[\varphi:U\subset M\longrightarrow\widetilde U\subset\R^n\]
e a cada ponto $x\in U$ associa um isomorfismo linear:
\[\varphi^M_x:T_xM\longrightarrow\R^n.\]
\end{itemize}
A respeito do par de regras acima, considere a seguinte condição:
para toda variedade diferenciável $M$, para quaisquer cartas $\varphi:U\subset M\to\widetilde U\subset\R^n$,
$\psi:V\subset M\to\widetilde V\subset\R^n$ e para todo ponto $x\in U\cap V$, o seguinte diagrama é comutativo:
\begin{equation}\label{eq:phipsidif}
\vcenter{\xymatrix{%
&T_xM\ar[dl]_{\varphi^M_x}\ar[dr]^{\psi^M_x}\\
\R^n\ar[rr]_{\dd(\psi\circ\varphi^{-1})^\diamondsuit_{\varphi(x)}}&&\R^n}}
\end{equation}
ou seja:
\[\dd(\psi\circ\varphi^{-1})^\diamondsuit_{\varphi(x)}\circ\varphi^M_x=\psi^M_x.\]
O objetivo do exercício é mostrar a unicidade, a menos de isomorfismos, de um par de regras $(x,M)\mapsto T_xM$,
$(x,\varphi)\mapsto\varphi^M_x$ satisfazendo a condição acima (a existência é demonstrada pela apresentação
da construção de espaço tangente em termos de classes de equivalência de curvas). Mais precisamente, sejam:
\[(x,M)\longmapsto T_xM,\quad(x,\varphi)\longmapsto\varphi^M_x,\qquad
(x,M)\longmapsto\overline T_xM,\quad(x,\varphi)\longmapsto\bar\varphi^M_x,\]
pares de regras satisfazendo a condição acima. Mostre que existe uma regra que a cada variedade diferenciável
$M$ e a cada ponto $x\in M$ associa um isomorfismo linear:
\[\tau^M_x:T_xM\longrightarrow\overline T_xM\]
tal que, para toda variedade diferenciável $M$, para toda carta
\[\varphi:U\subset M\longrightarrow\widetilde U\subset\R^n\]
e para todo ponto $x\in U$, seja comutativo o diagrama:
\begin{equation}\label{eq:diagtauiso}
\vcenter{\xymatrix{%
T_xM\ar[dr]_{\varphi^M_x}\ar[rr]^{\tau^M_x}&&\overline T_xM\ar[dl]^{\bar\varphi^M_x}\\
&\R^n}}
\end{equation}
ou seja:
\[\bar\varphi^M_x\circ\tau^M_x=\varphi^M_x\]
(sugestão: o diagrama \eqref{eq:diagtauiso} já diz como $\tau^M_x$ deve ser definida! Você tem que mostrar
apenas que $\tau^M_x$ não depende de $\varphi$).
\end{exercise}

\begin{exercise}
Sejam dadas:
\begin{itemize}
\item uma regra que a cada variedade diferenciável $M$ e a cada ponto $x\in M$ associa um espaço vetorial real
$T_xM$;
\item uma regra que a cada par de variedades diferenciáveis $M$, $N$, a cada função $f:M\to N$ de classe $C^\infty$
e a cada ponto $x\in M$ associa uma aplicação linear $\dd f_x:T_xM\to T_{f(x)}N$;
\item uma regra que a cada espaço vetorial real de dimensão finita $E$ e a cada ponto $x\in E$ associa um isomorfismo
linear $\sigma^E_x:T_xE\to E$.
\end{itemize}
A respeito da trinca de regras acima, considere as seguintes condições:
\begin{itemize}
\item[(a)] se $M$, $N$, $P$ são variedades diferenciáveis, $f:M\to N$, $g:N\to P$ são funções de classe
$C^\infty$ e $x\in M$ é um ponto então:
\[\dd(g\circ f)_x=\dd g_{f(x)}\circ\dd f_x;\]
\item[(b)] se $M$ é uma variedade diferenciável, $\Id:M\to M$ denota a aplicação identidade e $x\in M$ é um ponto
então $\dd(\Id)_x$ é a aplicação identidade de $T_xM$;
\item[(c)] se $M$ é uma variedade diferenciável, $U\subset M$ é uma subvariedade aberta, $i:U\to M$ denota
a aplicação inclusão e $x\in U$ é um ponto então $\dd i_x:T_xU\to T_xM$ é um isomorfismo linear;
\item[(d)] se $E$, $F$ são espaços vetoriais reais de dimensão finita, $U\subset E$ é um aberto,
$f:U\to F$ é uma aplicação de classe $C^\infty$ e $x\in U$ é um ponto então o diagrama \eqref{eq:duasdiffs} comuta.
\end{itemize}
O objetivo do exercício é mostrar a unicidade, a menos de isomorfismos, de uma trinca de regras $(x,M)\mapsto T_xM$,
$(x,f)\mapsto\dd f_x$, $(x,E)\mapsto\sigma^E_x$ satisfazendo as condições (a)---(d) acima (a existência é demonstrada pela apresentação
da construção de espaço tangente em termos de classes de equivalência de curvas). Mais precisamente, sejam:
\begin{gather*}
(x,M)\longmapsto T_xM,\quad(x,f)\longmapsto\dd f_x,\quad(x,E)\longmapsto\sigma^E_x,\\
(x,M)\longmapsto\overline T_xM,\quad(x,f)\longmapsto\bar\dd f_x,\quad(x,E)\longmapsto\bar\sigma^E_x,
\end{gather*}
trincas de regras satisfazendo as condições (a)---(d) acima. Você deve mostrar que existe uma regra que a cada
variedade diferenciável $M$ e a cada ponto $x\in M$ associa um isomorfismo linear:
\[\tau^M_x:T_xM\longrightarrow\overline T_xM\]
tal que:
\begin{itemize}
\item[(i)] dadas variedades diferenciáveis $M$, $N$, uma função $f:M\to N$ de classe $C^\infty$ e um ponto $x\in M$
então o diagrama:
\[\xymatrix@C+10pt{%
T_xM\ar[r]^-{\dd f_x}\ar[d]_{\tau^M_x}&T_{f(x)}N\ar[d]^{\tau^N_{f(x)}}\\
\overline T_xM\ar[r]_-{\bar\dd f_x}&\overline T_{f(x)}N}\]
comuta, i.e., $\bar\dd f_x\circ\tau^M_x=\tau^N_{f(x)}\circ\dd f_x$;
\item[(ii)] dado um espaço vetorial real de dimensão finita $E$ e um ponto $x\in E$ então o diagrama:
\[\xymatrix{%
T_xE\ar[dr]^{\sigma^E_x}\ar[dd]_{\tau^E_x}\\
&E\\
\overline T_xE\ar[ur]_{\bar\sigma^E_x}}\]
comuta, i.e., $\bar\sigma^E_x\circ\tau^E_x=\sigma^E_x$.
\end{itemize}
A estratégia sugerida é a seguinte: comece mostrando
que a trinca de regras $(x,M)\mapsto T_xM$, $(x,f)\mapsto\dd f_x$, $(x,E)\mapsto\sigma^E_x$ dá origem a uma regra:
\[(x,\varphi)\longmapsto\varphi^M_x\]
como a do Exercício~\ref{exe:axesptan1}; você pode então usar o resultado daquele exercício
para obter a regra desejada $(x,M)\mapsto\tau^M_x$. Dadas uma variedade diferenciável $M$, uma carta $\varphi:U\subset M\to\widetilde U\subset\R^n$
e um ponto $x\in U$, defina $\varphi^M_x:T_xM\to\R^n$ de modo que o diagrama:
\[\xymatrix@C+10pt{%
T_xU\ar[d]_{\dd\varphi_x}\ar[r]^{\dd i_x}&T_xM\ar@.[dr]^{\varphi^M_x}\\
T_{\varphi(x)}\widetilde U\ar[r]_{\dd\tilde\imath_{\varphi(x)}}&T_{\varphi(x)}\R^n\ar[r]_-{\sigma^{\R^n}_{\varphi(x)}}&\R^n}\]
seja comutativo, onde $i:U\to M$, $\tilde\imath:\widetilde U\to\R^n$ denotam as aplicações inclusão. Para mostrar
que $\varphi^M_x$ é um isomorfismo, use as condições (a) e (b) e conclua que $\dd\varphi_x$ é um isomorfismo.
Para mostrar que o diagrama \eqref{eq:phipsidif} comuta, a seguinte estratégia é conveniente: em primeiro lugar, mostre que se $W$
é um aberto contido em $U$ e $\varphi\vert_W:W\to\varphi(W)$ denota a restrição da carta $\varphi$ a $W$, então:
\[(\varphi\vert_W\!)^M_x=\varphi^M_x,\]
para todo $x\in W$. Em vista dessa última igualdade, você pode supor sem perda de generalidade, para provar
a comutatividade de \eqref{eq:phipsidif}, que as cartas $\varphi$ e $\psi$ possuam o mesmo domínio. Você vai precisar então
aplicar a condição (d) para a função de transição $\psi\circ\varphi^{-1}$.
Finalmente, você precisa mostrar que
a regra $(x,M)\mapsto\tau^M_x$ obtida usando o resultado do Exercício~\ref{exe:axesptan1} satisfaz as condições (i) e (ii). Para isso, é conveniente
mostrar primeiro que as regras $(x,f)\mapsto\dd f_x$, $(x,E)\mapsto\sigma^E_x$ podem ser pensadas como se fossem definidas a partir da regra
$(x,\varphi)\mapsto\varphi^M_x$, da seguinte forma: dadas variedades diferenciáveis $M$, $N$, uma função $f:M\to N$ de classe $C^\infty$, um ponto $x\in M$
e cartas $\varphi:U\subset M\to\widetilde U\subset\R^m$, $\psi:V\subset N\to\widetilde V\subset\R^n$ com $x\in U$ e $f(U)\subset V$ então o diagrama:
\[\xymatrix@C+30pt{%
T_xM\ar[r]^-{\dd f_x}\ar[d]_{\varphi^M_x}&T_{f(x)}N\ar[d]^{\varphi^N_{f(x)}}\\
\R^m\ar[r]_{\dd(\psi\circ f\circ\varphi^{-1})^\diamondsuit_{\varphi(x)}}&\R^n}\]
comuta. Além do mais, dados um espaço vetorial real $E$ de dimensão finita e um isomorfismo linear $\varphi:E\to\R^n$ (que é uma carta em $E$)
então, para todo $x\in E$, o diagrama:
\[\xymatrix{%
T_xE\ar[dr]_{\varphi^E_x}\ar[rr]^{\sigma^E_x}&&E\ar[dl]^\varphi\\
&\R^n}\]
comuta.
\end{exercise}

\end{section}

\begin{section}{Dia 18/08}

Recordamos os seguintes enunciados dos cursos de Cálculo no $\R^n$.

\begin{teo}[da função inversa]
Seja $f:U\to\R^n$ uma função de classe $C^\infty$ onde $U$ é um subconjunto aberto de $\R^n$. Se $x_0\in U$
é tal que a diferencial $\dd f(x_0):\R^n\to\R^n$ seja um isomorfismo então existe um aberto $V\subset\R^n$ com
$x_0\in V\subset U$, tal que $f(V)$ é aberto em $\R^n$ e $f\vert_V:V\to f(V)$ é um difeomorfismo de classe $C^\infty$.
\end{teo}

\begin{teo}[da função implícita]
Seja $f:U\to\R^n$ uma função de classe $C^\infty$, onde $U$ é um subconjunto aberto de $\R^m\times\R^n$. Sejam
$(x_0,y_0)\in U$ e $c=f(x_0,y_0)$. Se a diferencial $\frac{\partial f}{\partial y}(x_0,y_0):\R^n\to\R^n$
de $f$ com respeito à segunda variável no ponto $(x_0,y_0)$ é um isomorfismo então existem um subconjunto aberto
$V$ de $\R^m$, um subconjunto aberto $W$ de $\R^n$ e uma função $\alpha:V\to W$ de classe $C^\infty$ tais que
$x_0\in V$, $y_0\in W$, $V\times W\subset U$ e, para quaisquer $x\in V$, $y\in W$, temos:
\[f(x,y)=c\Longleftrightarrow y=\alpha(x).\]
\end{teo}

\begin{teo}[forma local das imersões]
Seja $f:U\to\R^n$ uma função de classe $C^\infty$, onde $U$ é um subconjunto aberto de $\R^m$. Seja $x_0\in U$
e suponha que $f$ seja uma imersão no ponto $x_0$ (i.e., que $\dd f(x_0)$ seja injetora). Então existem subconjuntos
abertos $V$, $\widetilde V$ de $\R^n$, um difeomorfismo $\phi:V\to\widetilde V$ de classe $C^\infty$ e um subconjunto
aberto $U'$ de $\R^m$ tais que $x_0\in U'\subset U$, $f(U')\subset V$ e:
\[\phi\big(f(z)\big)=(z,0)\in\R^m\times\R^{n-m}\cong\R^n,\]
para todo $z\in U'$.
\end{teo}

\begin{teo}[forma local das submersões]
Seja $f:U\to\R^n$ uma função de classe $C^\infty$, onde $U$ é um subconjunto aberto de $\R^m$. Seja $x_0\in U$
e suponha que $f$ seja uma submersão no ponto $x_0$ (i.e., que\/ $\dd f(x_0)$ seja sobrejetora). Então existem
subconjuntos abertos $V$, $\widetilde V$ de $\R^m$ e um difeomorfismo $\phi:V\to\widetilde V$ de classe $C^\infty$
tais que $x_0\in V\subset U$ e:
\[f\big(\phi^{-1}(z,z')\big)=z,\]
para todo $(z,z')\in\widetilde V\subset\R^m\cong\R^n\times\R^{m-n}$.
\end{teo}

\begin{teo}[do posto]
Seja $f:U\to\R^n$ uma função de classe $C^\infty$, onde $U$ é um subconjunto aberto de $\R^m$. Suponha que o posto
de $\dd f(x)$ seja igual a um certo número natural $r$, para todo ponto $x\in U$. Dado $x_0\in U$, existem então
subconjuntos abertos $V$, $\widetilde V$ de $\R^m$, subconjuntos abertos $W$, $\widetilde W$ de $\R^n$ e
difeomorfismos:
\[\phi:V\longrightarrow\widetilde V,\quad\lambda:W\longrightarrow\widetilde W\]
de classe $C^\infty$ tais que $x_0\in V\subset U$, $f(U)\subset W$ e:
\[\lambda\big[f\big(\phi^{-1}(z,z')\big)\big]=(z,0)\in\R^r\times\R^{n-r}\cong\R^n,\]
para todo $(z,z')\in\widetilde V\subset\R^m\cong\R^r\times\R^{m-r}$.
\end{teo}

Nos Exercícios~\ref{exe:invfunc} a \ref{exe:posto} pedimos para você generalizar para variedades os resultados dos teoremas acima
(a generalização do teorema da função inversa foi feita em aula).
\begin{exercise}[teorema da função inversa]\label{exe:invfunc}
Sejam $M$, $N$ variedades diferenciáveis e $f:M\to N$ uma função de classe $C^\infty$. Dado $x_0\in M$ tal que
$\dd f(x_0):T_{x_0}M\to T_{f(x_0)}N$ é um isomorfismo, mostre que existe um aberto $V$ em $M$ contendo $x_0$ tal que $f(V)$
é aberto em $N$ e $f\vert_V:V\to f(V)$ é um difeomorfismo de classe $C^\infty$.
\end{exercise}

\begin{exercise}[teorema da função implícita]
Sejam $M$, $N$, $P$ variedades diferenciáveis e $f:U\to P$ uma função de classe $C^\infty$, onde $U$ é uma subvariedade
aberta da variedade produto $M\times N$. Sejam $(x_0,y_0)\in U$ e $c=f(x_0,y_0)\in P$. Assuma que a diferencial parcial
$\frac{\partial f}{\partial y}(x_0,y_0):T_{y_0}N\to T_cP$ seja um isomorfismo (veja Exercício~\ref{exe:difparcial}).
Mostre que existem um subconjunto aberto $V$ de $M$, um subconjunto aberto $W$ de $N$ e uma função $\alpha:V\to W$
de classe $C^\infty$ tais que $x_0\in V$, $y_0\in W$, $V\times W\subset U$ e, para quaisquer $x\in V$, $y\in W$, temos:
\[f(x,y)=c\Longleftrightarrow y=\alpha(x).\]
\end{exercise}

\begin{exercise}[forma local das imersões]
Sejam $M$, $N$ variedades diferenciáveis e $f:M\to N$ uma função de classe $C^\infty$. Seja $x_0\in M$ e suponha
que $f$ seja uma imersão no ponto $x_0$ (i.e., que $\dd f(x_0)$ seja injetora). Mostre que, dada uma carta
$\varphi:U\subset M\to\widetilde U\subset\R^m$ com $x_0\in U$ então existem uma carta $\psi:V\subset N\to\widetilde V\subset\R^n$
e um subconjunto aberto $U'$ de $M$ com $x_0\in U'\subset U$, $f(U')\subset V$ e:
\[\psi\big[f\big(\varphi^{-1}(z)\big)\big]=(z,0)\in\R^m\times\R^{n-m}\cong\R^n,\]
para todo $z\in\varphi(U')$.
\end{exercise}

\begin{exercise}[forma local das submersões]
Sejam $M$, $N$ variedades diferenciáveis e $f:M\to N$ uma função de classe $C^\infty$. Seja $x_0\in M$ e suponha
que $f$ seja uma submersão no ponto $x_0$ (i.e., que $\dd f(x_0)$ seja sobrejetora). Mostre que, dada uma carta
$\psi:V\subset N\to\widetilde V\subset\R^n$ com $f(x_0)\in V$ então existe uma carta $\varphi:U\subset M\to\widetilde U\subset\R^m$
com $x_0\in U$, $f(U)\subset V$ e:
\[\psi\big[f\big(\varphi^{-1}(z,z')\big)\big]=z,\]
para todo $(z,z')\in\widetilde U\subset\R^m\cong\R^n\times\R^{m-n}$.
\end{exercise}

\begin{exercise}[teorema do posto]\label{exe:posto}
Sejam $M$, $N$ variedades diferenciáveis e $f:M\to N$ uma função de classe $C^\infty$. Suponha que o posto
de $\dd f(x)$ seja igual a um certo número natural $r$, para todo ponto $x\in M$. Dado $x_0\in M$, mostre que existem
cartas:
\[\varphi:U\subset M\longrightarrow\widetilde U\subset\R^m,\quad
\psi:V\subset N\longrightarrow\widetilde V\subset\R^n\]
tais que $x_0\in U$, $f(U)\subset V$ e:
\[\psi\big[f\big(\varphi^{-1}(z,z')\big)\big]=(z,0)\in\R^r\times\R^{n-r}\cong\R^n,\]
para todo $(z,z')\in\widetilde U\subset\R^m\cong\R^r\times\R^{m-r}$.
\end{exercise}

\begin{exercise}[recíproca do item~(d) do Exercício~\ref{exe:itensdif}]
Sejam $M$, $N$ variedades diferenciáveis e $f:M\to N$ uma função de classe $C^\infty$. Mostre que $f$ é um difeomorfismo
local se e somente se $\dd f(x)$ é um isomorfismo, para todo $x\in M$ (sugestão: use o teorema da função inversa).
\end{exercise}

\begin{exercise}\label{exe:imerlochomeo}
Sejam $M$, $N$ variedades diferenciáveis, $f:M\to N$ uma função de classe $C^\infty$ e suponha que $f$ seja uma imersão
num ponto $x_0\in M$. Mostre que existe um aberto $U$ em $M$ tal que $x_0\in U$ e $f\vert_U:U\to f(U)$ seja um homeomorfismo,
onde $f(U)$ é munido da topologia induzida por $N$ (sugestão: use a forma local das imersões, notando que a ``imersão padrão''
$z\mapsto(z,0)$ é um homeomorfismo sobre sua imagem). Note que {\em não está se dizendo\/} que se $f$ é uma imersão
então $f$ é um homeomorfismo local,
já que, em geral, $f(U)$ não será aberto em $N$ (em alguns casos, não será aberto nem mesmo na imagem de $f$, como veremos).
\end{exercise}

\begin{exercise}\label{exe:submersaberta}
Sejam $M$, $N$ variedades diferenciáveis e $f:M\to N$ uma função de classe $C^\infty$. Mostre que se $f$ é uma submersão
então $f$ é uma aplicação aberta (sugestão: use a forma local das submersões, notando que a ``submersão padrão''
$(z,z')\mapsto z$ é aberta).
\end{exercise}

\begin{exercise}
Sejam $E$, $F$ espaços vetoriais reais de dimensão finita e $\Lin(E,F)$ o espaço das transformações lineares
de $E$ para $F$. Mostre que a função:
\[\Lin(E,F)\ni T\longmapsto\posto(T)\]
é semi-contínua inferiormente\footnote{%
Para uma função $f$ definida num espaço topológico e tomando valores em $\R$ (ou tomando valores num conjunto totalmente
ordenado qualquer), diz-se que $f$ é {\em semi-contínua inferiormente\/} num ponto $x$ de seu domínio se dado
$c<f(x)$ então existe uma vizinhança de $x$ tal que $f(x')>c$, para todo $x'$ pertencente a essa vizinhança.},
i.e., se $T\in\Lin(E,F)$ tem posto $r$ então existe uma vizinhança de $T$ em $\Lin(E,F)$ tal que qualquer $T'$ pertencente
a essa vizinhança tem posto maior ou igual a $r$. Conclua que o conjunto das transformações lineares $T\in\Lin(E,F)$
de {\em posto máximo\/} (i.e., cujo posto é o maior possível, ou seja, igual a $\min\{\Dim(E),\Dim(F)\}$)
é aberto em $\Lin(E,F)$ (note que, se $\Dim(E)\le\Dim(F)$, então $T:E\to F$ tem posto máximo se e somente se $T$ for injetora
e, se $\Dim(E)\ge\Dim(F)$, então $T:E\to F$ tem posto máximo se e somente se $T$ for sobrejetora).
Conclua também que se $M$, $N$ são variedades diferenciáveis e $f:M\to N$ é uma função de classe $C^\infty$ então o conjunto
dos pontos $x\in M$ em que $\dd f(x)$ tem posto máximo é aberto em $M$ (sugestão: considere uma representação
$\tilde f=\psi\circ f\circ\varphi^{-1}$ de $f$ usando sistemas coordenadas $\varphi$, $\psi$ e note que a função
$z\mapsto\dd\tilde f(z)$ é contínua).
\end{exercise}

\end{section}

\begin{section}{Dia 23/08}

Os Exercícios de \ref{exe:secaolocal} a \ref{exe:uniqembed} foram resolvidos em aula.

\medskip

Sejam $M$, $N$ variedades diferenciáveis e $q:M\to N$ uma função de classe $C^\infty$. Uma {\em seção local\/} de $q$ é uma função $s:U\to M$ de classe
$C^\infty$, onde $U$ é um aberto de $N$, tal que $q\circ s$ é a aplicação identidade de $U$.
\begin{exercise}\label{exe:secaolocal}
Sejam $M$, $N$ variedades diferenciáveis e $q:M\to N$ uma função de classe $C^\infty$. Dado $x\in M$, mostre que $q$ é uma submersão no ponto $x$
se e somente se existe uma seção local $s:U\to M$ de $q$ tal que $q(x)\in U$ e $s\big(q(x)\big)=x$ (sugestão: se a seção local $s$ existe, diferencie
a igualdade $q\circ s=\Id_U$ no ponto $x$ para ver que $\dd q_x$ tem uma inversa à direita e é portanto sobrejetora. Reciprocamente, se $q$ é uma submersão
no ponto $x$, use a forma local das submersões e note que a ``submersão padrão'' $(z,z')\mapsto z$ admite seções locais).
\end{exercise}

Sejam $M$, $N$ variedades diferenciáveis e $\phi:M\to N$ uma função de classe $C^\infty$. Uma {\em retração local\/} para $\phi$ é uma função
$r:V\to U$ de classe $C^\infty$ tal que $U$ é aberto em $M$, $V$ é aberto em $N$, $\phi(U)\subset V$ e $r\circ\phi\vert_U$ é a aplicação identidade de $U$.
\begin{exercise}\label{exe:retracaolocal}
Sejam $M$, $N$ variedades diferenciáveis e $\phi:M\to N$ uma função de classe $C^\infty$. Dado $x\in M$, mostre que $\phi$ é uma imersão no ponto $x$
se e somente se existe uma retração local $r:V\to U$ para $\phi$ tal que $x\in U$ (sugestão: se a retração local $r$ existe, diferencie a igualdade
$r\circ\phi\vert_U=\Id_U$ no ponto $x$ para ver que $\dd\phi_x$ tem uma inversa à esquerda e é portanto injetora. Reciprocamente, se $\phi$ é uma imersão no
ponto $x$, use a forma local das imersões e note que a ``imersão padrão'' $z\mapsto(z,0)$ admite uma retração local).
\end{exercise}

\begin{exercise}\label{exe:passquoc}
Sejam $M$, $N$, $P$ variedades diferenciáveis e $q:M\to N$ uma submersão sobrejetora de classe $C^\infty$. Dada uma função $f:N\to P$, mostre que
$f$ é de classe $C^\infty$ se e somente se $f\circ q$ é de classe $C^\infty$ (sugestão: use o resultado do Exercício~\ref{exe:secaolocal}).
\[\xymatrix{%
M\ar[dr]^{f\circ q}\ar[d]_q\\
N\ar[r]_f&P}\]
\end{exercise}

\begin{exercise}\label{exe:lowercounterdomain}
Sejam $M$, $N$, $P$ variedades diferenciáveis e $\phi:M\to N$ uma imersão de classe $C^\infty$. Se $f:P\to M$ é uma função contínua, mostre que
$f$ é de classe $C^\infty$ se e somente se $\phi\circ f$ é de classe $C^\infty$ (sugestão: use o resultado do Exercício~\ref{exe:retracaolocal}).
Conclua que se $\phi$ é um mergulho então, para {\em qualquer\/} função $f:P\to M$ (que não precisa ser assumida contínua {\it a priori}) temos
que $f$ é de classe $C^\infty$ se e somente se $\phi\circ f$ é de classe $C^\infty$.
\[\xymatrix{%
&N\\
P\ar[ur]^{\phi\circ f}\ar[r]_f&M\ar[u]_\phi}\]
\end{exercise}

\begin{exercise}
Sejam $M$ uma variedade diferenciável e $\sim$ uma relação de equivalência em $M$. Se $\mathcal A$, $\mathcal A'$ são atlas maximais no conjunto quociente
$M/{\sim}$ que fazem da aplicação quociente $q:M\to M/{\sim}$ uma submersão de classe $C^\infty$, mostre que $\mathcal A=\mathcal A'$ (sugestão:
mostre que a aplicação identidade de $(M/{\sim},\mathcal A)$ para $(M/{\sim},\mathcal A')$ é um difeomorfismo de classe $C^\infty$ usando o resultado
do Exercício~\ref{exe:passquoc}).
\end{exercise}

\begin{exercise}\label{exe:uniqembed}
Sejam $M$ uma variedade diferenciável e $N$ um subconjunto de $M$. Sejam $\mathcal A$, $\mathcal A'$ atlas maximais em $N$. Mostre que:
\begin{itemize}
\item[(a)] se $\mathcal A$ e $\mathcal A'$ fazem da aplicação inclusão $i:N\to M$ um mergulho de classe $C^\infty$ então $\mathcal A=\mathcal A'$;
\item[(b)] se $\mathcal A$ e $\mathcal A'$ definem a mesma topologia em $N$ e fazem da inclusão $i:N\to M$ uma imersão de classe $C^\infty$ então
$\mathcal A=\mathcal A'$ (sugestão para ambos os itens: mostre que a aplicação identidade de $(N,\mathcal A)$ para $(N,\mathcal A')$ é um difeomorfismo
de classe $C^\infty$ usando o resultado do Exercício~\ref{exe:lowercounterdomain}).
\end{itemize}
\end{exercise}

\begin{exercise}
Sejam $M$, $N$ variedades diferenciáveis e $q:M\to N$ uma submersão sobrejetora de classe $C^\infty$ (por exemplo, $N$ poderia o conjunto
quociente $M/{\sim}$, onde $\sim$ é uma relação de equivalência em $M$, e $N$ é munida de um atlas maximal que faz da aplicação quociente $q$ uma submersão de classe
$C^\infty$). Mostre que a topologia definida pelo atlas de $N$ coincide com a topologia quociente definida por $q$ (sugestão: use o resultado dos
Exercícios~\ref{exe:contopensur} e \ref{exe:submersaberta}).
\end{exercise}

\begin{exercise}\label{exe:imerlocmerg}
Sejam $M$, $N$ variedades diferenciáveis e $f:M\to N$ uma imersão de classe $C^\infty$. Mostre que todo ponto de $M$ pertence a um aberto $U$
relativo a $M$ tal que $f\vert_U:U\to N$ é um mergulho (sugestão: isso segue diretamente do resultado do Exercício~\ref{exe:imerlochomeo}).
\end{exercise}

\begin{exercise}\label{exe:condmergulho}
Sejam $M$, $N$ variedades diferenciáveis e $f:M\to N$ uma função. Suponha que para cada $x\in M$ exista um aberto $V$ em $N$ contendo $f(x)$ tal que
$f^{-1}(V)$ é aberto em $M$ e tal que a restrição:
\[f\vert_{f^{-1}(V)}:f^{-1}(V)\longrightarrow N\]
seja um mergulho de classe $C^\infty$. Mostre que $f$ é um mergulho de classe $C^\infty$ (veja só: em vista do resultado do Exercício~\ref{exe:imerlocmerg},
toda imersão é localmente um mergulho, de modo que se $f$ é localmente um mergulho não segue que $f$ é um mergulho. No entanto, se $f$ é ``localmente
um mergulho'' no sentido mais forte considerado no enunciado deste exercício, então segue que $f$ é mesmo um mergulho!).
\end{exercise}

\begin{exercise}\label{exe:imagemimersaomerg}
Sejam $M$, $N_0$ variedades diferenciáveis e $f:N_0\to M$ uma função injetora. Suponha $N=f(N_0)$ munido do único atlas maximal
$\mathcal A$ que faz da bijeção $f:N_0\to N$ um difeomorfismo de classe $C^\infty$ (veja Exercício~\ref{exe:empurraatlas}). Mostre que $(N,\mathcal A)$
é uma subvariedade imersa de $M$ (resp., uma subvariedade mergulhada de $M$) se e somente se $f:N_0\to M$ é uma imersão (resp., um mergulho) de classe $C^\infty$ (sugestão:
a aplicação inclusão de $(N,\mathcal A)$ em $M$ é a composição do inverso do difeomorfismo $f:N_0\to(N,\mathcal A)$ com $f:N_0\to M$).
\[\xymatrix{%
&M\\
N_0\ar[ur]^f\ar[r]_f^{\scriptscriptstyle\cong}&N\ar[u]_{\text{inclusão}}}\]
\end{exercise}

\begin{exercise}
Seja $(N,\mathcal A)$ uma subvariedade imersa de uma variedade diferenciável $M$. Mostre que:
\begin{itemize}
\item[(a)] a topologia em $N$ definida pelo atlas $\mathcal A$ é mais fina do que (i.e., contém a) topologia induzida por $M$ em $N$ (sugestão:
a aplicação inclusão de $(N,\mathcal A)$ em $M$ é de classe $C^\infty$ e portanto contínua);
\item[(b)] $(N,\mathcal A)$ é uma subvariedade mergulhada de $M$ se e somente se a topologia definida por $\mathcal A$ em $N$ {\em coincide\/}
com a topologia induzida por $M$ em $N$.
\end{itemize}
\end{exercise}

\begin{exercise}\label{exe:maisimerso}
Considere a variedade $N_0$ obtida pela união disjunta das variedades $\R$ e $\R\setminus\{0\}$ (veja Exercício~\ref{exe:somavariedades});
como conjunto, $N_0$ é:
\[N_0=\big(\R\times\{1\}\!\big)\cup\big((\R\setminus\{0\})\times\{2\}\!\big).\]
Defina $f:N_0\to\R^2$, $g:N_0\to\R^2$ fazendo:
\[f(t,1)=(t,0),\quad f(s,2)=(0,s),\qquad g(t,1)=(0,t),\quad g(s,2)=(s,0),\]
para todos $t\in\R$, $s\in\R\setminus\{0\}$.
\begin{itemize}
\item[(a)] Mostre que:
\[f(N_0)=g(N_0)=(\R\times\{0\})\cup(\{0\}\times\R);\]
defina $N=f(N_0)=g(N_0)$.
\item[(b)] Mostre que $f$ e $g$ são imersões injetoras de classe $C^\infty$. Sejam $\mathcal A$ o atlas maximal em $N$ que faz de
$f:N_0\to N$ um difeomorfismo de classe $C^\infty$ e $\mathcal A'$ o atlas maximal em $N$ que faz de $g:N_0\to N$ um difeomorfismo de classe $C^\infty$.
Conclua que $(N,\mathcal A)$ e $(N,\mathcal A')$ são subvariedades imersas de $\R^2$.
\item[(c)] Descreva um sistema fundamental de vizinhanças para a origem em $N$ com respeito à topologia definida pelo atlas $\mathcal A$,
com respeito à topologia definida pelo atlas $\mathcal A'$ e com respeito à topologia induzida por $\R^2$ em $N$. Conclua que essas três topologias
são diferentes e que $(N,\mathcal A)$, $(N,\mathcal A')$ não são subvariedades mergulhadas de $\R^2$.
\item[(d)] Descreva explicitamente a função representada pela flecha horizontal no diagrama comutativo:
\[\xymatrix{%
&\R^2\\
N_0\ar[ur]^f\ar[rr]&&N_0\ar[ul]_g}\]
Note que essa função não é um homeomorfismo e baseie nesse fato um outro argumento mostrando que os atlas $\mathcal A$ e $\mathcal A'$ definem topologias diferentes
no conjunto $N$.
\end{itemize}
\end{exercise}

\end{section}

\begin{section}{Dia 25/08}

Seja $M$ uma variedade diferenciável. Recorde que entendemos por uma {\em subvariedade imersa\/} de $M$ (resp., uma {\em subvariedade mergulhada\/} de $M$)
uma variedade diferenciável $(N,\mathcal A)$ tal que $N$ é um subconjunto de $M$ e a aplicação inclusão de $(N,\mathcal A)$ em $M$ é uma imersão (resp.,
um mergulho) de classe $C^\infty$. {\em O termo ``subvariedade'' (sem qualificação) significará sempre ``subvariedade mergulhada''}.
Se $N$ é um subconjunto de $M$ então, em vista do resultado do item~(a) do Exercício~\ref{exe:uniqembed}, existe no máximo
um atlas maximal $\mathcal A$ em $N$ fazendo de $(N,\mathcal A)$ uma subvariedade de $M$. Diremos então que um subconjunto $N$ de $M$ é uma {\em subvariedade\/}
de $M$ quando {\em existe\/} um atlas maximal $\mathcal A$ em $N$ fazendo de $(N,\mathcal A)$ uma subvariedade de $M$ e nesse caso
identificamos o conjunto $N$ com a variedade $(N,\mathcal A)$.
Para subvariedades imersas, não é uma boa idéia identificar a subvariedade imersa $(N,\mathcal A)$ com o conjunto subjacente $N$, já que não há unicidade
para o atlas maximal $\mathcal A$.

\begin{exercise}\label{exe:mergrestated}
Sejam $M$, $N_0$ variedades diferenciáveis e seja $f:N_0\to M$ uma função. Mostre que $f$ é um mergulho de classe $C^\infty$
se e somente se $N=f(N_0)$ é uma subvariedade de $M$ e $f:N_0\to N$ é um difeomorfismo de classe $C^\infty$ (sugestão:
observe que esse enunciado é nada mais que uma reformulação da parte do Exercício~\ref{exe:imagemimersaomerg} que se refere a mergulhos,
usando as convenções terminológicas introduzidas no início desta seção).
\end{exercise}

\begin{exercise}\label{exe:difeosubvar}
Sejam $M$, $M'$ variedades diferenciáveis e $\varphi:M\to M'$ um difeomorfismo de classe $C^\infty$. Se $N$ é uma subvariedade de $M$, mostre
que $\varphi(N)$ é uma subvariedade de $M'$ e que $\varphi\vert_N:N\to\varphi(N)$ é um difeomorfismo de classe $C^\infty$
(sugestão: $\varphi\vert_N:N\to M'$ é um mergulho).
\end{exercise}

\begin{exercise}\label{exe:locmanifold}
Sejam $M$ um conjunto, $M=\bigcup_{i\in I}U_i$ uma cobertura de $M$ e $\mathcal A_i$ um atlas maximal no conjunto $U_i$, de modo que todas as variedades
diferenciáveis $(U_i,\mathcal A_i)$ tenham a mesma dimensão. Suponha que, para todos $i,j\in I$, $U_i\cap U_j$ seja aberto em $U_i$ com respeito à topologia
definida pelo atlas $\mathcal A_i$, que $U_i\cap U_j$ seja aberto em $U_j$ com respeito à topologia definida pelo atlas $\mathcal A_j$ e que os atlas
maximais induzidos em $U_i\cap U_j$ por $\mathcal A_i$ e por $\mathcal A_j$ sejam iguais (recorde Exercício~\ref{exe:subvaraberta}
para notação):
\begin{equation}\label{eq:atlasAiAj}
\mathcal A_i\vert_{U_i\cap U_j}=\mathcal A_j\vert_{U_i\cap U_j},\quad i,j\in I.
\end{equation}
Mostre que:
\begin{itemize}
\item[(a)] existe um único atlas maximal $\mathcal A$ em $M$ que faz de $(U_i,\mathcal A_i)$ uma subvariedade aberta de $(M,\mathcal A)$, para todo $i\in I$
(sugestão 1: use o resultado do item~(a) do Exercício~\ref{exe:manifoldcharts} com $N_i=(U_i,\mathcal A_i)$ e $\varphi_i$ igual à aplicação identidade de $U_i$;
sugestão 2: mostre que $\bigcup_{i\in I}\mathcal A_i$ é um atlas em $M$ e tome $\mathcal A$ como sendo o atlas maximal que contém esse atlas. Para mostrar
a compatibilidade de $\varphi\in\mathcal A_i$ com $\psi\in\mathcal A_j$, observe que as restrições de $\varphi$ e de $\psi$ a $\Dom(\varphi)\cap\Dom(\psi)$ pertencem
ao atlas \eqref{eq:atlasAiAj}. Note que, uma vez resolvido este exercício usando a segunda sugestão, você também poderia
obter o resultado do item~(a) do Exercício~\ref{exe:manifoldcharts} como corolário!);
\item[(b)] a topologia $\tau_{\mathcal A}$ definida pelo atlas $\mathcal A$ do item~(a) é a única que faz de $U_i$ um subconjunto aberto de $M$ e que
induz a topologia $\tau_{\mathcal A_i}$ em $U_i$, para todo $i\in I$ (sugestão: para a unicidade,
use o resultado do item~(a) do Exercício~\ref{exe:cartopol}).
\end{itemize}
\end{exercise}

\begin{exercise}[localidade da noção de subvariedade]\label{exe:subvarlocal}
Sejam $M$ uma variedade diferenciável e $N$ um subconjunto de $M$, munido da topologia induzida por $M$. Mostre que:
\begin{itemize}
\item[(a)] se $N$ é uma subvariedade de $M$ e $U$ é aberto em $N$ então $U$ é uma subvariedade de $N$ (sugestão: se $\mathcal A$ é o atlas maximal
de $N$, considere o atlas $\mathcal A\vert_U$);
\item[(b)] se $N=\bigcup_{i\in I}U_i$, cada $U_i$ é aberto em $N$, cada $U_i$ é uma subvariedade de $M$ e todas as variedades $U_i$ têm a mesma dimensão
então $N$ é uma subvariedade de $M$ (sugestão: use o resultado do item~(a) do Exercício~\ref{exe:locmanifold} para
obter um atlas maximal em $N$ e use o resultado do item~(b) do Exercício~\ref{exe:locmanifold} para mostrar que a
topologia definida por esse atlas coincide com a topologia induzida por $M$ em $N$; para mostrar que $U_i$ e $U_j$ induzem o mesmo
atlas maximal em $U_i\cap U_j$, use a unicidade do atlas maximal que faz de $U_i\cap U_j$ uma subvariedade de $M$).
\end{itemize}
\end{exercise}

\begin{exercise}
Seja $M$ uma variedade diferenciável. Mostre que $N$ é uma subvariedade de $M$ de dimensão $k$ se e somente se existe uma família $(\phi_i)_{i\in I}$,
onde cada $\phi_i:A_i\to M$ é um mergulho de classe $C^\infty$ definido num aberto $A_i$ de $\R^k$, a imagem de cada $\phi_i$ é um aberto de $N$
com respeito à topologia induzida por $M$ e $N=\bigcup_{i\in I}\phi_i(A_i)$ (sugestão: se $N$ é uma subvariedade de $M$, obtenha os mergulhos $\phi_i$
considerando inversas de cartas de $N$; se os mergulhos $\phi_i$ são dados, mostre que $N$ é uma subvariedade
de $M$ usando o resultado do item~(b) do Exercício~\ref{exe:subvarlocal}).
\end{exercise}

\begin{exercise}
Sejam $M$ uma variedade diferenciável e $N$ uma subvariedade de $M$ de dimensão $k$. Mostre que o atlas maximal de $N$ consiste precisamente das aplicações
bijetoras $\varphi:U\to\widetilde U$ tais que $U$ é um aberto de $N$ (na topologia induzida por $M$), $\widetilde U$ é um aberto de $\R^k$ e
$\varphi^{-1}:\widetilde U\to M$ é um mergulho de classe $C^\infty$ (sugestão: recorde do Exercício~\ref{exe:difeocarta} que o atlas
maximal de $N$ consiste precisamente dos difeomorfismos $\varphi:U\to\widetilde U$ de classe $C^\infty$, onde $U$
é aberto em $N$ e $\widetilde U$ é aberto em $\R^k$; use também o resultado do Exercício~\ref{exe:mergrestated}).
\end{exercise}

Se $(N,\mathcal A)$ é uma subvariedade imersa de uma variedade diferenciável $M$ então, para cada $x\in N$, a diferencial $\dd i_x:T_xN\to T_xM$ da aplicação
inclusão $i:N\to M$ é injetora, já que $i$ é uma imersão. {\em Usamos a aplicação $\dd i_x$ para identificar $T_xN$ com um subespaço de $T_xM$}.
\begin{exercise}\label{exe:difdomcodom}\
\begin{itemize}
\item[(a)] Sejam $M$, $N$ variedades diferenciáveis, $f:M\to N$ uma aplicação de classe $C^\infty$ e $(P,\mathcal A)$ uma subvariedade imersa de $M$.
Mostre que $f\vert_P:P\to N$ é de classe $C^\infty$ e que, para todo $x\in P$, se identificamos $T_xP$ com um subespaço de $T_xM$, então
$\dd(f\vert_P)_x$ é a restrição a $T_xP$ da diferencial $\dd f_x$ (sugestão: $f\vert_P$ é a composição de $f$ com a aplicação inclusão de $P$ em $M$).
\item[(b)] Sejam $M$, $N$ variedades diferenciáveis, $(P,\mathcal A)$ uma subvariedade imersa de $N$ e $f^0:M\to P$ uma aplicação de classe $C^\infty$.
Seja $f:M\to N$ a aplicação que difere de $f^0$ apenas pelo contra-domínio. Mostre que $f$ é de classe $C^\infty$ e que, para todo $x\in M$, se identificamos
$T_{f(x)}P$ com um subespaço de $T_{f(x)}N$ então as diferenciais $\dd f^0_x$ e $\dd f_x$ diferem apenas pelo contra-domínio (sugestão: $f$ é a composição
de $f^0$ com a aplicação inclusão de $P$ em $N$).
\end{itemize}
\end{exercise}

\begin{exercise}
Sejam $M$, $N_0$ variedades diferenciáveis e $f:N_0\to M$ um mergulho de classe $C^\infty$. Se $N=f(N_0)$,
mostre que, para todo $x\in N_0$, o subespaço de $T_{f(x)}M$ que é identificado com o espaço tangente $T_{f(x)}N$
é igual à imagem da diferencial $\dd f_x:T_xN_0\to T_{f(x)}M$ (sugestão: $f:N_0\to N$ é um difeomorfismo de classe $C^\infty$).
Enuncie e mostre um resultado similar a esse em que se supõe apenas que $f$ é uma imersão injetora de classe $C^\infty$.
\end{exercise}

\begin{exercise}\label{exe:esptandifeo}
Sejam $M$, $M'$ variedades diferenciáveis e $\varphi:M\to M'$ um difeomorfismo de classe $C^\infty$. Se $N$ é
uma subvariedade de $M$ então a imagem $N'=\varphi(N)$ de $N$ por $\varphi$
é uma subvariedade de $M'$ (Exercício~\ref{exe:difeosubvar}).
Dado $x\in N$, mostre que o subespaço de $T_{\varphi(x)}M'$ que é identificado com o espaço tangente $T_{\varphi(x)}N'$
coincide com a imagem pela diferencial
\[\dd\varphi_x:T_xM\longrightarrow T_{\varphi(x)}M'\]
do subespaço de $T_xM$ que é identificado
com o espaço tangente $T_xN$, i.e., a menos das identificações apropriadas, temos:
\[\dd\varphi_x(T_xN)=T_{\varphi(x)}N'.\]
\end{exercise}

\begin{exercise}
Sejam $M$ uma variedade diferenciável e $N$ uma subvariedade de $M$. Dado $x\in N$, mostre que o subespaço
de $T_xM$ que é identificado com o espaço tangente
$T_xN$ coincide com o conjunto $\mathcal V_x\subset T_xM$
de todos os vetores da forma $\gamma'(0)$, onde $\gamma:I\to M$ é uma curva
de classe $C^\infty$ tal que $I$ é aberto em $\R$, $0\in I$, $\gamma(0)=x$ e $\gamma(I)\subset N$.
Se $(N,\mathcal A)$ é apenas uma subvariedade imersa de $M$, mostre que $T_xN$ está contido em $\mathcal V_x$
e que essa inclusão {\em pode\/} não ser uma igualdade (veja o que ocorre com as subvariedades imersas que aparecem
no Exercício~\ref{exe:maisimerso}).
\end{exercise}

\begin{exercise}[subvariedades de mesma dimensão]
Sejam $M$, $N$ variedades diferenciáveis com a mesma dimensão e $f:N\to M$ uma imersão injetora de classe $C^\infty$. Mostre que $f(N)$ é um aberto de $M$ e que
$f\vert_N:N\to f(N)$ é um difeomorfismo de classe $C^\infty$ (sugestão: use o teorema da função inversa). Conclua que as subvariedades imersas de $M$
que têm a mesma dimensão que $M$ são (automaticamente mergulhadas e) são precisamente as subvariedades abertas de $M$.
\end{exercise}

Dadas variedades diferenciáveis $M$, $N$ e uma aplicação $f:M\to N$ de classe $C^\infty$, recorde que $x\in M$
é um {\em ponto regular\/} (resp, {\em ponto crítico}) de $f$ se $f$ for (resp., não for) uma submersão no ponto
$x$. Dizemos que $y\in N$ é um {\em valor regular\/} (resp., {\em valor crítico}) de $f$ se {\em todo\/} ponto $x\in f^{-1}(y)$
for regular para $f$ (resp., se {\em existe\/} um ponto $x\in f^{-1}(y)$ que é crítico para $f$).
\begin{exercise}[imagem inversa de valor regular]\label{exe:valorregular}
Sejam $M$, $N$ variedades diferenciáveis, $f:M\to N$ uma aplicação de classe $C^\infty$ e $y\in N$ um valor regular
para $f$. Mostre que $f^{-1}(y)$ é uma subvariedade de $M$ com dimensão $\Dim(M)-\Dim(N)$ e que, para todo $x\in f^{-1}(y)$,
o espaço tangente $T_x[f^{-1}(y)]$ é (identificado com o seguinte subespaço de $T_xM$):
\[T_x[f^{-1}(y)]=\Ker(\dd f_x)\]
(sugestão: use a forma local das submersões).
\end{exercise}

\begin{exercise}
Dados números reais não nulos $a_i$, $i=1,\ldots,n$, mostre que a {\em quádrica}:
\[\big\{x\in\R^n:a_1x_1^2+\cdots+a_nx_n^2=1\big\}\]
é uma subvariedade de $\R^n$ com dimensão $n-1$ e determine o seu espaço tangente num ponto qualquer
(sugestão: use o resultado do Exercício~\ref{exe:valorregular}). Em particular,
a esfera unitária:
\[S^{n-1}=\big\{x\in\R^n:x_1^2+\cdots+x_n^2=1\big\}\]
é uma subvariedade de $\R^n$ com dimensão $n-1$. Nós sempre consideraremos $S^{n-1}$ munida do atlas maximal que faz dela uma subvariedade
de $\R^n$.
\end{exercise}

Sejam $M$ uma variedade diferenciável e $N$ um subconjunto de $M$. Uma carta $\varphi:U\subset M\to\widetilde U\subset\R^n$
de $M$ é dita uma {\em carta de subvariedade $k$-dimensional\/} (ou, simplesmente, {\em carta de subvariedade})
para $N$ se:
\[\varphi(U\cap N)=\widetilde U\cap\R^k,\]
onde identificamos $\R^k$ com $\R^k\times\{0\}^{n-k}\subset\R^n$.
\begin{exercise}\label{exe:quasecartasubvar}
Sejam $M$ uma variedade diferenciável e $N$ um subconjunto de $M$. Se $\varphi:U\subset M\to\widetilde U\subset\R^n$
é uma carta de $M$ tal que $\varphi(U\cap N)$ é um aberto relativo a $\R^k\cong\R^k\times\{0\}^{n-k}$, mostre que
existe um aberto $U'$ em $M$ contido em $U$ tal que $U\cap N=U'\cap N$ e tal que $\varphi\vert_{U'}:U'\to\varphi(U')$
é uma carta de subvariedade $k$-dimensional para $N$ (sugestão: tome $U'$ igual a
$\varphi^{-1}\big(\varphi(U\cap N)\times\R^{n-k}\big)$).
\end{exercise}

\begin{exercise}
Sejam $M$ uma variedade diferenciável e $N$ um subconjunto de $M$. Mostre que $N$ é uma subvariedade de $M$ com $\Dim(N)=k$
se e somente se todo ponto de $N$ pertence ao domínio de uma carta de subvariedade $k$-dimensional para $N$ (sugestão: usando a forma local
das imersões para a aplicação inclusão de $N$ em $M$, você obtém, para cada $x\in N$, um aberto $V\ni x$ relativo à topologia definida pelo atlas de $N$ e
uma carta $\varphi:U\subset M\to\widetilde U\subset\R^n$ de $M$ tal que $V\subset U$ e tal que $\varphi(V)$ é um aberto de $\R^k\cong\R^k\times\{0\}^{n-k}$;
usando o resultado do Exercício~\ref{exe:quasecartasubvar} --- com $V$ no lugar de $N$ --- você consegue substituir $\varphi$ por uma carta ``menor'' de modo a garantir
que $\varphi(V)=\widetilde U\cap\R^k$. Finalmente, você vai precisar usar o fato que $V$ é um aberto na topologia induzida por $M$ em $N$ para
substituir $\varphi$ por uma carta ainda ``menor'' tal que $V=U\cap N$).
\end{exercise}

\begin{exercise}
Sejam $M$ uma variedade diferenciável e $N$ uma subvariedade de $M$. Se $\varphi:U\subset M\to\widetilde U\subset\R^n$ é uma carta de subvariedade
$k$-dimensional para $N$, mostre que:
\[\varphi\vert_{U\cap N}:U\cap N\longrightarrow\widetilde U\cap\R^k\]
é uma carta de $N$. Mostre também que, para todo $x\in U\cap N$, a diferencial $\dd\varphi_x:T_xM\to\R^n$ leva (o subespaço de $T_xM$ identificado com) $T_xN$
sobre o subespaço $\R^k\cong\R^k\times\{0\}^{n-k}$ de $\R^n$ (sugestão: use os resultados dos Exercícios~\ref{exe:difeosubvar} e \ref{exe:esptandifeo}).
\end{exercise}

\begin{exercise}[recíproca local do Exercício~\ref{exe:valorregular}]
Sejam $M$ uma variedade diferenciável de dimensão $n$ e $N$ uma subvariedade de $M$ de dimensão $k$. Dado
$x\in N$, mostre que existem um aberto $U$ em $M$ contendo $x$ e uma submersão $f:U\to\R^{n-k}$ de classe $C^\infty$
(em particular, $0\in\R^{n-k}$ é valor regular de $f$) tal que $f^{-1}(0)=U\cap N$ (sugestão: use uma carta de subvariedade
para $N$ em torno de $x$).
\end{exercise}

\begin{exercise}[gráficos de aplicações]
Sejam $M$, $N$ variedades diferenciáveis e $f:M\to N$ uma aplicação de classe $C^\infty$. Mostre que o gráfico de
$f$:
\[\Gr(f)=\big\{\big(x,f(x)\big):x\in M\big\}\]
é uma subvariedade de $M\times N$ cujo espaço tangente num ponto $\big(x,f(x)\big)$, $x$ em $M$, é (identificado com)
o gráfico da diferencial $\dd f_x$ (sugestão: a aplicação $x\mapsto\big(x,f(x)\big)$ é um mergulho).
\end{exercise}

\begin{exercise}
Sejam $M$, $N$, $P$ variedades diferenciáveis, $\phi:N\to M$ um mergulho de classe $C^\infty$ e $f:P\to N$ uma função. Mostre que
$f$ é um mergulho de classe $C^\infty$ se e somente se $\phi\circ f$ é um mergulho de classe $C^\infty$.
\[\xymatrix{%
&M\\
P\ar[ur]^{\phi\circ f}\ar[r]_f&N\ar[u]_\phi}\]
Conclua que se $N$ é uma subvariedade
de $M$ e $P$ é um subconjunto de $N$ então $P$ é uma subvariedade de $N$ se e somente se $P$ é uma subvariedade de $M$ (sendo que o atlas maximal em $P$
que faz de $P$ uma subvariedade de $N$ é igual ao atlas maximal que faz de $P$ uma subvariedade de $M$).
\end{exercise}

\begin{exercise}
Sejam $M$, $N$, $M'$, $N'$ variedades diferenciáveis e $f:N\to M$, $f':N'\to M'$ funções. Considere a função $f\times f':N\times N'\to M\times M'$
definida por $(f\times f')(x,y)=\big(f(x),f'(y)\big)$, para todos $x\in N$, $y\in N'$. Mostre que:
\begin{itemize}
\item[(a)] se $f$ e $f'$ são imersões (resp., submersões) de classe $C^\infty$ então $f\times f'$ é uma imersão (resp., uma submersão) de classe $C^\infty$;
\item[(b)] se $f$ e $f'$ são mergulhos de classe $C^\infty$ então $f\times f'$ é um mergulho de classe $C^\infty$ (para mostrar isso, você vai ter que verificar
que a topologia induzida por $M\times M'$ em $f(N)\times f'(N')$ coincide com a topologia produto de $f(N)\times f'(N')$);
\item[(c)] se $N$ é uma subvariedade imersa (resp., mergulhada) de $M$ e $N'$ é uma subvariedade imersa (resp., mergulhada) de $M'$ então
$N\times N'$ é uma subvariedade imersa (resp., mergulhada) de $M\times M'$.
\end{itemize}
\end{exercise}

\begin{exercise}\label{exe:toro}
Seja $S^1\subset\R^2\cong\C$ o círculo unitário e considere a aplicação:
\[q:\R^n\ni(\theta_1,\ldots,\theta_n)\longmapsto(e^{2\pi i\theta_1},\ldots,e^{2\pi i\theta_n})\in(S^1)^n.\]
Mostre que $q$ é uma submersão\footnote{%
Na verdade, $q$ é até mesmo um difeomorfismo local.}
sobrejetora de classe $C^\infty$ e que a relação de equivalência definida por $q$ em seu domínio é a mesma
relação de equivalência que o subgrupo aditivo $\Z^n$ define em $\R^n$, i.e.:
\[q(\theta)=q(\theta')\Longleftrightarrow\theta-\theta'\in\Z^n,\]
para todos $\theta,\theta'\in\R^n$. Conclua que $q$ induz (por passagem ao quociente) uma bijeção de $\R^n/\Z^n$ sobre $(S^1)^n$ e que se $\R^n/\Z^n$ é munido
do atlas maximal que faz dessa bijeção um difeomorfismo de classe $C^\infty$ então a aplicação quociente $\R^n\to\R^n/\Z^n$ é uma submersão de classe $C^\infty$, i.e.,
$\R^n/\Z^n$ é uma variedade quociente de $\R^n$ que é difeomorfa ao {\em toro $n$-dimensional\/} $(S^1)^n$.
\end{exercise}

\begin{exercise}
Sejam dados $r,R>0$ com $r<R$ e defina $f:\R^2\to\R^3$ fazendo:
\[f(\theta_1,\theta_2)=R(\cos\theta_1,\sen\theta_1,0)+r\cos\theta_2(\cos\theta_1,\sen\theta_1,0)+r\sen\theta_2(0,0,1),\]
para todos $\theta_1,\theta_2\in\R$. Mostre que $f$ é uma imersão de classe $C^\infty$ e que $f$ define em seu domínio a mesma relação de equivalência
que a aplicação $q:\R^2\to(S^1)^2$ considerada no Exercício~\ref{exe:toro}, i.e.:
\[f(\theta)=f(\theta')\Longleftrightarrow\theta-\theta'\in\Z^2,\]
para todos $\theta,\theta'\in\R^2$. Conclua que existe uma única função $\bar f:(S^1)^2\to\R^3$ tal que o diagrama:
\[\xymatrix{%
\R^2\ar[dr]^f\ar[d]_q\\
(S^1)^2\ar[r]_{\bar f}&\R^3}\]
comuta, i.e., tal que $\bar f\circ q=f$. Mostre que $\bar f$ é um mergulho de classe $C^\infty$ (sugestão: para ver que a aplicação $\bar f$ é um homeomorfismo sobre sua imagem, tenha
em mente que $(S^1)^2$ é compacto). Determine o espaço tangente à imagem de $f$ num ponto $f(\theta_1,\theta_2)$.
\end{exercise}

\begin{exercise}
Considere o subconjunto:
\[N=\big(\{0\}\times\left[0,1\right[\big)\cup\big(\left[0,1\right[\times\{0\}\!\big)\]
de $\R^2$.
\begin{itemize}
\item[(a)] Seja $f:\left]-1,1\right[\to\R^2$ definida por $f(t)=(t,0)$ para $0\le t<1$ e por $f(t)=(0,-t)$, para $-1<t<0$. Mostre que $f$
é uma bijeção sobre $N$ e que se $\mathcal A$ é o atlas maximal em $N$ que faz de $f$ um difeomorfismo de classe $C^\infty$ então $\mathcal A$ define em $N$
a topologia induzida por $\R^2$, mas a aplicação inclusão de $(N,\mathcal A)$ em $\R^2$ não é de classe $C^\infty$.
\item[(b)] Seja $g:\R\to\R^2$ definida por $g(t)=(e^{-\frac1t},0)$, para $t>0$, por $g(t)=(0,e^{\frac1t})$, para $t<0$ e por $g(0)=(0,0)$.
Mostre que $g$ é uma bijeção sobre $N$ e que se $\mathcal A'$ é o atlas maximal em $N$ que faz de $g$ um difeomorfismo de classe $C^\infty$ então
$\mathcal A'$ define em $N$ a topologia induzida por $\R^2$ e a aplicação inclusão de $(N,\mathcal A')$ em $\R^2$ é de classe $C^\infty$, mas não é uma imersão.
\item[(c)] Se $\gamma:I\to\R^2$ é uma função derivável definida num aberto $I\subset\R$ com $\gamma(I)\subset N$ e se $t_0\in I$ é tal que
$\gamma(t_0)=(0,0)$, mostre que $\gamma'(t_0)=0$ (sugestão: ambas as funções coordenadas de $\gamma$ assumem um mínimo em $t_0$). Conclua que não existe
um atlas maximal em $N$ que faz de $N$ uma subvariedade imersa de $\R^2$ de dimensão $1$ (no entanto, podemos fazer de $N$ uma variedade de dimensão zero
e nesse caso ela {\em é\/} uma subvariedade imersa de $\R^2$!).
\end{itemize}
\end{exercise}

\begin{exercise}
Se $M$ é uma variedade diferenciável, $N$ é um subconjunto de $M$ e $\mathcal A$, $\mathcal A'$ são atlas maximais em $N$ que fazem de $N$
uma subvariedade mergulhada de $M$, então vimos que $\mathcal A=\mathcal A'$. Se ambos os atlas fazem de $N$ uma subvariedade imersa de $M$, vimos que não necessariamente
ocorre $\mathcal A=\mathcal A'$. É possível que tenhamos atlas maximais distintos $\mathcal A$, $\mathcal A'$ em $N$, um deles fazendo de $N$ uma subvariedade
mergulhada de $M$ e o outro fazendo de $N$ uma subvariedade imersa de $M$? Vamos analisar a questão neste exercício.
\begin{itemize}
\item[(a)] Sejam $\mathcal A$, $\mathcal A'$ atlas maximais em $N$, sendo que $(N,\mathcal A)$ é uma subvariedade mergulhada de $M$
e $(N,\mathcal A')$ é uma subvariedade imersa de $M$. Mostre que, se $\Dim(N,\mathcal A)=\Dim(N,\mathcal A')$, então $\mathcal A=\mathcal A'$ (sugestão:
mostre primeiro que a aplicação identidade de $(N,\mathcal A')$ para $(N,\mathcal A)$ é uma imersão de classe $C^\infty$ e depois conclua que ela é
um difeomorfismo de classe $C^\infty$ usando o teorema da função inversa).
\[\xymatrix{%
&M\\
(N,\mathcal A')\ar[ur]^{\text{inclusão}}\ar[r]_{\Id}&(N,\mathcal A)\ar[u]_{\text{inclusão}}}\]
\item[(b)] Sejam $M=N=\R^2$, $\mathcal A$ o atlas maximal usual de $\R^2$ e $\mathcal A'$ o atlas maximal em $\R^2$ obtido quando consideramos
$\R^2$ como o produto de $\R$ munido de seu atlas maximal usual por $\R$ munido do atlas maximal que faz de $\R$ uma variedade de dimensão zero. Note
que $(\R^2,\mathcal A')$ é uma variedade de dimensão $1$ e que $\R\times\{t\}$ é aberto com respeito à topologia definida por $\mathcal A'$, para todo
$t\in\R$. Mostre que $(N,\mathcal A')$ é uma subvariedade imersa de $(M,\mathcal A)$ (sendo que, obviamente, $(N,\mathcal A)$ é uma subvariedade
mergulhada de $(M,\mathcal A)$). Note que $(N,\mathcal A')$ não possui base enumerável de abertos! É possível mostrar que esse fenômeno estranho
não poderia ocorrer se considerássemos apenas variedades com base enumerável de abertos (veja o item~(c) do
Exercício~\ref{exe:naosubnaosobre}).
\end{itemize}
\end{exercise}

\begin{exercise}
Seja $q:\R^2\to(S^1)^2$ a submersão de classe $C^\infty$ considerada no Exercício~\ref{exe:toro}. Dado um número real $\alpha$, considere a aplicação:
\[f:\R\ni t\longmapsto q(t,\alpha t)\in(S^1)^2.\]
Mostre que:
\begin{itemize}
\item[(a)] $f$ é uma imersão de classe $C^\infty$;
\item[(b)] se $\alpha$ é irracional então $f$ é injetora e sua imagem, munida do atlas maximal que faz de $f:\R\to f(\R)$ um difeomorfismo de classe $C^\infty$,
é uma subvariedade imersa do toro $(S^1)^2$ difeomorfa a $\R$;
\item[(c)] se $\alpha=\frac ab$, com $a,b\in\Z$, $\mdc(a,b)=1$, então $f$ é periódica com período $b$ e induz por passagem ao quociente
um mergulho de classe $C^\infty$ do círculo $S^1\cong\R/(b\Z)$ no toro $(S^1)^2$. Conclua que a imagem de $f$ é nesse caso uma subvariedade do toro
$(S^1)^2$ que é difeomorfa ao círculo $S^1$ (observação: para identificar $S^1$ com $\R/(b\Z)$ você vai precisar usar a aplicação
$\R\ni\theta\mapsto e^{2\pi i\frac\theta b}\in S^1$).
\end{itemize}
\end{exercise}

\end{section}

\begin{section}{Dia 30/08}

\begin{exercise}[grupo ortogonal]\label{exe:On}
Sejam $M_n(\R)$ o espaço vetorial das matrizes reais $n\times n$ e $\Sym(n,\R)$ o subespaço formado pelas matrizes simétricas.
\begin{itemize}
\item[(a)] Mostre que a função $f:M_n(\R)\to\Sym(n,\R)$ dada por $f(A)=A^\transp A$, $A\in M_n(\R)$, está bem definida e é de classe
$C^\infty$, onde $A^\transp$ denota a transposta da matriz $A$.
\item[(b)] Mostre que a diferencial de $f$ é dada por:
\[\dd f(A)H=H^\transp A+A^\transp H=A^\transp H+(A^\transp H)^\transp,\]
para quaisquer $A,H\in M_n(\R)$.
\item[(c)] Mostre que se $A\in M_n(\R)$ é invertível então $A$ é um ponto regular para $f$, i.e., $\dd f(A):M_n(\R)\to\Sym(n,\R)$ é sobrejetora
(sugestão: dada $S\in\Sym(n,\R)$, tome $H$ tal que $A^\transp H=\frac S2$).
\item[(d)] Mostre que a matriz identidade $\I\in\Sym(n,\R)$ é um valor regular para $f$ e conclua que o {\em grupo ortogonal}:
\[\Or(n)=\big\{A\in M_n(\R):A^\transp A=\I\big\}\]
é uma subvariedade de $M_n(\R)$.
\item[(e)] Mostre que o espaço tangente $T_\I\!\Or(n)$ a $\Or(n)$ na matriz identidade coincide com o espaço das matrizes anti-simétricas $n\times n$
e que a dimensão de $\Or(n)$ é $\frac12n(n-1)$.
\item[(f)] Dada $A\in\Or(n)$, mostre que o espaço tangente $T_A\!\Or(n)$ é igual a $L_A\big(T_\I\!\Or(n)\big)$, onde $L_A:M_n(\R)\to M_n(\R)$ denota
a {\em translação à esquerda\/} definida por $L_A(X)=AX$, $X\in M_n(\R)$ (sugestão: $L_A$ é um difeomorfismo de $M_n(\R)$ que leva $\Or(n)$ sobre
$\Or(n)$; use o resultado do Exercício~\ref{exe:esptandifeo}).
\item[(g)] Mostre que $\Or(n)$ é um subconjunto fechado e limitado (e portanto compacto) de $M_n(\R)$.
\item[(h)] Mostre que o {\em grupo especial ortogonal}:
\[\SO(n)=\big\{A\in\Or(n):\det(A)=1\big\}\]
é um subconjunto aberto e fechado de $\Or(n)$ e portanto também é uma subvariedade compacta de $M_n(\R)$ (sugestão: a restrição da função determinante
a $\Or(n)$ toma valores em $\{-1,1\}$).
\end{itemize}
\end{exercise}

\begin{exercise}[grupo ortogonal complexo]\label{exe:OnC}
Repita tudo que você fez no Exercício~\ref{exe:On} trocando $\R$ por $\C$, i.e., mostre que o {\em grupo ortogonal complexo}:
\[\Or(n,\C)=\big\{A\in M_n(\C):A^\transp A=\I\big\}\]
é uma subvariedade do espaço vetorial $M_n(\C)$ das matrizes complexas $n\times n$ (você deve pensar em $M_n(\C)$ como um espaço vetorial {\em real\/} de dimensão
$2n^2$) e que o {\em grupo especial ortogonal complexo}:
\[\SO(n,\C)=\big\{A\in\Or(n,\C):\det(A)=1\big\}\]
é uma subvariedade aberta e fechada de $\Or(n,\C)$. A dimensão de $\Or(n,\C)$ é $n(n-1)$ em vez de $\frac12n(n-1)$. Observe também que $\Or(n,\C)$
é fechado em $M_n(\C)$, mas {\em não é limitado\/} (por que?) e portanto não é compacto.
\end{exercise}

\begin{exercise}[grupo unitário]
Dada uma matriz complexa $A\in M_n(\C)$, denote por $A^*$ a sua {\em matriz adjunta}, i.e., o complexo conjugado da matriz transposta $A^\transp$.
O {\em grupo unitário\/} é definido por:
\[\Ur(n)=\big\{A\in M_n(\C):A^*A=\I\big\}.\]
Repita o que você fez no Exercício~\ref{exe:OnC} (trocando $A^\transp$ por $A^*$) para mostrar que $\Ur(n)$ é uma subvariedade de $M_n(\C)$ cujo
espaço tangente $T_\I\!\Ur(n)$ na matriz identidade é o espaço $\big\{H\in M_n(\C):H^*=-H\big\}$ formado pelas matrizes anti-hermiteanas. Mostre que a dimensão
de $\Ur(n)$ é $n^2$ e que, diferentemente de $\Or(n,\C)$, $\Ur(n)$ é compacto.
\end{exercise}

\begin{exercise}[grupo especial linear]\label{exe:SLnR}
Denote por $\frac{\partial\det}{\partial X_{ij}}(X)$ a derivada parcial no ponto $X\in M_n(\R)$ da função determinante:
\[\det:M_n(\R)\longrightarrow\R\]
com respeito à entrada $X_{ij}$ da matriz. Mostre que:
\begin{itemize}
\item[(a)] $\frac{\partial\det}{\partial X_{ij}}(X)=C_{ij}(X)$, onde $C_{ij}(X)$ é o cofator correspondente à entrada $X_{ij}$ da matriz $X$,
i.e., $C_{ij}(X)$ é $(-1)^{i+j}$ vezes o determinante da matriz obtida de $X$ pela remoção da linha $i$ e da coluna $j$
(sugestão: escreva a fórmula para $\det(X)$ usando expansão por cofatores ao longo da $i$-ésima linha);
\item[(b)] $X\in M_n(\R)$ é um ponto regular da função determinante se e somente se $X$ tem posto maior ou igual a $n-1$;
\item[(c)] o único valor crítico da função determinante é $0\in\R$ (exceto\footnote{%
Na verdade, para $n=0$, se convencionamos que o determinante da matriz vazia é igual a $1$, então o único valor crítico para a função
determinante seria $1\in\R$.} para $n=1$, em que nem mesmo $0$ é crítico);
\item[(d)] o {\em grupo especial linear}:
\[\SL(n,\R)=\big\{A\in M_n(\R):\det(A)=1\big\}\]
é uma subvariedade (fechada, mas não compacta) de $M_n(\R)$ de dimensão $n^2-1$;
\item[(e)] a diferencial da função determinante num ponto $X\in M_n(\R)$ é dada por:
\[\dd(\det)(X)H=\sum_{i,j=1}^n\frac{\partial\det}{\partial X_{ij}}(X)H_{ij}=\sum_{i,j=1}^nC_{ij}(X)H_{ij},\quad H\in M_n(\R),\]
e no ponto $X=\I$ a diferencial da função determinante é a função traço $\tr:M_n(\R)\to\R$;
\item[(f)] o espaço tangente $T_\I\!\SL(n,\R)$ a $\SL(n,\R)$ na matriz identidade é $\big\{H\in M_n(\R):\tr(H)=0\big\}$
e, para $A\in\SL(n,\R)$, o espaço tangente a $\SL(n,\R)$ no ponto $A$ é:
\[T_A\SL(n,\R)=L_A\big(T_\I\!\SL(n,\R)\big),\]
onde $L_A$ denota a translação à esquerda $X\mapsto AX$.
\end{itemize}
\end{exercise}

\begin{exercise}[grupo especial linear complexo]
Repita tudo que você fez no Exercício~\ref{exe:SLnR} trocando $\R$ por $\C$ para mostrar que o
{\em grupo especial linear complexo}:
\[\SL(n,\C)=\big\{A\in M_n(\C):\det(A)=1\big\}\]
é uma subvariedade de $M_n(\C)$ de dimensão $2(n^2-1)$ cujo espaço tangente na matriz identidade é
$\big\{H\in M_n(\C):\tr(H)=0\big\}$ (no caso complexo, a igualdade $\frac{\partial\det}{\partial X_{ij}}(X)=C_{ij}(X)$
significa o seguinte: quando consideramos a função determinante como função apenas da entrada $X_{ij}$ da matriz, obtemos
uma função de $\C$ em $\C$ cuja diferencial num ponto $X_{ij}\in\C$ é a transformação linear dada por multiplicação
pelo número complexo $C_{ij}(X)$; note que a diferencial $\dd(\det)(X):M_n(\C)\to\C$ é, não só linear sobre $\R$,
mas também {\em linear sobre $\C$}, de modo que, para um dado $X$, essa diferencial é ou nula ou sobrejetora,
como no caso real).
\end{exercise}

\begin{exercise}[grupo especial unitário]\
\begin{itemize}
\item[(a)] Dada $A\in\Ur(n)$, mostre que o determinante $\det(A)$ tem módulo igual a $1$.
\item[(b)] Mostre que a função determinante $\det:\Ur(n)\to S^1\subset\C\cong\R^2$ é de classe $C^\infty$.
\item[(c)] Dada $A\in\Ur(n)$, considere o diagrama comutativo:
\[\xymatrix@C+5pt{%
\Ur(n)\ar[r]^{L_A}\ar[d]_{\det}&\Ur(n)\ar[d]^{\det}\\
S^1\ar[r]_{L_{\det(A)}}&S^1}\]
onde $L_A:X\mapsto AX$ e $L_{\det(A)}:z\mapsto\det(A)z$ denotam translações à esquerda. Diferencie o diagrama
para mostrar que a função:
\[\det:\Ur(n)\longrightarrow S^1\]
tem posto constante (sugestão: observe que as translações à esquerda são difeomorfismos).
\item[(d)] Mostre que $\det:\Ur(n)\to S^1$ é uma submersão no ponto $\I\in\Ur(n)$ e conclua, usando o resultado
do item~(c), que ela é uma submersão.
\item[(e)] Mostre que o {\em grupo especial unitário}:
\[\SU(n)=\big\{A\in\Ur(n):\det(A)=1\big\}\]
é uma subvariedade compacta de $M_n(\C)$ de dimensão $n^2-1$ cujo espaço tangente na matriz identidade é:
\[T_\I\!\SU(n)=\big\{H\in M_n(\C):H^*=-H,\ \tr(H)=0\big\}.\]
\end{itemize}
\end{exercise}

\begin{rem}
O grupo ortogonal (real) $\Or(n)$ é formado pelas matrizes que representam na base canônica de $\R^n$
os operadores lineares em $\R^n$ que preservam o produto interno canônico de $\R^n$. O grupo ortogonal complexo
$\Or(n,\C)$ é formado pelas matrizes que representam na base canônica de $\C^n$ os operadores lineares (sobre $\C$)
em $\C^n$ que preservam o produto bilinear complexo:
\begin{equation}\label{eq:prodbilincompl}
\C^n\times\C^n\ni(z,w)\longmapsto\sum_{j=1}^nz_jw_j\in\C.
\end{equation}
Já o grupo unitário $\Ur(n)$ é formado pelas matrizes que representam na base canônica de $\C^n$ os operadores
lineares (sobre $\C$) em $\C^n$ que preservam o produto interno (sesqui-linear) canônico de $\C^n$:
\[\C^n\times\C^n\ni(z,w)\longmapsto\sum_{j=1}^nz_j\bar w_j\in\C.\]
O grupo especial linear real $\SL(n,\R)$ é formado pelas matrizes que representam na base canônica de $\R^n$ os
operadores lineares em $\R^n$ que preservam orientação e volume. Os grupos $\Or(n)$, $\SO(n)$ e $\SL(n,\R)$
são subgrupos do {\em grupo linear geral real}:
\[\GL(n,\R)=\big\{A\in M_n(\R):\det(A)\ne0\big\}\]
que é um subconjunto aberto de $M_n(\R)$ e os grupos $\Or(n,\C)$, $\SO(n,\C)$, $\SL(n,\C)$, $\Ur(n)$ e $\SU(n)$
são subgrupos do {\em grupo linear geral complexo}:
\[\GL(n,\C)=\big\{A\in M_n(\C):\det(A)\ne0\big\}\]
que é um subconjunto aberto de $M_n(\C)$.
\end{rem}

\begin{defin}
Um {\em grupo de Lie\/} é um grupo $G$ munido de um atlas maximal (de modo que $G$ é uma variedade diferenciável)
tal que a multiplicação de $G$:
\begin{equation}\label{eq:multG}
G\times G\ni(x,y)\longmapsto xy\in G
\end{equation}
e a inversão:
\begin{equation}\label{eq:invG}
G\ni x\longmapsto x^{-1}\in G
\end{equation}
sejam aplicações de classe $C^\infty$.
\end{defin}
Se um grupo $G$ é munido apenas de uma topologia e se as funções \eqref{eq:multG} e \eqref{eq:invG} são contínuas,
diz-se que $G$ é um {\em grupo topológico}. Obviamente, grupos de Lie (munidos da topologia definida pelo atlas)
são grupos topológicos.

\begin{exercise}
Mostre que:
\begin{gather*}
\Or(n),\quad\SO(n),\quad\Or(n,\C),\quad\SO(n,\C),\quad\SL(n,\R),\quad\SL(n,\C),\\
\Ur(n),\quad\SU(n),\quad\GL(n,\R),\quad\GL(n,\C)
\end{gather*}
são grupos de Lie. Mostre também que o círculo unitário $S^1$, munido da estrutura de
grupo induzida pela multiplicação do plano complexo, é um grupo de Lie.
\end{exercise}

\begin{exercise}\label{exe:quaternios}
A {\em álgebra dos quatérnios\/}
é o espaço vetorial real $\R^4$ munido do produto bilinear definido da seguinte forma: denotamos por $1$, $i$, $j$, $k$
a base canônica de $\R^4$ e fazemos:
\begin{gather*}
q1=1q=q,\quad q\in\{1,i,j,k\},\quad ij=-ji=k,\quad ki=-ik=j,\\
jk=-kj=i,\quad i^2=j^2=k^2=-1.
\end{gather*}
A álgebra dos quatérnios é uma álgebra associativa com unidade $1$. Vamos denotá-la por $\IH$. A {\em conjugação\/}
em $\IH$ é a transformação linear $q\mapsto\bar q$ que fixa a unidade e troca o sinal de $i$, de $j$ e de $k$. Para $q_1,q_2\in\IH$, vale que:
\begin{equation}\label{eq:barq1q2}
\overline{q_1q_2}=\bar q_2\bar q_1,
\end{equation}
e para $q\in\IH$, vale que:
\begin{equation}\label{eq:qbarq}
q\bar q=\bar qq=\vert q\vert^2,
\end{equation}
onde $\vert q\vert$ denota a norma Euclideana usual de $\R^4$. Segue de \eqref{eq:qbarq} que $\IH$ é uma {\em álgebra com divisão},
i.e., todo elemento não nulo de $\IH$ possui um inverso multiplicativo. Segue de \eqref{eq:barq1q2} e de \eqref{eq:qbarq} que:
\[\vert q_1q_2\vert=\vert q_1\vert\vert q_2\vert,\quad q_1,q_2\in\IH.\]
Mostre que a esfera unitária $S^3$,
munida da estrutura de grupo induzida pela multiplicação de $\IH$, é um grupo de Lie.
\end{exercise}

\begin{exercise}
Dado um grupo de Lie $G$ e um elemento $g\in G$, denotamos por $L_g:G\to G$, $R_g:G\to G$ respectivamente a
{\em translação à esquerda\/} $L_g:x\mapsto gx$ e a {\em translação à direita\/} $R_g:x\mapsto xg$.
\begin{itemize}
\item[(a)] Mostre que $L_g$ e $R_g$ são difeomorfismos de classe $C^\infty$, para todo $g\in G$ (sugestão:
a inversa de $L_g$ é $L_{g^{-1}}$ e a inversa de $R_g$ é $R_{g^{-1}}$).
\item[(b)] Denote por $m:G\times G\to G$ a função \eqref{eq:multG} e por $1\in G$ o elemento neutro. Notando que:
\[m(x,y)=1\Longleftrightarrow y=x^{-1},\quad x,y\in G,\]
use o resultado do item~(a) e o teorema da função implícita para mostrar que o fato que a função \eqref{eq:invG}
é de classe $C^\infty$ na verdade {\em segue\/} do fato que a função \eqref{eq:multG} é de classe $C^\infty$
(embora a maioria dos textos exija que \eqref{eq:invG} seja de classe $C^\infty$, vemos que a exigência
é supérflua!).
\end{itemize}
\end{exercise}

\begin{exercise}
Mostre que:
\begin{itemize}
\item[(a)] se $G$ é um grupo topológico tal que $\{1\}$ é um subconjunto fechado de $G$ então $G$ é Hausdorff (sugestão: um espaço topológico
$G$ é Hausdorff se e somente se a diagonal é um subconjunto fechado de $G\times G$; a diagonal de $G$ é a imagem inversa de $\{1\}$ pela aplicação
$(x,y)\mapsto xy^{-1}$);
\item[(b)] todo grupo de Lie é Hausdorff (sugestão: use o resultado do item~(a) do Exercício~\ref{exe:manifoldtop}).
\end{itemize}
\end{exercise}

Veja o Apêndice~\ref{sec:Grassmanniana} para uma exposição (e alguns exercícios) sobre Grassmannianas.

\end{section}

\begin{section}{Dia 01/09}

\begin{exercise}\label{exe:projradial}
Mostre que a projeção radial:
\begin{equation}\label{eq:projradial}
\R^n\setminus\{0\}\ni x\longmapsto\frac x{\Vert x\Vert}\in S^{n-1}
\end{equation}
é uma submersão de classe $C^\infty$ (sugestão: use o resultado do Exercício~\ref{exe:secaolocal}; note que, para todo $R>0$, a função
$x\mapsto Rx$ é uma seção da projeção radial).
\end{exercise}

O {\em espaço projetivo real $\R P^{n-1}$ de dimensão $n-1$\/} é a Grassmanniana $G_1(\R^n)$ de subespaços unidimensionais de $\R^n$. Dado
$v\in\R^n$, denotamos por $\R v$ o subespaço gerado por $v$, de modo que $\R v\in\R P^{n-1}$, para $v\ne0$.
\begin{exercise}\label{exe:cartasRP}
Dado $i\in\{1,\ldots,n\}$, mostre que a aplicação:
\[\R^{n-1}\ni(x_1,\ldots,x_{n-1})\longmapsto\R(x_1,\ldots,x_{i-1},1,x_i,\ldots,x_{n-1})\in\R P^{n-1}\]
é um difeomorfismo de classe $C^\infty$ sobre um subconjunto aberto de $\R P^{n-1}$ (sugestão: considere a carta $\phi_{E_0,E_1}$ em
$G_1(\R^n)$ onde $E_0=\R e_i$ e $E_1$ é gerado por $\{e_1,\ldots,e_n\}\setminus\{e_i\}$, sendo $e_1$, \dots, $e_n$ a base canônica de $\R^n$).
As inversas dessas aplicações constituem portanto um atlas contido no atlas maximal de $\R P^{n-1}$.
\end{exercise}

\begin{exercise}
Mostre que a aplicação:
\begin{equation}\label{eq:projSRP}
S^{n-1}\ni x\longmapsto\R x\in\R P^{n-1}
\end{equation}
é um difeomorfismo local (sugestão: considere a restrição dessa aplicação a conjuntos da forma
$\big\{x\in S^{n-1}:x_i>0\big\}$, $\big\{x\in S^{n-1}:x_i<0\big\}$ e use o atlas em $\R P^{n-1}$ definido
no Exercício~\ref{exe:cartasRP}). Conclua, usando também o resultado do Exercício~\ref{exe:projradial},
que a aplicação:
\begin{equation}\label{eq:projRRPn}
\R^n\setminus\{0\}\ni x\longmapsto\R x\in\R P^{n-1}
\end{equation}
é uma submersão de classe $C^\infty$ (sugestão: observe que \eqref{eq:projRRPn} é a composição de \eqref{eq:projradial} com \eqref{eq:projSRP}).
\end{exercise}

\begin{exercise}\label{exe:quocRP}
Seja $\R^*$ o grupo multiplicativo dos números reais não nulos e $\R_+$ o subgrupo de $\R^*$ formado pelos números reais positivos.
Considere a ação de $\R^*$ em $\R^n$ definida por $(\lambda,x)\mapsto\lambda x$. No que segue, as ações consideradas são todas restrições
dessa ação de $\R^*$ em $\R^n$. Mostre que:
\begin{itemize}
\item[(a)] a relação de equivalência determinada pela projeção radial \eqref{eq:projradial} em seu domínio é a mesma relação de equivalência
definida pela ação de $\R_+$ em $\R^n\setminus\{0\}$ e conclua que $S^{n-1}$ é difeomorfa à variedade quociente $\big(\R^n\setminus\{0\}\big)/\R_+$;
\item[(b)] a relação de equivalência determinada pela aplicação \eqref{eq:projRRPn} em seu domínio é a mesma relação de equivalência definida pela
ação de $\R^*$ em $\R^n\setminus\{0\}$ e conclua que $\R P^{n-1}$ é difeomorfo à variedade quociente $\big(\R^n\setminus\{0\}\big)/\R^*$;
\item[(c)] a relação de equivalência determinada pela aplicação \eqref{eq:projSRP} em seu domínio é a mesma relação de equivalência definida pela
ação de $\{-1,1\}$ em $S^{n-1}$ e conclua que $\R P^{n-1}$ é difeomorfo à variedade quociente $S^{n-1}/\{-1,1\}$.
\end{itemize}
\end{exercise}

\begin{exercise}\label{exe:RP1S1}
Mostre que:
\begin{itemize}
\item[(a)] a aplicação $S^1\ni z\mapsto z^2\in S^1$ é um difeomorfismo local e conclua que a variedade quociente
$S^1/\{-1,1\}$ é difeomorfa a $S^1$;
\item[(b)] o espaço projetivo real $\R P^1$ é difeomorfo ao círculo $S^1$ (sugestão: use o resultado do item~(c)
do Exercício~\ref{exe:quocRP}).
\end{itemize}
\end{exercise}

O {\em espaço projetivo complexo $\C P^{n-1}$\/} é a Grassmanniana complexa $G_1(\C^n)$ de subespaços unidimensionais complexos
de $\C^n$. Dado $v\in\C^n$, denotamos por $\C v$ o subespaço gerado por $v$, de modo que $\C v\in\C P^{n-1}$, para $v\ne0$.
A dimensão\footnote{%
No entanto, $\C P^{n-1}$ é também uma {\em variedade complexa\/} de dimensão $n-1$, mas não falamos em
variedades complexas neste curso.} de $\C P^{n-1}$ é $2(n-1)$ (veja Exercício~\ref{exe:Grasscompl}).
\begin{exercise}\label{exe:cartasCP}
Repita o que você fez no Exercício~\ref{exe:cartasRP} trocando $\R$ por $\C$ para mostrar que, para todo
$j=1,\ldots,n$, a aplicação:
\[\C^{n-1}\ni(z_1,\ldots,z_{n-1})\longmapsto\C(z_1,\ldots,z_{j-1},1,z_j,\ldots,z_{n-1})\in\C P^{n-1}\]
é um difeomorfismo de classe $C^\infty$ sobre um subconjunto aberto de $\C P^{n-1}$. As inversas dessas aplicações
constituem portanto (após identificação de $\C^{n-1}$ com $\R^{2(n-1)}$) um atlas contido no atlas maximal de $\C P^{n-1}$.
\end{exercise}

\begin{exercise}
Vamos identificar $\C^n$ com $\R^{2n}$ (por exemplo, identificando $x+iy\in\C^n$, $x,y\in\R^n$, com $(x,y)\in\R^{2n}$),
de modo que a esfera unitária de $\C^n$ identifica-se com $S^{2n-1}$. Mostre que a aplicação:
\begin{equation}\label{eq:projSCP}
\C^n\supset S^{2n-1}\ni z\longmapsto\C z\in \C P^{n-1}
\end{equation}
é uma submersão de classe $C^\infty$ (sugestão: use o atlas em $\C P^{n-1}$ definido no Exercício~\ref{exe:cartasCP}
e considere seções locais de \eqref{eq:projSCP} definidas por $\C z\mapsto\lambda\frac z{\Vert z\Vert}$,
onde $\lambda\in\C$ é um número complexo unitário fixado e o representante $z$ do espaço unidimensional $\C z$ é escolhido
de modo que sua $j$-ésima coordenada seja igual a $1$). Conclua, usando também o resultado do Exercício~\ref{exe:projradial},
que a aplicação:
\begin{equation}\label{eq:projCCPn}
\C^n\setminus\{0\}\ni z\longmapsto\C z\in\C P^{n-1}
\end{equation}
é uma submersão de classe $C^\infty$.
\end{exercise}

\begin{exercise}\label{exe:quocCP}
Seja $\C^*$ o grupo multiplicativo dos números complexos não nulos.
Considere a ação de $\C^*$ em $\C^n$ definida por $(\lambda,z)\mapsto\lambda z$. No que segue, as ações consideradas são todas restrições
dessa ação de $\C^*$ em $\C^n$. Mostre que:
\begin{itemize}
\item[(a)] a relação de equivalência determinada pela aplicação \eqref{eq:projCCPn} em seu domínio é a mesma relação de equivalência definida pela
ação de $\C^*$ em $\C^n\setminus\{0\}$ e conclua que $\C P^{n-1}$ é difeomorfo à variedade quociente $\big(\C^n\setminus\{0\}\big)/\C^*$;
\item[(b)] a relação de equivalência determinada pela aplicação \eqref{eq:projSCP} em seu domínio é a mesma relação de equivalência definida pela
ação de $S^1$ em $S^{2n-1}\subset\C^n$ e conclua que $\C P^{n-1}$ é difeomorfo à variedade quociente $S^{2n-1}/S^1$.
\end{itemize}
\end{exercise}

\begin{exercise}[plano com ponto no infinito]\label{exe:Ccupinf}
Seja $\infty$ um elemento qualquer fora do plano complexo $\C$ e considere os sistemas de coordenadas $\varphi_0$,
$\varphi_\infty$ no conjunto $\C\cup\{\infty\}$ definidos da seguinte forma: $\varphi_0$ é a aplicação identidade
de $\C$ para $\C\cong\R^2$ e $\varphi_\infty:\big(\C\setminus\{0\}\big)\cup\{\infty\}\to\C\cong\R^2$ é definido por
$\varphi_\infty(z)=\frac1z$, para $z\in\C\setminus\{0\}$ e $\varphi_\infty(\infty)=0$. Mostre que:
\begin{itemize}
\item[(a)] $\{\varphi_0,\varphi_\infty\}$ é um atlas em $\C\cup\{\infty\}$;
\item[(b)] munido do atlas maximal que contém $\{\varphi_0,\varphi_\infty\}$, o conjunto $\C\cup\{\infty\}$ é uma variedade
diferenciável difeomorfa à esfera $S^2$ (sugestão: defina uma bijeção entre $\C\cup\{\infty\}$ e $S^2$
usando uma projeção estereográfica; você vai ter que fazer um pouco de conta para checar que essa bijeção é um difeomorfismo
mesmo, especialmente em torno do ponto $\infty$, onde você vai ter que usar a carta $\varphi_\infty$).
\end{itemize}
\end{exercise}

\begin{exercise}
Considere a aplicação $\phi:\C\cup\{\infty\}\to\C P^1$ definida por $\phi(z)=\C(z,1)$, para $z\in\C$ e
por $\phi(\infty)=\C(1,0)$. Se $\C\cup\{\infty\}$ é munido do atlas definido no Exercício~\ref{exe:Ccupinf},
mostre que $\phi$ é um difeomorfismo de classe $C^\infty$ (sugestão: use o atlas em $\C P^1$ definido no
Exercício~\ref{exe:cartasCP}). Conclua que $\C P^1$ é difeomorfo à esfera $S^2$ e, usando o resultado do item~(b)
do Exercício~\ref{exe:quocCP}, conclua também que a esfera $S^2$ é difeomorfa à variedade quociente $S^3/S^1$.
A submersão $S^3\to S^3/S^1\cong S^2$ é conhecida como {\em fibração de Hopf}.
\end{exercise}

\begin{exercise}\label{exe:GkFGkE}
Seja $E$ um espaço vetorial (real ou complexo) de dimensão finita, $F$ um subespaço de $E$ e $k$ um número natural menor ou igual à dimensão de $F$.
Mostre que a Grassmanniana (real ou complexa, respectivamente) $G_k(F)$ é uma subvariedade da Grassmanniana $G_k(E)$ (sugestão: considere cartas
$\phi_{E_0,E_1}$ em $G_k(E)$ associadas a decomposições $E=E_0\oplus E_1$ tais que $E_0\subset F$; qual é a imagem de $G_k(F)$ por $\phi_{E_0,E_1}$?).
\end{exercise}

\begin{rem}
O plano projetivo real $\R P^2$ é um modelo para a {\em geometria projetiva plana real}. As {\em retas projetivas\/} em $\R P^2$ são as subvariedades
$G_1(\pi)$, sendo $\pi$ um subespaço bidimensional de $\R^3$ (veja Exercício~\ref{exe:GkFGkE}). Obviamente, a variedade $G_1(\pi)$ é difeomorfa
a $G_1(\R^2)=\R P^1$ e é portanto difeomorfa ao círculo $S^1$ (veja Exercício~\ref{exe:RP1S1}). A aplicação (veja Exercício~\ref{exe:cartasRP}):
\begin{equation}\label{eq:R2RP2}
\R^2\ni(x,y)\longmapsto\R(x,y,1)\in\R P^2
\end{equation}
é usada para identificar o plano Euclideano $\R^2$ com um subconjunto aberto do plano projetivo $\R P^2$. Os pontos de $\R P^2$ que não pertencem a
(esse subconjunto de $\R P^2$ que é identificado com) $\R^2$ são chamados {\em pontos impróprios}. O conjunto de todos os pontos impróprios é a reta
projetiva $G_1\big(\R^2\times\{0\}\big)$, chamada de {\em reta imprópria}. Uma reta projetiva que não é a reta imprópria é a união de uma reta
de $\R^2$ (ou melhor, da imagem de uma reta de $\R^2$ pela aplicação \eqref{eq:R2RP2}) com precisamente um ponto impróprio (o {\em ponto no infinito\/}
correspondente àquela reta Euclideana). Duas retas projetivas distintas se cortam em precisamente um ponto: duas retas distintas não impróprias que
correspondem a retas Euclideanas não paralelas cortam-se num ponto do plano Euclideano; duas retas distintas não impróprias que correspondem
a retas Euclideanas paralelas cortam-se num ponto impróprio; toda reta não imprópria corta a reta imprópria num ponto impróprio.
A reta imprópria, sendo uma subvariedade não aberta de $\R P^2$, tem interior vazio e portanto o aberto de $\R P^2$ identificado com $\R^2$ é denso.
Já que $\R P^2$ é compacto (sendo um quociente da esfera $S^2$), vemos que o plano projetivo $\R P^2$ é uma {\em compactificação\/} de $\R^2$, i.e.,
$\R P^2$ é um espaço topológico compacto que contém uma cópia homeomorfa densa de $\R^2$.
\end{rem}

\begin{exercise}\label{exe:naosubnaosobre}
Sejam $M$, $N$ variedades diferenciáveis e $f:M\to N$ uma aplicação de classe $C^\infty$ que tenha posto constante
(i.e., o posto de $\dd f(x)$ não depende de $x\in M$), mas não seja uma submersão.
\begin{itemize}
\item[(a)] Mostre que todo ponto de $M$ possui uma vizinhança $U$ tal que $f(U)$ tem interior vazio em $N$
(sugestão: use o teorema do posto, i.e., o resultado do Exercício~\ref{exe:posto}).
\item[(b)] Assuma que $M$ tem base enumerável de abertos e que $N$ é Hausdorff. Use o teorema de Baire (Teorema~\ref{thm:Baire}) e o resultado
do item~(a) para mostrar que a imagem de $f$ tem interior vazio em $N$.
\item[(c)] Conclua que se $f:M\to N$ é uma imersão de classe $C^\infty$, $M$ tem base enumerável
de abertos, $N$ é Hausdorff e se $\Dim(M)<\Dim(N)$ então a imagem de $f$ tem interior vazio em $N$.
\end{itemize}
A hipótese de que $N$ é Hausdorff é na verdade desnecessária, mas facilita a demonstração (veja o Apêndice~\ref{sec:Baire} para um roteiro
da demonstração que não usa a hipótese de que $N$ é Hausdorff).
\end{exercise}

\begin{exercise}\label{exe:betacterk}
Sejam $G$ um grupo de Lie, $M$ uma variedade diferenciável e:
\[G\times M\ni(g,x)\longmapsto g\cdot x\in M\]
uma ação de classe $C^\infty$. Dado $x\in M$, denote por $\beta_x:G\to M$ a aplicação $g\mapsto g\cdot x$
e, dado $g\in G$, denote por $\gamma_g:M\to M$ a aplicação $x\mapsto g\cdot x$. Dados $g\in G$, $x\in M$, considere
o diagrama comutativo:
\[\xymatrix{%
G\ar[r]^{L_g}\ar[d]_{\beta_x}&G\ar[d]^{\beta_x}\\
M\ar[r]_{\gamma_g}&M}\]
onde $L_g:h\mapsto gh$ denota a translação à esquerda.
Diferencie esse diagrama para mostrar que a aplicação $\beta_x$ tem posto constante (sugestão $L_g$ e $\gamma_g$
são difeomorfismos).
\end{exercise}

\begin{exercise}\label{exe:betaxsubmers}
Sejam $G$ um grupo de Lie com base enumerável de abertos, $M$ uma variedade diferenciável Hausdorff e:
\[G\times M\ni(g,x)\longmapsto g\cdot x\in M\]
uma ação {\em transitiva\/} de classe $C^\infty$. Mostre que, para todo $x\in M$, a aplicação $\beta_x:G\ni g\mapsto g\cdot x\in M$
é uma submersão de classe $C^\infty$ (sugestão: use os resultados dos Exercícios~\ref{exe:naosubnaosobre} e \ref{exe:betacterk};
como no Exercício~\ref{exe:naosubnaosobre}, a hipótese de que $M$ é Hausdorff é na verdade desnecessária). Conclua
que, para todo $x\in M$, a variedade $M$ é difeomorfa à variedade quociente $G/G_x$ formada pelas coclasses à esquerda $gG_x$, $g\in G$, do
grupo de isotropia $G_x$ de $x$ definido por:
\[G_x=\big\{g\in G:g\cdot x=x\big\}.\]
\end{exercise}

\begin{exercise}
Considere a ação canônica do grupo linear geral $\GL(n,\R)$ na Grassmanniana $G_k(\R^n)$
(veja Apêndice~\ref{sec:Grassmanniana}). Mostre que tanto essa ação como sua restrição ao grupo ortogonal
$\Or(n)$ são transitivas. Conclua que a Grassmanniana $G_k(\R^n)$ é compacta e é difeomorfa às variedades quociente:
\[\GL(n,\R)/\GL(n,\R;k),\quad\Or(n)/\big(\!\Or(k)\times\Or(n-k)\big),\]
onde $\GL(n,\R;k)$ é o subgrupo de $\GL(n,\R)$ definido por:
\begin{multline}\label{eq:GLnRk}
\GL(n,\R;k)=\Big\{\begin{pmatrix}A&C\\0&B\end{pmatrix}:A\in\GL(k,\R),\ B\in\GL(n-k,\R),\\
C\in M_{k\times(n-k)}(\R)\Big\}
\end{multline}
e $\Or(k)\times\Or(n-k)$ é identificado com o subgrupo de $\Or(n)$ definido por:
\begin{equation}\label{eq:OkOn-k}
\Or(k)\times\Or(n-k)\cong\Big\{\begin{pmatrix}A&0\\0&B\end{pmatrix}:A\in\Or(k),\ B\in\Or(n-k)\Big\}.
\end{equation}
\end{exercise}

\begin{exercise}\label{exe:esferaquocO}
Considere a ação do grupo ortogonal $\Or(n)$ na esfera $S^{n-1}$ definida por:
\begin{equation}\label{eq:acaoOnSn}
\Or(n)\times S^{n-1}\ni(A,x)\longmapsto Ax\in S^{n-1}.
\end{equation}
Mostre que essa ação é transitiva e conclua que $S^{n-1}$ é difeomorfa à variedade quociente $\Or(n)/\Or(n-1)$,
onde $\Or(n-1)$ é identificado com o subgrupo de $\Or(n)$ formado pelas matrizes da forma:
\[\begin{pmatrix}1&0\\0&B\end{pmatrix},\quad B\in\Or(n-1).\]
Para $n\ge2$, mostre que a restrição da ação \eqref{eq:acaoOnSn} ao grupo especial ortogonal $\SO(n)$ também é transitiva
e conclua que $S^{n-1}$ é difeomorfa à variedade quociente $\SO(n)/\SO(n-1)$, onde $\SO(n-1)$ é identificado
com um subgrupo de $\SO(n)$ como acima.
\end{exercise}

\begin{exercise}
Considere a ação canônica do grupo linear geral complexo $\GL(n,\C)$ na Grassmanniana complexa $G_k(\C^n)$.
Mostre que tanto essa ação como sua restrição ao grupo unitário
$\Ur(n)$ são transitivas. Conclua que a Grassmanniana complexa $G_k(\C^n)$ é compacta e é difeomorfa às variedades quociente:
\[\GL(n,\C)/\GL(n,\C;k),\quad\Ur(n)/\big(\!\Ur(k)\times\Ur(n-k)\big),\]
onde $\GL(n,\C;k)$ é o subgrupo de $\GL(n,\C)$ definido de modo análogo a \eqref{eq:GLnRk} e
$\Ur(k)\times\Ur(n-k)$ é identificado com o subgrupo de $\Ur(n)$ definido de modo análogo a \eqref{eq:OkOn-k}.
\end{exercise}

\begin{rem}
A ação do grupo ortogonal complexo $\Or(n,\C)$ na Grassmanniana complexa $G_k(\C^n)$ {\em não é\/} transitiva em geral!
De fato, se $W$ é um subespaço complexo de $\C^n$ então a restrição do produto bilinear complexo \eqref{eq:prodbilincompl} a $W$
é uma forma bilinear complexa cujo posto\footnote{%
Se $B$ é uma forma bilinear num espaço vetorial $W$ então o {\em núcleo\/} de $B$ é o subespaço $\Ker(B)$ de $W$ formado pelos vetores $w\in W$ tais que
$B(w,u)=0$, para todo $u\in W$. O {\em posto\/} de $B$ é a dimensão de $W/\Ker(B)$.}
$r(W)$ depende de $W$. Se $W,W'\in G_k(\C^n)$ são tais que $r(W)\ne r(W')$ então a ação de um elemento
de $\Or(n,\C)$ não pode levar $W$ sobre $W'$. Por exemplo, se $W=\C(1,i)\in G_1(\C^2)$ e $W'=\C(1,0)\in G_1(\C^2)$ então $r(W)=0$ e $r(W')=1$.
\end{rem}

\begin{exercise}\label{exe:esferaquocU}
Considere a ação do grupo unitário $\Ur(n)$ na esfera $S^{2n-1}$ (vista como um subconjunto de $\C^n$) definida por:
\begin{equation}\label{eq:acaoUnSn}
\Ur(n)\times S^{2n-1}\ni(A,z)\longmapsto Az\in S^{2n-1}.
\end{equation}
Mostre que essa ação é transitiva e conclua que $S^{2n-1}$ é difeomorfa à variedade quociente $\Ur(n)/\Ur(n-1)$,
onde $\Ur(n-1)$ é identificado com o subgrupo de $\Ur(n)$ formado pelas matrizes da forma:
\[\begin{pmatrix}1&0\\0&B\end{pmatrix},\quad B\in\Ur(n-1).\]
Para $n\ge2$, mostre que a restrição da ação \eqref{eq:acaoUnSn} ao grupo especial unitário $\SU(n)$ também é transitiva
e conclua que $S^{2n-1}$ é difeomorfa à variedade quociente $\SU(n)/\SU(n-1)$, onde $\SU(n-1)$ é identificado
com um subgrupo de $\SU(n)$ como acima.
\end{exercise}

\begin{exercise}
O resultado dos Exercícios~\ref{exe:esferaquocO} e \ref{exe:esferaquocU} implicam que o círculo $S^1$ é difeomorfo a $\SO(2)$ e que a esfera $S^3$
é difeomorfa a $\SU(2)$. Na verdade, algo mais forte pode ser mostrado. Mostre que:
\begin{itemize}
\item[(a)] a aplicação:
\[\SO(2)\ni\begin{pmatrix}a&-b\\b&a\end{pmatrix}\longmapsto a+bi\in S^1\subset\C\]
é um difeomorfismo de classe $C^\infty$ e um isomorfismo de grupos;
\item[(b)] a aplicação (recorde Exercício~\ref{exe:quaternios}):
\[\SU(2)\ni\begin{pmatrix}z&-\bar w\\w&\bar z\end{pmatrix}\longmapsto z+jw\in S^3\subset\IH\]
é um difeomorfismo de classe $C^\infty$ e um isomorfismo de grupos, onde $\bar z$ denota o conjugado do número complexo $z$ (sugestão:
a conta fica mais fácil se você usar a identidade $zj=j\bar z$, $z\in\C$).
\end{itemize}
\end{exercise}

\begin{exercise}
Sejam $G$ um grupo de Lie com base enumerável de abertos, $M$ um conjunto e seja dada uma ação transitiva $G\times M\to M$ de $G$ em $M$.
Se $\mathcal A$, $\mathcal A'$ são atlas maximais em $M$ que fazem dessa ação uma aplicação de classe $C^\infty$, mostre que $\mathcal A=\mathcal A'$
(sugestão: use o resultado do Exercício~\ref{exe:betaxsubmers}). Conclua que o atlas maximal que definimos na Grassmanniana (real ou complexa) é o único que
faz da ação canônica do grupo linear geral (real ou complexo, respectivamente)
uma aplicação de classe $C^\infty$ (veja Apêndice~\ref{sec:Grassmanniana}).
\end{exercise}

\end{section}

\begin{section}{Dia 13/09}
\label{sec:1309}

Dada uma variedade diferenciável $M$, então seu {\em fibrado tangente\/} é o conjunto:
\[TM=\bigcup_{x\in M}\big(\{x\}\times T_xM\big)\]
e a {\em projeção canônica\/} de $TM$ é a aplicação $\pi:TM\to M$ definida por:
\[\pi(x,v)=x,\quad x\in M,\ v\in T_xM.\]
Dadas variedades diferenciáveis $M$, $N$ e uma aplicação $f:M\to N$ de classe $C^\infty$, então a {\em diferencial\/} de $f$ é a aplicação $\dd f:TM\to TN$ definida por:
\[\dd f(x,v)=\dd f_x(v),\quad x\in M,\ v\in T_xM.\]
Se $M$ é um aberto de um espaço vetorial real de dimensão finita $E$ então identificamos $T_xM$ com $E$, para todo $x\in M$, e portanto identificamos
o fibrado tangente $TM$ com $M\times E$. Dados um aberto $M$ de um espaço vetorial real de dimensão finita $E$, um aberto $N$ de um espaço vetorial real de
dimensão finita $F$ e uma aplicação $f:M\to N$ de classe $C^\infty$ então a diferencial $\dd f:TM\to TN$ (definida como acima) é a aplicação:
\begin{equation}\label{eq:dfEF}
M\times E\ni(x,v)\longmapsto\big(f(x),\dd f_x(v)\big)\in N\times F.
\end{equation}
Observamos que, em cursos de Cálculo no $\R^n$, é usual denotar por $\dd f$ a aplicação $M\ni x\mapsto\dd f_x\in\Lin(E,F)$, que não coincide exatamente
com a aplicação \eqref{eq:dfEF}. Nós usaremos a notação $\dd f$ tanto para a aplicação:
\[E\supset M\ni x\longmapsto\dd f_x\in\Lin(E,F)\]
como para a aplicação \eqref{eq:dfEF} e esperamos que o pequeno conflito de notação não crie confusão.

Recorde que se $N$ é uma subvariedade (possivelmente imersa) de uma variedade diferenciável $M$ então, para cada $x\in N$, identificamos $T_xN$ com um subespaço
de $T_xM$, de modo que o fibrado tangente $TN$ é identificado com um subconjunto de $TM$. Em particular, se $U$ é um aberto de $M$ então identificamos $TU$ com
um subconjunto de $TM$; obviamente:
\[TU=\pi^{-1}(U),\]
onde $\pi:TM\to M$ denota a projeção canônica.

\begin{exercise}\label{exe:basicodf}
Sejam $M$, $N$ variedades diferenciáveis e $f:M\to N$ uma aplicação de classe $C^\infty$.
\begin{itemize}
\item[(a)] Se $P$ é uma subvariedade (possivelmente imersa) de $M$, mostre que $\dd(f\vert_P)=\dd f\vert_{TP}$ (sugestão: use o resultado
do item~(a) do Exercício~\ref{exe:difdomcodom}).
\item[(b)] Se $P$ é uma subvariedade (possivelmente imersa) de $N$, $f(M)\subset P$ e a aplicação $f_0:M\to P$ que difere de $f$ apenas pelo contra-domínio
é de classe $C^\infty$, mostre que $\dd f_0:TM\to TP$ e $\dd f:TM\to TN$ diferem apenas pelo contra-domínio (sugestão: use o resultado
do item~(b) do Exercício~\ref{exe:difdomcodom}).
\item[(c)] Se $P$ é uma variedade diferenciável e $g:N\to P$ é uma aplicação de classe $C^\infty$, mostre que $\dd(g\circ f)=\dd g\circ\dd f$.
\item[(d)] Mostre que a diferencial da aplicação identidade de $M$ é a aplicação identidade de $TM$.
\item[(e)] Se $f$ é um difeomorfismo, mostre que $\dd f:TM\to TN$ é uma bijeção cuja inversa é $\dd(f^{-1})$.
\end{itemize}
\end{exercise}

\begin{exercise}[atlas no fibrado tangente]\label{exe:fibradotangente}
Seja $(M,\mathcal A)$ uma variedade diferenciável. Mostre que:
\begin{equation}\label{eq:atlasTM}
\big\{\dd\varphi:\varphi\in\mathcal A\big\}
\end{equation}
é um atlas no fibrado tangente $TM$ (sugestão: para escrever a função de transição entre $\dd\varphi$ e $\dd\psi$, $\varphi,\psi\in\mathcal A$,
use o resultado dos itens~(c) e (e) do Exercício~\ref{exe:basicodf}).
A partir de agora, consideramos $TM$ sempre munido do atlas maximal que contém \eqref{eq:atlasTM}.
\end{exercise}

\begin{exercise}
Seja $M$ uma variedade diferenciável. Mostre que a projeção canônica $\pi:TM\to M$
é uma submersão de classe $C^\infty$ (sugestão: considere a representação de $\pi$ com respeito às cartas $\dd\varphi$ e $\varphi$,
onde $\varphi$ é uma carta de $M$).
\end{exercise}

\begin{exercise}
Sejam $M$ uma variedade diferenciável e $U$ uma subvariedade aberta de $M$. Mostre que $TU$ é uma subvariedade aberta de $TM$ (observe que você não deve
verificar apenas que $TU$ é um subconjunto aberto de $TM$ --- o que segue diretamente da continuidade da projeção canônica --- mas também que
o atlas maximal que $TU$ possui por ser o fibrado tangente de $U$ coincide com o atlas maximal que $TU$ herda de $TM$).
\end{exercise}

\begin{exercise}
Seja $E$ um espaço vetorial real de dimensão finita. Identificando o fibrado tangente $TE$ com $E\times E$, mostre que
o atlas maximal que $TE$ possui por ser o fibrado tangente de $E$ coincide com a estrutura diferenciável canônica do espaço vetorial $E\times E$
(sugestão: se $\varphi:E\to\R^n$ é um isomorfismo linear então $\dd\varphi$ é uma carta do fibrado tangente $TE$; verifique que $\dd\varphi$ é linear
e conclua que $\dd\varphi$ define a estrutura diferenciável canônica do espaço vetorial $E\times E$).
\end{exercise}

\begin{exercise}
Seja $M$ uma variedade diferenciável. Mostre que:
\begin{itemize}
\item[(a)] se $M$ é Hausdorff então $TM$ é Hausdorff (sugestão: dados $v,w\in TM$ distintos, considere os casos $\pi(v)=\pi(w)$ e $\pi(v)\ne\pi(w)$; no
segundo caso, use a continuidade de $\pi$);
\item[(b)] se $M$ possui base enumerável de abertos então $TM$ possui base enumerável de abertos (sugestão: observe que uma variedade diferenciável admite
base enumerável de abertos se e somente se seu atlas maximal contém um atlas enumerável).
\end{itemize}
\end{exercise}

\begin{exercise}
Sejam $M$, $N$ variedades diferenciáveis e $f:M\to N$ uma aplicação de classe $C^\infty$. Dadas cartas:
\[\varphi:U\subset M\longrightarrow\widetilde U\subset\R^m,\quad\psi:V\subset N\longrightarrow\widetilde V\subset\R^n\]
tais que $f(U)\subset V$, seja $\tilde f=\psi\circ f\circ\varphi^{-1}$ a representação de $f$. Mostre que
a representação de $\dd f:TM\to TN$ com respeito às cartas $\dd\varphi$ e $\dd\psi$ é:
\begin{equation}\label{eq:dtildef}
\dd\tilde f:\widetilde U\times\R^m\ni(x,u)\longmapsto\big(\tilde f(x),\dd\tilde f_x(u)\big)\in\widetilde V\times\R^n
\end{equation}
(sugestão: use o resultado do item~(c) do Exercício~\ref{exe:basicodf}). Conclua que $\dd f$ é uma aplicação de classe $C^\infty$.
\end{exercise}

\begin{exercise}\label{exe:dimersubmer}
Sejam $M$, $N$ variedades diferenciáveis e $f:M\to N$ uma aplicação de classe $C^\infty$. Mostre que:
\begin{itemize}
\item[(a)] se $f$ é uma imersão então $\dd f:TM\to TN$ é uma imersão;
\item[(b)] se $f$ é uma submersão então $\dd f:TM\to TN$ é uma submersão
\end{itemize}
(sugestão: considere uma representação $\tilde f$ de $f$ usando cartas e a representação \eqref{eq:dtildef} de $\dd f$ usando as cartas correspondentes
nos fibrados tangentes. Uma maneira de resolver o exercício é diferenciar \eqref{eq:dtildef} para verificar diretamente que se $\tilde f$ é uma imersão
então \eqref{eq:dtildef} é uma imersão e que se $\tilde f$ é uma submersão então \eqref{eq:dtildef} é uma submersão. Uma outra opção é usar cartas cuja existência
é dada pelas formas locais das imersões/submersões).
\end{exercise}

\begin{exercise}
Sejam $M$ uma variedade diferenciável e $N$ uma subvariedade imersa de $M$. Mostre que $TN$ é uma subvariedade imersa de $TM$ (sugestão:
use o resultado do item~(a) do Exercício~\ref{exe:dimersubmer}).
\end{exercise}

\begin{exercise}\label{exe:Tsubvar}
Sejam $M$ uma variedade diferenciável e $N$ uma subvariedade mergulhada de $M$. Seja $\varphi:U\subset M\to\widetilde U\subset\R^n$ uma carta de subvariedade
para $N$, i.e.:
\[\varphi(U\cap N)=\widetilde U\cap\big(\R^k\times\{0\}^{n-k}\big).\]
Mostre que:
\[\dd\varphi(TU\cap TN)=(\widetilde U\times\R^n)\cap\big(\R^k\times\{0\}^{n-k}\times\R^k\times\{0\}^{n-k}\big).\]
Conclua que $TN$ é uma subvariedade mergulhada de $TM$ (note que não estamos afirmando apenas que o subconjunto $TN$ de $TM$ admite um atlas maximal
que faz da inclusão de $TN$ em $TM$ um mergulho, mas também que tal atlas maximal coincide com aquele que $TN$ possui por ser o fibrado tangente de $N$).
\end{exercise}

\begin{exercise}\label{exe:dmergulho}
Sejam $M$, $N$ variedades diferenciáveis e $f:M\to N$ um mergulho de classe $C^\infty$. Mostre que $\dd f:TM\to TN$ é um mergulho de classe $C^\infty$
(sugestão: note que $f(M)$ é uma subvariedade de $N$ e que a aplicação $f_0:M\to f(M)$ que difere de $f$ apenas pelo contra-domínio é um difeomorfismo de classe
$C^\infty$. Note também que $\dd f$ é a composição de $\dd f_0$ com a inclusão de $T\big(f(M)\big)$ em $TN$, que
$\dd f_0$ é um difeomorfismo e use o resultado do Exercício~\ref{exe:Tsubvar}).
\end{exercise}

\begin{rem}
É possível obter o resultado do item~(a) do Exercício~\ref{exe:dimersubmer} como corolário do resultado do Exercício~\ref{exe:dmergulho}, notando
que uma aplicação é uma imersão se e somente se for localmente um mergulho.
\end{rem}

\begin{exercise}\label{exe:TMNTMTN}
Sejam $M$, $N$ variedades diferenciáveis e $\pi^1:M\times N\to M$, $\pi^2:M\times N\to N$ as projeções. Mostre que a aplicação:
\[(\dd\pi^1,\dd\pi^2):T(M\times N)\longrightarrow TM\times TN\]
é um difeomorfismo de classe $C^\infty$ (sugestão: em primeiro lugar, use o resultado do Exercício~\ref{exe:esptanprod} para mostrar que
$(\dd\pi^1,\dd\pi^2)$ é uma bijeção. Em segundo lugar, mostre que $(\dd\pi^1,\dd\pi^2)$ é um difeomorfismo local usando a seguinte estratégia:
se $\varphi$ é uma carta de $M$ e $\psi$ é uma carta de $N$ então $\varphi\times\psi$ é uma carta
de $M\times N$, $\dd(\varphi\times\psi)$ é uma carta de $T(M\times N)$ e $(\dd\varphi)\times(\dd\psi)$ é uma carta de $TM\times TN$. Escreva a representação
da função $(\dd\pi^1,\dd\pi^2)$ com respeito às cartas $\dd(\varphi\times\psi)$ e $(\dd\varphi)\times(\dd\psi)$).
\end{exercise}

Um {\em campo vetorial\/} numa variedade diferenciável $M$ é uma função:
\[X:M\longrightarrow TM\]
tal que\footnote{%
A rigor, $X(x)$ está em $\{x\}\times T_xM$, mas identificamos $\{x\}\times T_xM$ com $T_xM$, sempre que não houver perigo de confusão.}
$X(x)\in T_xM$ para todo $x\in M$, i.e., tal que
$\pi\circ X$ é a aplicação identidade de $M$, onde $\pi:TM\to M$ denota a projeção canônica.
Note que se $M$ é um aberto de um espaço vetorial real de dimensão finita $E$ então (identificando $TM$ com $M\times E$) um campo vetorial
em $M$ é uma aplicação da forma:
\begin{equation}\label{eq:campoaberto}
M\ni x\longmapsto\big(x,X(x)\big)\in M\times E,
\end{equation}
onde $X:M\to E$ é uma função. Em cursos de Cálculo no $\R^n$, é a aplicação $X:M\to E$ que é chamada de ``campo vetorial'' no aberto $M$ de $E$,
e não a aplicação \eqref{eq:campoaberto}. Quando $M$ é um aberto de um espaço vetorial real de dimensão finita $E$, nós criaremos então um novo
pequeno conflito terminológico, pois chamaremos de ``campo vetorial'' em $M$ tanto a aplicação \eqref{eq:campoaberto} como a aplicação $X:M\to E$.

\begin{exercise}
Sejam $M$ uma variedade diferenciável e $X$ um campo vetorial de classe $C^\infty$ em $M$. Mostre que a aplicação $X:M\to TM$ é um mergulho
(sugestão: note que $\pi$ é uma inversa à esquerda de classe $C^\infty$ para $X$. Para mostrar que $X$ é uma imersão, use o resultado do Exercício~\ref{exe:retracaolocal}).
\end{exercise}

\begin{defin}\label{thm:defpullbackvecfield}
Sejam $M$, $N$ variedades diferenciáveis e $\phi:M\to N$ uma aplicação de classe $C^\infty$. Se $\dd\phi_x:T_xM\to T_{f(x)}N$ é um isomorfismo para todo $x\in M$
(i.e., se $\phi$ é um difeomorfismo local) e se $X$ é um campo vetorial em $N$ então o {\em pull-back\/} $\phi^*X$ é o campo vetorial em $M$ definido por:
\[(\phi^*X)(x)=\dd\phi_x^{-1}\big[X\big(\phi(x)\big)\big],\quad x\in M.\]
Se $\phi:M\to N$ é um difeomorfismo e $X$ é um campo vetorial em $M$, então o {\em push-forward\/} $\phi_*X$ é o campo vetorial em $N$ definido por:
\[(\phi_*X)(y)=\dd\phi_{\phi^{-1}(y)}\big[X\big(\phi^{-1}(y)\big)\big],\quad y\in N.\]
\end{defin}
Note que o push-forward $\phi_*X$ é precisamente o pull-back $(\phi^{-1})^*X$ de $X$ pela aplicação inversa $\phi^{-1}:N\to M$.

\smallskip

Se $M$ é uma variedade diferenciável, $X:M\to TM$ é um campo vetorial em $M$ e $U$ é um aberto de $M$ então $X\vert_U$ (com o contra-domínio
alterado para a subvariedade aberta $TU$ de $TM$) é um campo vetorial em $U$. Dada uma outra variedade diferenciável $N$ e um difeomorfismo $\phi:U\to N$
de classe $C^\infty$ então nós escreveremos simplesmente $\phi_*X$ para denotar o push-forward $\phi_*(X\vert_U)$.

\begin{exercise}\label{exe:pbpscampos}
Sejam $M$, $N$, $P$ variedades diferenciáveis e $\phi:M\to N$, $\lambda:N\to P$ difeomorfismos locais de classe $C^\infty$. Mostre que:
\begin{itemize}
\item[(a)] se $X$ é um campo vetorial em $P$ então $(\lambda\circ\phi)^*X=\phi^*\lambda^*X$;
\item[(b)] se $\phi$ e $\lambda$ são difeomorfismos e $X$ é um campo vetorial em $M$ então $(\lambda\circ\phi)_*X=\lambda_*\phi_*X$.
\end{itemize}
\end{exercise}

\begin{exercise}\label{exe:Xcartas}
Sejam $M$ uma variedade diferenciável, $X:M\to TM$ um campo vetorial e $\varphi:U\subset M\to\widetilde U\subset\R^n$ uma carta. Temos que
o push-forward $\varphi_*X$ é um campo vetorial:
\[\varphi_*X:\widetilde U\ni u\longmapsto\big(u,\widetilde X(u)\big)\in\widetilde U\times\R^n\]
na variedade $\widetilde U$. Mostre que a aplicação $\varphi_*X:\widetilde U\to\widetilde U\times\R^n$ é precisamente a representação da aplicação $X:M\to TM$
com respeito às cartas $\varphi$ e $\dd\varphi$. Conclua que, dado um atlas $\mathcal A$ em $M$ (contido no atlas maximal de $M$) então um campo vetorial
$X:M\to TM$ é de classe $C^\infty$ se e somente se $\varphi_*X$ é de classe $C^\infty$, para qualquer $\varphi\in\mathcal A$.
\end{exercise}

\begin{exercise}
Sejam $M$, $N$ variedades diferenciáveis e $\phi:M\to N$ uma aplicação de classe $C^\infty$. Mostre que:
\begin{itemize}
\item[(a)] se $\phi$ é um difeomorfismo local e $X$ é um campo vetorial de classe $C^\infty$ em $N$ então $\phi^*X$ é um campo vetorial de classe $C^\infty$ em $M$
(sugestão: considere a representação de $\phi$ com respeito a cartas e use o resultado do Exercício~\ref{exe:Xcartas});
\item[(b)] se $\phi$ é um difeomorfismo e $X$ é um campo vetorial de classe $C^\infty$ em $M$ então $\phi_*X$ é um campo vetorial de classe $C^\infty$ em $N$
(sugestão: use o resultado do item~(a) para $\phi^{-1}$).
\end{itemize}
\end{exercise}

\begin{exercise}\label{exe:XMCMmodulo}
Seja $M$ uma variedade diferenciável. O conjunto de todas as funções $f:M\to\R$ é uma álgebra comutativa com unidade, munida das operações
de soma e multiplicação de funções definidas ponto a ponto. O conjunto de todos os campos vetoriais em $M$ é um espaço vetorial real e um módulo
sobre a álgebra das funções $f:M\to\R$, sendo as operações em questão definidas ponto a ponto. Mostre que:
\begin{itemize}
\item[(a)] o conjunto $C^\infty(M)$ de todas as funções $f:M\to\R$ de classe $C^\infty$ é uma subálgebra da álgebra de todas
as funções $f:M\to\R$ (além do mais, se $E$ é um espaço vetorial real de dimensão finita então o conjunto de todas as funções $f:M\to E$
de classe $C^\infty$ é um subespaço do espaço vetorial real de todas as funções em $M$ a valores em $E$, com operações definidas ponto a ponto);
\item[(b)] o conjunto $\mathfrak X(M)$ de todos os campos vetoriais em $M$ de classe $C^\infty$ é um subespaço do espaço vetorial de todos os campos
vetoriais em $M$ e o produto $fX$ está em $\mathfrak X(M)$, se $f\in C^\infty(M)$ e $X\in\mathfrak X(M)$. Conclua que $\mathfrak X(M)$ é um $C^\infty(M)$-módulo
(sugestão: use o resultado do Exercício~\ref{exe:Xcartas}).
\end{itemize}
\end{exercise}

\begin{exercise}\label{exe:restrcampoimersa}
Sejam $M$, $N$ variedades diferenciáveis e $\phi:N\to M$ uma imersão de classe $C^\infty$. Seja $X:M\to TM$ um campo vetorial de classe $C^\infty$ e suponha que
$X\big(\phi(x)\big)\in\dd\phi_x(T_xN)$, para todo $x\in N$. Seja $X_0:N\to TN$ o único campo vetorial tal que:
\[\dd\phi_x\big(X_0(x)\big)=X\big(\phi(x)\big),\]
para todo $x\in N$. Mostre que $X_0$ é de classe $C^\infty$ (sugestão: seja $U\subset N$ um aberto tal que $\phi\vert_U$ seja um mergulho, de modo que
também $\dd\phi\vert_{TU}$ é um mergulho, pelo resultado do Exercício~\ref{exe:dmergulho}. Mostre que $X_0\vert_U$ é de classe $C^\infty$ usando o fato que o diagrama:
\[\xymatrix@C+10pt{%
M\ar[r]^-X&TM\\
U\ar[u]^{\phi\vert_U}\ar[r]_{X_0\vert_U}&TU\ar[u]_{\dd\phi\vert_{TU}}}\]
comuta). Conclua (considerando o caso particular em que $\phi$ é a aplicação inclusão)
que se $N$ é uma subvariedade imersa de $M$ e se $X:M\to TM$ é um campo vetorial de classe $C^\infty$ tal que $X(x)\in T_xN$, para todo $x\in N$,
então $X\vert_N:N\to TN\subset TM$ é um campo vetorial de classe $C^\infty$ em $N$.
\end{exercise}

\end{section}

\begin{section}{Dia 15/09}

\begin{exercise}\label{exe:bilintang}
Sejam $M$ uma variedade diferenciável e $x\in M$ um ponto. Suponha que para cada carta $\varphi:U\subset M\to\widetilde U\subset\R^n$ em $M$
com $x\in U$ seja dada uma aplicação bilinear $B^\varphi\in\Lin_2(\R^n,\R)$. Mostre que as seguintes condições são equivalentes:
\begin{itemize}
\item[(a)] o pull-back $(\dd\varphi_x)^*B^\varphi\in\Lin_2(T_xM,\R)$ de $B^\varphi$ por $\dd\varphi_x:T_xM\to\R^n$ não depende da carta $\varphi$;
\item[(b)] dadas cartas $\varphi:U\subset M\to\widetilde U\subset\R^n$, $\psi:V\subset M\to\widetilde V\subset\R^n$ com $x\in U\cap V$ então,
denotando por $\alpha=\psi\circ\varphi^{-1}$ a função de transição de $\varphi$ para $\psi$, vale que $(\dd\alpha_{\varphi(x)})^*B^\psi=B^\varphi$
\end{itemize}
(sugestão: use o resultado do Exercício~\ref{exe:pullbackcomposta}).
\end{exercise}

\begin{exercise}\label{exe:Hessf}
Sejam $M$ uma variedade diferenciável, $f:M\to\R$ uma função de classe $C^\infty$ e $x\in M$ um ponto. Dada uma carta:
\[\varphi:U\subset M\longrightarrow\widetilde U\subset\R^n\]
em $M$ com $x\in U$, denote por $f^\varphi:\widetilde U\to\R$ a representação $f^\varphi=f\circ\varphi^{-1}$ de $f$ com respeito à carta $\varphi$
e denote por $B^\varphi\in\Lin_2(\R^n,\R)$ a aplicação bilinear:
\[B^\varphi=\dd^2f^\varphi\big(\varphi(x)\big).\]
Mostre que, dada uma carta $\psi:V\subset M\to\widetilde V\subset\R^n$, se denotamos por $\alpha=\psi\circ\varphi^{-1}$ a função de transição
de $\varphi$ para $\psi$ então
as funções $f^\varphi$ e $f^\psi\circ\alpha$ coincidem sobre o domínio $\varphi(U\cap V)$ de $\alpha$. Diferenciando $f^\psi\circ\alpha$
duas vezes no ponto $\varphi(x)$, conclua que:
\[B^\varphi=(\dd\alpha_{\varphi(x)})^*B^\psi+\dd f^\psi\big(\psi(x)\big)\circ\dd^2\alpha\big(\varphi(x)\big).\]
Conclua também, usando o resultado do Exercício~\ref{exe:bilintang}, que a aplicação bilinear $(\dd\varphi_x)^*B^\varphi$ é independente da carta $\varphi$ se
$\dd f(x)=0$ (na verdade, vale também a recíproca dessa afirmação,
i.e., se $(\dd\varphi_x)^*B^\varphi$ é independente de $\varphi$ então $\dd f(x)=0$, mas a demonstração
é um pouco mais difícil! Você pode tentar provar essa recíproca usando a seguinte sugestão: dada uma carta $\varphi$ qualquer em torno de $x$, você pode escolher um
difeomorfismo $\alpha$ qualquer e definir uma nova carta $\psi$ fazendo $\psi=\alpha\circ\varphi$. Mostre que o difeomorfismo $\alpha$ pode ser escolhido
de modo que $\dd^2\alpha\big(\varphi(x)\big)$ seja uma forma bilinear simétrica arbitrária; para isso, tome $\alpha$ como sendo a restrição a uma vizinhança
de $\varphi(x)$ de um polinômio de grau $2$).
\end{exercise}

\begin{defin}
Se $M$ é uma variedade diferenciável, $f:M\to\R$ é uma função de classe $C^\infty$ e $x\in M$ é um {\em ponto crítico\/} de $f$, i.e., um ponto
tal que $\dd f(x)=0$ então a aplicação bilinear $(\dd\varphi_x)^*B^\varphi\in\Lin_2(T_xM,\R)$ cuja independência da carta $\varphi$ foi estabelecida
no Exercício~\ref{exe:Hessf} é chamada a {\em Hessiana\/} (ou {\em diferencial segunda}) de $f$ no ponto crítico $x$ e é denotada por $\dd^2f(x)$.
\end{defin}

\begin{exercise}\label{exe:lintang}
Sejam $M$ uma variedade diferenciável e $x\in M$ um ponto. Suponha que para cada carta $\varphi:U\subset M\to\widetilde U\subset\R^n$ em $M$
com $x\in U$ seja dado um operador linear $L^\varphi:\R^n\to\R^n$. Mostre que as seguintes condições são equivalentes:
\begin{itemize}
\item[(a)] o operador linear $\dd\varphi_x^{-1}\circ L^\varphi\circ\dd\varphi_x:T_xM\to T_xM$ não depende da carta $\varphi$;
\item[(b)] dadas cartas $\varphi:U\subset M\to\widetilde U\subset\R^n$, $\psi:V\subset M\to\widetilde V\subset\R^n$ com $x\in U\cap V$ então,
denotando por $\alpha=\psi\circ\varphi^{-1}$ a função de transição de $\varphi$ para $\psi$,
vale que $\dd\alpha_{\varphi(x)}\circ L^\varphi=L^\psi\circ\dd\alpha_{\varphi(x)}$.
\end{itemize}
\end{exercise}

\begin{exercise}\label{exe:linearizarX}
Sejam $M$ uma variedade diferenciável, $X:M\to TM$ um campo vetorial de classe $C^\infty$ e $x\in M$ um ponto. Dada uma carta:
\[\varphi:U\subset M\longrightarrow\widetilde U\subset\R^n\]
denote por $X^\varphi:\widetilde U\to\R^n$ o campo vetorial $X^\varphi=\varphi_*X$ e denote
por $L^\varphi$ o operador linear em $\R^n$ definido por:
\[L^\varphi=\dd X^\varphi\big(\varphi(x)\big).\]
Mostre que, dada uma carta $\psi:V\subset M\to\widetilde V\subset\R^n$, se denotamos por $\alpha=\psi\circ\varphi^{-1}$ a função de transição
de $\varphi$ para $\psi$ então:
\begin{equation}\label{eq:Xvarphipsi}
\dd\alpha_z\big(X^\varphi(z)\big)=X^\psi\big(\alpha(z)\big),
\end{equation}
para todo $z$ no domínio $\varphi(U\cap V)$ de $\alpha$, i.e., $\alpha_*X^\varphi$ é a restrição de $X^\psi$ à imagem de $\alpha$ (sugestão:
use o resultado do Exercício~\ref{exe:pbpscampos}). Diferencie \eqref{eq:Xvarphipsi} no ponto $\varphi(x)$ para mostrar que:
\[H+\dd\alpha_{\varphi(x)}\circ L^\varphi=L^\psi\circ\dd\alpha_{\varphi(x)},\]
onde $H:\R^n\to\R^n$ é definido por:
\[H(v)=\dd^2\alpha_{\varphi(x)}\big[v,X^\varphi\big(\varphi(x)\big)\big],\quad v\in\R^n.\]
Conclua, usando o resultado do Exercício~\ref{exe:lintang}, que o operador linear:
\begin{equation}\label{eq:conjdvarphiLvarphi}
\dd\varphi_x^{-1}\circ L^\varphi\circ\dd\varphi_x
\end{equation}
não depende da carta $\varphi$ se $X(x)=0$ (na verdade, vale também a recíproca dessa afirmação, i.e., se \eqref{eq:conjdvarphiLvarphi} é independente
de $\varphi$ então $X(x)=0$, mas a demonstração é um pouco mais difícil! Veja a sugestão dada no enunciado do Exercício~\ref{exe:Hessf}).
\end{exercise}

\begin{defin}
Se $M$ é uma variedade diferenciável, $X:M\to TM$ é um campo vetorial de classe $C^\infty$ e $x\in M$ é uma {\em singularidade\/} de $X$, i.e., um ponto
tal que $X(x)=0$ então o operador linear \eqref{eq:conjdvarphiLvarphi} em $T_xM$ cuja independência da carta $\varphi$ foi estabelecida
no Exercício~\ref{exe:linearizarX} é chamado a {\em linearização\/} do campo vetorial $X$ na singularidade $x$.
\end{defin}

\end{section}

\begin{section}{Dia 13/10}

Sejam $M$ uma variedade diferenciável e $(E_x)_{x\in M}$ uma família de espaços vetoriais reais de dimensão $k<+\infty$. Uma {\em trivialização local\/}
(ou, abreviadamente, uma {\em trivialização}) da família $(E_x)_{x\in M}$ consiste de uma família $\alpha=(\alpha_x)_{x\in U}$, onde $U$ é um aberto de $M$ e
$\alpha_x:\R^k\to E_x$ é um isomorfismo linear
para todo $x\in U$. O aberto $U=\Dom(\alpha)$ é denominado o {\em domínio\/} da trivialização local $\alpha$. Dadas trivializações locais $\alpha=(\alpha_x)_{x\in U}$ e
$\beta=(\beta_x)_{x\in V}$ de $(E_x)_{x\in M}$, então a {\em função de transição\/}
de $\alpha$ para $\beta$ é definida por:
\[g_{\alpha\beta}:U\cap V\ni x\longmapsto\alpha_x^{-1}\circ\beta_x\in\GL(k,\R).\]
As trivializações locais $\alpha$ e $\beta$ são ditas {\em compatíveis\/} se a função de transição $g_{\alpha\beta}$ é de classe $C^\infty$.
Um {\em atlas de trivializações\/} de $(E_x)_{x\in M}$ é um conjunto $\mathfrak A$ de trivializações locais de $(E_x)_{x\in M}$, duas a duas
compatíveis, cujos domínios cobrem $M$. Um atlas de trivializações $\mathfrak A$ de $(E_x)_{x\in M}$ é dito {\em maximal\/} se não está contido propriamente em nenhum
atlas de trivializações de $(E_x)_{x\in M}$.

\begin{exercise}\label{exe:alphabetagamma}
Mostre que se $\alpha=(\alpha_x)_{x\in U}$, $\beta=(\beta_x)_{x\in V}$ e $\gamma=(\gamma_x)_{x\in W}$ são trivializações locais de $(E_x)_{x\in M}$ então:
\[g_{\alpha\gamma}(x)=g_{\alpha\beta}(x)\circ g_{\beta\gamma}(x),\]
para todo $x\in U\cap V\cap W$.
\end{exercise}

Se $\mathfrak A$ é um conjunto de trivializações locais de $(E_x)_{x\in M}$ então dizemos que uma trivialização local $\alpha$ de $(E_x)_{x\in M}$ é
{\em compatível\/} com $\mathfrak A$ se for compatível com qualquer $\beta\in\mathfrak A$.
\begin{exercise}\label{exe:alphabetafrakA}
Sejam $\alpha$, $\beta$ trivializações locais de $(E_x)_{x\in M}$ e $\mathfrak A$ um conjunto de trivializações locais de $(E_x)_{x\in M}$ tal que a união
dos domínios dos elementos de $\mathfrak A$ contém $\Dom(\alpha)\cap\Dom(\beta)$. Mostre que se $\alpha$ e $\beta$ são compatíveis com $\mathfrak A$ então
$\alpha$ é compatível com $\beta$ (sugestão: use o resultado do Exercício~\ref{exe:alphabetagamma}).
\end{exercise}

\begin{exercise}
Seja $\mathfrak A$ um atlas de trivializações de $(E_x)_{x\in M}$.
\begin{itemize}
\item[(a)] Mostre que uma trivialização local $\alpha$ de $(E_x)_{x\in M}$ é compatível com $\mathfrak A$ se e somente se $\mathfrak A\cup\{\alpha\}$ é um atlas
de trivializações.
\item[(b)] Mostre que $\mathfrak A$ é um atlas maximal de trivializações de $(E_x)_{x\in M}$ se e somente se qualquer trivialização local de $(E_x)_{x\in M}$
compatível com $\mathfrak A$ pertence a $\mathfrak A$.
\end{itemize}
\end{exercise}

\begin{exercise}
Seja $\mathfrak A$ um atlas de trivializações de $(E_x)_{x\in M}$. Denote por $\mathfrak A_{\max}$ o conjunto de todas as trivializações
locais de $(E_x)_{x\in M}$ que são compatíveis com $\mathfrak A$. Mostre que:
\begin{itemize}
\item[(a)] $\mathfrak A_{\max}$ é um atlas de trivializações de $(E_x)_{x\in M}$ que contém $\mathfrak A$ (sugestão: use o resultado do Exercício~\ref{exe:alphabetafrakA});
\item[(b)] se $\mathfrak A'$ é um atlas de trivializações de $(E_x)_{x\in M}$ que contém $\mathfrak A$ então $\mathfrak A'\subset\mathfrak A_{\max}$
(isto é, $\mathfrak A_{\max}$ é o {\em maior\/} atlas de trivializações de $(E_x)_{x\in M}$ que contém $\mathfrak A$);
\item[(c)] $\mathfrak A_{\max}$ é o único atlas maximal de trivializações de $(E_x)_{x\in M}$ que contém $\mathfrak A$.
\end{itemize}
\end{exercise}

\begin{defin}
Um {\em fibrado vetorial\/} sobre uma variedade diferenciável $M$ consiste de uma família $(E_x)_{x\in M}$ de espaços vetoriais reais de dimensão $k<+\infty$
e de um atlas maximal $\mathfrak A$ de trivializações de $(E_x)_{x\in M}$. A variedade diferenciável $M$ é dita a {\em base\/} do fibrado vetorial,
o espaço vetorial $E_x$ é dito a {\em fibra\/} sobre o ponto $x\in M$, a união disjunta:
\[E=\bigcup_{x\in M}\big(\{x\}\times E_x\big)\]
é dita o {\em espaço total\/} do fibrado vetorial e a aplicação $\pi:E\to M$ dada por $\pi(x,e)=x$, $x\in M$, $e\in E_x$ é dita a {\em projeção\/}
do fibrado vetorial. A dimensão comum $k$ das fibras é dita o {\em posto\/} do fibrado vetorial.
\end{defin}
Normalmente, identifica-se um fibrado vetorial $\big((E_x)_{x\in M},\mathfrak A\big)$ com o seu espaço total $E$ ou com a projeção $\pi:E\to M$.
Quando falamos de uma ``trivialização local'' (ou, abreviadamente, ``trivialização'') de um fibrado vetorial {\em subentende-se que estamos falando de uma
trivialização local que pertence ao atlas maximal dado $\mathfrak A$}. Também, em muitos casos, identificamos a fibra $E_x$ com o subconjunto
$\{x\}\times E_x$ do espaço total $E$.

\begin{exercise}[fibrado tangente]
Dada uma carta $\varphi:U\subset M\to\widetilde U\subset\R^n$ de uma variedade diferenciável $M$, mostre que:
\begin{equation}\label{eq:defalphavarphi}
\alpha^\varphi_x=\dd\varphi_x^{-1}:\R^n\longrightarrow T_xM,\quad x\in U,
\end{equation}
define uma trivialização $\alpha^\varphi$ da família $(T_xM)_{x\in M}$. Mostre que:
\begin{equation}\label{eq:alphavarphi}
\big\{\alpha^\varphi:\text{$\varphi$ carta de $M$}\big\}
\end{equation}
é um atlas de trivializações de $(T_xM)_{x\in M}$ (sugestão: se $\varphi$ e $\psi$ são cartas de $M$ então a função de transição de $\alpha^\varphi$
para $\alpha^\psi$ é dada por:
\[g_{\alpha^\varphi\alpha^\psi}(x)=\dd(\varphi\circ\psi^{-1})\big(\psi(x)\big),\]
para todo $x\in\Dom(\varphi)\cap\Dom(\psi)$).
A família $(T_xM)_{x\in M}$, munida do atlas maximal que contém \eqref{eq:alphavarphi},
é um fibrado vetorial sobre $M$ que chamamos o {\em fibrado tangente\/} de $M$. O espaço total desse fibrado vetorial é denotado por $TM$
e coincide justamente com o fibrado tangente definido na Seção~\ref{sec:1309}.
\end{exercise}

\begin{exercise}\label{exe:restrtriv}
Sejam $\big((E_x)_{x\in M},\mathfrak A\big)$ um fibrado vetorial e $\alpha=(\alpha_x)_{x\in U}\in\mathfrak A$. Mostre que se $V$ é um aberto de $M$
contido em $U$ então $\alpha\vert_V=(\alpha_x)_{x\in V}$ pertence a $\mathfrak A$ (sugestão: mostre que $\alpha\vert_V$ é compatível com $\mathfrak A$).
\end{exercise}

\begin{exercise}[restrição de um fibrado vetorial a uma subvariedade]\label{exe:restrfibvet}
Sejam $\big((E_x)_{x\in M},\mathfrak A\big)$ um fibrado vetorial e $N$ uma subvariedade (possivelmente imersa) de $M$. Para cada $\alpha=(\alpha_x)_{x\in U}$
em $\mathfrak A$, considere a trivialização local:
\[\alpha\vert_{U\cap N}=(\alpha_x)_{x\in U\cap N}\]
da família $(E_x)_{x\in N}$. Mostre que a coleção:
\[\mathfrak A\vert_N=\big\{\alpha\vert_{\Dom(\alpha)\cap N}:\alpha\in\mathfrak A\big\}\]
é um atlas de trivializações locais de $(E_x)_{x\in N}$. Dessa forma, a família $(E_x)_{x\in N}$ munida do atlas maximal que contém $\mathfrak A\vert_N$
é um fibrado vetorial sobre $N$ cujo espaço total é denotado por:
\[E\vert_N=\pi^{-1}(N)=\bigcup_{x\in N}\big(\{x\}\times E_x\big),\]
onde denotamos por $E$ o espaço total de $(E_x)_{x\in M}$ e por $\pi:E\to M$ a projeção. Mostre que se $N$ é uma subvariedade aberta de $M$ então $\mathfrak A\vert_N$
consiste precisamente dos elementos de $\mathfrak A$ que têm domínio contido em $N$ (sugestão: use o resultado do Exercício~\ref{exe:restrtriv})
e que nesse caso $\mathfrak A\vert_N$ é um atlas {\em maximal\/} de trivializações de $(E_x)_{x\in N}$ (sugestão: mostre que se uma trivialização
local de $(E_x)_{x\in N}$ é compatível com $\mathfrak A\vert_N$ então ela é compatível com $\mathfrak A$).
\end{exercise}

\begin{exercise}
Mostre que se $U$ é uma subvariedade aberta de uma variedade diferenciável $M$ então o fibrado vetorial $TM\vert_U$ (veja
Exercício~\ref{exe:restrfibvet}) é o fibrado tangente $TU$ da variedade $U$ (sugestão: verifique que se $\varphi$ é uma carta de $U$ então a trivialização local
$\alpha^\varphi$ de $TU$ pertence ao atlas maximal de trivializações de $TM\vert_U$).
\end{exercise}

\begin{exercise}
Dada uma trivialização local $\alpha=(\alpha_x)_{x\in U}$ de uma família $(E_x)_{x\in M}$ então definimos:
\begin{equation}\label{eq:tildealpha}
\tilde\alpha:E\vert_U\ni(x,e)\longmapsto\big(x,\alpha_x^{-1}(e)\big)\in U\times\R^k,
\end{equation}
onde $E\vert_U=\bigcup_{x\in U}\big(\{x\}\times E_x\big)$. Mostre que a aplicação $\tilde\alpha$ é uma bijeção. Dada uma outra trivialização
local $\beta=(\beta_x)_{x\in V}$ de $(E_x)_{x\in M}$, mostre que a composição:
\[\tilde\alpha\circ\tilde\beta^{-1}:(U\cap V)\times\R^k\longrightarrow(U\cap V)\times\R^k\]
é dada por:
\[(\tilde\alpha\circ\tilde\beta^{-1})(x,h)=\big(x,g_{\alpha\beta}(x)\cdot h\big),\quad x\in U\cap V,\ h\in\R^k,\]
onde $g_{\alpha\beta}$ denota a função de transição de $\alpha$ para $\beta$.
Conclua que $\alpha$ é compatível com $\beta$ se e somente se $\tilde\alpha\circ\tilde\beta^{-1}$ é de classe $C^\infty$ (sugestão:
para mostrar que se $\tilde\alpha\circ\tilde\beta^{-1}$ é de classe $C^\infty$ então $\alpha$ é compatível com $\beta$, tenha em mente que
para mostrar que a aplicação $g_{\alpha\beta}$ é de classe $C^\infty$ é suficiente mostrar que as colunas da matriz que representa $g_{\alpha\beta}(x)$
são funções de classe $C^\infty$ de $x$).
\end{exercise}

\begin{exercise}[estrutura de variedade no espaço total]\label{exe:espacototal}
Seja $\pi:E\to M$ um fibrado vetorial. Use o resultado do item~(a) do Exercício~\ref{exe:manifoldcharts} para mostrar que existe um único atlas maximal $\mathcal A$ (de cartas)
no conjunto $E$ que faz de $E\vert_U$ um aberto de $E$ e da aplicação \eqref{eq:tildealpha} um difeomorfismo de classe $C^\infty$, para qualquer
trivialização local $\alpha=(\alpha_x)_{x\in U}$ de $E$. Mostre também que, dado um atlas $\mathfrak A$ de trivializações de $\pi:E\to M$ (contido no maximal)
então $\mathcal A$ é também o único atlas maximal em $E$ tal que $E\vert_U$ é um aberto de $E$ e tal que a aplicação \eqref{eq:tildealpha} é um difeomorfismo de classe
$C^\infty$ para qualquer $\alpha=(\alpha_x)_{x\in U}$ pertencente a $\mathfrak A$.
A partir de agora, consideramos o espaço total de um fibrado vetorial sempre munido do atlas maximal $\mathcal A$.
\end{exercise}

\begin{exercise}
Mostre que o atlas maximal (de cartas) definido no fibrado tangente pelo Exercício~\ref{exe:espacototal} coincide com aquele definido pelo Exercício~\ref{exe:fibradotangente}
(sugestão: se $TM$ é munido do atlas maximal definido pelo Exercício~\ref{exe:fibradotangente}, se $\varphi$ é uma carta de $M$
e $\alpha=\alpha^\varphi$ é definida em \eqref{eq:defalphavarphi}, mostre que a aplicação $\tilde\alpha$ definida em \eqref{eq:tildealpha}
é um difeomorfismo com domínio aberto).
\end{exercise}

\begin{exercise}[o fibrado trivial]\label{exe:fibradotrivial}
Sejam $M$ uma variedade diferenciável e $E_0$ um espaço vetorial real de dimensão $k<+\infty$. Considere a família constante $(E_x)_{x\in M}$
tal que $E_x=E_0$, para todo $x\in M$. Dado um isomorfismo linear $\alpha_0:\R^k\to E_0$, denotamos também por $\alpha_0$ a trivialização
de $(E_x)_{x\in M}$ que associa o isomorfismo $\alpha_0:\R^k\to E_x=E_0$ a cada $x\in M$.
\begin{itemize}
\item[(a)] Mostre que a coleção formada pelas trivializações $\alpha_0$ associadas a isomorfismos lineares $\alpha_0:\R^k\to E_0$ é um atlas
de trivializações de $(E_x)_{x\in M}$. O fibrado vetorial obtido pela consideração do atlas maximal que contém tal atlas de trivializações
é chamado o {\em fibrado trivial\/} com base $M$ e fibra $E_0$.
\item[(b)] O espaço total do fibrado trivial com base $M$ e fibra $E_0$ é $M\times E_0$. Mostre que o atlas maximal (de cartas) desse
espaço total (definido como no Exercício~\ref{exe:espacototal}) coincide com aquele da variedade produto $M\times E_0$ (sugestão:
$\tilde\alpha_0=\Id\times\alpha_0^{-1}:M\times E_0\to M\times\R^k$).
\end{itemize}
\end{exercise}

\begin{exercise}\label{exe:fibramergulhada}
Seja $\pi:E\to M$ um fibrado vetorial.
Mostre que para qualquer $x\in M$ a fibra $E_x$ é uma subvariedade mergulhada do espaço total $E$ e que a estrutura de variedade que $E_x$ herda de $E$
coincide com a estrutura de variedade canônica do espaço vetorial $E_x$ (sugestão: se $\alpha$ é uma trivialização local de $E$ cujo domínio $U$ contém
$x$ então o difeomorfismo \eqref{eq:tildealpha} leva $E_x$ --- ou melhor, $\{x\}\times E_x$ --- sobre a subvariedade mergulhada
$\{x\}\times\R^k$ de $U\times\R^k$. Note que \eqref{eq:tildealpha} restringe-se a uma bijeção de $\{x\}\times E_x\cong E_x$ em $\{x\}\times\R^k\cong\R^k$ que
é um difeomorfismo se $E_x$ estiver munido da estrutura de variedade que herda de $E$ e, sendo essa bijeção um isomorfismo linear, ela também
é um difeomorfismo se $E_x$ estiver munido da sua estrutura de variedade canônica).
\end{exercise}

\begin{exercise}
Mostre que a projeção $\pi:E\to M$ de um fibrado vetorial é uma submersão de classe $C^\infty$ (sugestão: se $\alpha=(\alpha_x)_{x\in U}$ é uma trivialização
de $E$ então a restrição de $\pi$ ao aberto $E\vert_U$ é a composição de $\tilde\alpha$ com a primeira projeção do produto $U\times\R^k$).
\end{exercise}

\begin{exercise}
Seja $\pi:E\to M$ um fibrado vetorial. Mostre que:
\begin{itemize}
\item[(a)] se $M$ é Hausdorff então também $E$ é Hausdorff (sugestão: se dois pontos de $E$ estão em fibras distintas então suas projeções em $M$ podem ser separadas por abertos
disjuntos e se dois pontos de $E$ estão na mesma fibra então eles pertencem a um aberto do espaço total $E$ da forma $E\vert_U$, sendo $U$ o
domínio de uma trivialização local de $E$; note que $E\vert_U\cong U\times\R^k$ é Hausdorff);
\item[(b)] se $M$ tem base enumerável de abertos então também $E$ tem base enumerável de abertos (sugestão: pela propriedade de Lindelöf, $M$ pode ser
coberta por uma quantidade enumerável de domínios $U$ de trivializações locais de $E$ e daí cobrimos $E$ com uma quantidade enumerável de abertos $E\vert_U\cong U\times\R^k$
que possuem base enumerável de abertos).
\end{itemize}
\end{exercise}

\begin{exercise}
Seja $\pi:E\to M$ um fibrado vetorial. Se $U$ é um aberto de $M$, mostre que $E\vert_U$ é uma subvariedade aberta de $E$ (note que você não deve mostrar
apenas que $E\vert_U$ é um subconjunto aberto de $E$ --- o que segue diretamente da continuidade da projeção --- mas também que o atlas maximal
que $E\vert_U$ herda de $E$ coincide com aquele que $E\vert_U$ possui por ser o espaço total de um fibrado vetorial sobre $U$).
\end{exercise}

\begin{exercise}
Sejam $\pi:E\to M$ um fibrado vetorial e $N$ uma subvariedade imersa de $M$. Mostre que:
\begin{itemize}
\item[(a)] a inclusão de $E\vert_N$ em $E$ é uma imersão, de modo que $E\vert_N$ é uma subvariedade imersa de $E$ (sugestão: se $\alpha=(\alpha_x)_{x\in U}$
é uma trivialização de $E$ e $\beta=\alpha\vert_{U\cap N}$ então o seguinte diagrama:
\[\xymatrix@C+20pt{%
E\vert_{U\cap N}\ar[r]^{\text{inclusão}}\ar[d]_{\tilde\beta}&E\vert_U\ar[d]^{\tilde\alpha}\\
(U\cap N)\times\R^k\ar[r]_-{\text{inclusão}}&U\times\R^k}\]
comuta);
\item[(b)] se $N$ é mergulhada em $M$ então a inclusão de $E\vert_N$ em $E$ é um mergulho, de modo que $E\vert_N$ é uma subvariedade mergulhada de $E$
(sugestão: use o resultado do Exercício~\ref{exe:condmergulho} notando que $E\vert_{U\cap N}$ é a imagem inversa pela inclusão $E\vert_N\to E$
do aberto $E\vert_U$ de $E$).
\end{itemize}
\end{exercise}

\begin{defin}
Uma {\em seção\/} de um fibrado vetorial $\pi:E\to M$ é uma aplicação $s:M\to E$ tal que $s(x)\in E_x$, para todo $x\in M$, i.e., tal que $\pi\circ s$
é a aplicação identidade de $M$. Se $U$ é um aberto de $M$ e $s:U\to E$ é uma aplicação tal que $s(x)\in E_x$ para todo $x\in U$ (i.e., $s$ é uma seção
do fibrado vetorial $E\vert_U$) então dizemos que $s$ é uma {\em seção local\/} de $E$.
\end{defin}
Note que uma seção do fibrado tangente de uma variedade diferenciável é a mesma coisa que um campo vetorial nessa variedade diferenciável.

\begin{exercise}\label{exe:reprsecao}
Seja dada uma seção $s:M\to E$ de um fibrado vetorial $\pi:E\to M$ de posto $k$ e seja $\alpha=(\alpha_x)_{x\in U}$ uma trivialização local de $E$; a aplicação:
\[U\ni x\longmapsto\alpha_x^{-1}\big(s(x)\big)\in\R^k\]
é chamada a {\em representação\/} de $s$ com respeito à trivialização $\alpha$. Se $\mathfrak A$ é um atlas de trivializações locais de $E$ (contido no
maximal), mostre que uma seção $s:M\to E$ é de classe $C^\infty$ se e somente se sua representação com respeito a qualquer $\alpha\in\mathfrak A$ é uma aplicação
de classe $C^\infty$ (sugestão: considere a composição de $s$ com o difeomorfismo \eqref{eq:tildealpha}). Conclua que se $\alpha=(\alpha_x)_{x\in U}$
é uma trivialização local de $E$, $v:U\to\R^k$ é uma função e $s:U\to E$ é a seção local definida por:
\[s(x)=\alpha_x\big(v(x)\big),\quad x\in U\]
então $s$ é de classe $C^\infty$ se e somente se $v$ é de classe $C^\infty$ (sugestão: o conjunto unitário $\{\alpha\}$ é um atlas de trivializações de $E\vert_U$ e
$v$ é a representação de $s$ com respeito a $\alpha$).
\end{exercise}

\begin{exercise}
Seja $\pi:E\to M$ um fibrado vetorial. O conjunto de todas as seções de $E$ é um espaço vetorial real e um módulo
sobre a álgebra das funções $f:M\to\R$, sendo as operações em questão definidas ponto a ponto. Mostre que o conjunto $\Gamma(E)$
de todas as seções de classe $C^\infty$ de $E$ é um subespaço do espaço vetorial de todas as seções de $E$ e que o produto $fs$ está em $\Gamma(E)$
se $f\in C^\infty(M)$ e $s\in\Gamma(E)$; conclua que $\Gamma(E)$ é um $C^\infty(M)$-módulo (sugestão: use o resultado do Exercício~\ref{exe:reprsecao}).
\end{exercise}

\begin{exercise}\label{exe:trivsecoes}
Seja $\pi:E\to M$ um fibrado vetorial de posto $k$ e denote por $(e_i)_{i=1}^k$ a base canônica de $\R^k$. Dada uma trivialização local $\alpha=(\alpha_x)_{x\in U}$
de $E$ então, em vista do resultado do Exercício~\ref{exe:reprsecao}, as seções locais $s_i:U\to E$ definidas por:
\begin{equation}\label{eq:sialphaei}
s_i(x)=\alpha_x(e_i),\quad x\in U,\ i=1,\ldots,k
\end{equation}
são de classe $C^\infty$ e, além do mais, é claro que
$\big(s_i(x)\big)_{i=1}^k$ é uma base de $E_x$, para todo $x\in U$. Reciprocamente, dados um subconjunto aberto $U$ de $M$ e seções locais $s_i:U\to E$, $i=1,\ldots,k$,
de classe $C^\infty$ tais que $\big(s_i(x)\big)_{i=1}^k$ é uma base de $E_x$ para todo $x\in U$, mostre que a única
trivialização local $\alpha$ com domínio $U$ satisfazendo \eqref{eq:sialphaei} pertence ao atlas maximal $\mathfrak A$ do fibrado vetorial $E$ (sugestão:
mostre que $\alpha$ é compatível com $\mathfrak A$; para isso, note que se $\beta\in\mathfrak A$ então a $i$-ésima coluna da matriz de $g_{\beta\alpha}(x)$
coincide com o valor em $x$ da representação de $s_i$ com respeito a $\beta$).
\end{exercise}

\begin{rem}
Seja $M$ uma variedade diferenciável. Em vista do resultado do Exercício~\ref{exe:trivsecoes}, uma trivialização local do fibrado tangente $TM$ com domínio
num aberto $U$ de $M$ determina (e é determinada por) uma $n$-upla de campos vetoriais $X_i:U\to TM$, $i=1,\ldots,n$, de classe $C^\infty$ em $U$ tais que
$\big(X_i(x)\big)_{i=1}^n$ é uma base de $T_xM$, para todo $x\in U$. Uma tal $n$-upla de campos vetoriais é também chamada um {\em referencial local\/}
em $M$. Como veremos adiante, nem toda trivialização local de $TM$ é determinada por
uma carta de $M$ (como em \eqref{eq:defalphavarphi}).
\end{rem}

\begin{defin}
Seja $\mathfrak F$ uma regra que associa a cada espaço vetorial real de dimensão finita $E$ um espaço vetorial real de dimensão finita $\mathfrak F(E)$
e a cada isomorfismo linear $T:E\to F$ entre espaços vetoriais reais de dimensão finita $E$, $F$, um isomorfismo linear $\mathfrak F(T):\mathfrak F(E)\to\mathfrak F(F)$.
Assuma que, dados espaços vetoriais reais de dimensão finita $E$, $F$, $G$ e isomorfismos lineares $T:E\to F$, $S:F\to G$ então\footnote{%
Note que segue de \eqref{eq:frakFfunc} que se $\Id$ é a aplicação identidade de um espaço vetorial real de dimensão finita $E$ então $\mathfrak F(\Id)$ é
a aplicação identidade de $\mathfrak F(E)$ (basta fazer $S=T=\Id$ em \eqref{eq:frakFfunc}). Segue também então que $\mathfrak F(T^{-1})=\mathfrak F(T)^{-1}$.}:
\begin{equation}\label{eq:frakFfunc}
\mathfrak F(S\circ T)=\mathfrak F(S)\circ\mathfrak F(T).
\end{equation}
Uma tal regra $\mathfrak F$ chama-se um {\em funtor\/} da categoria dos espaços vetorias reais de dimensão finita em si mesma\footnote{%
Na definição geral de funtor é necessário incluir a condição de que o morfismo identidade seja levado no morfismo identidade. No caso em questão
isso não foi necessário pois estamos considerando uma categoria em que todo morfismo é um isomorfismo.}
(os morfismos da categoria
sendo os isomorfismos lineares). Tal funtor é dito {\em de classe $C^\infty$\/} se, dados espaços vetoriais reais de dimensão finita $E$, $F$, então
a aplicação $T\mapsto\mathfrak F(T)$, definida no conjunto dos isomorfismos lineares de $E$ para $F$, tomando valores no conjunto dos isomorfismos lineares
de $\mathfrak F(E)$ para $\mathfrak F(F)$, é de classe $C^\infty$. Nós usaremos simplesmente o termo ``funtor de classe $C^\infty$'' para se referir
a um funtor de classe $C^\infty$ da categoria dos espaços vetoriais reais de dimensão finita em si mesma (sendo os morfismos da categoria como explicado acima).
\end{defin}

\begin{exercise}[aplicando um funtor a um fibrado vetorial]\label{exe:funcvecbund}
Seja $\mathfrak F$ um funtor de classe $C^\infty$ e seja $\pi:E\to M$ um fibrado vetorial de posto $k$; denote por $l$ a dimensão de $\mathfrak F(\R^k$).
Mostre que, para cada trivialização local $\alpha=(\alpha_x)_{x\in U}$ de $E$ e para cada isomorfismo linear $i:\R^l\to\mathfrak F(\R^k)$, a família:
\[\mathfrak F(\alpha;i)=\big(\mathfrak F(\alpha_x)\circ i\big)_{x\in U}\]
é uma trivialização local de:
\begin{equation}\label{eq:frakFEx}
\big(\mathfrak F(E_x)\big)_{x\in M}.
\end{equation}
Mostre também que o conjunto de todas as trivializações locais da forma $\mathfrak F(\alpha;i)$
(com $\alpha$ uma trivialização local de $E$ e $i:\R^l\to\mathfrak F(\R^k)$ um isomorfismo) é um atlas de trivializações de \eqref{eq:frakFEx}.
A família \eqref{eq:frakFEx}, munida do atlas maximal que contém tal atlas, é um fibrado vetorial sobre $M$ cujo espaço total é denotado por
$\mathfrak F(E)$.
\end{exercise}

\begin{exercise}\label{exe:exemplosfuntor}
No que segue, $E$ sempre denota um espaço vetorial real de dimensão finita e $T$ um isomorfismo linear entre espaços vetoriais reais de dimensão finita.
Mostre que os seguintes são funtores de classe $C^\infty$ (veja Apêndice~\ref{sec:tensores} para notações):
\begin{itemize}
\item[(a)] $\mathfrak F(E)=\Lin_{(p,q)}(E,\R)=\big(\bigotimes_pE^*\big)\otimes\big(\bigotimes_qE\big)$, $\mathfrak F(T)=T_*$;
\smallskip
\item[(b)] $\mathfrak F(E)=\Lin_k^{\mathrm a}(E,\R)=\bigwedge_kE^*$, $\mathfrak F(T)=T_*$;
\smallskip
\item[(c)] $\mathfrak F(E)=\Lin_k^{\mathrm a}(E^*,\R)=\bigwedge_kE$, $\mathfrak F(T)=T_*$;
\smallskip
\item[(d)] $\mathfrak F(E)=\Lin_k^{\mathrm s}(E,\R)=\bigvee_kE^*$, $\mathfrak F(T)=T_*$;
\smallskip
\item[(e)] $\mathfrak F(E)=\Lin_k^{\mathrm s}(E^*,\R)=\bigvee_kE$, $\mathfrak F(T)=T_*$;
\smallskip
\item[(f)] $\mathfrak F(E)=\bigoplus_kE$ (a soma direta de $k$ cópias de $E$), $\mathfrak F(T)=\bigoplus_kT$
\end{itemize}
(sugestão para o item~(a): dados espaços vetoriais reais de dimensão finita $E$, $F$, tenha em mente que a aplicação:
\[\rho:\Lin(E,F)^k\longrightarrow\Lin\!\big(\!\Lin_k(F,\R),\Lin_k(E,\R)\big)\]
definida por:
\begin{multline*}
\big[\rho(T_1,\ldots,T_k)(\mathfrak t)\big](x_1,\ldots,x_k)=\mathfrak t\big(T_1(x_1),\ldots,T_k(x_k)\big),\\
T_1,\ldots,T_k\in\Lin(E,F),\ \mathfrak t\in\Lin_k(F,\R),\ x_1,\ldots,x_k\in E,
\end{multline*}
é multilinear e portanto de classe $C^\infty$. Escreva a aplicação $T\mapsto T_*$ em termos de uma aplicação similar a $\rho$. Para os itens~(b)---(e)
tenha também em mente que as aplicações:
\[T_*:\bigwedge_kE^*\longrightarrow\bigwedge_kF^*,\quad T_*:\bigvee_kE^*\longrightarrow\bigvee_kF^*\]
são restrições da aplicação $T_*:\bigotimes_kE^*\to\bigotimes_kF^*$ e que as aplicações:
\[T_*:\bigwedge_kE\longrightarrow\bigwedge_kF,\quad T_*:\bigvee_kE\longrightarrow\bigvee_kF\]
são restrições da aplicação $T_*:\bigotimes_kE\to\bigotimes_kF$. Para o item~(f), note que a aplicação $T\mapsto\bigoplus_kT$ é linear).
\end{exercise}

\begin{rem}\label{thm:remfuncE}
Em vista dos resultados dos Exercícios~\ref{exe:funcvecbund} e \ref{exe:exemplosfuntor}, dado um fibrado vetorial $\pi:E\to M$, obtemos novos fibrados vetoriais
sobre $M$ que serão denotados por:
\begin{gather}\label{eq:otimespqE}
\big(\bigotimes_pE^*\big)\otimes\big(\bigotimes_qE\big),\\
\notag\bigwedge_kE^*,\quad\bigwedge_kE,\quad\bigvee_kE^*,\quad\bigvee_kE,\quad\bigoplus_kE.
\end{gather}
Alguns casos particulares de \eqref{eq:otimespqE}
são notáveis (veja Exemplo~\ref{exa:guardachuva}). Para $p=q=0$, \eqref{eq:otimespqE} é sempre o fibrado trivial $M\times\R$
(veja Exercício~\ref{exe:fibradotrivial}). Para $p=0$, $q=1$, podemos identificar \eqref{eq:otimespqE} com o próprio fibrado vetorial $E$.
Para $p=1$, $q=0$, \eqref{eq:otimespqE} é o {\em dual\/} $E^*$ de $E$.
O dual $TM^*$ do fibrado tangente $TM$ é conhecido também como o {\em fibrado cotangente\/} de $M$.
\end{rem}

\noindent\lower15pt\hbox{\dbend}\hspace{10pt}\parbox[t]{330pt}{\begin{rem}
A rigor, dado um espaço vetorial real de dimensão finita $E$, usando a construção de \eqref{eq:otimespqE} dada na Subseção~\ref{sub:pqtensores},
temos que \eqref{eq:otimespqE} é igual ao bidual $E^{**}$ se $p=0$, $q=1$. Para um dado fibrado vetorial $E$, o seu bidual $E^{**}$
é {\em isomorfo\/} a $E$. Ocorre que um {\em isomorfismo natural\/} entre dois funtores $\mathfrak F$, $\mathfrak G$ de classe $C^\infty$
induz, para cada fibrado vetorial $E$, um isomorfismo entre os fibrados vetoriais $\mathfrak F(E)$ e $\mathfrak G(E)$.
\end{rem}}

\begin{exercise}
Sejam $\mathfrak F$ um funtor de classe $C^\infty$, $\pi:E\to M$ um fibrado vetorial e $N$ uma subvariedade (possivelmente imersa) de $M$.
Mostre que os fibrados vetoriais $\mathfrak F(E)\vert_N$ e $\mathfrak F(E\vert_N)$ coincidem (sugestão: se $\alpha$ é uma trivialização local
de $E$ e $i:\R^l\to\mathfrak F(\R^k)$ é um isomorfismo, note que $\mathfrak F(\alpha;i)\vert_{U\cap N}=\mathfrak F(\alpha\vert_{U\cap N};i)$,
onde $U=\Dom(\alpha)$).
\end{exercise}

\begin{exercise}
Seja $\mathfrak F$ um funtor de classe $C^\infty$ e seja $\pi:E\to M$ um fibrado vetorial de posto $k$.
Dada uma trivialização local $\alpha=(\alpha_x)_{x\in U}$ de $E$,
mostre que a aplicação $\mathfrak F(\tilde\alpha):\mathfrak F(E)\vert_U\to U\times\mathfrak F(\R^k)$ definida por:
\begin{equation}\label{eq:Ftildealpha}
\mathfrak F(\tilde\alpha)(x,\epsilon)=\big(x,\mathfrak F(\alpha_x)^{-1}(\epsilon)\big),\quad x\in U,\ \epsilon\in\mathfrak F(E_x),
\end{equation}
é um difeomorfismo de classe $C^\infty$ (sugestão: seja $i:\R^l\to\mathfrak F(\R^k)$ um isomorfismo e considere a trivialização local
$\beta=\mathfrak F(\alpha;i)$ de $\mathfrak F(E)$. Considere o difeomorfismo $\tilde\beta$ associado a $\beta$ e note que o diagrama:
\[\xymatrix{%
&\mathfrak F(E)\vert_U\ar[dl]_{\tilde\beta}\ar[dr]^{\mathfrak F(\tilde\alpha)}\\
U\times\R^l\ar[rr]_{\Id\times i}&&U\times\mathfrak F(\R^k)}\]
comuta).
\end{exercise}

\begin{exercise}\label{exe:secaoFEsmooth}
Seja $\mathfrak F$ um funtor de classe $C^\infty$ e seja $\pi:E\to M$ um fibrado vetorial de posto $k$. Dada uma seção $\epsilon:M\to\mathfrak F(E)$
de $\mathfrak F(E)$ e um atlas de trivializações $\mathfrak A$ de $E$ (contido no maximal), mostre que $\epsilon$ é de classe $C^\infty$ se e somente se
para qualquer $\alpha=(\alpha_x)_{x\in U}\in\mathfrak A$ a aplicação:
\begin{equation}\label{eq:repreepsilonF}
U\ni x\longmapsto\mathfrak F(\alpha_x)^{-1}\big(\epsilon(x)\big)\in\mathfrak F(\R^k)
\end{equation}
é de classe $C^\infty$ (sugestão: considere a composição de $\epsilon$ com o difeomorfismo \eqref{eq:Ftildealpha}).
\end{exercise}

\begin{exercise}\label{exe:funtorincfuntor}
Sejam $\mathfrak F$, $\mathfrak G$ funtores de classe $C^\infty$ tais que\footnote{%
As condições (i) e (ii) dizem que a inclusão define uma {\em transformação natural\/} do funtor $\mathfrak G$ no funtor $\mathfrak F$.}:
\begin{itemize}
\item[(i)] $\mathfrak G(E)$ é um subespaço de $\mathfrak F(E)$, para todo espaço vetorial real de dimensão finita $E$;
\item[(ii)] dados espaços vetoriais reais de dimensão finita $E$, $F$ e um isomorfismo linear $T:E\to F$ então $\mathfrak G(T)$ é a restrição
de $\mathfrak F(T)$.
\end{itemize}
Mostre que, para todo fibrado vetorial $\pi:E\to M$, o espaço total $\mathfrak G(E)$ é uma subvariedade mergulhada do espaço total $\mathfrak F(E)$
(sugestão: mostre que a inclusão de $\mathfrak G(E)$ em $\mathfrak F(E)$ é um mergulho usando o resultado do Exercício~\ref{exe:condmergulho}.
Para isso, note também que se $\alpha=(\alpha_x)_{x\in U}$ é uma trivialização local de $E$ então o diagrama:
\[\xymatrix@C+20pt{%
\mathfrak F(E)\vert_U\ar[r]^-{\mathfrak F(\tilde\alpha)}&U\times\mathfrak F(\R^k)\\
\mathfrak G(E)\vert_U\ar[u]^{\text{inclusão}}\ar[r]_-{\mathfrak G(\tilde\alpha)}&U\times\mathfrak G(\R^k)\ar[u]_{\text{inclusão}}}\]
comuta e que $\mathfrak G(E)\vert_U$ é a imagem inversa do aberto $\mathfrak F(E)\vert_U$ pela inclusão $\mathfrak G(E)\to\mathfrak F(E)$).
\end{exercise}

\begin{exercise}\label{exe:espacototalmerg}
Use o resultado do Exercício~\ref{exe:funtorincfuntor} para mostrar que se $\pi:E\to M$ é um fibrado vetorial então os espaços totais
$\bigwedge_kE^*$ e $\bigvee_kE^*$ são subvariedades mergulhadas de $\bigotimes_kE^*$ (e, similarmente, os espaços totais
$\bigwedge_kE$ e $\bigvee_kE$ são subvariedades mergulhadas de $\bigotimes_kE$).
\end{exercise}

\end{section}

\begin{section}{Dia 18/10}

Um {\em campo tensorial $p$ vezes covariante e $q$ vezes contravariante}
(ou, mais abreviadamente, um {\em $(p,q)$-campo tensorial\/} ou, ainda mais abreviadamente, um {\em campo tensorial}) sobre uma variedade diferenciável
$M$ é uma seção:
\[\mathfrak t:M\longrightarrow\Big(\bigotimes_pTM^*\Big)\otimes\Big(\bigotimes_qTM\Big)\]
do fibrado vetorial $\big(\bigotimes_pTM^*\big)\otimes\big(\bigotimes_qTM\big)$ (veja Observação~\ref{thm:remfuncE}). Se $\mathfrak t$ é puramente covariante
(i.e., se $q=0$) ou se $\mathfrak t$ é puramente contravariante (i.e., se $p=0$), dizemos
que $\mathfrak t$ é {\em simétrico\/} (resp., {\em anti-simétrico}) se $\mathfrak t(x)$ for simétrico (resp., anti-simétrico) para todo $x\in M$.
Um campo tensorial simétrico $p$ vezes covariante é uma seção do fibrado vetorial $\bigvee_pTM^*$ e um campo tensorial simétrico $q$ vezes contravariante
é uma seção do fibrado vetorial $\bigvee_qTM$; similarmente, um campo tensorial anti-simétrico $p$ vezes covariante é uma seção do fibrado vetorial
$\bigwedge_pTM^*$ e um campo tensorial anti-simétrico $q$ vezes contravariante é uma seção do fibrado vetorial $\bigwedge_qTM$. Em vista
do resultado do Exercício~\ref{exe:espacototalmerg} (e também do resultado do Exercício~\ref{exe:lowercounterdomain}), vemos que quando consideramos
a propriedade de um campo tensorial simétrico $p$ vezes covariante $\mathfrak t$ ser de classe $C^\infty$ não faz diferença se usamos $\bigotimes_pTM^*$
ou $\bigvee_pTM^*$ como seu contra-domínio; observações similares são válidas para campos tensoriais puramente contravariantes ou para campos tensoriais anti-simétricos.


\begin{defin}
Um campo tensorial anti-simétrico $k$ vezes covariante sobre uma variedade diferenciável é também chamado uma {\em $k$-forma diferencial}.
\end{defin}

\begin{rem}
Campos vetoriais sobre uma variedade $M$ podem ser identificados com $(0,1)$-campos tensoriais sobre $M$ e funções $f:M\to\R$ podem
ser identificadas com $(0,0)$-campos tensoriais sobre $M$ (veja Observação~\ref{thm:remfuncE}).
\end{rem}

\begin{rem}
Se $E$ é um espaço vetorial real de dimensão finita e $M$ é um aberto de $E$ então o fibrado tangente $TM$ identifica-se com o fibrado trivial
$M\times E$ e, para qualquer funtor $\mathfrak F$ de classe $C^\infty$, o fibrado vetorial $\mathfrak F(TM)$ identifica-se com o fibrado trivial
$M\times\mathfrak F(E)$ (verifique!). Em particular, um $(p,q)$-campo tensorial $\mathfrak t$ sobre $M$ identifica-se nesse caso com uma aplicação:
\[\mathfrak t:M\subset E\longrightarrow\Big(\bigotimes_pE^*\Big)\otimes\Big(\bigotimes_qE\Big).\]
\end{rem}

\begin{defin}\label{thm:defpullbacktensfield}
Sejam $M$, $N$ variedades diferenciáveis e $\phi:M\to N$ uma aplicação de classe $C^\infty$. Se $\mathfrak t$ é um campo tensorial puramente covariante
sobre $N$ então seu {\em pull-back\/} por $\phi$ é o campo tensorial $\phi^*\mathfrak t$ sobre $M$ (do mesmo tipo que $\mathfrak t$) definido por
(veja a definição de pull-back na Subseção~\ref{sub:multilinear}):
\[(\phi^*\mathfrak t)(x)=\dd\phi_x^*\big[\mathfrak t\big(\phi(x)\big)\big],\quad x\in M.\]
Se $\mathfrak t$ é um campo tensorial qualquer em $N$ (não necessariamente puramente covariante), definimos o pull-back $\phi^*\mathfrak t$
da mesma maneira, desde que $\phi$ seja um difeomorfismo local (veja a definição de pull-back para $(p,q)$-tensores na Subseção~\ref{sub:pqtensores}).
Se $\mathfrak t$ é um campo tensorial qualquer em $M$ e $\phi:M\to N$ é um difeomorfismo
de classe $C^\infty$, então definimos o {\em push-forward\/} de $\mathfrak t$ por $\phi$ fazendo:
\[\phi_*\mathfrak t=(\phi^{-1})^*\mathfrak t.\]
\end{defin}
Se $M$, $N$ são variedades diferenciáveis, $U$ é um subconjunto aberto de $M$, $\phi:U\to N$ é um difeomorfismo de classe $C^\infty$ e $\mathfrak t$ é um campo
tensorial sobre $M$, usamos também a notação $\phi_*\mathfrak t$ para denotar o push-forward da restrição $\mathfrak t\vert_U$ pela aplicação $\phi$.

\begin{exercise}
Mostre que um pull-back ou um push-forward de um campo tensorial simétrico é ainda simétrico e que um pull-back ou um push-forward
de um campo tensorial anti-simétrico é ainda anti-simétrico (sugestão: use o resultado do item~(b) do Exercício~\ref{exe:sigmapullback}).
\end{exercise}

\begin{exercise}
Se campos vetoriais sobre uma variedade diferenciável são identificados com $(0,1)$-campos tensoriais sobre essa variedade, mostre que as noções
de pull-back e push-forward dadas na Definição~\ref{thm:defpullbacktensfield} coincidem com aquelas dadas na Definição~\ref{thm:defpullbackvecfield}
(sugestão: resolva primeiro o Exercício~\ref{exe:pullbackpqpartic}).
\end{exercise}

\begin{rem}\label{thm:obspullback00}
Se $f:N\to\R$ é um $(0,0)$-campo tensorial então o pull-back de $f$ por uma aplicação $\phi:M\to N$ de classe $C^\infty$
nada mais é que a aplicação composta $f\circ\phi$.
Similarmente, se $f:M\to\R$ é um $(0,0)$-campo tensorial então o push-forward de $f$ por um difeomorfismo $\phi:M\to N$ de classe $C^\infty$ nada
mais é que a aplicação composta $f\circ\phi^{-1}$.
\end{rem}

\begin{exercise}\label{exe:pullbackcompostafield}
Sejam $M$, $N$, $P$ variedades diferenciáveis e $\phi:M\to N$, $\lambda:N\to P$ funções de classe $C^\infty$. Dado um campo tensorial $\mathfrak t$
em $P$, mostre que:
\[(\lambda\circ\phi)^*\mathfrak t=\phi^*\lambda^*\mathfrak t,\]
onde devemos supor que $\phi$ e $\lambda$ são difeomorfismos locais caso $\mathfrak t$ não seja puramente covariante. Se $\mathfrak t$ é um campo
tensorial em $M$ e $\phi$, $\lambda$ são difeomorfismos, mostre que:
\[(\lambda\circ\phi)_*\mathfrak t=\lambda_*\phi_*\mathfrak t\]
(sugestão: use o resultado do Exercício~\ref{exe:pullbackcomposta} e sua reformulação para $(p,q)$-tensores dada na Subseção~\ref{sub:pqtensores}).
\end{exercise}

\begin{exercise}\label{exe:crittensorsmooth}
Seja $\mathcal A$ um atlas (de cartas) de uma variedade diferenciável $M$ (contido no maximal). Mostre que um campo tensorial $\mathfrak t$ sobre $M$ é
de classe $C^\infty$ se e somente se $\varphi_*\mathfrak t$ é de classe $C^\infty$, para qualquer $\varphi\in\mathcal A$ (sugestão: use o resultado
do Exercício~\ref{exe:secaoFEsmooth} com $E=TM$ e com $\mathfrak A=\big\{\alpha^\varphi:\varphi\in\mathcal A\big\}$, onde $\alpha^\varphi$ é a trivialização
local de $TM$ associada à carta $\varphi$. Se $\epsilon=\mathfrak t$ e $\alpha=\alpha^\varphi$, verifique que a aplicação \eqref{eq:repreepsilonF} é
$(\varphi_*\mathfrak t)\circ\varphi$).
\end{exercise}

\begin{exercise}\label{exe:pullbacksmoothmanifold}
Sejam $M$, $N$ variedades diferenciáveis, $\phi:M\to N$ uma aplicação de classe $C^\infty$ e $\mathfrak t$ um campo tensorial de classe $C^\infty$ sobre $N$.
Mostre que $\phi^*\mathfrak t$ é também de classe $C^\infty$, onde devemos supor que $\phi$ é um difeomorfismo local se $\mathfrak t$ não for puramente covariante
(sugestão: use os resultados dos Exercícios~\ref{exe:pullbackcompostafield} e \ref{exe:crittensorsmooth}. No caso em que $\phi$ é um difeomorfismo local
é conveniente tomar $\mathcal A$ como sendo um atlas formado por cartas de $M$ da forma $\varphi\circ\phi\vert_U$, com $U$ um aberto de $M$ que é levado
difeomorficamente por $\phi$ sobre um aberto $\phi(U)$ que é o domínio de uma carta $\varphi$ de $N$.
No caso em que $\mathfrak t$ é puramente covariante e $\phi$ não é necessariamente um difeomorfismo
local você precisará considerar a representação de $\phi$ com respeito a cartas de $M$, $N$ e usar o resultado do Exercício~\ref{exe:pullbacksmoothvecspa}).
Enuncie e prove o resultado análogo para o push-forward.
\end{exercise}

Se $\mathfrak t$ é um $(p,q)$-campo tensorial sobre uma variedade diferenciável $M$ e $\mathfrak t'$ é um $(p',q')$-campo tensorial sobre $M$
então definimos um $(p+p',q+q')$-campo tensorial $\mathfrak t\otimes\mathfrak t'$
sobre $M$, fazendo (veja a Subseção~\ref{sub:pqtensores} para a definição de produto tensorial de $(p,q)$-tensores):
\[(\mathfrak t\otimes\mathfrak t')(x)=\mathfrak t(x)\otimes\mathfrak t'(x),\quad x\in M.\]
Similarmente, se $\mathfrak t$, $\mathfrak t'$ são campos tensoriais simétricos (resp., an\-ti-si\-mé\-tri\-cos), definimos um campo tensorial
simétrico (resp., anti-simétrico) $\mathfrak t\vee\mathfrak t'$ (resp., $\mathfrak t\wedge\mathfrak t'$) fazendo
$(\mathfrak t\vee\mathfrak t')(x)=\mathfrak t(x)\vee\mathfrak t'(x)$ (resp., $(\mathfrak t\wedge\mathfrak t')(x)=\mathfrak t(x)\wedge\mathfrak t'(x)$),
para todo $x\in M$.

\begin{exercise}\label{exe:pullbacktensfield}
Sejam $M$, $N$ variedades diferenciáveis e $\phi:M\to N$ uma aplicação de classe $C^\infty$. Se $\mathfrak t$, $\mathfrak t'$ são campos tensoriais em $N$, mostre
que:
\[\phi^*(\mathfrak t\otimes\mathfrak t')=(\phi^*\mathfrak t)\otimes(\phi^*\mathfrak t'),\]
onde devemos supor que $\phi$ é um difeomorfismo local se $\mathfrak t$ e $\mathfrak t'$ não forem puramente covariantes (sugestão: use o resultado do
Exercício~\ref{exe:pullbackhomo} e sua reformulação para $(p,q)$-tensores dada na Subseção~\ref{sub:pqtensores}).
Se $\mathfrak t$ e $\mathfrak t'$ são simétricos, mostre também que:
\[\phi^*(\mathfrak t\vee\mathfrak t')=(\phi^*\mathfrak t)\vee(\phi^*\mathfrak t')\]
e se $\mathfrak t$, $\mathfrak t'$ são anti-simétricos
mostre que:
\[\phi^*(\mathfrak t\wedge\mathfrak t')=(\phi^*\mathfrak t)\wedge(\phi^*\mathfrak t')\]
(sugestão: use o resultado do item~(d) do Exercício~\ref{exe:sigmapullback}).
Enuncie e prove os resultados análogos para o push-forward.
\end{exercise}

\begin{exercise}
Se $\mathfrak t$, $\mathfrak t'$ são campos tensoriais de classe $C^\infty$ sobre uma variedade diferenciável $M$, mostre que $\mathfrak t\otimes\mathfrak t'$
é de classe $C^\infty$. Se $\mathfrak t$ e $\mathfrak t'$ são simétricos mostre que $\mathfrak t\vee\mathfrak t'$ é de classe $C^\infty$ e
se $\mathfrak t$ e $\mathfrak t'$ são anti-simétricos mostre que $\mathfrak t\wedge\mathfrak t'$ é de classe $C^\infty$ (sugestão: use os resultados
dos Exercícios~\ref{exe:crittensorsmooth} e \ref{exe:pullbacktensfield}).
\end{exercise}

Sejam $\mathfrak t$ um $(p,q)$-campo tensorial sobre uma variedade diferenciável $M$, $X_1$, \dots, $X_p$ campos vetoriais sobre $M$
e $\alpha_1$, \dots, $\alpha_q$ $1$-formas sobre $M$. Denotamos por $\mathfrak t(X_1,\ldots,X_p,\alpha_1,\ldots,\alpha_q):M\to\R$ a função definida
por:
\[M\ni x\longmapsto\mathfrak t(x)\big(X_1(x),\ldots,X_p(x),\alpha_1(x),\ldots,\alpha_q(x)\big)\in\R.\]
\begin{exercise}\label{exe:pushaval}
Sejam $M$, $N$ variedades diferenciáveis, $\phi:M\to N$ um difeomorfismo local, $\mathfrak t$ um $(p,q)$-campo tensorial sobre $N$, $X_1$, \dots, $X_p$
campos vetoriais sobre $N$ e $\alpha_1$, \dots, $\alpha_q$ $1$-formas sobre $N$. Mostre que:
\begin{align*}
\phi^*\big(\mathfrak t(X_1,\ldots,X_p,\alpha_1,\ldots,\alpha_q)\big)&=
\mathfrak t(X_1,\ldots,X_p,\alpha_1,\ldots,\alpha_q)\circ\phi\\
&=(\phi^*\mathfrak t)(\phi^*X_1,\ldots,\phi^*X_p,\phi^*\alpha_1,\ldots,\phi^*\alpha_q).
\end{align*}
Enuncie e prove o resultado análogo para o push-forward.
\end{exercise}

\begin{exercise}\label{exe:avalsmooth}
Se $\mathfrak t$ é um $(p,q)$-campo tensorial de classe $C^\infty$ sobre uma variedade diferenciável $M$, $X_1$, \dots, $X_p$ são campos vetoriais
de classe $C^\infty$ sobre $M$ e $\alpha_1$, \dots, $\alpha_q$ são $1$-formas de classe $C^\infty$ sobre $M$, mostre que a função
$\mathfrak t(X_1,\ldots,X_p,\alpha_1,\ldots,\alpha_q)$ é de classe $C^\infty$ (sugestão: use os resultados dos Exercícios~\ref{exe:crittensorsmooth} e \ref{exe:pushaval}).
\end{exercise}

Se $\mathfrak t$ é um campo tensorial $p$ vezes covariante sobre uma variedade diferenciável $M$ e $X$ é um campo vetorial sobre $M$ então o {\em produto interior\/}
$i_X\mathfrak t$ é o campo tensorial $p-1$ vezes covariante sobre $M$ definido por (recorde Definição~\ref{thm:defprodint}):
\[(i_X\mathfrak t)(x)=i_{X(x)}\mathfrak t(x),\quad x\in M.\]
Evidentemente, se $\mathfrak t$ é simétrico (resp., anti-simétrico) então também $i_X\mathfrak t$ é simétrico (resp., anti-simétrico).

\begin{exercise}\label{exe:propiXomega}
Sejam $\mathfrak t$ um campo tensorial $k$ vezes covariante e $\mathfrak t'$ um campo tensorial $l$ vezes covariante sobre uma variedade diferenciável $M$.
Dado um campo vetorial $X$ sobre $M$, mostre que:
\begin{itemize}
\item[(a)] se $\mathfrak t$, $\mathfrak t'$ são simétricos então:
\[i_X(\mathfrak t\vee\mathfrak t')=(i_X\mathfrak t)\vee\mathfrak t'+\mathfrak t\vee(i_X\mathfrak t');\]
\item[(b)] se $\mathfrak t$, $\mathfrak t'$ são anti-simétricos então:
\[i_X(\mathfrak t\wedge\mathfrak t')=(i_X\mathfrak t)\wedge\mathfrak t'+(-1)^k\mathfrak t\wedge(i_X\mathfrak t')\]
\end{itemize}
(sugestão: use o resultado do Lema~\ref{thm:lemiveewedge}).
\end{exercise}

\begin{exercise}\label{exe:pulliXt}
Sejam $M$, $N$ variedades diferenciáveis e $\phi:M\to N$ um difeomorfismo local. Se $\mathfrak t$ é um campo tensorial puramente covariante
sobre $N$ e $X$ é um campo vetorial sobre $N$, mostre que:
\[\phi^*(i_X\mathfrak t)=i_{(\phi^*X)}(\phi^*\mathfrak t).\]
Enuncie e prove o resultado análogo para o push-forward.
\end{exercise}

\begin{exercise}
Seja $\mathfrak t$ um campo tensorial puramente covariante de classe $C^\infty$ sobre uma variedade diferenciável $M$ e seja $X$ um campo vetorial
de classe $C^\infty$ sobre $M$. Mostre que $i_X\mathfrak t$ é de classe $C^\infty$ (sugestão: use o resultado dos Exercícios~\ref{exe:crittensorsmooth}
e \ref{exe:pulliXt}).
\end{exercise}

\begin{exercise}\label{exe:difextindcarta}
Sejam $M$ uma variedade diferenciável e $\omega$ uma $k$-forma de classe $C^\infty$ sobre $M$. Dado um ponto $x\in M$ e cartas
$\varphi:U\subset M\to\widetilde U\subset\R^n$, $\psi:V\subset M\to\widetilde V\subset\R^n$ com $x\in U\cap V$, mostre que:
\[[\varphi^*\dd(\varphi_*\omega)](x)=[\psi^*\dd(\psi_*\omega)(x)]\in\Lin_{k+1}^{\mathrm a}(T_xM,\R)=\bigwedge_{k+1}T_xM^*,\]
onde $\dd$ denota diferenciação exterior definida como no Apêndice~\ref{sec:difext}
(sugestão: use o resultado do item~(e) do Exercício~\ref{exe:propdifext} sendo $\phi$ igual à função de transição
entre as cartas $\varphi$ e $\psi$).
\end{exercise}

\begin{defin}\label{thm:defdifomegavar}
Seja $\omega$ uma $k$-forma diferencial de classe $C^\infty$ sobre uma variedade diferenciável $M$. A {\em diferencial exterior\/}
de $\omega$ é a $(k+1)$-forma diferencial $\dd\omega$ sobre $M$ definida por:
\begin{equation}\label{eq:defdomegavar}
\dd\omega(x)=[\varphi^*\dd(\varphi_*\omega)](x),
\end{equation}
para todo $x\in M$, onde $\varphi$ é uma carta qualquer cujo domínio contém $x$ e $\dd$ denota diferenciação exterior definida como no Apêndice~\ref{sec:difext}
(o fato que o lado direito de \eqref{eq:defdomegavar} não depende da carta
$\varphi$ é precisamente o resultado do Exercício~\ref{exe:difextindcarta}).
\end{defin}

\begin{exercise}
Mostre que se $\omega$ é uma $k$-forma diferencial de classe $C^\infty$ sobre um aberto $U$ de um espaço vetorial real de dimensão finita $E$ então
a diferencial exterior $\dd\omega$ dada pela Definição~\ref{thm:defdifomegavar} coincide com aquela definida no Apêndice~\ref{sec:difext}
(sugestão: considere uma carta $\varphi$ em $U$ --- por exemplo, a restrição de um isomorfismo linear entre $E$ e $\R^n$ --- e use o resultado do
item~(e) do Exercício~\ref{exe:propdifext} com $\phi=\varphi$).
\end{exercise}

\begin{exercise}
Mostre que a diferencial exterior de uma forma diferencial de classe $C^\infty$ sobre uma variedade diferenciável é ainda de classe $C^\infty$ (sugestão:
use o resultado do Exercício~\ref{exe:pullbacksmoothmanifold}).
\end{exercise}

\begin{exercise}[propriedades da diferencial exterior]\label{exe:propdifextvar}
Seja $M$ uma variedade diferenciável.
\begin{itemize}
\item[(a)] Se $f:M\to\R$ é uma $0$-forma (i.e., uma função a valores reais) de classe $C^\infty$, mostre que a diferencial exterior de $f$ coincide
com a sua diferencial comum.
\item[(b)] Mostre que a diferenciação exterior define uma aplicação linear do espaço vetorial real das $k$-formas de classe $C^\infty$ em $M$ no espaço
vetorial real das $(k+1)$-formas de classe $C^\infty$ em $M$.
\item[(c)] Se $\omega$ é uma $k$-forma de classe $C^\infty$ em $M$, mostre que $\dd(\dd\omega)=0$.
\item[(d)] Se $\omega$ é uma $k$-forma de classe $C^\infty$ em $M$ e $\lambda$ é uma $l$-forma de classe $C^\infty$ em $M$, mostre que:
\[\dd(\omega\wedge\lambda)=(\dd\omega)\wedge\lambda+(-1)^k\omega\wedge(\dd\lambda).\]
\item[(e)] Se $N$ é uma variedade diferenciável, $\phi:M\to N$ é uma aplicação de classe $C^\infty$
e $\omega$ é uma $k$-forma de classe $C^\infty$ em $N$, mostre que:
\[\dd(\phi^*\omega)=\phi^*\dd\omega\]
\end{itemize}
(sugestão: use cartas e os resultados correspondentes que aparecem no enunciado do Exercício~\ref{exe:propdifext}).
\end{exercise}

Quando se trabalha com campos tensoriais em variedades, é costumeiro usar as seguintes notações: sejam $M$ uma variedade
diferenciável e:
\[\varphi:U\subset M\longrightarrow\widetilde U\subset\R^n\]
uma carta em $M$. Denota-se por $x_i:U\to\R$, $i=1,\ldots,n$, as funções
coordenadas de $\varphi$, de modo que $\varphi=(x_1,\ldots,x_n)$. Os campos vetoriais de classe $C^\infty$ em $U$ obtidos pelo pull-back
por $\varphi$ da base canônica de $\R^n$ (pensamos nos vetores da base canônica de $\R^n$ como campos vetoriais constantes em $\R^n$) são denotados por:
\begin{equation}\label{eq:deldelxi}
\frac{\partial}{\partial x_1},\ldots,\frac{\partial}{\partial x_n}.
\end{equation}
As diferenciais (exteriores ou comuns) das funções $x_i:U\to\R$ nos dão $1$-formas de classe $C^\infty$:
\begin{equation}\label{eq:dxi}
\dd x_1,\ldots,\dd x_n
\end{equation}
sobre o aberto $U$. Note que $\dd x_i\big(\frac{\partial}{\partial x_j}\big)=1$ se $i=j$ e $\dd x_i\big(\frac{\partial}{\partial x_j}\big)=0$ se
$i\ne j$, de modo que a avaliação num ponto $x\in U$ das $1$-formas \eqref{eq:dxi} nos dão a base dual da base de $T_xM$ obtida pela avaliação em $x$
dos campos vetoriais \eqref{eq:deldelxi}. O push-forward das $1$-formas \eqref{eq:dxi} por $\varphi$ nos dão a base dual da base canônica de $\R^n$
(pensadas como $1$-formas constantes em $\R^n$).

Usando os campos vetoriais \eqref{eq:deldelxi} e as $1$-formas \eqref{eq:dxi} nós podemos construir as seguintes famílias de campos tensoriais de classe $C^\infty$ sobre o domínio $U$ da
carta $\varphi$ (identificamos, como sempre, campos vetoriais com $(0,1)$-campos tensoriais):
\begin{gather}
\label{eq:basepqtensoresfield}
\dd x_{i_1}\otimes\cdots\otimes\dd x_{i_p}\otimes\frac{\partial}{\partial x_{j_1}}\otimes\cdots\otimes\frac{\partial}{\partial x_{j_q}},\\[5pt]
\notag\qquad\qquad\qquad\qquad\qquad\qquad\qquad\qquad\qquad i_1,\ldots,i_p,j_1,\ldots,j_q=1,\ldots,n,\\[10pt]
\label{eq:baseptensoressimfield}
\dd x_{i_1}\vee\cdots\vee\dd x_{i_p},\quad 1\le i_1\le\cdots\le i_p\le n,\\
\label{eq:baseqtensoressimfield}
\frac{\partial}{\partial x_{j_1}}\vee\cdots\vee\frac{\partial}{\partial x_{j_q}},\quad 1\le j_1\le\cdots\le j_q\le n,\\
\label{eq:baseptensoresantfield}
\dd x_{i_1}\wedge\cdots\wedge\dd x_{i_p},\quad 1\le i_1<\cdots<i_p\le n,\\
\label{eq:baseqtensoresantfield}
\frac{\partial}{\partial x_{j_1}}\wedge\cdots\wedge\frac{\partial}{\partial x_{j_q}},\quad 1\le j_1<\cdots<j_q\le n.
\end{gather}
Em vista dos resultados dos Exercícios~\ref{exe:Lintensor}, \ref{exe:Linsbigvee} e \ref{exe:Linabigwedge}, as
famílias \eqref{eq:basepqtensoresfield}, \eqref{eq:baseptensoressimfield}, \eqref{eq:baseqtensoressimfield}, \eqref{eq:baseptensoresantfield}
e \eqref{eq:baseqtensoresantfield}, quando avaliadas num ponto $x\in U$, nos dão bases dos espaços $\big(\bigotimes_pT_xM^*\big)\otimes\big(\bigotimes_qT_xM\big)$,
$\bigvee_pT_xM^*$, $\bigvee_qT_xM$, $\bigwedge_pT_xM^*$ e $\bigwedge_qT_xM$, respectivamente. Usando novamente os resultados de tais e\-xer\-cí\-cios, obtemos que:
\begin{multline*}
\mathfrak t=\sum_{i_1,\ldots,i_p=1}^n\sum_{j_1,\ldots,j_q=1}^n\mathfrak t\Big(\frac{\partial}{\partial x_{i_1}},\ldots,
\frac{\partial}{\partial x_{i_p}},\dd x_{j_1},\ldots,\dd x_{j_q}\Big)\\
\dd x_{i_1}\otimes\cdots\otimes\dd x_{i_p}\otimes\frac{\partial}{\partial x_{j_1}}
\otimes\cdots\otimes\frac{\partial}{\partial x_{j_q}},
\end{multline*}
se $\mathfrak t$ é um $(p,q)$-campo tensorial sobre $U$, que:
\[\mathfrak t=\sum_{1\le i_1\le\cdots\le i_p\le n}
\frac1{\langle i\rangle}\,\mathfrak t\Big(\frac{\partial}{\partial x_{i_1}},\ldots,\frac{\partial}{\partial x_{i_p}}\Big)\;\dd x_{i_1}\vee\cdots\vee\dd x_{i_p},\]
se $\mathfrak t$ é um $(p,0)$-campo tensorial simétrico sobre $U$, que:
\[\mathfrak t=\sum_{1\le j_1\le\cdots\le j_q\le n}
\frac1{\langle j\rangle}\,\mathfrak t(\dd x_{j_1},\ldots,\dd x_{j_q})\;\frac{\partial}{\partial x_{j_1}}\vee\cdots\vee\frac{\partial}{\partial x_{j_q}},\]
se $\mathfrak t$ é um $(0,q)$-campo tensorial simétrico sobre $U$, que:
\begin{equation}\label{eq:formascoordenadas}
\mathfrak t=\sum_{1\le i_1<\cdots<i_p\le n}
\mathfrak t\Big(\frac{\partial}{\partial x_{i_1}},\ldots,\frac{\partial}{\partial x_{i_p}}\Big)\;\dd x_{i_1}\wedge\cdots\wedge\dd x_{i_p},
\end{equation}
se $\mathfrak t$ é um $(p,0)$-campo tensorial anti-simétrico sobre $U$ e que:
\[\mathfrak t=\sum_{1\le j_1<\cdots<j_q\le n}
\mathfrak t(\dd x_{j_1},\ldots,\dd x_{j_q})\;\frac{\partial}{\partial x_{j_1}}\wedge\cdots\wedge\frac{\partial}{\partial x_{j_q}},\]
se $\mathfrak t$ é um $(0,q)$-campo tensorial anti-simétrico sobre $U$.

\begin{exercise}
Se uma $k$-forma $\omega$ de classe $C^\infty$ no domínio de uma carta $\varphi=(x_1,\ldots,x_n)$ é dada por:
\[\omega=\sum_{1\le i_1<\cdots<i_k\le n}\omega_{i_1\ldots i_k}\;\dd x_{i_1}\wedge\cdots\wedge\dd x_{i_k},\]
onde $\omega_{i_1\ldots i_k}$ são funções (automaticamente\footnote{%
Note que $\omega_{i_1\ldots i_k}=\omega\big(\frac{\partial}{\partial x_{i_1}},\ldots,\frac{\partial}{\partial x_{i_k}}\big)$ e use o resultado
do Exercício~\ref{exe:avalsmooth}.}
de classe $C^\infty$) sobre o domínio de $\varphi$,
use os resultados dos itens~(a)---(d) do Exercício~\ref{exe:propdifextvar} para mostrar que:
\[\dd\omega=\sum_{1\le i_1<\cdots<i_k\le n}\dd\omega_{i_1\ldots i_k}\wedge\dd x_{i_1}\wedge\cdots\wedge\dd x_{i_k}\]
(sugestão: note que o produto de uma função escalar por uma $k$-forma pode ser pensado como o produto exterior de uma $0$-forma por uma $k$-forma).
\end{exercise}

\end{section}

\begin{section}{Dia 25/10}

\begin{exercise}\label{exe:exp1x}\
\begin{itemize}
\item[(a)] Seja $f:\left[a,b\right[\to\R$ uma função contínua, derivável em $\left]a,b\right[$. Mostre que se o limite $\lim_{t\to a}f'(t)$ existe
então $f$ é derivável no ponto $a$ e:
\[f'(a)=\lim_{t\to a}f'(t)\]
(sugestão: reescreva o quociente $\frac{f(a+h)-f(a)}h$ usando o teorema do valor médio).
\item[(b)] Seja $f:\left]0,+\infty\right[\to\R$ uma função da forma:
\[f(x)=\frac{p(x)}{q(x)}\,e^{-\frac1x},\quad x>0,\]
onde $p$, $q$ são polinômios e $q$ não tem raízes em $\left]0,+\infty\right[$. Mostre que $\lim_{x\to0}f(x)=0$ e que:
\[f'(x)=\frac{p_1(x)}{q_1(x)}\,e^{-\frac1x},\quad x>0,\]
onde $p_1$, $q_1$ são polinômios e $q_1$ não tem raízes em $\left]0,+\infty\right[$.
\item[(c)] Mostre que a função $f:\R\to\R$ definida por $f(x)=e^{-\frac1x}$ para $x>0$ e $f(x)=0$ para $x\le0$ é de classe $C^\infty$.
\end{itemize}
\end{exercise}

\begin{exercise}\label{exe:corteRn}\
\begin{itemize}
\item[(a)] Mostre que existe uma função de $\R$ em $\R$ não negativa, não identicamente nula, de classe $C^\infty$ e com suporte\footnote{%
O {\em suporte\/} de uma função é o fecho do conjunto dos pontos nos quais a função é diferente de zero.} compacto (sugestão:
seja $f$ a função definida no item~(c) do Exercício~\ref{exe:exp1x}. Considere o produto de $f$ por uma função cujo gráfico
é a reflexão do gráfico de $f$ numa reta paralela ao eixo das ordenadas).
\item[(b)] Mostre que para quaisquer números reais $a$, $b$ com $0<b<a$ existe uma função $f:\R\to\R$ de classe $C^\infty$ que vale $1$
no intervalo $[-b,b]$, que tem suporte contido no intervalo $\left]-a,a\right[$ e cuja imagem está contida no intervalo $[0,1]$
(sugestão: primeiro use o resultado do item~(a) para mostrar que existe uma função $g:\R\to\R$ de classe
$C^\infty$ com suporte contido em $\left]-a,a\right[$, que vale zero em $[-b,b]$, que é não negativa em $\left]-\infty,0\right]$,
não positiva em $\left[0,+\infty\right[$ e tal que:
\[\int_{-a}^0g=1,\quad\int_0^ag=-1.\]
Depois, defina $f(x)=\int_{-\infty}^xg(t)\,\dd t$).
\item[(c)] Mostre que para quaisquer números reais $a$, $b$ com $0<b<a$ existe uma função $f:\R^n\to\R$ de classe $C^\infty$ que vale $1$
na bola fechada de centro na origem e raio $b$, que tem suporte contido na bola aberta de centro na origem e raio $a$ e
cuja imagem está contida no intervalo $[0,1]$ (sugestão: se $f$ é a função do item~(b),
considere a função $x\mapsto f(\Vert x\Vert)$, onde $\Vert x\Vert$ denota a norma Euclideana de $x$).
\end{itemize}
\end{exercise}

\begin{exercise}[função de corte]\label{exe:cutoff}
Sejam $M$ uma variedade diferenciável Hausdorff, $U$ um subconjunto aberto de $M$ e $p$ um ponto de $U$. Mostre que existe uma função
$\phi:M\to\R$ de classe $C^\infty$ que tem suporte contido em $U$, vale $1$ numa vizinhança de $p$ e tem imagem contida no intervalo $[0,1]$ (sugestão:
seja $\varphi$ uma carta em $M$ cujo domínio está contido em $U$, cuja imagem é uma bola aberta de centro na origem e raio $a$ e tal que $\varphi(p)=0$.
Seja $f$ uma função como aquela dada pelo item~(c) do Exercício~\ref{exe:corteRn}. Defina $\phi$ como sendo $f\circ\varphi$ no domínio de $\varphi$
e como sendo zero fora do domínio de $\varphi$. Para mostrar que $\phi$ é de classe $C^\infty$, escreva $M$ como união de dois abertos restrita aos quais
a função $\phi$ é de classe $C^\infty$, a saber, o domínio de $\varphi$ e o complementar em $M$ de $\varphi^{-1}(\supp f)$, onde $\supp f$ denota o suporte
de $f$. Note que $\varphi^{-1}(\supp f)$ é compacto e portanto fechado, {\em tendo em vista que $M$ é Hausdorff\/}!).
\end{exercise}

\begin{exercise}
Seja $M$ a variedade diferenciável (não Hausdorff!) definida no Exercício~\ref{exe:duasorigens} com $A=\R\setminus\{0\}$. Mostre que para qualquer
função contínua $\phi:M\to\R$ vale que $\phi\big(q(0,0)\big)=\phi\big(q(0,1)\big)$ (sugestão: a função $\phi\circ q:\R\times\{0,1\}\to\R$ é contínua e
$(\phi\circ q)(x,0)=(\phi\circ q)(x,1)$ para todo $x\ne0$). Conclua que se $p=q(0,0)$ e $U$ é o aberto $q\big(\R\times\{0\}\big)$ então
não existe uma função $\phi$ como aquela cujo resultado do Exercício~\ref{exe:cutoff} afirma existir (sugestão: $q(0,1)\not\in U$).
\end{exercise}

\begin{exercise}\label{exe:difdada}
Sejam $M$ uma variedade diferenciável Hausdorff e $p\in M$ um ponto.
\begin{itemize}
\item[(a)] Mostre que se $U$ é um subconjunto aberto de $M$ contendo $p$ e se $f_0:U\to\R$ é uma função de classe $C^\infty$ então existe uma função
$f:M\to\R$ de classe $C^\infty$ que coincide com $f_0$ numa vizinhança de $p$ (sugestão: pelo resultado do Exercício~\ref{exe:cutoff} existe
uma função $\phi:M\to\R$ de classe $C^\infty$ que vale $1$ numa vizinhança de $p$ e tem suporte contido em $U$. Seja $f$ tal que $f$ é igual a $f_0\phi$
em $U$ e $f$ é igual a zero fora de $U$).
\item[(b)] Mostre que dados $c\in\R$ e um funcional linear $\alpha\in T_pM^*$ então existe uma função $f:M\to\R$ de classe $C^\infty$ tal
que $f(p)=c$ e $\dd f(p)=\alpha$ (sugestão: primeiro use uma carta $\varphi$ cujo domínio é uma vizinhança aberta $U$ de $p$ para obter uma função
$f_0:U\to\R$ de classe $C^\infty$ tal que $f_0(p)=c$ e $\dd f_0(p)=\alpha$; por exemplo, faça com que a representação de $f_0$ com respeito a $\varphi$
seja a soma de um funcional linear com uma constante. Use então o resultado do item~(a)).
\end{itemize}
\end{exercise}

\begin{defin}\label{thm:defXdef}
Sejam $M$ uma variedade diferenciável, $X:M\to TM$ um campo vetorial em $M$ e $f:M\to E$ uma função de classe $C^\infty$ tomando valores num espaço
vetorial real de dimensão finita $E$. Denotamos por $X(f)$ a função:
\[X(f):M\ni x\longmapsto\dd f_x\big(X(x)\big)\in E.\]
\end{defin}

\begin{exercise}\label{exe:propXf}
Sejam $X:M\to TM$ um campo vetorial numa variedade diferenciável $M$ e $E$ um espaço vetorial real de dimensão finita.
\begin{itemize}
\item[(a)] Mostre que a aplicação $f\mapsto X(f)$ definida no espaço vetorial real das funções $f:M\to E$ de classe $C^\infty$ e tomando valores no espaço vetorial
real das funções em $M$ a valores em $E$ é linear.
\item[(b)] Mostre que se $f:M\to\R$, $g:M\to\R$ são funções de classe $C^\infty$ então:
\[X(fg)=X(f)g+fX(g).\]
\item[(c)] Mostre que se $N$ é uma variedade diferenciável, $\phi:N\to M$ é um difeomorfismo local e $f:M\to E$ é uma função de classe $C^\infty$ então:
\[X(f)\circ\phi=(\phi^*X)(f\circ\phi).\]
Enuncie e mostre um resultado análogo para o push-forward.
\item[(d)] Mostre que se $X$ e $f:M\to E$ são de classe $C^\infty$ então também $X(f)$ é de classe $C^\infty$ (sugestão: use o resultado do item~(c)
para um difeomorfismo que é uma carta de $M$).
\item[(e)] Mostre que se $f:M\to E$ é de classe $C^\infty$, $E'$ é um espaço vetorial real de dimensão finita e $L:E\to E'$ é uma aplicação linear então:
\[X(L\circ f)=L\circ X(f).\]
\end{itemize}
\end{exercise}

\begin{exercise}
Seja $X:M\to TM$ um campo vetorial numa variedade diferenciável $M$. Se $M$ é Hausdorff e $X(f)\in C^\infty(M)$ para toda $f\in C^\infty(M)$, mostre
que $X$ é de classe $C^\infty$ (sugestão: sejam $p\in M$ e:
\[\varphi=(\varphi_1,\ldots,\varphi_n):U\subset M\longrightarrow\widetilde U\subset\R^n\]
uma carta em $M$ tal que $p\in U$. Use o resultado do item~(a) do Exercício~\ref{exe:difdada} para encontrar uma função
$f_i\in C^\infty(M)$ que coincide com $\varphi_i$ numa
vizinhança aberta $V_i$ de $p$; seja $V=\bigcap_{i=1}^nV_i$. Mostre que $(\varphi_*X)\circ\varphi:U\to\R^n$ e $\big(X(f_1),\ldots,X(f_n)\big):M\to\R^n$
coincidem sobre $V$ e conclua que $X\vert_V$ é de classe $C^\infty$).
\end{exercise}

\begin{defin}
Seja $A$ uma álgebra (i.e., $A$ é um espaço vetorial munido de uma multiplicação $A\times A\ni(a,b)\mapsto ab\in A$ que é bilinear). Uma {\em derivação\/}
de $A$ é um operador linear $D:A\to A$ tal que:
\[D(ab)=D(a)b+aD(b),\]
para quaisquer $a,b\in A$. Denotamos por $\Der(A)$ o conjunto das derivações de $A$.
\end{defin}
Evidentemente, $\Der(A)$ é um subespaço do espaço vetorial dos operadores lineares em $A$.
Em vista dos resultados dos itens~(a), (b) e (d) do Exercício~\ref{exe:propXf} temos que, para qualquer campo vetorial $X$ de classe $C^\infty$
numa variedade diferenciável $M$, a aplicação:
\begin{equation}\label{eq:dercampo}
C^\infty(M)\ni f\longmapsto X(f)\in C^\infty(M)
\end{equation}
é uma derivação da álgebra $C^\infty(M)$.

\medskip

O objetivo do restante desta seção é mostrar o seguinte:
\begin{teo}\label{thm:camposderivacoes}
Se $M$ é uma variedade diferenciável Hausdorff e $D$ é uma derivação da álgebra $C^\infty(M)$ então existe um único campo vetorial $X$ em $M$ tal que
$D(f)=X(f)$, para toda $f\in C^\infty(M)$. O campo vetorial $X$ é de classe $C^\infty$.
\end{teo}

\begin{exercise}\label{exe:XfYfXigualY}
Sejam $X$, $Y$ campos vetoriais numa variedade diferenciável Hausdorff $M$. Mostre que se $X(f)=Y(f)$ para qualquer $f\in C^\infty(M)$ então $X=Y$
(sugestão: use o resultado do item~(b) do Exercício~\ref{exe:difdada} para concluir que $\alpha\big(X(p)\big)=\alpha\big(Y(p)\big)$ para todo
$p\in M$ e todo $\alpha\in T_pM^*$). Conclua que é válida a afirmação do enunciado do Teorema~\ref{thm:camposderivacoes} a respeito da unicidade do campo $X$.
\end{exercise}

\begin{exercise}\label{exe:Dconst}
Sejam $A$ uma álgebra com unidade $1\in A$ (i.e., $a1=1a=a$, para todo $a\in A$)
e $D$ uma derivação de $A$. Mostre que $D(1)=0$ (sugestão: quanto é $D(1\cdot1)$?).
Conclua que se $M$ é uma variedade diferenciável e $D$ é uma derivação de $C^\infty(M)$ então $D(f)=0$ para qualquer função constante $f:M\to\R$.
\end{exercise}

\begin{exercise}\label{exe:pigeramideal}
Seja $U$ um subconjunto aberto de $\R^n$ estrelado em torno de um ponto $x^0\in U$ (i.e., para todo $x\in U$ temos que o segmento ligando $x^0$ a $x$ está contido em $U$).
Se $f:U\to\R$ é uma função de classe $C^\infty$, mostre que existem funções $a_i:U\to\R$, $i=1,\ldots,n$ de classe $C^\infty$ tais que:
\[f(x)=f(x^0)+\sum_{i=1}^na_i(x)(x_i-x^0_i),\]
para todo $x\in U$ e tais que $a_i(x^0)=\frac{\partial f}{\partial x_i}(x^0)$, $i=1,\ldots,n$ (sugestão: note que:
\[f(x)=f(x^0)+\int_0^1\frac{\dd}{\dd t}\big[f\big((1-t)x^0+tx\big)\big]\,\dd t,\quad x\in U.\]
Expanda o integrando em termos das derivadas parciais de $f$
usando a regra da cadeia. Para mostrar que a função $a_i$ é de classe $C^\infty$ você vai precisar do seguinte resultado dos
cursos de Cálculo no $\R^n$: se $\alpha:W\to\R$ é uma função de classe $C^\infty$
num aberto $W$ de $\R\times\R^n$ que contém $[a,b]\times U$, onde $U$ é um aberto de $\R^n$, então a função $U\ni x\mapsto\int_a^b\alpha(t,x)\,\dd t$
é de classe $C^\infty$).
\end{exercise}

\begin{exercise}\label{exe:valenoconvexo}
Mostre que o Teorema~\ref{thm:camposderivacoes} vale se $M$ é um subconjunto aberto convexo de $\R^n$ (sugestão: seja $\pi_i:M\to\R$, $i=1,\ldots,n$
a restrição a $M$ da $i$-ésima projeção de $\R^n$ e defina um campo vetorial
$X=(X_1,\ldots,X_n)$ em $M$ fazendo $X_i=D(\pi_i)$, $i=1,\ldots,n$. Dada $f\in C^\infty(M)$ e $x^0\in M$,
use o resultado do Exercício~\ref{exe:pigeramideal} para escrever:
\[f=f(x^0)+\sum_{i=1}^na_i(\pi_i-x^0_i),\]
com $a_1,\ldots,a_n\in C^\infty(M)$.
Aplique $D$ aos dois lados dessa igualdade e avalie o resultado no ponto $x^0$. Use também o resultado do Exercício~\ref{exe:Dconst}).
\end{exercise}

\begin{exercise}\label{exe:difeomorfasXf}
Sejam $M$, $M'$ variedades diferenciáveis difeomorfas. Mostre que se o Teorema~\ref{thm:camposderivacoes} vale para $M$ então ele também vale para $M'$
(sugestão: use o resultado do item~(c) do Exercício~\ref{exe:propXf}. Tenha em mente que se $\phi:M\to M'$ é um difeomorfismo então
$f\mapsto f\circ\phi$ define um isomorfismo da álgebra $C^\infty(M')$ na álgebra $C^\infty(M)$).
\end{exercise}

\begin{exercise}\label{exe:derlocal}
Sejam $M$ uma variedade diferenciável Hausdorff e $D$ uma derivação de $C^\infty(M)$. Mostre que $D$ é {\em local}, i.e., se $f,g\in C^\infty(M)$
coincidem sobre um subconjunto aberto $U$ de $M$ então também $D(f)$ e $D(g)$ coincidem sobre $U$ (sugestão: se $h=f-g$, mostre que $D(h)$ se anula
sobre $U$. Dado $p\in U$, use o resultado do Exercício~\ref{exe:cutoff} para obter uma função $\phi\in C^\infty(M)$ com suporte contido em $U$
tal que $\phi(p)=1$ e calcule $D(\phi h)$ no ponto $p$. Note que $\phi h=0$).
\end{exercise}

\begin{exercise}
Sejam $M$ uma variedade diferenciável Hausdorff, $D$ uma derivação de $C^\infty(M)$ e $U$ um subconjunto aberto de $M$. Mostre que existe uma derivação
$D^U$ da álgebra $C^\infty(U)$ tal que:
\begin{equation}\label{eq:DUD}
D^U(f\vert_U)=D(f)\vert_U,
\end{equation}
para qualquer $f\in C^\infty(M)$ (sugestão: para $f_0\in C^\infty(U)$ e $p\in U$, defina:
\[D^U(f_0)(p)=D(f)(p),\]
onde $f\in C^\infty(M)$ coincide com $f_0$ numa vizinhança de $p$; a existência de $f$ é dada pelo resultado do item~(a) do Exercício~\ref{exe:difdada} e o fato
que $D(f)(p)$ não depende da escolha de $f$ segue do resultado do Exercício~\ref{exe:derlocal}. Para mostrar que $D^U(f_0)$ é de classe $C^\infty$,
note que $D^U(f_0)$ e $D(f)$ coincidem sobre todo o interior da vizinhança de $p$ onde $f$ e $f_0$ coincidem).
\end{exercise}

\begin{exercise}\label{exe:teocampderpedacos}
Sejam $M$ uma variedade diferenciável Hausdorff e:
\[M=\bigcup_{U\in\mathcal U}U\]
uma cobertura aberta de $M$. Mostre que se o Teorema~\ref{thm:camposderivacoes}
vale para qualquer $U\in\mathcal U$ então o Teorema~\ref{thm:camposderivacoes} vale para $M$ (sugestão: dada uma derivação $D$ de $C^\infty(M)$,
escolha para cada aberto $U\in\mathcal U$ uma derivação $D^U$ de $C^\infty(U)$ satisfazendo \eqref{eq:DUD} e seja $X^U$ o campo vetorial de classe
$C^\infty$ em $U$ tal que $X^U(f)=D^U(f)$, para toda $f\in C^\infty(U)$. Mostre que para $U,V\in\mathcal U$ os campos vetoriais $X^U$ e $X^V$
coincidem sobre $U\cap V$; para isso, escolha $p\in U\cap V$, $\alpha\in T_pM^*$, $f$ como no enunciado do item~(b) do Exercício~\ref{exe:difdada}
e compare $X^U(f\vert_U)$ com $X^V(f\vert_V)$ no ponto $p$. Seja $X$ o campo vetorial em $M$ que coincide com $X^U$ em $U$, para cada $U\in\mathcal U$,
e mostre que $X(f)=D(f)$ para toda $f\in C^\infty(M)$).
\end{exercise}

\begin{proof}[Demonstração do Teorema~\ref{thm:camposderivacoes}]
A afirmação sobre unicidade foi demonstrada no Exercício~\ref{exe:XfYfXigualY}. A afirmação sobre existência segue dos resultados dos Exercícios~\ref{exe:valenoconvexo},
\ref{exe:difeomorfasXf} e \ref{exe:teocampderpedacos}, tendo em mente que $M$ pode ser coberta por abertos que são difeomorfos a abertos convexos de $\R^n$.
\end{proof}

\begin{exercise}\label{exe:isoDerXM}
Em vista do Teorema~\ref{thm:camposderivacoes} temos, para uma dada variedade diferenciável Hausdorff $M$, uma aplicação bijetora:
\begin{equation}\label{eq:isodercampos}
\mathfrak X(M)\longrightarrow\Der\!\big(C^\infty(M)\big)
\end{equation}
que associa a cada campo vetorial $X\in\mathfrak X(M)$ a derivação \eqref{eq:dercampo} de $C^\infty(M)$. Mostre que a aplicação \eqref{eq:isodercampos}
é um isomorfismo de espaços vetoriais reais.
\end{exercise}

\begin{exercise}\label{exe:Amodulo}
Se $A$ é uma álgebra associativa e comutativa, mostre que se $D$ é uma derivação de $A$ e $a\in A$ então:
\[aD:A\ni b\longmapsto a\big(D(b)\big)\in A\]
é uma derivação de $A$. Conclua que o espaço vetorial $\Der(A)$, munido da operação definida acima, é um $A$-módulo.
\end{exercise}

\begin{exercise}\label{exe:isoAmodulos}
Seja $M$ uma variedade diferenciável Hausdorff. Se o espaço vetorial $\Der\!\big(C^\infty(M)\big)$ é munido da estrutura de $C^\infty(M)$-módulo definida pelo Exercício~\ref{exe:Amodulo},
mostre que \eqref{eq:isodercampos} é um isomorfismo de $C^\infty(M)$-módulos (recorde o Exercício~\ref{exe:XMCMmodulo} para a estrutura de $C^\infty(M)$-módulo
de $\mathfrak X(M)$).
\end{exercise}

\end{section}

\begin{section}{Dia 27/10}

\begin{exercise}\label{exe:comutadorderivacoes}
Sejam $A$ uma álgebra e $D_1$, $D_2$ derivações de $A$. Mostre que o comutador $[D_1,D_2]=D_1\circ D_2-D_2\circ D_1$ também é uma derivação de $A$.
\end{exercise}

\begin{exercise}\label{exe:colchete}
Seja $M$ uma variedade diferenciável Hausdorff. Dados campos vetoriais $X$, $Y$ em $M$ de classe $C^\infty$, mostre que existe um único campo vetorial
$Z$ em $M$ tal que:
\[Z(f)=X\big(Y(f)\big)-Y\big(X(f)\big),\]
para qualquer $f\in C^\infty(M)$ e mostre que o campo vetorial $Z$ é de classe $C^\infty$ (sugestão: use o Teorema~\ref{thm:camposderivacoes} e o resultado do
Exercício~\ref{exe:comutadorderivacoes}).
\end{exercise}

\begin{defin}
Se $M$ é uma variedade diferenciável Hausdorff e $X$, $Y$ são campos vetoriais de classe $C^\infty$ em $M$ então o campo vetorial $Z$ em $M$ de classe $C^\infty$
definido pelo resultado do Exercício~\ref{exe:colchete} é denotado por $[X,Y]$ e é chamado o {\em Colchete de Lie\/} de $X$ por $Y$.
\end{defin}
Temos:
\begin{equation}\label{eq:defcolchete}
[X,Y](f)=X\big(Y(f)\big)-Y\big(X(f)\big),
\end{equation}
para todos $X,Y\in\mathfrak X(M)$ e toda $f\in C^\infty(M)$.

\begin{exercise}
Seja $A$ uma álgebra. Mostre que a aplicação:
\[\Der(A)\times\Der(A)\ni(D_1,D_2)\longmapsto[D_1,D_2]\in\Der(A)\]
é bilinear (sobre o corpo de escalares da álgebra) e anti-simétrica.
Conclua usando o resultado do Exercício~\ref{exe:isoDerXM} que, dada uma variedade diferenciável Hausdorff $M$, então a aplicação:
\[\mathfrak X(M)\times\mathfrak X(M)\ni(X,Y)\longmapsto[X,Y]\in\mathfrak X(M)\]
é bilinear (sobre o corpo de escalares $\R$) e anti-simétrica.
\end{exercise}

\begin{exercise}
Seja $A$ uma álgebra associativa e comutativa e considere $\Der(A)$ munido da estrutura de $A$-módulo definida no Exercício~\ref{exe:Amodulo}.
Mostre que:
\[[D_1,aD_2]=D_1(a)D_2+a[D_1,D_2],\]
para quaisquer $a\in A$, $D_1,D_2\in\Der(A)$. Conclua usando o resultado do Exercício~\ref{exe:isoAmodulos} que, dada uma variedade diferenciável Hausdorff $M$,
então:
\[[X,fY]=X(f)Y+f[X,Y],\]
para quaisquer $X,Y\in\mathfrak X(M)$, $f\in C^\infty(M)$.
\end{exercise}

\begin{exercise}
Sejam $M$ uma variedade diferenciável Hausdorff, $X$, $Y$ campos vetoriais de classe $C^\infty$ em $M$ e $E$ um espaço vetorial real de dimensão finita.
Mostre que a igualdade \eqref{eq:defcolchete} vale também se $f:M\to E$ é uma aplicação de classe $C^\infty$ a valores em $E$
(sugestão: seja $\alpha\in E^*$ um funcional linear arbitrário,
use o fato que \eqref{eq:defcolchete} vale se trocamos $f$ por $\alpha\circ f:M\to\R$ e use o resultado do item~(e) do Exercício~\ref{exe:propXf}).
\end{exercise}

\begin{defin}
Sejam $M$, $N$ variedades diferenciáveis e $\phi:M\to N$ uma aplicação de classe $C^\infty$. Um campo vetorial $X$ em $M$ é dito {\em $\phi$-relacionado\/}
com um campo vetorial $Y$ em $N$ quando:
\[Y\big(\phi(x)\big)=\dd\phi_x\big(X(x)\big),\]
para todo $x\in M$.
\end{defin}

\begin{exercise}\label{exe:phirelpullback}
Sejam $M$, $N$ variedades diferenciáveis e $\phi:M\to N$ um difeomorfismo local. Dados um campo vetorial $X$ em $M$ e um campo vetorial $Y$ em $N$, mostre
que $X$ e $Y$ são $\phi$-relacionados se e somente se $X=\phi^*Y$.
\end{exercise}

\begin{exercise}\label{exe:phirelder}
Sejam $M$, $N$ variedades diferenciáveis e $\phi:M\to N$ uma aplicação de classe $C^\infty$. Suponha que $N$ seja Hausdorff.
Dados um campo vetorial $X$ em $M$ e um campo vetorial $Y$ em $N$, mostre que $X$ é $\phi$-relacionado com $Y$ se e somente se:
\begin{equation}\label{eq:YfXfphi}
Y(f)\circ\phi=X(f\circ\phi),
\end{equation}
para qualquer $f\in C^\infty(N)$ (sugestão: assumindo \eqref{eq:YfXfphi}, mostre que:
\[\alpha\big[Y\big(\phi(x)\big)\big]=\alpha\big[\dd\phi_x\big(X(x)\big)\big],\]
para todo $x\in M$ e todo $\alpha\in T_{\phi(x)}N^*$, usando o resultado do item~(b) do Exercício~\ref{exe:difdada}).
\end{exercise}

\begin{exercise}\label{exe:phirelLiebrack}
Sejam $M$, $N$ variedades diferenciáveis Hausdorff, $X'$, $Y'$ campos vetoriais em $M$ e $X$, $Y$ campos vetoriais em $N$. Se $\phi:M\to N$
é uma aplicação de classe $C^\infty$, $X'$ é $\phi$-relacionado com $X$ e $Y'$ é $\phi$-relacionado com $Y$, mostre que $[X',Y']$ é $\phi$-relacionado
com $[X,Y]$ (sugestão: use o resultado do Exercício~\ref{exe:phirelder}). Conclua usando o resultado do Exercício~\ref{exe:phirelpullback}
que se $\phi$ é um difeomorfismo local então:
\begin{equation}\label{eq:Liebrackpback}
\phi^*[X,Y]=[\phi^*X,\phi^*Y].
\end{equation}
\end{exercise}

\begin{rem}
Um caso particular interessante de \eqref{eq:Liebrackpback} é o seguinte: se $\phi$ é a aplicação inclusão de um subconjunto aberto então o pull-back
por $\phi$ nada mais é que restrição a esse aberto, de modo que:
\[[X\vert_U,Y\vert_U]=[X,Y]\vert_U,\]
se $X$, $Y$ são campos vetoriais de classe $C^\infty$ numa variedade diferenciável Hausdorff $M$ e $U$ é um subconjunto aberto de $M$. Mais
geralmente, se $N$ é uma subvariedade (possivelmente imersa\footnote{%
O fato que a aplicação inclusão de $N$ em $M$ é contínua garante que $N$ é automaticamente Hausdorff se $M$ o for.})
de $M$ e $X(x)\in T_xN$, $Y(x)\in T_xN$ para qualquer $x\in N$ então $X\vert_N$,
$Y\vert_N$ (com o contra-domínio alterado para $TN\subset TM$) são campos vetoriais de classe $C^\infty$ em $N$ (veja Exercício~\ref{exe:restrcampoimersa}) e, tomando
$\phi$ como sendo a aplicação inclusão de $N$ em $M$, o resultado do Exercício~\ref{exe:phirelLiebrack} nos dá que $[X,Y](x)\in T_xN$ para todo $x\in N$ e que:
\[[X\vert_N,Y\vert_N]=[X,Y]\vert_N.\]
\end{rem}

\begin{exercise}\label{exe:LiebrackRn}
Sejam $U$ um aberto de um espaço vetorial real de dimensão finita $E$ e $X:U\to E$, $Y:U\to E$ campos vetoriais de classe $C^\infty$ em $U$.
Dada uma função $f:U\to\R$ de classe $C^\infty$, mostre que:
\[X\big(Y(f)\big)(x)=\dd^2f_x\big(X(x),Y(x)\big)+\dd f_x\big[\dd Y_x\big(X(x)\big)\big],\]
para todo $x\in U$ (sugestão: calcule a derivada direcional de $x\mapsto\dd f_x\big(Y(x)\big)$ num ponto $x$ na direção de $X(x)$ usando
o resultado do Exercício~\ref{exe:regraproduto} e tendo em mente a bilinearidade da aplicação $E^*\times E\ni(\alpha,v)\mapsto\alpha(v)\in\R$).
Usando a simetria de $\dd^2f_x$, conclua que:
\[\big[X\big(Y(f)\big)-Y\big(X(f)\big)\big](x)=\dd f_x\big[\dd Y_x\big(X(x)\big)-\dd X_x\big(Y(x)\big)\big],\]
para todo $x\in U$ e que o colchete $[X,Y]$ é dado por:
\[[X,Y](x)=\dd Y_x\big(X(x)\big)-\dd X_x\big(Y(x)\big),\quad x\in U,\]
ou seja\footnote{%
Quando escrevemos $X(Y)$ estamos pensando em $X$ como campo vetorial na variedade $U$ e em $Y$ como uma função a valores no espaço vetorial fixo $E$
definida na variedade $U$.}:
\begin{equation}\label{eq:LiebackRn}
[X,Y]=X(Y)-Y(X).
\end{equation}
\end{exercise}

\begin{exercise}
Sejam $M$ uma variedade diferenciável e:
\[\varphi:U\subset M\longrightarrow\widetilde U\subset\R^n\]
uma carta em $M$. Denote por:
\begin{equation}\label{eq:deldelxi2}
\frac{\partial}{\partial x_i},\quad i=1,\ldots,n,
\end{equation}
o campo vetorial em $U$ que é o pull-back por $\varphi$ do campo vetorial constante e igual ao $i$-ésimo vetor da base canônica de $\R^n$.
Mostre que\footnote{%
Não precisamos supor que $M$ seja Hausdorff pois certamente $U$ é Hausdorff, sendo homeomorfo a $\widetilde U$.}:
\[\Big[\frac{\partial}{\partial x_i},\frac{\partial}{\partial x_j}\Big]=0,\]
para todos $i,j=1,\ldots,n$ (sugestão: use \eqref{eq:Liebrackpback} e \eqref{eq:LiebackRn}).
\end{exercise}

\begin{exercise}\label{exe:phiprojetavel}
Sejam $M$, $N$ variedades diferenciáveis e $\phi:M\to N$ uma submersão sobrejetora de classe $C^\infty$. Um campo vetorial $X$ em $M$
é dito {\em projetável por $\phi$\/} se:
\[\dd\phi_x\big(X(x)\big)=\dd\phi_y\big(X(y)\big),\]
para quaisquer $x,y\in M$ tais que $\phi(x)=\phi(y)$. Mostre que $X$ é projetável por $\phi$ se e somente se existe um campo vetorial $Y$
em $N$ que é $\phi$-relacionado com $X$ e que tal campo $Y$ é único se existir. Mostre também que $Y$ é de classe $C^\infty$ se $X$ for de classe
$C^\infty$ (sugestão: note que $Y\circ\phi=\dd\phi\circ X$ e use o resultado do Exercício~\ref{exe:passquoc}).
\end{exercise}

\begin{exercise}[campos canônicos do toro]
Considere o toro $n$-dimensional $(S^1)^n$ e o difeomorfismo local $q:\R^n\to(S^1)^n$ definido no Exercício~\ref{exe:toro}.
Mostre que um campo vetorial $X:\R^n\to\R^n$ em $\R^n$ é projetável por $q$ se e somente se:
\[X(\theta)=X(\theta'),\]
para quaisquer $\theta,\theta'\in\R^n$ tais que $\theta-\theta'\in\Z^n$ (sugestão: note que:
\[\dd q_\theta(v)=2\pi(ie^{2\pi i\theta_1}v_1,\ldots,ie^{2\pi i\theta_n}v_n),\]
para quaisquer $\theta,v\in\R^n$, de modo que $\dd q_\theta=\dd q_{\theta'}$ se $\theta-\theta'\in\Z^n$).
Conclua que o campo vetorial constante e igual ao $j$-ésimo vetor da base canônica de $\R^n$ é projetável por $q$
e denote por:
\[\frac{\partial}{\partial\theta_j},\quad j=1,\ldots,n,\]
o único campo vetorial no toro que é $q$-relacionado com tal campo constante (veja Exercício~\ref{exe:phiprojetavel}). Mostre que:
\[\Big[\frac{\partial}{\partial\theta_j},\frac{\partial}{\partial\theta_k}\Big]=0,\]
para todos $j,k=1,\ldots,n$ (sugestão: uma opção é usar o resultado do Exercício~\ref{exe:phirelLiebrack}. Outra opção é notar que se $U$ é um aberto
de $\R^n$ onde o difeomorfismo local $q$ é injetor então os campos $\frac{\partial}{\partial\theta_j}$, quando restritos ao aberto $q(U)$ do toro,
são os campos \eqref{eq:deldelxi2} associados à carta $(q\vert_U)^{-1}:q(U)\to U$).
\end{exercise}

\newpage

\noindent\lower15pt\hbox{\dbend}\hspace{10pt}\parbox[t]{330pt}{\begin{exercise}
O Teorema~\ref{thm:camposderivacoes} não vale em geral para variedades que não são Hausdorff e não se generaliza para variedades de dimensão infinita e nem mesmo
para variedades de classe $C^k$ com $k<+\infty$, de modo que é importante ter uma forma de definir o colchete de Lie sem usar o Teorema~\ref{thm:camposderivacoes}.
Mostre o seguinte:
\begin{lem}\label{thm:lemLiebrackgeral}
Se $M$ é uma variedade diferenciável e $X$, $Y$ são campos vetoriais de classe $C^\infty$ em $M$ então existe um único campo vetorial
$Z$ em $M$ tal que:
\[Z(f)=X\big(Y(f)\big)-Y\big(X(f)\big),\]
para qualquer função $f:\Dom(f)\to\R$ de classe $C^\infty$ definida num subconjunto aberto $\Dom(f)$ de $M$ (estamos usando a notação $X(f)$ mesmo quando $X$ e $f$
não têm o mesmo domínio e estamos subentendendo, obviamente, que tanto $X$ quanto $f$ devem ser restritos ao domínio comum de $X$ e $f$). Além do mais,
o campo vetorial $Z$ é de classe $C^\infty$.
\end{lem}
Define-se então $[X,Y]=Z$. Para demonstrar o Lema~\ref{thm:lemLiebrackgeral}, siga o seguinte roteiro:
\begin{itemize}
\item[(a)] o lema vale se a variedade $M$ é um aberto de $\R^n$ (proceda como no Exercício~\ref{exe:LiebrackRn});
\item[(b)] se $M$ e $M'$ são difeomorfas e o lema vale para $M$ então ele também vale para $M'$ (use o resultado do item~(c) do Exercício~\ref{exe:propXf});
\item[(c)] se $M=\bigcup_{U\in\mathcal U}U$ é uma cobertura aberta de $M$ e o lema vale para qualquer aberto de $M$ que está contido
em algum $U\in\mathcal U$ então o lema vale para $M$;
\item[(d)] por (a), (b) e (c), o lema vale para qualquer variedade diferenciável $M$.
\end{itemize}
Você também pode adaptar o resultado do Exercício~\ref{exe:phirelder} para variedades não Hausdorff (exigindo que \eqref{eq:YfXfphi} seja válida
para funções de classe $C^\infty$ em {\em subconjuntos abertos\/} de $N$) e demonstrar o resultado do Exercício~\ref{exe:phirelLiebrack} também para variedades
não Hausdorff.
\end{exercise}}

\end{section}

\begin{section}{Dia 03/11}

\begin{defin}
Sejam $M$ uma variedade diferenciável e $X$ um campo vetorial em $M$. Uma {\em curva integral\/} de $X$ é uma função
$\gamma:I\to M$ de classe $C^\infty$, definida num intervalo aberto $I\subset\R$, tal que:
\[\gamma'(t)=X\big(\gamma(t)\big),\]
para todo $t\in I$.
\end{defin}

Note que, se $M$ é um aberto de $\R^n$, então a condição de que $\gamma$ seja uma curva integral de $X$ é nada mais que uma equação
diferencial ordinária para $\gamma$. Assumiremos o seguinte resultado demonstrado em alguns
cursos de equações diferenciais ordinárias:
\begin{teo}[existência, unicidade e dependência $C^\infty$ das condições iniciais]\label{thm:EDO}
Seja $X:U\to\R^n$ uma função de classe $C^\infty$ definida num subconjunto aberto $U$ de $\R^n$ (i.e., $X$ é um campo
vetorial de classe $C^\infty$ em $U$).
\begin{itemize}
\item[(a)] Dados $t_0\in\R$, $x_0\in U$ então
existem $\varepsilon>0$, uma vizinhança aberta $V$ de $x_0$ em $U$ e uma função
$\gamma:\left]t_0-\varepsilon,t_0+\varepsilon\right[\times V\to U$ de classe $C^\infty$ tal que $\gamma(t_0,x)=x$ para todo $x\in V$ e tal que:
\[\left]t_0-\varepsilon,t_0+\varepsilon\right[\ni t\longmapsto\gamma(t,x)\in U\]
é uma curva integral de $X$, para todo $x\in V$.
\item[(b)] Se $I\subset\R$ é um intervalo aberto, $\gamma:I\to U$, $\mu:I\to U$
são curvas integrais de $X$ e se $\gamma(t)=\mu(t)$ para algum $t\in I$ então $\gamma=\mu$.
\end{itemize}
\end{teo}

\begin{exercise}\label{exe:curintphirel}
Sejam $M$, $N$ variedades diferenciáveis, $\phi:M\to N$ uma aplicação de classe $C^\infty$, $X$ um campo vetorial em $M$
e $Y$ um campo vetorial em $N$. Se $X$ e $Y$ são $\phi$-relacionados e $\gamma:I\to M$ é uma curva integral de $X$, mostre
que $\phi\circ\gamma$ é uma curva integral de $Y$.
\end{exercise}

\begin{exercise}[existência de curvas integrais]\label{exe:fluxolocal}
Sejam $M$ uma variedade diferenciável e $X$ um campo vetorial de classe $C^\infty$ em $M$. Dados $t_0\in\R$, $x_0\in M$, mostre que existem $\varepsilon>0$,
uma vizinhança aberta $V$ de $x_0$ em $M$ e uma função $\gamma:\left]t_0-\varepsilon,t_0+\varepsilon\right[\times V\to M$ de classe $C^\infty$ tal que
$\gamma(t_0,x)=x$ para todo $x\in V$ e tal que:
\[\left]t_0-\varepsilon,t_0+\varepsilon\right[\ni t\longmapsto\gamma(t,x)\in M\]
é uma curva integral de $X$, para todo $x\in V$ (sugestão: seja:
\[\varphi:U\subset M\longrightarrow\widetilde U\subset\R^n\]
uma carta em $M$ com $x_0\in U$
e use o item~(a) do Teorema~\ref{thm:EDO} para o campo vetorial $\varphi_*X$ no aberto $\widetilde U$ e o
resultado do Exercício~\ref{exe:curintphirel},
tendo em mente que $\varphi_*X$ e $X$ são $\varphi^{-1}$-relacionados).
\end{exercise}

\begin{exercise}[unicidade de curvas integrais]\label{exe:uniccurvainteg}
Sejam $M$ uma variedade diferenciável Hausdorff, $X$ um campo vetorial de classe $C^\infty$ em $M$ e:
\[\gamma:I\longrightarrow M,\quad\mu:J\longrightarrow M\]
curvas integrais de $X$. Se existe $t\in I\cap J$ tal que $\gamma(t)=\mu(t)$, mostre que $\gamma\vert_{I\cap J}=\mu\vert_{I\cap J}$ (sugestão: considere
o conjunto $C=\big\{t\in I\cap J:\gamma(t)=\mu(t)\big\}$ e mostre que $C$ é aberto e fechado em $I\cap J$. Para mostrar que $C$ é aberto, dado $t\in C$,
use uma carta $\varphi$ cujo domínio contém $\gamma(t)=\mu(t)$ e o item~(b) do Teorema~\ref{thm:EDO} para o campo vetorial $\varphi_*X$, tendo
em mente também o resultado do Exercício~\ref{exe:curintphirel}.
Para mostrar que $C$ é fechado em $I\cap J$, use a continuidade de $\gamma$ e $\mu$ e o fato que $M$ é Hausdorff).
\end{exercise}

\begin{exercise}
Sejam $A$ um subconjunto aberto de $\R$ e $M$ a variedade diferenciável definida no Exercício~\ref{exe:duasorigens}.
Consideramos a identificação:
\[T_{(x,i)}\big(\R\times\{0,1\}\big)\cong T_x\R\times T_i\{0,1\}\cong\R\times\{0\}\cong\R,\quad(x,i)\in\R\times\{0,1\},\]
de modo que campos vetoriais (resp., de classe $C^\infty$) em $\R\times\{0,1\}$ são identificados com funções (resp.,
de classe $C^\infty$) a valores reais em $\R\times\{0,1\}$.
\begin{itemize}
\item[(a)] Mostre que para todo $x\in A$ temos $\dd q_{(x,0)}=\dd q_{(x,1)}$ (sugestão: para $i\in\{0,1\}$,
a diferencial da aplicação $\lambda_i:\R\ni x\mapsto(x,i)\in\R\times\{0,1\}$ em qualquer ponto de $\R$ é a
aplicação identidade de $\R$. As aplicações $q\circ\lambda_0$ e $q\circ\lambda_1$ coincidem sobre o aberto $A$
e portanto suas diferenciais em qualquer ponto de $A$ são iguais).
\item[(b)] Mostre que um campo vetorial $X:\R\times\{0,1\}\to\R$ em $\R\times\{0,1\}$ é projetável por $q$ se e somente se $X(x,0)=X(x,1)$, para todo $x\in A$
(recorde Exercício~\ref{exe:phiprojetavel}; tenha em mente que $q$ é uma submersão sobrejetora de classe $C^\infty$
já que é até mesmo um difeomorfismo local sobrejetor).
\item[(c)] Considere o campo vetorial $X:\R\times\{0,1\}\ni(x,i)\mapsto1$ em $\R\times\{0,1\}$ e seja $Y$ o único campo vetorial (de classe $C^\infty$) em $M$ tal que
$X$ e $Y$ são $q$-relacionados. Use o resultado do Exercício~\ref{exe:curintphirel} para concluir que $\gamma_i:\R\ni t\mapsto q(t,i)\in M$, $i=0,1$,
é uma curva integral de $Y$. Note que $\gamma_0(t)=\gamma_1(t)$ quando $t\in A$ e $\gamma_0(t)\ne\gamma_1(t)$ quando $t\not\in A$, de modo que
se $A\ne\emptyset$ e $A\ne\R$ (justamente o caso em que $M$ não é Hausdorff!) então $\gamma_0$ e $\gamma_1$ coincidem em algum instante e não são iguais.
\end{itemize}
\end{exercise}

\begin{exercise}[invariância por translação temporal]\label{exe:transltemp}
Sejam $M$ uma variedade diferenciável e $X$ um campo vetorial em $M$. Se $\gamma:I\to M$ é uma curva integral de $X$ e
$s\in\R$, mostre que a {\em translação temporal\/} de $\gamma$ por $s$ definida por:
\[I-s\ni t\longmapsto\gamma(t+s)\in M\]
é uma curva integral de $X$, onde $I-s=\big\{t-s:t\in I\big\}$.
\end{exercise}

Dado um campo vetorial $X$ numa variedade diferenciável $M$, consideramos no conjunto das curvas integrais de $X$ a relação
de ordem parcial $\preceq$ definida por $\gamma\preceq\mu$ se e somente se $\mu$ é uma extensão de $\gamma$ (i.e., $\Dom(\gamma)\subset\Dom(\mu)$
e $\gamma$ é a restrição de $\mu$ a $\Dom(\gamma)$). Uma {\em curva integral maximal\/} de $X$ é um elemento maximal do conjunto das curvas integrais
de $X$, i.e., uma curva integral $\gamma$ de $X$ que não se estende a uma curva integral $\mu$ de $X$ tal que $\mu\ne\gamma$.

\begin{exercise}\label{exe:translcurintmax}
Se $X$ é um campo vetorial numa variedade diferenciável $M$, mostre que uma translação temporal de uma curva integral maximal de $X$
é uma curva integral maximal de $X$ (sugestão: use o resultado do Exercício~\ref{exe:transltemp}).
\end{exercise}

\begin{exercise}[existência e unicidade para curvas integrais maximais]\label{exe:maxintcurve}
Sejam $M$ uma variedade diferenciável Hausdorff e $X$ um campo vetorial de classe $C^\infty$ em $M$.
\begin{itemize}
\item[(a)] Se $\Gamma$ é um conjunto não vazio de curvas integrais de $X$ tal que existe $t$ em $\bigcap_{\gamma\in\Gamma}\Dom(\gamma)$ com
$\gamma(t)=\mu(t)$ para quaisquer $\gamma,\mu\in\Gamma$, mostre que a união\footnote{%
Estamos pensando aqui em uma função como um conjunto de pares ordenados. Note que se $\mathcal F$ é um conjunto de funções e se $f$ e $g$ coincidem
em $\Dom(f)\cap\Dom(g)$ para quaisquer $f,g\in\mathcal F$ então $\bigcup_{f\in\mathcal F}f$ é uma função cujo domínio é $\bigcup_{f\in\mathcal F}\Dom(f)$.}
$\bigcup_{\gamma\in\Gamma}\gamma$ é uma curva integral de $X$ (sugestão: use o resultado do Exercício~\ref{exe:uniccurvainteg}).
\item[(b)] Se $\gamma:I\to M$ e $\mu:J\to M$ são curvas integrais maximais de $X$ e existe $t\in I\cap J$ tal que $\gamma(t)=\mu(t)$, mostre que
$\gamma=\mu$ (sugestão: pelo resultado do item~(a), $\gamma\cup\mu$ é uma curva integral de $X$).
\item[(c)] Dados $t_0\in\R$, $x_0\in M$, mostre que o conjunto das curvas integrais $\gamma$ de $X$ tais que $t_0\in\Dom(\gamma)$
e $\gamma(t_0)=x_0$ possui um maior elemento, i.e., existe uma curva integral $\gamma$ de $X$ satisfazendo $t_0\in\Dom(\gamma)$, $\gamma(t_0)=x_0$ e que estende
qualquer curva integral de $X$ satisfazendo essa condição (sugestão: considere a união de todas as curvas integrais $\mu$ de $X$ tais que $t_0\in\Dom(\mu)$,
$\mu(t_0)=x_0$ e use o resultado do item~(a)). Mostre que $\gamma$ é a única curva integral maximal de $X$ satisfazendo $t_0\in\Dom(\gamma)$ e $\gamma(t_0)=x_0$.
Conclua que toda curva integral de $X$ se estende de modo único a uma curva integral maximal de $X$.
\end{itemize}
\end{exercise}

\begin{exercise}
Sejam $M$ uma variedade diferenciável Hausdorff e $X$ um campo vetorial de classe $C^\infty$ em $M$. Se $\gamma$ e $\mu$ são curvas integrais
maximais de $X$ e se as imagens de $\gamma$ e de $\mu$ não são disjuntas, mostre que $\mu$ é uma translação temporal de $\gamma$ e conclua
que $\gamma$ e $\mu$ possuem a mesma imagem (sugestão: use o resultado do item~(b) do Exercício~\ref{exe:maxintcurve} e o resultado
do Exercício~\ref{exe:translcurintmax}). Conclua também, usando o resultado do item~(c) do Exercício~\ref{exe:maxintcurve}, que as imagens das curvas
integrais maximais de $X$ constituem uma partição da variedade $M$.
\end{exercise}

\begin{exercise}
Sejam $M$ uma variedade diferenciável Hausdorff e $X$ um campo vetorial de classe $C^\infty$ em $M$. Seja $\gamma:\left]a,b\right[\to M$ uma curva
integral maximal de $X$ com $-\infty\le a<b<+\infty$. Se $(t_n)_{n\ge1}$ é uma seqüência em $\left]a,b\right[$ tal que $\lim_{n\to+\infty}t_n=b$,
mostre que a seqüência $\big(\gamma(t_n)\big)_{n\ge1}$ não converge em $M$ (sugestão: suponha por absurdo que:
\[\lim_{n\to+\infty}\gamma(t_n)=x_0\in M\]
e sejam $\varepsilon>0$, $V$ uma vizinhança aberta de $x_0$ em $M$ tais que para todo $x\in V$ existe uma curva integral
$\mu:\left]-\varepsilon,\varepsilon\right[\to M$
de $X$ tal que $\mu(0)=x$; a existência de $\varepsilon$ e $V$ é garantida pelo resultado do Exercício~\ref{exe:fluxolocal}. Seja $n\ge1$ tal que
$t_n>b-\varepsilon$ e tal que $\gamma(t_n)\in V$; sejam $\mu:\left]-\varepsilon,\varepsilon\right[\to M$ uma curva integral de $X$ tal que
$\mu(0)=\gamma(t_n)$ e $\lambda$ a translação temporal de $\mu$ tal que $\lambda(t_n)=\mu(0)$. Use o resultado do item~(a)
do Exercício~\ref{exe:maxintcurve} para concluir que $\gamma\cup\lambda$ é uma extensão
de $\gamma$ definida num intervalo que contém
$\left]a,t_n+\varepsilon\right[\varsupsetneq\left]a,b\right[$, o que contradiz
a maximalidade de $\gamma$).
Conclua que para todo subconjunto compacto $K$ de $M$ existe $\delta>0$ tal que $\gamma(t)\not\in K$ para todo $t\in\left]b-\delta,b\right[$ (isso
significa que $\lim_{t\to b}\gamma(t)$ é igual ao ponto no infinito do {\em compactificado de Alexandroff\/} de $M$).
Enuncie e prove resultados análogos para o caso em que $a>-\infty$.
\end{exercise}

\begin{defin}
Seja $X$ um campo vetorial de classe $C^\infty$ numa variedade diferenciável Hausdorff $M$. O {\em fluxo\/} (ou {\em fluxo máximo}) de $X$ é a função:
\[F^X:\Dom(F^X)\subset\R\times M\longrightarrow M\]
definida da seguinte forma: o domínio $\Dom(F^X)$ de $F^X$ é o conjunto dos pares $(t,x)\in\R\times M$ tais que $t$ pertence ao domínio da única curva integral
maximal $\gamma_x$ de $X$ tal que $\gamma_x(0)=x$; o valor $F^X(t,x)$ de $F^X$ no ponto $(t,x)$ é $\gamma_x(t)$.
\end{defin}

\begin{exercise}\label{exe:FsFtx}
Seja $X$ um campo vetorial de classe $C^\infty$ numa variedade diferenciável Hausdorff $M$. Dados $s,t\in\R$ e $x\in M$, mostre que se $(t,x)$ pertence a
$\Dom(F^X)$ então:
\[\big(s,F^X(t,x)\big)\in\Dom(F^X)\Longleftrightarrow(s+t,x)\in\Dom(F^X)\]
e que se uma dessas duas condições equivalentes for satisfeita então:
\[F^X\big(s,F^X(t,x)\big)=F^X(s+t,x)\]
(sugestão: se $\gamma$ é a curva integral maximal de $X$ tal que $\gamma(0)=x$ então, pelo resultado do Exercício~\ref{exe:translcurintmax}, a translação temporal
$\mu:u\mapsto\gamma(t+u)$ de $\gamma$ é a curva integral maximal de $X$ tal que $\mu(0)=F^X(t,x)$).
\end{exercise}

O objetivo dos próximos exercícios é demonstrar o seguinte:
\begin{teo}[do fluxo]\label{thm:fluxo}
Se $F^X$ é o fluxo de um campo vetorial $X$ de classe $C^\infty$ numa variedade diferenciável Hausdorff $M$ então o domínio de $F^X$ é aberto em $\R\times M$
e $F^X$ é de classe $C^\infty$.
\end{teo}

\begin{exercise}\label{exe:prepfluxo}
Seja $X$ um campo vetorial de classe $C^\infty$ numa variedade diferenciável Hausdorff $M$.
Seja $A\subset\R\times M$ a união de todos os subconjuntos abertos $B$ de $\R\times M$ tais que $B\subset\Dom(F^X)$ e $F^X\vert_B$ é de classe $C^\infty$.
Mostre que:
\begin{itemize}
\item[(a)] $A$ é um subconjunto aberto de $\R\times M$, $A\subset\Dom(F^X)$ e $F^X\vert_A$ é de classe $C^\infty$;
\item[(b)] $\{0\}\times M\subset A$ (sugestão: use o resultado do Exercício~\ref{exe:fluxolocal});
\item[(c)] dados $t,s\in\R$ e $x\in M$, se $(t,x)\in A$ e $\big(s,F^X(t,x)\big)\in A$ então $(s+t,x)\in A$ (sugestão:
seja:
\[B=\big\{(u,x)\in\R\times M:\text{$(t,x)\in A$ e $\big(u-t,F^X(t,x)\big)\in A$}\big\}.\]
Mostre que $B$ é aberto em $\R\times M$
e use o resultado do Exercício~\ref{exe:FsFtx} para mostrar que $B\subset\Dom(F^X)$ e que $F^X\vert_B$ é de classe $C^\infty$. Note que
$(s+t,x)\in B\subset A$).
\end{itemize}
\end{exercise}

\begin{exercise}
Demonstre o Teorema~\ref{thm:fluxo} seguindo os seguintes passos: seja $A$ como no enunciado do Exercício~\ref{exe:prepfluxo}. Você deve demonstrar que
$A=\Dom(F^X)$. Para isso, seja $x\in M$, denote por $I_x$ o domínio da curva integral maximal $\gamma:I_x\to M$ de $X$ com $\gamma(0)=x$ e seja:
\[J_x=\big\{t\in\R:(t,x)\in A\big\}\subset I_x.\]
Mostre que $J_x$ é aberto e fechado em $I_x$. Para mostrar que $J_x$ é fechado em $I_x$, seja $u\in I_x$ pertencente ao fecho de $J_x$. Sejam
$\varepsilon>0$ e $V$ uma vizinhança aberta de $\gamma(u)$ em $M$ tais que $\left]-\varepsilon,\varepsilon\right[\times V\subset A$; observe que
existe $t\in J_x$ tal que $\gamma(t)\in V$ e $\vert u-t\vert<\varepsilon$. Aplique o resultado do item~(c) do Exercício~\ref{exe:prepfluxo} com $s=u-t$
para concluir que $(u,x)\in A$.
\end{exercise}

Seja $X$ um campo vetorial de classe $C^\infty$ numa variedade diferenciável Hausdorff $M$. Para cada $t\in\R$, sejam:
\[\Dom(F^X_t)=\big\{x\in M:(t,x)\in\Dom(F)\big\}\]
e:
\[F^X_t:\Dom(F^X_t)\ni x\longmapsto F^X(t,x)\in M.\]
Pelo Teorema~\ref{thm:fluxo}, o domínio $\Dom(F^X_t)$ de $F^X_t$ é aberto em $M$ e a aplicação $F^X_t$ é de classe $C^\infty$.
O resultado do Exercício~\ref{exe:FsFtx} nos diz que, dados $s,t\in\R$ e $x\in\Dom(F^X_t)$ então:
\[F^X_t(x)\in\Dom(F^X_s)\Longleftrightarrow x\in\Dom(F^X_{t+s})\]
e se uma dessas duas condições equivalentes for satisfeita temos:
\begin{equation}\label{eq:FXtpluss}
F^X_{t+s}(x)=F^X_s\big(F^X_t(x)\big).
\end{equation}
Note que $\Dom(F^X_0)=M$ e que $F^X_0$ é a aplicação identidade de $M$. O campo vetorial $X$ é dito {\em completo\/} se toda curva integral maximal
de $X$ estiver definida em $\R$, i.e., se $\Dom(F^X)=\R\times M$. Nesse caso, as observações acima nos dizem
que o fluxo $F^X$ define uma ação de classe $C^\infty$ do grupo de Lie $(\R,+)$ na variedade diferenciável $M$.

\begin{exercise}\label{exe:Ftdifeo}
Seja $X$ um campo vetorial de classe $C^\infty$ numa variedade diferenciável Hausdorff $M$. Mostre que, para qualquer $t\in\R$, a imagem de $F^X_t$ é
igual ao domínio de $F^X_{-t}$ e que $F^X_t$, $F^X_{-t}$ são difeomorfismos mutuamente inversos de classe $C^\infty$.
\end{exercise}

\begin{defin}
Seja $M$ uma variedade diferenciável. Um {\em campo vetorial dependente do tempo\/} em $M$ é uma aplicação:
\[X:\Dom(X)\subset\R\times M\longrightarrow TM,\]
tal que $X(t,x)\in T_xM$ para todo $(t,x)\in\Dom(X)$, onde $\Dom(X)$ é um subconjunto aberto de $\R\times M$. Se $X$ é um campo vetorial
dependente do tempo em $M$ então uma {\em curva integral\/} de $X$ é uma aplicação $\gamma:I\to M$ de classe $C^\infty$
tal que $\big(t,\gamma(t)\big)\in\Dom(X)$ e:
\[\gamma'(t)=X\big(t,\gamma(t)\big),\]
para todo $t\in I$, onde $I\subset\R$ é um intervalo aberto. Como no caso em que $X$ é um campo vetorial comum (independente do tempo), uma
{\em curva integral maximal\/} de $X$ é um elemento maximal do conjunto das curvas integrais
de $X$, i.e., uma curva integral $\gamma$ de $X$ que não se estende a uma curva integral $\mu$ de $X$ tal que $\mu\ne\gamma$.
\end{defin}

\begin{exercise}\label{exe:XdeptimeX}
Sejam $M$ uma variedade diferenciável e $X$ um campo vetorial dependente do tempo de classe $C^\infty$ em $M$. Seja $\widetilde X$ o campo vetorial
na variedade $\Dom(X)$ (que é uma subvariedade aberta de $\R\times M$) definido por\footnote{%
Note que para mostrar que $\widetilde X:\Dom(X)\to T\big(\Dom(X)\big)\subset T(\R\times M)$ é de classe $C^\infty$ pode-se usar o resultado
do Exercício~\ref{exe:TMNTMTN}.}:
\begin{multline}\label{eq:widetildeX}
\widetilde X(t,x)=\big(1,X(t,x)\big)\in T_{(t,x)}(\R\times M)\cong\R\times T_xM,\\
(t,x)\in\Dom(X).
\end{multline}
Dado $(t_0,x_0)\in\Dom(X)$, mostre que se $\gamma:\Dom(\gamma)\to M$ é uma curva integral de $X$ satisfazendo $t_0\in\Dom(\gamma)$ e $\gamma(t_0)=x_0$ então:
\begin{equation}\label{eq:mutildeX}
\tilde\gamma:\Dom(\tilde\gamma)=\Dom(\gamma)-t_0\ni t\longmapsto(t+t_0,\gamma(t+t_0)\big)\in\R\times M
\end{equation}
é uma curva integral de $\widetilde X$ satisfazendo $0\in\Dom(\tilde\gamma)$ e $\tilde\gamma(0)=(t_0,x_0)$. Reciprocamente,
se $\tilde\gamma:\Dom(\tilde\gamma)\to\R\times M$ é uma curva
integral do campo vetorial $\widetilde X$ satisfazendo $0\in\Dom(\tilde\gamma)$ e $\tilde\gamma(0)=(t_0,x_0)$, mostre que
existe uma (obviamente única)
curva integral $\gamma:\Dom(\gamma)\to M$ de $X$ satisfazendo $t_0\in\Dom(\gamma)$ e $\gamma(t_0)=x_0$
tal que $\tilde\gamma$ é dada por \eqref{eq:mutildeX}.
\end{exercise}

\begin{exercise}\label{exe:maxintcurvetime}\
\begin{itemize}
\item[(a)] Mostre que o resultado do Exercício~\ref{exe:uniccurvainteg} vale se $X$ é um campo vetorial dependente do tempo de classe $C^\infty$ (sugestão:
use o resultado do Exercício~\ref{exe:XdeptimeX}).
\item[(b)] Mostre que o resultado do Exercício~\ref{exe:maxintcurve} vale se $X$ é um campo vetorial dependente do tempo de classe $C^\infty$,
onde no item~(c) daquele exercício deve-se acrescentar a hipótese de que $(t_0,x_0)$ pertença a $\Dom(X)$ (sugestão: siga o mesmo roteiro que você
usou para resolver o Exercício~\ref{exe:maxintcurve}. Para resolver o item~(c), você precisará usar o resultado do
Exercício~\ref{exe:XdeptimeX} para mostrar que existe uma curva integral $\mu$ de $X$ satisfazendo $\mu(t_0)=x_0$).
\item[(c)] Se $X$ é um campo vetorial dependente do tempo de classe $C^\infty$ numa variedade diferenciável Hausdorff $M$, $(t_0,x_0)\in\Dom(X)$
e $\gamma$ é a curva integral maximal de $X$ tal que $t_0\in\Dom(\gamma)$ e $\gamma(t_0)=x_0$, mostre que a curva $\tilde\gamma$
definida em \eqref{eq:mutildeX} é a curva integral maximal satisfazendo $\tilde\gamma(0)=(t_0,x_0)$ do campo vetorial $\widetilde X$ definido em
\eqref{eq:widetildeX}.
\end{itemize}
\end{exercise}

Diferentemente do que ocorre com campos vetoriais comuns (independentes do tempo), uma translação temporal (veja Exercício~\ref{exe:transltemp})
de uma curva integral de um campo vetorial dependente do tempo não é em geral uma curva integral desse campo. Em virtude disso, para definir
uma noção de fluxo de campos vetoriais dependentes do tempo que encapsule todas as curvas integrais maximais do campo, não é suficiente (como no
caso de campos vetoriais comuns) se restringir a condições iniciais da forma $\gamma(0)=x$.
\begin{defin}
Seja $X$ um campo vetorial dependente do tempo de classe $C^\infty$ numa variedade diferenciável Hausdorff $M$. O {\em fluxo\/} (ou {\em fluxo máximo}) de
$X$ é a função:
\[F^X:\Dom(F^X)\subset\R\times\R\times M\longrightarrow M\]
definida da seguinte forma: o domínio $\Dom(F^X)$ de $F^X$ é o conjunto das trincas $(t,t_0,x)\in\R\times\R\times M$ tais que $(t_0,x)\in\Dom(X)$
e $t$ pertence ao domínio da única curva integral maximal $\gamma_{(t_0,x)}$ de $X$ satisfazendo $\gamma_{(t_0,x)}(t_0)=x$; o valor $F^X(t,t_0,x)$ de $F^X$ no
ponto $(t,t_0,x)$ é $\gamma_{(t_0,x)}(t)$.
\end{defin}

\begin{exercise}[teorema do fluxo para campos dependentes do tempo]\label{exe:flowtimedep}
Seja $X$ um campo vetorial dependente do tempo de classe $C^\infty$ numa variedade diferenciável Hausdorff $M$. Se $\widetilde X$ é definido como em
\eqref{eq:widetildeX}, mostre que:
\[\Dom(F^X)=\big\{(t,t_0,x)\in\R\times\R\times M:\big(t-t_0,(t_0,x)\big)\in\Dom(F^{\widetilde X})\big\}\]
e que:
\[\big(t,F^X(t,t_0,x)\big)=F^{\widetilde X}\big(t-t_0,(t_0,x)\big),\]
para todo $(t,t_0,x)\in\Dom(F^X)$ (sugestão: use o resultado do item~(c) do Exercício~\ref{exe:maxintcurvetime}).
Conclua que $\Dom(F^X)$ é aberto em $\R\times\R\times M$ e que $F^X$ é de classe $C^\infty$.
\end{exercise}

\begin{exercise}
Seja $X$ um campo vetorial de classe $C^\infty$ numa variedade diferenciável Hausdorff $M$ e defina um campo vetorial dependente do tempo $X_0:\R\times M\to TM$
fazendo $X_0(t,x)=X(x)$, para todo $(t,x)\in\R\times M$. Mostre que:
\[\Dom(F^{X_0})=\big\{(t,t_0,x)\in\R\times\R\times M:(t-t_0,x)\in\Dom(F^X)\big\}\]
e que:
\[F^{X_0}(t,t_0,x)=F^X(t-t_0,x),\]
para todo $(t,t_0,x)\in\Dom(F^{X_0})$ (sugestão: $X$ e $X_0$ possuem as mesmas curvas integrais; use o resultado do Exercício~\ref{exe:translcurintmax}).
\end{exercise}

\begin{exercise}[generalização do resultado do Exercício~\ref{exe:FsFtx} para campos dependentes do tempo]\label{exe:FFtimedep}
Seja $X$ um campo vetorial dependente do tempo de classe $C^\infty$ numa variedade diferenciável Hausdorff $M$. Dados $t_0,t,s\in\R$ e dado $x\in M$,
mostre que se $(t,t_0,x)$ pertence a $\Dom(F^X)$ então:
\[\big(s,t,F^X(t,t_0,x)\big)\in\Dom(F^X)\Longleftrightarrow(s,t_0,x)\in\Dom(F^X)\]
e que se uma dessas duas condições equivalentes for satisfeita então:
\[F^X\big(s,t,F^X(t,t_0,x)\big)=F^X(s,t_0,x).\]
\end{exercise}

\medskip

Se $X$ é um campo vetorial dependente do tempo de classe $C^\infty$ numa variedade diferenciável Hausdorff $M$ então, para todos $t_0,t\in\R$, definimos:
\[F^X_{t,t_0}:\Dom(F^X_{t,t_0})\ni x\longmapsto F^X(t,t_0,x)\in M,\]
onde:
\[\Dom(F^X_{t,t_0})=\big\{x\in M:(t,t_0,x)\in\Dom(F^X)\big\}.\]
Segue do resultado do Exercício~\ref{exe:flowtimedep} que $\Dom(F^X_{t,t_0})$ é aberto em $M$ e que $F^X_{t,t_0}$ é de classe $C^\infty$.
O resultado do Exercício~\ref{exe:FFtimedep} nos diz que, dados $t_0,t,s\in\R$ e $x\in\Dom(F^X_{t,t_0})$ então:
\[F^X_{t,t_0}(x)\in\Dom(F^X_{s,t})\Longleftrightarrow x\in\Dom(F^X_{s,t_0})\]
e se uma dessas duas condições equivalentes for satisfeita temos:
\[F^X_{s,t_0}(x)=F^X_{s,t}\big(F^X_{t,t_0}(x)\big).\]
Note também que:
\[\Dom(F^X_{t_0,t_0})=\big\{x\in M:(t_0,x)\in\Dom(X)\big\}\]
e que $F^X_{t_0,t_0}$ é a aplicação identidade de $\Dom(F^X_{t_0,t_0})$.

\begin{exercise}[generalização do resultado do Exercício~\ref{exe:Ftdifeo} para campos dependentes do tempo]
Seja $X$ um campo vetorial dependente do tempo de classe $C^\infty$ numa variedade diferenciável Hausdorff $M$. Mostre que, para quaisquer $t_0,t\in\R$,
a imagem de $F^X_{t,t_0}$ é igual ao domínio de $F^X_{t_0,t}$ e que $F^X_{t,t_0}$, $F^X_{t_0,t}$ são difeomorfismos mutuamente inversos de classe $C^\infty$.
\end{exercise}

\begin{exercise}[teorema do fluxo com parâmetro]
Sejam $M$, $\Lambda$ variedades diferenciáveis Hausdorff\footnote{%
A conclusão do exercício (de que $F^X$ tem domínio aberto e é de classe $C^\infty$) pode na verdade ser demonstrada sem a hipótese de que $\Lambda$
seja Hausdorff, pois podemos escrever $\Lambda$ como uma união $\Lambda=\bigcup_{i\in I}\Lambda_i$ de abertos $\Lambda_i$
que são Hausdorff (domínios de cartas, por exemplo), considerar a restrição $X_i$ de $X$ a $\Dom(X)\cap(\R\times M\times\Lambda_i)$
e depois observar que $F^X=\bigcup_{i\in I}F^{X_i}$.} e:
\[X:\Dom(X)\subset\R\times M\times\Lambda\longrightarrow TM\]
uma aplicação de classe $C^\infty$ tal que:
\[X(t,x,\lambda)\in T_xM,\]
para todo $(t,x,\lambda)\in\Dom(X)$, onde $\Dom(X)$ é um subconjunto aberto de $\R\times M\times\Lambda$. Assim, para
cada $\lambda\in\Lambda$, temos que:
\[X_\lambda:\Dom(X_\lambda)\ni(t,x)\longmapsto X(t,x,\lambda)\in TM\]
é um campo vetorial dependente do tempo de classe $C^\infty$ em $M$, onde:
\[\Dom(X_\lambda)=\big\{(t,x)\in\R\times M:(t,x,\lambda)\in\Dom(X)\big\}.\]
Defina:
\[F^X:\Dom(F^X)\subset\R\times\R\times M\times\Lambda\longrightarrow M\]
fazendo:
\[F^X(t,t_0,x,\lambda)=F^{X_\lambda}(t,t_0,x),\]
para todo $(t,t_0,x,\lambda)\in\Dom(F^X)$, onde:
\[\Dom(F^X)=\big\{(t,t_0,x,\lambda)\in\R\times\R\times M\times\Lambda:(t,t_0,x)\in\Dom(F^{X_\lambda})\big\}.\]
Seja $X^\Lambda$ o campo vetorial dependente do tempo de classe $C^\infty$ na variedade $M\times\Lambda$ definido por:
\[X^\Lambda:\Dom(X)\ni(t,x,\lambda)\longmapsto\big(X(t,x,\lambda),0\big)\in T_{(x,\lambda)}(M\times\Lambda)\cong T_xM\times T_\lambda\Lambda.\]
Mostre que o domínio do fluxo de $X^\Lambda$ é $\Dom(F^X)$ e que:
\[F^{X^{\Lambda}}(t,t_0,x,\lambda)=\big(F^X(t,t_0,x,\lambda),\lambda\big),\]
para todo $(t,t_0,x,\lambda)\in\Dom(F^X)$ (sugestão: as curvas integrais de $X^\Lambda$ são precisamente as curvas da forma
$t\mapsto\big(\gamma(t),\lambda)$, onde $\lambda\in\Lambda$ e $\gamma$ é uma curva integral de $X_\lambda$).
Conclua que $\Dom(F^X)$ é aberto em $\R\times\R\times M\times\Lambda$ e que $F^X$ é de classe $C^\infty$.
\end{exercise}

\end{section}

\begin{section}{Dia 08/11}

\begin{exercise}[generalização do resultado do Exercício~\ref{exe:reprsecao} para seções com parâmetros]\label{exe:smoothsecpar}
Sejam $\pi:E\to M$ um fibrado vetorial de posto $k$, $\Lambda$ uma variedade diferenciável e:
\[s:\Dom(s)\subset\Lambda\times M\longrightarrow E\]
uma função tal que $s(\lambda,x)\in E_x$, para todo $(\lambda,x)\in\Dom(s)$, onde $\Dom(s)$ é um subconjunto aberto de $\Lambda\times M$.
Se $\mathfrak A$ é um atlas de trivializações de $E$ (contido no maximal), mostre que $s$ é de classe $C^\infty$ se e somente se a aplicação:
\[\Dom(s)\cap(\Lambda\times U)\ni(\lambda,x)\longmapsto\alpha_x^{-1}\big(s(\lambda,x)\big)\in\R^k\]
é de classe $C^\infty$, para qualquer $\alpha=(\alpha_x)_{x\in U}\in\mathfrak A$ (sugestão:
considere a composição de $s$ com o difeomorfismo \eqref{eq:tildealpha}).
\end{exercise}

\begin{exercise}[generalização do resultado do Exercício~\ref{exe:secaoFEsmooth} para seções com parâmetros]\label{exe:secaoFEsmoothpar}
Sejam $\mathfrak F$ um funtor de classe $C^\infty$, $\pi:E\to M$ um fibrado vetorial de posto $k$ e $\Lambda$ uma variedade diferenciável.
Seja:
\[\epsilon:\Dom(\epsilon)\subset\Lambda\times M\longrightarrow\mathfrak F(E)\]
uma função tal que $\epsilon(\lambda,x)\in\mathfrak F(E_x)$, para todo $(\lambda,x)\in\Dom(\epsilon)$, onde $\Dom(\epsilon)$ é um subconjunto
aberto de $\Lambda\times M$.
Dado um atlas de trivializações $\mathfrak A$ de $E$ (contido no maximal), mostre que $\epsilon$ é de classe $C^\infty$ se e somente se
para qualquer $\alpha=(\alpha_x)_{x\in U}\in\mathfrak A$ a aplicação:
\begin{equation}\label{eq:Falphalambda}
\Dom(\epsilon)\cap(\Lambda\times U)\ni(\lambda,x)\longmapsto\mathfrak F(\alpha_x)^{-1}\big(\epsilon(\lambda,x)\big)\in\mathfrak F(\R^k)
\end{equation}
é de classe $C^\infty$ (sugestão: considere a composição de $\epsilon$ com o difeomorfismo \eqref{eq:Ftildealpha}).
\end{exercise}

\begin{exercise}[generalização do resultado do Exercício~\ref{exe:crittensorsmooth} para campos tensoriais com parâmetros]\label{exe:critensmoothpar}
Sejam $M$, $\Lambda$ variedades diferenciáveis e:
\[\tau:\Dom(\tau)\subset\Lambda\times M\longrightarrow\Big(\bigotimes_pTM^*\Big)\otimes\Big(\bigotimes_qTM\Big)\]
uma função tal que $\tau(\lambda,x)\in\big(\bigotimes_pT_xM^*\big)\otimes\big(\bigotimes_qT_xM\big)$, para todo $(\lambda,x)$ em $\Dom(\tau)$,
onde $\Dom(\tau)$ é um aberto de $\Lambda\times M$. Para cada $\lambda\in\Lambda$, denote por:
\[\tau_\lambda:\Dom(\tau_\lambda)\ni x\longmapsto\tau(\lambda,x)\in\big(\bigotimes_pTM^*\big)\otimes\big(\bigotimes_qTM\big)\]
o campo tensorial definido sobre o aberto:
\[\Dom(\tau_\lambda)=\big\{x\in M:(\lambda,x)\in\Dom(\tau)\big\}.\]
Seja $\mathcal A$ um atlas de $M$ (contido no maximal). Mostre que a função $\tau$ é de classe $C^\infty$ se e somente se
a função:
\begin{equation}\label{eq:varphitlambda}
(\Id\times\varphi)\big[\Dom(\tau)\cap\big(\Lambda\times\Dom(\varphi)\big)\big]\ni(\lambda,u)\longmapsto(\varphi_*\tau_\lambda)(u)
\end{equation}
é de classe $C^\infty$, para qualquer $\varphi\in\mathcal A$, onde $\Id$ denota a aplicação identidade de $\Lambda$ (sugestão: use o resultado
do Exercício~\ref{exe:secaoFEsmoothpar} com $E=TM$ e com $\mathfrak A=\big\{\alpha^\varphi:\varphi\in\mathcal A\big\}$, onde $\alpha^\varphi$ é a trivialização
local de $TM$ associada à carta $\varphi$. Se $\epsilon=\tau$ e $\alpha=\alpha^\varphi$, verifique que a aplicação \eqref{eq:Falphalambda} é
igual à composição de \eqref{eq:varphitlambda} com $\Id\times\varphi$).
\end{exercise}

\begin{exercise}[Exercício~\ref{exe:pullbacksmoothmanifold} com parâmetros]\label{exe:pullbackparametros}
Sejam $\Lambda$, $\Theta$, $M$, $N$ variedades diferenciáveis, seja $\phi:\Dom(\phi)\subset\Lambda\times M\to N$ uma função de classe $C^\infty$
definida num subconjunto aberto $\Dom(\phi)$ de $\Lambda\times M$ e seja:
\[\tau:\Dom(\tau)\subset\Theta\times N\longrightarrow\Big(\bigotimes_pTN^*\Big)\otimes\Big(\bigotimes_qTN\Big)\]
uma função de classe $C^\infty$ definida num subconjunto aberto $\Dom(\tau)$ de $\Theta\times N$ tal que
$\tau(\theta,y)\in\big(\bigotimes_pT_yN^*\big)\otimes\big(\bigotimes_qT_yN\big)$, para todo $(\theta,y)\in\Dom(\tau)$.
Para cada $\lambda\in\Lambda$ considere a aplicação de classe $C^\infty$:
\[\phi_\lambda:\Dom(\phi_\lambda)\ni x\longmapsto\phi(\lambda,x)\in N\]
definida sobre o aberto:
\[\Dom(\phi_\lambda)=\big\{x\in M:(\lambda,x)\in\Dom(\phi)\big\}\]
e para cada $\theta\in\Theta$ considere o campo tensorial de classe $C^\infty$:
\[\tau_\theta:\Dom(\tau_\theta)\ni y\longmapsto\tau(\theta,y)\in\Big(\bigotimes_pTN^*\Big)\otimes\Big(\bigotimes_qTN\Big)\]
definido sobre o aberto:
\[\Dom(\tau_\theta)=\big\{y\in N:(\theta,y)\in\Dom(\tau)\big\}.\]
Se $q\ne0$, assuma que $\phi_\lambda$ é um difeomorfismo local, para todo $\lambda\in\Lambda$.
Mostre que a aplicação:
\[(\lambda,\theta,x)\longmapsto(\phi_\lambda^*\tau_\theta)(x)\in\Big(\bigotimes_pTM^*\Big)\otimes\Big(\bigotimes_qTM\Big)\]
definida sobre o aberto:
\[\big\{(\lambda,\theta,x)\in\Lambda\times\Theta\times M:\text{$(\lambda,x)\in\Dom(\phi)$ e $\big(\theta,\phi(\lambda,x)\big)\in\Dom(\tau)$}\big\}\]
é de classe $C^\infty$ (sugestão: use cartas nas variedades $M$, $N$ e o resultado do Exercício~\ref{exe:critensmoothpar} --- com $\Lambda\times\Theta$
no lugar de $\Lambda$ --- para reduzir o problema ao caso em que
$M$ é um aberto de $\R^m$ e $N$ é um aberto de $\R^n$; evidentemente, $m=n$ se $q\ne0$. O caso em que $M$ é um aberto de $\R^m$ e $N$ é um aberto de $\R^n$
pode ser resolvido de modo similar ao Exercício~\ref{exe:pullbacksmoothvecspa}).
\end{exercise}

\begin{defin}
Se $E$ é um fibrado vetorial sobre uma variedade diferenciável $M$ então uma {\em seção dependente do tempo\/} de $E$ é uma aplicação:
\[s:\Dom(s)\subset\R\times M\longrightarrow E\]
tal que $s(t,x)\in E_x$, para todo $(t,x)\in\Dom(s)$, onde $\Dom(s)$ é um subconjunto aberto de $\R\times M$. Para cada $t\in\R$,
temos que a aplicação:
\[s_t:\Dom(s_t)\ni x\longmapsto s(t,x)\in E\]
é uma seção local de $E$ definida sobre o aberto:
\[\Dom(s_t)=\big\{x\in M:(t,x)\in\Dom(s)\big\}.\]
\end{defin}
Se $s$ é uma seção dependente do tempo de classe $C^\infty$ de $E$, denotamos por $\frac{\dd}{\dd t}s_t$ a seção local de $E$ definida por:
\[\frac{\dd}{\dd t}s_t:\Dom(s_t)\ni x\longmapsto\frac{\dd}{\dd t}[s_t(x)]\in E.\]
Tenha em mente que, para $x\in M$, a aplicação $t\mapsto s_t(x)$ (definida num subconjunto aberto de $\R$) é uma curva de classe $C^\infty$
no espaço vetorial real de dimensão finita $E_x$ (usamos aqui o resultado do Exercício~\ref{exe:fibramergulhada}),
de modo que sua derivada num dado instante é um elemento de $E_x$.

\begin{exercise}\label{exe:ddtst}
Se $s$ é uma seção dependente do tempo de classe $C^\infty$ de um fibrado vetorial $E$ de posto $k$ sobre uma variedade $M$, mostre que:
\[\Dom(s)\ni(t,x)\longmapsto\Big(\frac{\dd}{\dd t}s_t\Big)(x)=\frac{\dd}{\dd t}[s_t(x)]\in E\]
é uma seção dependente do tempo de classe $C^\infty$ de $E$ (sugestão: use o resultado do Exercício~\ref{exe:smoothsecpar} tendo em mente que,
se $\alpha$ é uma trivialização local de $E$ então:
\begin{equation}\label{eq:alphaddt}
\alpha_x^{-1}\Big(\frac{\dd}{\dd t}[s_t(x)]\Big)=\frac{\dd}{\dd t}\big[\alpha_x^{-1}\big(s_t(x)\big)\big]\in\R^k,
\end{equation}
para todo $(t,x)\in\big(\R\times\Dom(\alpha)\big)\cap\Dom(s)$).
\end{exercise}

\begin{exercise}\label{exe:ddtcomutXft}
Sejam $X$ um campo vetorial dependente do tempo de classe $C^\infty$
numa variedade diferenciável $M$ e $f\in C^\infty(M)$. Para cada $t\in\R$, considere o campo vetorial:
\[X_t:\Dom(X_t)\ni x\longmapsto X(t,x)\in TM\]
definido sobre o aberto:
\[\Dom(X_t)=\big\{x\in M:(t,x)\in\Dom(X)\big\}.\]
Mostre que a função:
\[\Dom(X)\ni(t,x)\longmapsto\big(X_t(f)\big)(x)\in\R\]
é de classe $C^\infty$ e que:
\[\frac{\dd}{\dd t}\big(X_t(f)\big)=\Big(\frac{\dd}{\dd t}X_t\Big)(f).\]
\end{exercise}

\begin{exercise}\label{exe:ddtXft}
Sejam $M$ uma variedade diferenciável, $X$ um campo vetorial de classe $C^\infty$ em $M$ e $f:\Dom(f)\subset\R\times M\to\R$ uma função de classe $C^\infty$
definida num subconjunto aberto $\Dom(f)$ de $\R\times M$; para cada $t\in\R$, denote por $f_t$ a função $f_t:\Dom(f_t)\ni x\mapsto f(t,x)\in\R$ definida
sobre o aberto:
\[\Dom(f_t)=\big\{x\in M:(t,x)\in\Dom(f)\big\}.\]
\begin{itemize}
\item[(a)] Mostre que a função $\Dom(f)\ni(t,x)\mapsto X(f_t)(x)\in\R$ é de classe $C^\infty$ (sugestão: uma opção é simplesmente usar uma carta em $M$.
Outra opção é usar o resultado do item~(d) do Exercício~\ref{exe:propXf}, notando que $X(f_t)(x)=X_0(f)(t,x)$, onde $X_0$ é o campo vetorial em $\R\times M$ definido
por $X_0(t,x)=\big(0,X(x)\big)$, para todo $(t,x)\in\R\times M$).
\item[(b)] Mostre que $\frac{\dd}{\dd t}[X(f_t)]=X\big(\frac{\dd}{\dd t}f_t\big)$ (sugestão: use uma carta em $M$ para reduzir o problema
ao caso em que $M$ é um aberto de $\R^n$ e nesse caso use o teorema de Schwarz).
\end{itemize}
\end{exercise}

\begin{exercise}\label{exe:dddtomegat}
Sejam $M$ uma variedade diferenciável e $\omega$ uma $k$-forma diferencial dependente do tempo de classe $C^\infty$ em $M$, i.e., $\omega$ é uma seção
dependente do tempo de classe $C^\infty$ do fibrado vetorial $\bigwedge_kTM^*$. Para cada $t\in\R$, denote por $\omega_t$ a $k$-forma diferencial:
\[\omega_t:\Dom(\omega_t)\ni x\longmapsto\omega(t,x)\in\bigwedge_kTM^*\]
definida sobre o aberto:
\[\Dom(\omega_t)=\big\{x\in M:(t,x)\in\Dom(\omega)\big\}.\]
Mostre que:
\[\frac{\dd}{\dd t}(\dd\omega_t)=\dd\Big(\frac{\dd}{\dd t}\,\omega_t\Big),\]
onde $\dd$ denota diferenciação exterior (sugestão: use uma carta em $M$ para reduzir o problema
ao caso em que $M$ é um aberto de $\R^n$ e nesse caso use o teorema de Schwarz).
\end{exercise}

\begin{exercise}\label{exe:ddtpullback}
Sejam $M$, $N$ variedades diferenciáveis, $\phi:M\to N$ uma aplicação de classe $C^\infty$ e $\tau$ um $(p,q)$-campo tensorial
dependente do tempo de classe $C^\infty$ em $N$, i.e., $\tau$ é uma seção dependente do tempo de classe $C^\infty$
do fibrado vetorial $\big(\bigotimes_pTN^*\big)\otimes\big(\bigotimes_qTN\big)$. Para cada $t\in\R$, denote por $\tau_t$ o
$(p,q)$-campo tensorial:
\[\tau_t:\Dom(\tau_t)\ni x\longmapsto\tau(t,x)\in\Big(\bigotimes_pTN^*\Big)\otimes\Big(\bigotimes_qTN\Big)\]
definido sobre o aberto:
\[\Dom(\tau_t)=\big\{x\in N:(t,x)\in\Dom(\tau)\big\}.\]
Se $q\ne0$, assuma que $\phi$ é um difeomorfismo local.
Mostre que:
\[\frac{\dd}{\dd t}(\phi^*\tau_t)=\phi^*\Big(\frac{\dd}{\dd t}\,\tau_t\Big)\]
(sugestão: para $x\in M$, a expressão $\dd\phi(x)^*\big[\tau_t\big(\phi(x)\big)\big]$ é linear em $\tau_t\big(\phi(x)\big)$).
\end{exercise}

Sejam $M$ uma variedade diferenciável Hausdorff, $X$ um campo vetorial de classe $C^\infty$ em $M$ e $\tau$ um $(p,q)$-campo tensorial
de classe $C^\infty$ sobre $M$. Em vista do teorema do fluxo (Teorema~\ref{thm:fluxo}) e do resultado do Exercício~\ref{exe:pullbackparametros},
temos que:
\[\R\times M\supset\Dom(F^X)\ni(t,x)\longmapsto[(F^X_t)^*\tau](x)\in\Big(\bigotimes_pTM^*\Big)\otimes\Big(\bigotimes_qTM\Big)\]
é um $(p,q)$-campo tensorial dependente do tempo de classe $C^\infty$ em $M$ (note também que, em virtude do resultado do Exercício~\ref{exe:Ftdifeo},
$F^X_t$ é um difeomorfismo, de modo que o pull-back por $F^X_t$ está sempre bem definido). Assim, em vista do resultado do Exercício~\ref{exe:ddtst},
temos que:
\[\R\times M\supset\Dom(F^X)\ni(t,x)\longmapsto\Big[\frac{\dd}{\dd t}[(F^X_t)^*\tau]\Big](x)\]
é também um $(p,q)$-campo tensorial dependente do tempo de classe $C^\infty$ em $M$. A {\em derivada de Lie\/} $\mathbb L_X\tau$ é o
$(p,q)$-campo tensorial de classe $C^\infty$ em $M$ definido por:
\[\mathbb L_X\tau=\left.\frac{\dd}{\dd t}[(F^X_t)^*\tau]\right\vert_{t=0},\]
i.e., o valor de $\mathbb L_X\tau$ num ponto $x$ de $M$ é o valor de $\frac{\dd}{\dd t}[(F^X_t)^*\tau]$ no ponto $x$ em $t=0$.

\begin{exercise}
Sejam $M$ uma variedade diferenciável Hausdorff, $X$ um campo vetorial de classe $C^\infty$ em $M$ e $\tau$ um $(p,q)$-campo tensorial
de classe $C^\infty$ em $M$. Suponha que $p=0$ ou $q=0$. Mostre que se $\tau$ é simétrico então $\mathbb L_X\tau$ também
é simétrico e que se $\tau$ é anti-simétrico então também $\mathbb L_X\tau$ é anti-simétrico (sugestão: os tensores simétricos e os tensores anti-simétricos
num espaço vetorial fixado constituem subespaços do espaço de todos os tensores e a derivada de uma curva num subespaço está nesse subespaço).
\end{exercise}

\begin{exercise}[derivada de Lie de uma função escalar]\label{exe:derLieescalar}
Sejam $M$ uma variedade diferenciável Hausdorff, $X$ um campo vetorial de classe $C^\infty$ em $M$ e $f:M\to\R$ uma função de classe $C^\infty$
(que pode ser pensada como um $(0,0)$-campo tensorial em $M$). Mostre que:
\[\mathbb L_Xf=X(f)\]
(sugestão: tenha em mente a Observação~\ref{thm:obspullback00}).
\end{exercise}

\begin{exercise}[derivada de Lie de um campo vetorial]\label{exe:derLievecfield}
Sejam $M$ uma variedade diferenciável Hausdorff e $X$, $Y$ campos vetoriais de classe $C^\infty$ em $M$ (pensamos em $Y$ como um $(0,1)$-campo tensorial
em $M$). Mostre que:
\[\mathbb L_XY=[X,Y]\]
(sugestão: em primeiro lugar, use o resultado do Exercício~\ref{exe:ddtcomutXft} para concluir que, para $f\in C^\infty(M)$, temos:
\[\Big[\frac{\dd}{\dd t}\big((F^X_t)^*Y\big)\Big](f)=\frac{\dd}{\dd t}\big[\big((F^X_t)^*Y\big)(f)\big].\]
Depois, use o resultado do Exercício~\ref{exe:phirelder} --- tendo em mente que $(F^X_t)^*Y$ e $Y$ são $F^X_t$-relacionados --- e também
o resultado do Exercício~\ref{exe:Ftdifeo} para obter:
\begin{equation}\label{eq:FXtYf}
\big((F^X_t)^*Y\big)(f)=\big(Y(f\circ F^X_{-t})\big)\circ F^X_t.
\end{equation}
Para calcular a derivada com respeito a $t$ da expressão que está no lado direito da igualdade \eqref{eq:FXtYf}, observe que o valor dessa expressão
num ponto $x\in M$ é $\phi(t,t)$, onde:
\[\phi(t_1,t_2)=\big[\big(Y(f\circ F^X_{-t_1})\big)\circ F^X_{t_2}\big](x).\]
O resultado do item~(a) do Exercício~\ref{exe:ddtXft} garante que $\phi$ é de classe $C^\infty$ (o domínio de $\phi$ é um subconjunto aberto de $\R^2$
que contém os pares $(t,t)$ tais que $x$ está no domínio de $F^X_t$).
Tenha em mente que:
\[\left.\frac{\dd}{\dd t}\phi(t,t)\right\vert_{t=0}=\frac{\partial\phi}{\partial t_1}(0,0)+\frac{\partial\phi}{\partial t_2}(0,0).\]
O resultado do item~(b) do Exercício~\ref{exe:ddtXft}
nos ensina a calcular $\frac{\partial\phi}{\partial t_1}(0,0)$).
\end{exercise}

\begin{exercise}\label{exe:Liederlint}
Sejam $M$ uma variedade diferenciável Hausdorff e $X$ um campo vetorial de classe $C^\infty$ em $M$. Mostre que a aplicação
$\tau\mapsto\mathbb L_X\tau$ é um operador linear no espaço vetorial real dos $(p,q)$-campos tensoriais em $M$ de classe $C^\infty$.
\end{exercise}

\begin{exercise}[regras do produto para derivada de Lie]\label{exe:Lieregraprod}
Sejam $M$ uma variedade diferenciável Hausdorff e $X$ um campo vetorial de classe $C^\infty$ em $M$.
\begin{itemize}
\item[(a)] Se $\tau$ é um $(p,q)$-campo tensorial de classe $C^\infty$ em $M$ e $\tau'$ é um $(p',q')$-campo tensorial de classe
$C^\infty$ em $M$, mostre que:
\[\mathbb L_X(\tau\otimes\tau')=(\mathbb L_X\tau)\otimes\tau'+\tau\otimes(\mathbb L_X\tau').\]
Se $p=p'=0$ ou $q=q'=0$ e $\tau$, $\tau'$ são simétricos, mostre que:
\[\mathbb L_X(\tau\vee\tau')=(\mathbb L_X\tau)\vee\tau'+\tau\vee(\mathbb L_X\tau')\]
e se $p=p'=0$ ou $q=q'=0$ e $\tau$, $\tau'$ são anti-simétricos, mostre que:
\[\mathbb L_X(\tau\wedge\tau')=(\mathbb L_X\tau)\wedge\tau'+\tau\wedge(\mathbb L_X\tau')\]
(sugestão: use o resultado do Exercício~\ref{exe:pullbacktensfield} e o fato que as operações $\otimes$, $\vee$ e $\wedge$ em espaços
de tensores num espaço vetorial fixado são bilineares).
\item[(b)] Se $\tau$ é um $(p,q)$-campo tensorial de classe $C^\infty$ em $M$, $Y_1$, \dots, $Y_p$ são campos vetoriais de classe $C^\infty$ em $M$
e $\alpha_1$, \dots, $\alpha_q$ são $1$-formas de classe $C^\infty$ em $M$, mostre que:
\begin{align*}
\mathbb L_X\big(\tau(Y_1,\ldots,Y_p,&\,\alpha_1,\ldots,\alpha_q)\big)=(\mathbb L_X\tau)(Y_1,\ldots,Y_p,\alpha_1,\ldots,\alpha_q)\\
&+\sum_{i=1}^p\tau(Y_1,\ldots,Y_{i-1},\mathbb L_XY_i,Y_{i+1},\ldots,Y_p,\alpha_1,\ldots,\alpha_q)\\
&+\sum_{i=1}^q\tau(Y_1,\ldots,Y_p,\alpha_1,\ldots,\alpha_{i-1},\mathbb L_X\alpha_i,\alpha_{i+1},\ldots,\alpha_q)
\end{align*}
(sugestão: use o resultado do Exercício~\ref{exe:pushaval}, mais o fato que a operação de avaliação de tensores em vetores e funcionais lineares
de um espaço vetorial fixado é multilinear).
\item[(c)] Se $\tau$ é um campo tensorial puramente covariante de classe $C^\infty$ em $M$ e $Y$ é um campo vetorial de classe $C^\infty$ em $M$, mostre que:
\[\mathbb L_X(i_Y\tau)=i_{(\mathbb L_XY)}\tau+i_Y\mathbb L_X\tau\]
(sugestão: use o resultado do Exercício~\ref{exe:pulliXt}, mais o fato que o produto interior define uma operação bilinear entre vetores e tensores de um espaço
vetorial fixado).
\end{itemize}
\end{exercise}

\begin{exercise}\label{exe:formulasderLie}
Sejam $M$ uma variedade diferenciável Hausdorff, $X$ um campo vetorial de classe $C^\infty$ em $M$ e $\tau$ um $(p,q)$-campo tensorial
de classe $C^\infty$ em $M$. Sejam também dados campos vetoriais $Y_1$, \dots, $Y_p$ de classe $C^\infty$ em $M$ e $1$-formas $\alpha_1$, \dots, $\alpha_q$
de classe $C^\infty$ em $M$. Use os resultados dos Exercícios~\ref{exe:derLieescalar}, \ref{exe:derLievecfield} e do item~(b) do Exercício~\ref{exe:Lieregraprod}
para escrever uma fórmula para
$(\mathbb L_X\tau)(Y_1,\ldots,Y_p,\alpha_1,\ldots,\alpha_q)$ em termos da operação introduzida na Definição~\ref{thm:defXdef}, dos colchetes
de Lie $[X,Y_i]$ e das derivadas de Lie $\mathbb L_X\alpha_i$. Observe também que se $Y$ é um campo vetorial de classe $C^\infty$ em $M$ então você
também pode escrever uma fórmula para $(\mathbb L_X\alpha_i)(Y)$ em termos da operação introduzida na Definição~\ref{thm:defXdef} e do colchete de Lie
$[X,Y]$.
\end{exercise}

\begin{exercise}
Sejam $M$ uma variedade diferenciável Hausdorff e $\tau$ um $(p,q)$-campo tensorial de classe $C^\infty$ em $M$. Mostre que a aplicação:
\[\mathfrak X(M)\ni X\longmapsto\mathbb L_X\tau\]
é linear sobre $\R$ (sugestão: não tente usar a definição de derivada de Lie! Use as fórmulas que você obteve no Exercício~\ref{exe:formulasderLie}).
\end{exercise}

\begin{exercise}\label{exe:Lieextdif}
Sejam $M$ uma variedade diferenciável Hausdorff, $X$ um campo vetorial de classe $C^\infty$ em $M$ e $\omega$ uma $k$-forma diferencial de classe $C^\infty$ em $M$.
Mostre que:
\[\mathbb L_X\dd\omega=\dd\mathbb L_X\omega,\]
onde $\dd$ denota diferenciação exterior (sugestão: use o resultado do item~(e) do Exercício~\ref{exe:propdifextvar} e o resultado
do Exercício~\ref{exe:dddtomegat}).
\end{exercise}

\begin{exercise}\label{exe:derLiepullback}
Sejam $M$, $N$ variedades diferenciáveis Hausdorff, $X$ um campo vetorial de classe $C^\infty$ em $M$, $Y$ um campo vetorial de classe $C^\infty$ em $N$,
$\phi:M\to N$ uma aplicação de classe $C^\infty$ e $\tau$ um $(p,q)$-campo tensorial de classe $C^\infty$ em $N$. Assuma que $X$ e $Y$ sejam $\phi$-relacionados
e, se $q\ne0$, assuma que $\phi$ seja um difeomorfismo local. Mostre que:
\[\phi^*\mathbb L_Y\tau=\mathbb L_X(\phi^*\tau)\]
(sugestão: use o resultado do Exercício~\ref{exe:curintphirel} para concluir que $F^Y_t\circ\phi$ e $\phi\circ F^X_t$ coincidem sobre o domínio de $F^X_t$.
Use também os resultados dos Exercícios~\ref{exe:ddtpullback} e \ref{exe:pullbackcompostafield}).
\end{exercise}

\begin{rem}\label{thm:obsderLieaberto}
Um caso particular notável do resultado do Exercício~\ref{exe:derLiepullback} ocorre quando $\phi$ é a aplicação inclusão de um subconjunto aberto.
Nesse caso, o resultado do exercício nos diz que a derivada de Lie comuta com restrições a abertos, i.e.:
\[\mathbb L_{X\vert_U}(\tau\vert_U)=(\mathbb L_X\tau)\vert_U,\]
onde $X$ é um campo vetorial de classe $C^\infty$ numa variedade diferenciável Hausdorff $M$, $\tau$ é um $(p,q)$-campo tensorial de classe $C^\infty$ em $M$
e $U$ é um subconjunto aberto de $M$.
\end{rem}

\begin{exercise}[derivada de Lie de formas diferenciais]\label{exe:diid}
Sejam $M$ uma variedade diferenciável Hausdorff, $\omega$ uma $k$-forma de classe $C^\infty$ em $M$ e $X$ um campo vetorial de classe $C^\infty$ em $M$. Demonstre a fórmula:
\[\mathbb L_X\omega=\dd i_X\omega+i_X\dd\omega\]
seguindo o seguinte roteiro:
\begin{itemize}
\item[(a)] se a fórmula vale para $\omega_1$ e $\omega_2$ então ela vale para $\omega_1+\omega_2$ (sugestão: use o resultado do Exercício~\ref{exe:Liederlint}
e do item~(b) do Exercício~\ref{exe:propdifextvar});
\item[(b)] se a fórmula vale para $\omega_1$ e $\omega_2$ então ela vale para $\omega_1\wedge\omega_2$ (sugestão: use o resultado do item~(a)
do Exercício~\ref{exe:Lieregraprod}, do item~(d) do Exercício~\ref{exe:propdifextvar} e do item~(b) do Exercício~\ref{exe:propiXomega});
\item[(c)] se a fórmula vale para uma certa $\omega$ então ela vale também para $\dd\omega$ (sugestão: use o resultado do Exercício~\ref{exe:Lieextdif}
e o resultado do item~(c) do Exercício~\ref{exe:propdifextvar});
\item[(d)] a fórmula vale se $k=0$ (sugestão: use o resultado do Exercício~\ref{exe:derLieescalar} e o resultado do item~(a) do Exercício~\ref{exe:propdifextvar});
\item[(e)] conclua que a fórmula vale em geral, observando que podemos trocar $M$ por um domínio de carta e que, num domínio de carta, qualquer forma diferencial
de classe $C^\infty$ é obtida a partir de $0$-formas de classe $C^\infty$ usando diferenciação exterior, produto exterior e somas (sugestão:
veja \eqref{eq:formascoordenadas} e tenha em mente que o produto de uma função escalar por uma forma diferencial é um caso particular do produto exterior.
Tenha em mente também a Observação~\ref{thm:obsderLieaberto}).
\end{itemize}
\end{exercise}

\begin{exercise}[fórmula de Cartan]
Sejam $M$ uma variedade diferenciável, $\omega$ uma $k$-forma de classe $C^\infty$ em $M$ e $X_1$, \dots, $X_{k+1}$ campos vetoriais de classe $C^\infty$
em $M$. Mostre que:
\begin{multline*}
\dd\omega(X_1,\ldots,X_{k+1})=\sum_{i=1}^{k+1}(-1)^{i+1}X_i\big(\omega(X_1,\ldots,\widehat{X_i},\ldots,X_{k+1})\big)\\
+\sum_{1\le i<j\le k+1}(-1)^{i+j}\,\omega\big([X_i,X_j],X_1,\ldots,\widehat{X_i},\ldots,\widehat{X_j},\ldots,X_{k+1}\big),
\end{multline*}
onde o chapéu indica que o termo foi omitido da seqüência (sugestão: podemos trocar $M$ por um domínio de carta, de modo que não há perda de generalidade
em se assumir que $M$ é Hausdorff. Use indução em $k$ e a fórmula para derivada de Lie de formas diferenciais que você provou no Exercício~\ref{exe:diid}
para expressar $i_{X_1}\dd\omega$ em termos de $\dd i_{X_1}\omega$ e $\mathbb L_{X_1}\omega$. Use também a fórmula para derivada de Lie que você obteve
no Exercício~\ref{exe:formulasderLie}).
\end{exercise}

\begin{exercise}\label{exe:ddtFXtstar}
Sejam $M$ uma variedade diferenciável Hausdorff, $X$ um campo vetorial de classe $C^\infty$ em $M$ e $\tau$ um $(p,q)$-campo tensorial de classe
$C^\infty$ em $M$. Mostre que:
\[\frac{\dd}{\dd t}[(F^X_t)^*\tau]=(F^X_t)^*\mathbb L_X\tau\]
(sugestão: em primeiro lugar, observe que:
\[\left.\frac{\dd}{\dd t}(F^X_t)^*\tau\,\right\vert_{t=t_0}=\left.\frac{\dd}{\dd t}(F^X_{t_0+t})^*\tau\,\right\vert_{t=0}\]
e que, por \eqref{eq:FXtpluss}, $F^X_{t_0+t}$ e $F^X_t\circ F^X_{t_0}$ coincidem sobre o domínio de $F^X_{t_0}$. Use também os resultados dos
Exercícios~\ref{exe:pullbackcompostafield} e \ref{exe:ddtpullback}).
\end{exercise}

\begin{exercise}
Sejam $M$ uma variedade diferenciável Hausdorff, $X$ um campo vetorial de classe $C^\infty$ em $M$ e $\tau$ um $(p,q)$-campo tensorial de classe
$C^\infty$ em $M$. Dizemos que $\tau$ é {\em invariante pelo fluxo de $X$\/} se $(F^X_t)^*\tau$ e $\tau$ coincidem sobre o domínio
de $F^X_t$, para todo $t\in\R$. Mostre que $\tau$ é invariante pelo fluxo de $X$ se e somente se $\mathbb L_X\tau=0$ (sugestão:
use o resultado do Exercício~\ref{exe:ddtFXtstar} e tenha em mente que $F^X_0$ é a aplicação identidade de $M$).
\end{exercise}

\noindent\lower15pt\hbox{\dbend}\hspace{10pt}\parbox[t]{330pt}{\begin{rem}
Na verdade, o desenvolvimento da teoria da derivada de Lie não depende da hipótese de que a variedade seja Hausdorff. Em primeiro lugar, o teorema
do fluxo não é realmente necessário, pois para se definir a derivada de Lie $\mathbb L_X\tau$ poderíamos ter usado, em vez do fluxo $F^X_t$, uma aplicação
$F_t$ tal que $F_0(x)=x$ e $\left.\frac{\dd}{\dd t}\big(F_t(x)\big)\right\vert_{t=0}=X(x)$, para todo $x\in M$. Em segundo lugar, poderíamos usar as
fórmulas que você obteve no Exercício~\ref{exe:formulasderLie} para definir a derivada de Lie. E, finalmente, como toda variedade diferenciável é localmente Hausdorff, podemos
definir a derivada de Lie (usando o fluxo, por exemplo) nos abertos que são Hausdorff e aí, observando que as derivadas de Lie obtidas nesses abertos
coincidem sobre as interseções dos mesmos, definir a derivada de Lie na variedade toda.
\end{rem}}

\end{section}

\begin{section}{Dia 10/11}

\begin{exercise}
Seja $\Phi:W\to\Lin(\R^m,\R^n)$ uma função de classe $C^\infty$ definida num subconjunto aberto $W$ de $\R^m\times\R^n$. Assuma que exista uma função
$f:U\to\R^n$ de classe $C^\infty$ num aberto $U$ de $\R^m$ tal que $\big(x,f(x)\big)\in W$ para todo $x\in U$ e tal que:
\[\dd f(x)=\Phi\big(x,f(x)\big),\]
para todo $x\in U$. Mostre que, para qualquer $(x,y)\in\R^m\times\R^n$ pertencente ao gráfico de $f$, vale que a aplicação bilinear:
\[\R^m\times\R^m\ni(v,w)\longmapsto\frac{\partial\Phi}{\partial x}(x,y)(v,w)+\frac{\partial\Phi}{\partial y}(x,y)\big(\Phi(x,y)v,w\big)\in\R^n\]
é simétrica.
\end{exercise}

\end{section}

\appendix

\begin{section}{A reta longa}

Neste apêndice pressupomos que o leitor tem uma familiaridade mínima com ordinais.

Seja $(X,{\le})$ um conjunto totalmente ordenado, i.e., a relação $\le$ é reflexiva, anti-simétrica, transitiva e quaisquer dois elementos de $X$ são comparáveis.
Como é usual, escrevemos $a<b$ quando $a\le b$ e $a\ne b$; escrevemos também $a\ge b$ (resp., $a>b$) quando $b\le a$ (resp., $b<a$).
Um subconjunto de $X$ é dito um {\em intervalo aberto\/} se for vazio, igual a $X$, ou de uma das formas:
\[\big\{x\in X:x<b\big\},\quad\big\{x\in X:x>a\big\},\quad\big\{x\in X:a<x<b\big\},\]
com $a,b\in X$. É fácil ver que a interseção de dois intervalos abertos é um intervalo aberto e portanto os intervalos abertos em $X$ constituem uma base
para uma topologia em $X$, chamada a {\em topologia da ordem}.

\begin{exercise}
Mostre que a topologia usual da reta $\R$ coincide com a topologia definida pela ordem usual.
\end{exercise}

\begin{exercise}\label{exe:subordenado}
Se $X$ é um conjunto totalmente ordenado e $Y$ é um subconjunto de $X$ então a ordem total de $X$ induz uma ordem total em $Y$. Infelizmente,
a topologia da ordem em $Y$ pode não coincidir com a topologia induzida por $X$ em $Y$ (por exemplo, seja $X=\R$, munido da ordem usual, e
$Y=\{0\}\cup\big\{1+\frac1n:n=1,2,\ldots\big\}$; note que $\{0\}$ é aberto em $Y$ com respeito à topologia induzida por $X$, mas não é aberto
com respeito à topologia da ordem em $Y$). Mostre que a topologia induzida por $X$ em $Y$ é mais fina do que a topologia da ordem em $Y$ e que
as duas topologias coincidem se $Y$ for um intervalo aberto em $X$ (cuidado: o resultado não é tão imediato quanto parece, pois não é evidente
{\it a priori\/} que um intervalo aberto de $X$ que esteja contido em $Y$ seja um intervalo aberto relativamente a $Y$).
\end{exercise}

\begin{exercise}
Mostre que a topologia da ordem num conjunto totalmente ordenado é sempre Hausdorff.
\end{exercise}

\begin{exercise}\label{exe:homeocresc}
Se $X$, $Y$ são conjuntos totalmente ordenados, munidos das respectivas topologias da ordem, e se $h:X\to Y$ é uma bijeção {\em crescente\/} (i.e.,
$x_1\le x_2$ implica $h(x_1)\le h(x_2)$, para quaisquer $x_1,x_2\in X$), mostre que $h$ é um homeomorfismo.
\end{exercise}

Considere o primeiro ordinal não enumerável $\aleph_1$ e o produto cartesiano $\aleph_1\times\left[0,1\right[$ munido da {\em ordem lexicográfica\/}
definida por:
\[(\alpha,t)\le(\beta,s)\Longleftrightarrow\text{$\alpha<\beta$ ou $(\text{$\alpha=\beta$ e $t\le s$})$},\]
para todos $\alpha,\beta\in\aleph_1$, $t,s\in\left[0,1\right[$. Considere o conjunto:
\[L=\big(\aleph_1\times\left[0,1\right[\big)\setminus\{(0,0)\}\]
obtido de $\aleph_1\times\left[0,1\right[$ pela remoção do ponto inicial $(0,0)$. O conjunto $L$, munido da ordem induzida por $\aleph_1\times\left[0,1\right[$
e da correspondente topologia da ordem é a {\em reta longa}.

\begin{exercise}
Mostre que a reta longa $L$ não possui base enumerável de abertos e não é nem mesmo separável, i.e., não admite subconjunto enumerável denso (sugestão:
considere os abertos $\{\alpha\}\times\left]0,1\right[\subset L$, $\alpha\in\aleph_1$).
\end{exercise}

O objetivo deste apêndice é mostrar que existe um atlas diferenciável em $L$ que define a topologia
de $L$, fazendo de $L$ uma variedade diferenciável Hausdorff de dimensão $1$ conexa, porém sem base enumerável de abertos (segue então de resultados padrão de topologia
geral que $L$ não é nem paracompacta nem metrizável).

A estratégia a ser usada é a seguinte: para cada ordinal não nulo $\alpha\in\aleph_1$, consideramos o intervalo aberto $I_\alpha$ formado por todos os elementos
de $L$ que são menores do que $(\alpha,0)$, i.e.:
\[I_\alpha=\big\{(\beta,t)\in L:\beta<\alpha\big\}.\]
Vamos construir por recursão transfinita uma família de bijeções crescentes:
\[h_\alpha:I_\alpha\longrightarrow\left]-\infty,0\right[,\quad\alpha\in\aleph_1\setminus\{0\},\]
de modo que os sistemas de coordenadas $h_\alpha$, $\alpha\in\aleph_1\setminus\{0\}$, sejam dois a dois compatíveis. Como:
\[L=\bigcup_{\substack{\alpha\in\aleph_1\\\alpha\ne0}}I_\alpha,\]
obteremos então que $\big\{h_\alpha:\alpha\in\aleph_1,\ \alpha\ne0\big\}$ é um atlas diferenciável em $L$. Como $I_\alpha$ é aberto
em $L$ e, em vista dos resultados dos Exercícios~\ref{exe:subordenado} e \ref{exe:homeocresc}, $h_\alpha$ é um homeomorfismo,
segue que esse atlas define a topologia de $L$. Note que, dados ordinais $\alpha$, $\beta$ com $0<\alpha<\beta<\aleph_1$ então, como $I_\alpha\subset I_\beta$,
o domínio da função de transição $h_\beta\circ h_\alpha^{-1}$ é todo o intervalo $\left]-\infty,0\right[$; além do mais, como $h_\beta$ é uma bijeção crescente,
a imagem $h_\beta(I_\alpha)$ da função de transição $h_\beta\circ h_\alpha^{-1}$
é o intervalo $\left]-\infty,h_\beta(\alpha,0)\right[\varsubsetneq\left]-\infty,0\right[$. Assim, a função de transição $h_\beta\circ h_\alpha^{-1}$
é um homeomorfismo crescente:
\[h_\beta\circ h_\alpha^{-1}:\left]-\infty,0\right[\longrightarrow\left]-\infty,h_\beta(\alpha,0)\right[\varsubsetneq\left]-\infty,0\right[.\]
A compatibilidade entre $h_\alpha$ e $h_\beta$ significa que esse homeomorfismo é também um difeomorfismo de classe $C^\infty$.
Observe que, dados ordinais $\alpha$, $\beta$, $\gamma$
com $0<\alpha\le\beta\le\gamma<\aleph_1$ então, já que o domínio $I_\beta$ de $h_\beta$ contém o domínio $I_\alpha$ de $h_\alpha$, segue que se $h_\alpha$
é compatível com $h_\beta$ e $h_\beta$ é compatível com $h_\gamma$ então $h_\alpha$ é compatível com $h_\gamma$.

\begin{exercise}[base da recursão]
Mostre que existe uma bijeção crescente $h_1:I_1\to\left]-\infty,0\right[$ (sugestão: note que $I_1=\{0\}\times\left]0,1\right[$).
\end{exercise}

\begin{exercise}[passo da recursão]
Seja $\alpha\in\aleph_1$, $\alpha\ne0$, um ordinal e seja dada uma bijeção crescente $h_\alpha:I_\alpha\to\left]-\infty,0\right[$. Denote
por $\alpha+1$ o sucessor de $\alpha$ e defina $h_{\alpha+1}:I_{\alpha+1}\to\left]-\infty,0\right[$ fazendo:
\[h_{\alpha+1}(\beta,t)=h_\alpha(\beta,t)-1,\]
se $(\beta,t)\in I_\alpha$ e:
\[h_{\alpha+1}(\beta,t)=t-1,\]
se $(\alpha,0)\le(\beta,t)<(\alpha+1,0)$ (i.e., se $\beta=\alpha$ e $t\in\left[0,1\right[$). Mostre que $h_{\alpha+1}$ é uma bijeção crescente compatível
com $h_\alpha$. Conclua que, se são dadas bijeções crescentes $h_\beta:I_\beta\to\left]-\infty,0\right[$ para $0<\beta<\alpha$, todas compatíveis
com $h_\alpha$, então todas são compatíveis com $h_{\alpha+1}$ também.
\end{exercise}

Para completar a construção por recursão das aplicações $h_\alpha$, precisamos mostrar o seguinte:
\begin{lem}\label{thm:lemaordlim}
Dado um ordinal limite $\alpha$ com $0<\alpha<\aleph_1$, e dadas bijeções crescentes $h_\beta:I_\beta\to\left]-\infty,0\right[$, $0<\beta<\alpha$,
duas a duas compatíveis, então existe uma bijeção crescente $h_\alpha:I_\alpha\to\left]-\infty,0\right[$ que é compatível com $h_\beta$, para todo $\beta$
com $0<\beta<\alpha$.
\end{lem}

A demonstração do Lema~\ref{thm:lemaordlim} depende de outros resultados, que são deixados como exercício.

\begin{exercise}\label{exe:cofomega}
Seja $\alpha$ um ordinal limite com $0<\alpha<\aleph_1$. Mostre que existe uma seqüência $(\alpha_n)_{n\ge0}$ de ordinais tal que
$\alpha_n<\alpha_{n+1}$, para todo $n\ge0$, e tal que $\sup_{n\ge0}\alpha_n=\alpha$ (sugestão: seja $n\mapsto\beta_n$ uma enumeração
do ordinal $\alpha$ e defina $\alpha_{n+1}$ como sendo o sucessor do máximo entre $\alpha_n$ e $\beta_n$).
\end{exercise}

\begin{exercise}\label{exe:extenddifeo}
Seja $u:\left]-\infty,a\right[\to\left]-\infty,b\right[$ um difeomorfismo crescente de classe $C^\infty$ com $a<0$. Dados $\bar a<a$ e $r>u(\bar a)$, mostre que existe
um difeomorfismo crescente $v$ de classe $C^\infty$ de $\left]-\infty,0\right[$ sobre um intervalo da forma $\left]-\infty,\bar b\right[$
que coincide com $u$ em $\left]-\infty,\bar a\right]$
e tal que $v(a)=r$ (sugestão: seja $m$ tal que $\bar a<m<a$ e tal que $u(m)<r$. Considere\footnote{%
Se você não sabe construir uma função assim, dê uma olhada nos Exercícios~\ref{exe:exp1x} e \ref{exe:corteRn}.}
uma função $\lambda:\left]-\infty,0\right[\to[0,1]$
de classe $C^\infty$ que vale $1$ em $\left]-\infty,\bar a\right]$ e vale $0$ em $\left[m,0\right[$. Escolha $k>0$ e defina $w$ sobre o intervalo
$\left]-\infty,0\right[$ fazendo $w=u'\lambda+k(1-\lambda)$, onde $u'$ denota a derivada de $u$ e entende-se que o produto $u'\lambda$ vale $0$ onde $u$ não está
definida. Tome $v$ como sendo a primitiva de $w$ tal que $v(\bar a)=u(\bar a)$. Mostre que é possível escolher $k>0$ de modo que $v(a)=r$).
\end{exercise}

\begin{exercise}\label{exe:colagemintervalos}
Para cada inteiro $n\ge0$, seja $\phi_n:\left]-\infty,0\right[\to\left]-\infty,a_n\right[$ um difeomorfismo crescente de classe $C^\infty$, onde
$a_n<0$. Mostre que:
\begin{itemize}
\item[(a)] existe uma seqüência $(g_n)_{n\ge1}$, onde $g_n$ é um difeomorfismo crescente de classe $C^\infty$ de $\left]-\infty,0\right[$ sobre
um intervalo da forma $\left]-\infty,b_n\right[$, tal que $g_{n+1}\circ\phi_n$ coincide com $g_n$ sobre $\left]-\infty,a_{n-1}\right]$, para todo $n\ge1$
e tal que $\sup_{n\ge1}g_n(a_{n-1})=0$
(sugestão: obtenha $g_{n+1}$ a partir de $g_n$ usando o resultado do Exercício~\ref{exe:extenddifeo} com $u=g_n\circ\phi_n^{-1}$ e $\bar a=\phi_n(a_{n-1})$);
\item[(b)] existe uma seqüência $(f_n)_{n\ge0}$, onde $f_n$ é um difeomorfismo crescente de classe $C^\infty$ de $\left]-\infty,0\right[$ sobre
um intervalo da forma $\left]-\infty,c_n\right[$, tal que $f_{n+1}\circ\phi_n=f_n$, para todo $n\ge0$, e tal que $\sup_{n\ge0}c_n=0$
(sugestão: tome $g_n$ como no item~(a) e defina $f_n=g_{n+1}\circ\phi_n$).
\end{itemize}
\end{exercise}

\begin{exercise}
Demonstre o Lema~\ref{thm:lemaordlim} seguindo o seguinte roteiro: seja $(\alpha_n)_{n\ge0}$ uma seqüência de ordinais como no Exercício~\ref{exe:cofomega} e defina:
\[\phi_n=h_{\alpha_{n+1}}\circ h_{\alpha_n}^{-1},\quad n\ge0.\]
Sejam $f_n$, $n\ge0$, como no item~(b) do Exercício~\ref{exe:colagemintervalos} e defina $h_\alpha$ fazendo:
\[h_\alpha\vert_{I_{\alpha_n}}=f_n\circ h_{\alpha_n},\quad n\ge0.\]
Conclua a demonstração provando que $h_\alpha:I_\alpha\to\left]-\infty,0\right[$ está bem definida, é uma bijeção crescente e é compatível com $h_\beta$,
para todo $\beta$ com $0<\beta<\alpha$ (sugestão: basta mostrar a compatibilidade de $h_\alpha$ com $h_{\alpha_n}$, para todo $n$).
\end{exercise}

\begin{exercise}
Mostre que a reta longa $L$ é conexa (sugestão: dois pontos de $L$ pertencem a $I_\alpha$, para algum $\alpha\in\aleph_1\setminus\{0\}$).
\end{exercise}

\end{section}

\begin{section}{A Grassmanniana}\label{sec:Grassmanniana}

Sejam $E$ um espaço vetorial real de dimensão $n<+\infty$ e $k$ um número natural menor ou igual a $n$. A {\em Grassmanniana\/}
$G_k(E)$ é o conjunto de todos os subespaços $k$-dimensionais de $E$. O objetivo deste apêndice é construir um atlas
maximal em $G_k(E)$ que faça da {\em ação canônica}:
\begin{equation}\label{eq:acaoGLEGkE}
\GL(E)\times G_k(E)\ni(A,W)\longmapsto A(W)\in G_k(E)
\end{equation}
do grupo linear geral $\GL(E)$ (i.e., o grupo dos isomorfismos lineares de $E$) em $G_k(E)$
uma aplicação de classe $C^\infty$.

Dada uma decomposição em soma direta $E=E_0\oplus E_1$ então, para cada transformação linear $T:E_0\to E_1$, o gráfico de $T$:
\[\Gr(T)=\big\{x+T(x):x\in E_0\big\}\]
é um subespaço de $E$ que tem a mesma dimensão que $E_0$ (já que a aplicação $x\mapsto x+T(x)$ define um isomorfismo de $E_0$ sobre $\Gr(T)$). Denotando
por $\Lin(E_0,E_1)$ o espaço vetorial das transformações lineares de $E_0$ para $E_1$ e assumindo $\Dim(E_0)=k$, temos que a aplicação:
\begin{equation}\label{eq:TGrT}
\Lin(E_0,E_1)\ni T\longmapsto\Gr(T)\in G_k(E)
\end{equation}
é injetiva.

\begin{exercise}\label{exe:condGrT}
Sejam $E$ um espaço vetorial e $E=E_0\oplus E_1$ uma decomposição em soma direta de $E$. Denote por $\pi_0:E\to E_0$, $\pi_1:E\to E_1$ as projeções
correspondentes. Dado um subespaço $W$ de $E$, mostre que as seguintes afirmações
são equivalentes:
\begin{itemize}
\item[(a)] $E=W\oplus E_1$;
\item[(b)] $\pi_0\vert_W:W\to E_0$ é um isomorfismo;
\item[(c)] existe uma transformação linear $T:E_0\to E_1$ tal que $W=\Gr(T)$.
\end{itemize}
Para demonstrar que (b) implica (c), mostre que $T$ é dada por:
\[T=(\pi_1\vert_W)\circ(\pi_0\vert_W)^{-1}.\]
\end{exercise}

Em vista do resultado do Exercício~\ref{exe:condGrT}, a imagem da aplicação injetiva \eqref{eq:TGrT} é o subconjunto:
\[G_k(E;E_1)=\big\{W\in G_k(E):E=W\oplus E_1\big\}\]
da Grassmanniana $G_k(E)$. A inversa de \eqref{eq:TGrT} define então uma bijeção:
\[\phi_{E_0,E_1}:G_k(E;E_1)\longrightarrow\Lin(E_0,E_1)\]
dada por:
\[\phi_{E_0,E_1}(W)=T,\]
onde $W=\Gr(T)$, $T\in\Lin(E_0,E_1)$. Nós definiremos um atlas em $G_k(E)$ considerando todas as aplicações $\phi_{E_0,E_1}$, onde $E_0$, $E_1$ são subespaços
de $E$ tais que $E=E_0\oplus E_1$ e $\Dim(E_0)=k$. Estritamente falando, $\phi_{E_0,E_1}$ não é um sistema de coordenadas em $G_k(E)$, já que não toma
valores num espaço $\R^N$; para se obter um sistema de coordenadas em $G_k(E)$, é necessário escolher um isomorfismo entre $\Lin(E_0,E_1)$
e $\R^N$ (aqui $N=\Dim(E_0)\Dim(E_1)=k(n-k)$) e compô-lo com $\phi_{E_0,E_1}$. No entanto, é mais elegante e prático obter um atlas maximal em $G_k(E)$
usando o resultado do Exercício~\ref{exe:manifoldcharts}, que nos permite usar ``sistemas de coordenadas'' a valores em outras variedades diferenciáveis
(que podem ser, por exemplo, espaços vetoriais reais de dimensão finita) para definir um atlas maximal em um conjunto.

Nós precisamos mostrar que, dadas decomposições $E=E_0\oplus E_1=E'_0\oplus E'_1$, com $\Dim(E_0)=\Dim(E'_0)=k$, então a função de transição
$\phi_{E'_0,E'_1}\circ\phi_{E_0,E_1}^{-1}$ é um difeomorfismo de classe $C^\infty$ de um aberto de $\Lin(E_0,E_1)$ sobre um aberto de $\Lin(E'_0,E'_1)$.
Nosso plano é o seguinte: sejam $W\in G_k(E;E_1)$ e $T=\phi_{E_0,E_1}(W)$, de modo que $W=\Gr(T)$. Dado um isomorfismo linear $A:E\to E$, nós escreveremos,
em termos de $T$ e $A$, uma condição necessária e suficiente para que $A(W)$ pertença ao domínio $G_k(E;E'_1)$ de $\phi_{E'_0,E'_1}$ e nós escreveremos
uma fórmula explícita para $\phi_{E'_0,E'_1}\big(A(W)\big)$ em termos de $A$ e $T$. Isso nos permitirá resolver dois problemas ao mesmo tempo: tomando
$A$ igual à aplicação identidade de $E$, obteremos uma expressão explícita para a função de transição de $\phi_{E_0,E_1}$ para $\phi_{E'_0,E'_1}$
(que nos permitirá mostrar que ela é um difeomorfismo de classe $C^\infty$ entre abertos); considerando $A\in\GL(E)$ arbitrário, nós obteremos também
uma prova de que a ação canônica \eqref{eq:acaoGLEGkE} é de classe $C^\infty$.

Denote por $\pi'_0:E\to E'_0$, $\pi'_1:E\to E'_1$ as projeções correspondentes à decomposição $E=E'_0\oplus E'_1$.
Em vista do resultado do Exercício~\ref{exe:condGrT}, $A(W)$ pertence a $G_k(E;E'_1)$ se e somente se $\pi'_0$ leva
$A(W)$ isomorficamente sobre $E'_0$; nesse caso, a aplicação $T'=\phi_{E'_0,E'_1}\big(A(W)\big)$ é dada por:
\[T'=(\pi'_1\vert_{A(W)})\circ(\pi'_0\vert_{A(W)})^{-1}.\]
Considere o isomorfismo $\mathfrak j:E_0\to A(W)$ definido por $\mathfrak j(x)=A\big(x+T(x)\big)$, $x\in E_0$.
Temos então que $A(W)$ pertence
a $G_k(E;E'_1)$ se e somente se $(\pi'_0\vert_{A(W)})\circ\mathfrak j:E_0\to E'_0$ é um isomorfismo e, nesse caso,
$T'$ é dada por:
\[T'=[(\pi'_1\vert_{A(W)})\circ\mathfrak j]\circ[(\pi'_0\vert_{A(W)})\circ\mathfrak j]^{-1}.\]
Para descrever $T'$ de forma mais explícita, consideramos a decomposição do operador linear $A:E\to E$ nas componentes:
\[A_{00}:E_0\longrightarrow E'_0,\quad A_{01}:E_1\longrightarrow E'_0,\quad
A_{10}:E_0\longrightarrow E'_1,\quad A_{11}:E_1\longrightarrow E'_1,\]
definidas por:
\[A_{00}=\pi'_0\circ A\vert_{E_0},\quad A_{01}=\pi'_0\circ A\vert_{E_1},\quad
A_{10}=\pi'_1\circ A\vert_{E_0},\quad A_{11}=\pi'_1\circ A\vert_{E_1}.\]
Se representamos $A$ por uma matriz, usando bases em $E$ compatíveis com as decomposições $E=E_0\oplus E_1$
e $E=E'_0\oplus E'_1$, então as representações matriciais das aplicações lineares $A_{ij}$, $i,j=0,1$ correspondem
precisamente a blocos da matriz que representa $A$, como esquematizado abaixo:
\[A=\begin{pmatrix}A_{00}&A_{01}\\A_{10}&A_{11}\end{pmatrix}.\]
É fácil ver que:
\[(\pi'_1\vert_{A(W)})\circ\mathfrak j=A_{10}+A_{11}\circ T,\quad
(\pi'_0\vert_{A(W)})\circ\mathfrak j=A_{00}+A_{01}\circ T.\]
Mostramos então que, para $W=\Gr(T)$, $T\in\Lin(E_0,E_1)$, temos que:
\[A(W)\in G_k(E;E'_1)\Longleftrightarrow\text{$A_{00}+A_{01}\circ T$ é invertível}\]
e que:
\[\phi_{E'_0,E'_1}\big(A(W)\big)=(A_{10}+A_{11}\circ T)\circ(A_{00}+A_{01}\circ T)^{-1}.\]
Já que o conjunto dos isomorfismos lineares é aberto no conjunto de todas as aplicações lineares
e as operações de soma, composição e inversão de aplicações lineares são de classe $C^\infty$, vê-se que
o conjunto:
\[\big\{T\in\Lin(E_0,E_1):A\big(\!\Gr(T)\big)\in G_k(E;E'_1)\big\}\]
é aberto em $\Lin(E_0,E_1)$ e que a função:
\[T\longmapsto\phi_{E'_0,E'_1}\big[A\big(\!\Gr(T)\big)\big]\in\Lin(E'_0,E'_1)\]
definida sobre esse conjunto é de classe $C^\infty$. Tomando $A$ igual à aplicação identidade de $E$, vemos em particular
que a função de transição:
\[\phi_{E'_0,E'_1}\circ\phi_{E_0,E_1}^{-1}\]
tem domínio aberto e é de classe $C^\infty$
(o mesmo vale, obviamente, para sua inversa, de modo que essa função de transição é então um difeomorfismo de classe
$C^\infty$ entre abertos). Como os domínios $G_k(E;E_1)$ das funções $\phi_{E_0,E_1}$ obviamente cobrem $G_k(E)$
(já que todo subespaço vetorial de $E$ é um somando direto), nós provamos a seguinte:
\begin{prop}\label{thm:propGrassm}
Existe um único atlas maximal na Grassmanniana $G_k(E)$ que faz de $G_k(E;E_1)$ um aberto de $G_k(E)$ e da aplicação
\[\phi_{E_0,E_1}:G_k(E;E_1)\longrightarrow\Lin(E_0,E_1)\]
um difeomorfismo de classe $C^\infty$, para qualquer decomposição
$E=E_0\oplus E_1$, com $\Dim(E_0)=k$. A dimensão de $G_k(E)$ é $k(n-k)$, onde $n=\Dim(E)$.\qed
\end{prop}

As aplicações $\GL(E)\ni A\longmapsto A_{ij}\in\Lin(E_j,E'_i)$, $i,j=0,1$, são (restrições ao aberto $\GL(E)$ de)
aplicações lineares e portanto são de classe $C^\infty$. Segue então que o conjunto:
\[\big\{(A,T)\in\GL(E)\times\Lin(E_0,E_1):A\big(\!\Gr(T)\big)\in G_k(E;E'_1)\big\}\]
é aberto em $\GL(E)\times\Lin(E_0,E_1)$ e que a função:
\[(A,T)\longmapsto\phi_{E'_0,E'_1}\big[A\big(\!\Gr(T)\big)\big]\in\Lin(E'_0,E'_1)\]
definida nesse conjunto é de classe $C^\infty$. Isso prova também a seguinte:
\begin{prop}
Se $G_k(E)$ é munido do atlas maximal dado pela Proposição~\ref{thm:propGrassm}, então a ação canônica
\eqref{eq:acaoGLEGkE} de $\GL(E)$ em $G_k(E)$ é uma aplicação de classe $C^\infty$.\qed
\end{prop}

\begin{exercise}\label{exe:LemaGrassHauss}
Seja $E$ um espaço vetorial real de dimensão finita e sejam $E=E_0\oplus E_1$, $E=E'_0\oplus E'_1$
decomposições de $E$ com $\Dim(E_0)=\Dim(E'_0)$. Mostre que o conjunto:
\[\big\{(T,T')\in\Lin(E_0,E_1)\times\Lin(E'_0,E'_1):\Gr(T)=\Gr(T')\big\}\]
é fechado em $\Lin(E_0,E_1)\times\Lin(E'_0,E'_1)$ (sugestão: esse conjunto é igual a:
\[\bigcap_{x\in E_0}\big\{(T,T')\in\Lin(E_0,E_1)\times\Lin(E'_0,E'_1):x+T(x)\in\Gr(T')\big\},\]
sendo que, para $(T,T')\in\Lin(E_0,E_1)\times\Lin(E'_0,E'_1)$ e $x\in E_0$, temos:
\[x+T(x)\in\Gr(T')\Longleftrightarrow\pi'_1\big(x+T(x)\big)=T'\big[\pi'_0\big(x+T(x)\big)\big],\]
onde $\pi'_0$, $\pi'_1$ denotam as projeções da decomposição $E=E'_0\oplus E'_1$).
\end{exercise}

\begin{exercise}
Use o resultado do Exercício~\ref{exe:LemaGrassHauss} para mostrar que a Grassmanniana $G_k(E)$ é Hausdorff.
\end{exercise}

\begin{exercise}\label{exe:Grasscompl}
Se $E$ é um espaço vetorial complexo de dimensão $n<+\infty$ e $k$ é um número natural menor ou igual a $n$, podemos considerar a {\em Grassmanniana complexa\/}
$G_k(E)$, que é o conjunto de todos os subespaços complexos de $E$ de dimensão $k$. Repita o que foi feito neste apêndice para obter um atlas maximal
em $G_k(E)$ que faz de $G_k(E)$ uma variedade diferenciável Hausdorff de dimensão\footnote{%
No entanto, a Grassmanniana complexa $G_k(E)$ é também uma {\em variedade complexa\/} de dimensão $k(n-k)$,
mas não falamos em variedades complexas neste curso.} $2k(n-k)$ tal que a ação canônica de $\GL(E)$ (o grupo dos isomorfismos
de $E$ que são lineares sobre $\C$) em $G_k(E)$ é uma aplicação de classe $C^\infty$ (observe que se $E=E_0\oplus E_1$, com $E_0$, $E_1$ subespaços
complexos de $E$ e se $\Dim(E_0)=k$ então o espaço $\Lin(E_0,E_1)$ das transformações $T:E_0\to E_1$ que são lineares sobre $\C$ é um espaço vetorial
complexo de dimensão $k(n-k)$ e um espaço vetorial real de dimensão $2k(n-k)$).
\end{exercise}

\begin{exercise}
Se $E$ é um espaço vetorial complexo de dimensão $n<+\infty$, denote por $E^\R$ o espaço vetorial real subjacente a $E$ (i.e., $E^\R$ é obtido de $E$ pela
restrição da operação de multiplicação por escalares de $E$ aos números reais), que é um espaço vetorial real de dimensão $2n$. Dado um número natural $k$
menor ou igual a $n$, então a Grassmanniana complexa $G_k(E)$ está contida na Grassmanniana real $G_{2k}(E^\R)$. Mostre que
$G_k(E)$ é uma subvariedade de $G_{2k}(E^\R)$ (sugestão: se $E=E_0\oplus E_1$ é uma decomposição de $E$, com $E_0$, $E_1$ subespaços complexos
de $E$ e $\Dim(E_0)=k$, considere a correspondente carta $\phi_{E_0,E_1}$ em $G_{2k}(E^\R)$, que toma valores no espaço $\Lin(E_0^\R,E_1^\R)$ das aplicações
$T:E_0\to E_1$ que são lineares sobre $\R$. Qual é a imagem de $G_k(E)$ por $\phi_{E_0,E_1}$?).
\end{exercise}

\end{section}

\begin{section}{Espaços de Baire}\label{sec:Baire}

Seja $X$ um espaço topológico. Um subconjunto $A$ de $X$ é dito {\em raro\/} (em inglês, {\em nowhere dense})
se seu fecho possui interior vazio (ou, equivalentemente, se $A$ está contido
num subconjunto fechado de $X$ que tem interior vazio). Dizemos que $A$ é {\em magro\/} se $A$ pode ser escrito como uma união enumerável de conjuntos raros.
Evidentemente, todo conjunto raro é magro, todo subconjunto de um conjunto raro (resp., magro) é raro (resp., magro) e a união enumerável de conjuntos magros
é magra. Diz-se que $X$ é um {\em espaço de Baire\/} se todo subconjunto magro de $X$ possui interior vazio.

O seguinte resultado é demonstrado em muitos cursos de topologia:
\begin{teo}[Baire]\label{thm:Baire}
Todo espaço topológico completamente metrizável (i.e., cuja topologia pode ser definida por uma métrica completa) é um espaço de Baire e todo espaço topológico
localmente compacto Hausdorff é um espaço de Baire.\qed
\end{teo}
Segue daí que toda variedade diferenciável Hausdorff (sendo localmente compacta Hausdorff) é um espaço de Baire. No entanto, a hipótese de que a variedade
seja Hausdorff é na verdade desnecessária, em vista dos resultados que enunciamos a seguir na forma de exercícios.

\begin{exercise}\label{exe:propmagroraro}
Sejam $X$ um espaço topológico e $Y$ um subespaço de $X$. Dado um subconjunto $A$ de $Y$, mostre que:
\begin{itemize}
\item[(a)] se $A$ é raro em $Y$ então $A$ é raro em $X$ (sugestão: se $U$ é um aberto em $X$ não vazio contido no fecho de $A$ em $X$ então $U$ corta
$Y$, pois corta $A$, e $U\cap Y$ está contido no fecho de $A$ relativo a $Y$);
\item[(b)] se $A$ é magro em $Y$ então $A$ é magro em $X$;
\item[(c)] se $Y$ é aberto em $X$ e $A$ é raro em $X$ então $A$ é raro em $Y$;
\item[(d)] se $Y$ é aberto em $X$ e $A$ é magro em $X$ então $A$ é magro em $Y$.
\end{itemize}
\end{exercise}

\begin{exercise}[localidade da noção de espaço de Baire]\label{exe:locBaire}
Seja $X$ um espaço topológico. Mostre que:
\begin{itemize}
\item[(a)] se $Y$ é aberto em $X$ e $X$ é um espaço de Baire então $Y$ é um espaço de Baire (sugestão: use o resultado do item~(b) do Exercício~\ref{exe:propmagroraro});
\item[(b)] se $X=\bigcup_{i\in I}U_i$ é uma cobertura aberta de $X$ e cada $U_i$ é um espaço de Baire então $X$ é um espaço de Baire (sugestão:
use o resultado do item~(d) do Exercício~\ref{exe:propmagroraro}).
\end{itemize}
\end{exercise}

\begin{exercise}\label{exe:variedadeBaire}
Use o resultado do Exercício~\ref{exe:locBaire} para mostrar que toda variedade diferenciável (e toda variedade topológica) é um espaço de Baire (mesmo
sem a hipótese de que sua topologia seja Hausdorff).
\end{exercise}

Vejamos agora que o resultado do Exercício~\ref{exe:naosubnaosobre} vale sem a hipótese de que $N$ seja Hausdorff.
\begin{exercise}
Sejam $M$, $N$ variedades diferenciáveis e $f:M\to N$ uma aplicação de classe $C^\infty$ que tenha posto constante, mas não seja uma submersão.
\begin{itemize}
\item[(a)] Mostre que todo ponto de $M$ possui uma vizinhança $U$ tal que $f(U)$ é raro em $N$ (sugestão: usando o teorema do posto, obtem-se que
$f(U)$ é raro relativamente ao domínio $V$ de uma certa carta de $N$; use então o resultado do item~(a) do Exercício~\ref{exe:propmagroraro}).
\item[(b)] Se $M$ tem base enumerável de abertos, mostre que a imagem de $f$ é magra em $N$ e (em vista do resultado do Exercício~\ref{exe:variedadeBaire})
tem interior vazio.
\end{itemize}
\end{exercise}

\begin{exercise}\label{exe:genericoBaire}
Sejam $X$ um espaço de Baire e $Y$ um subconjunto de $X$ cujo complementar é magro. Mostre que $Y$ é um espaço de Baire (sugestão: se $A$ é magro em $Y$, use
o fato que a união de $A$ com o complementar de $Y$ é magra em $X$).
\end{exercise}

\begin{exercise}
Diferentemente do que possa parecer, um subespaço fechado de um espaço de Baire nem sempre é um espaço de Baire.
Seja $X$ o complementar em $\R^2$ do conjunto $(\R\setminus\Q)\times\{0\}$.
\begin{itemize}
\item[(a)] Mostre que o complementar de $X$ em $\R^2$ é raro e use o resultado do Exercício~\ref{exe:genericoBaire} para concluir que $X$ é um espaço de Baire.
\item[(b)] Mostre que $\Q\times\{0\}$ é um subespaço fechado de $X$ que não é um espaço de Baire.
\end{itemize}
\end{exercise}

\begin{exercise}
Seja $X$ um conjunto infinito enumerável munido da {\em topologia cofinita}, i.e., os abertos de $X$ são o conjunto vazio e os conjuntos de complementar finito.
\begin{itemize}
\item[(a)] Mostre que todo subconjunto de $X$ é compacto e, em particular, $X$ é localmente compacto.
\item[(b)] Mostre que $X$ é T1, mas não é Hausdorff.
\item[(c)] Mostre que $X$ não é um espaço de Baire (sugestão: os subconjuntos unitários de $X$ são raros).
\end{itemize}
\end{exercise}

\end{section}

\begin{section}{Tensores}
\label{sec:tensores}

Este apêndice contém um curso expresso sobre tensores (apresentado na forma de uma grande lista de exercícios). O que é um tensor? Para um físico,
um tensor é uma hipermatriz (i.e., uma matriz com vários índices) que está associada a um certo sistema de coordenadas (uma base de um espaço vetorial)
e que se transforma, após uma mudança de base, por uma certa lei um tanto complicada. A forma matematicamente civilizada de se apresentar esse conceito é através
de aplicações multilineares: tais aplicações são representadas, com respeito a uma escolha de bases, por hipermatrizes e tais hipermatrizes se transformam,
após uma mudança de base, de acordo com a lei considerada pelos físicos. Por outro lado, do ponto de vista de um algebrista, produtos tensoriais de espaços vetoriais
(ou, mais geralmente, de módulos) são definidos através de uma certa propriedade universal e tensores são elementos de tais produtos tensoriais. Para
espaços vetorais de dimensão finita, os produtos tensoriais definidos através da propriedade universal podem ser identificados com certos espaços de aplicações
multilineares.

O apêndice está organizado da seguinte forma: na Subseção~\ref{sub:multilinear} estudamos propriedades de aplicações multilineares e certas operações
com aplicações multilineares (o produto tensorial de aplicações multilineares, o pull-back e o push-forward). Na Subseção~\ref{sub:univtensorial}
apresentamos a propriedade universal que os algebristas usam para definir produtos tensoriais de espaços vetoriais e mostramos que, para espaços vetoriais
de dimensão finita, certos espaços de aplicações multilineares satisfazem justamente essa propriedade universal.
Na Subseção~\ref{sub:pqtensores} consideramos um tipo particular de aplicação multilinear, os chamados $(p,q)$-tensores, que são particularmente importantes.
Na Subseção~\ref{sub:tensorsimetria} estudamos a ação de permutações sobre aplicações multilineares e definimos os conceitos de aplicações multilineares
simétricas e anti-simétricas e certas operações com tais tipos de aplicações multilineares (os produtos simétrico e exterior e o produto interior).
Finalmente, na Subseção~\ref{sub:univextsim}, mostramos que espaços de aplicações multilineares simétricas e anti-simétricas, no caso de
espaços vetoriais de dimensão finita,
também podem ser definidos em termos de propriedades universais (que definem os chamados produto simétrico e exterior de espaços vetoriais).

Em todo o apêndice trabalhamos apenas com espaços vetoriais reais, mas o corpo de escalares $\R$ poderia ser substituído por um corpo arbitrário de característica
zero, sem qualquer dificuldade. Muitas considerações poderiam ser generalizadas para corpos de escalares arbitrários (ou mesmo anéis comutativos de escalares),
mas isso geraria algumas complicações que queremos evitar. Na maior parte do tempo, não há nenhuma necessidade de se assumir que a dimensão dos espaços vetoriais
seja finita. Assim, a dimensão dos espaços vetoriais não é assumida finita, a menos que tal hipótese seja colocada explicitamente. Excepcionalmente, na
Subseção~\ref{sub:pqtensores} assumimos o tempo todo que a dimensão dos espaços vetoriais seja finita.

\subsection{Aplicações multilineares}\label{sub:multilinear}
Sejam $E_1$, \dots, $E_k$, $F$ espaços vetoriais e seja $\mathfrak t:E_1\times\cdots\times E_k\to F$ uma função. Dizemos que $\mathfrak t$ é
{\em multilinear\/} (ou {\em $k$-linear})
se for linear em cada variável, i.e., se dado $i\in\{1,\ldots,k\}$ e vetores $x_j\in E_j$, $j=1,\ldots,k$, $j\ne i$, então a aplicação:
\[E_i\ni x\longmapsto\mathfrak t(x_1,\ldots,x_{i-1},x,x_{i+1},\ldots,x_k)\in F\]
é linear. Denotamos por:
\[\Lin(E_1,\ldots,E_k;F)\]
o espaço de todas as aplicações
multilineares $\mathfrak t:E_1\times\cdots\times E_k\to F$, o qual é um subespaço do espaço vetorial de todas as funções $\mathfrak t:E_1\times\cdots\times E_k\to F$.
Denotamos também por:
\[\Lin_k(E,F)\]
o espaço das aplicações $k$-lineares $\mathfrak t:E\times\cdots\times E\to F$ (sendo $k$ cópias de $E$ no produto cartesiano $E\times\cdots\times E$).
O espaço $\Lin_0(E,F)$ é identificado\footnote{%
O produto cartesiano de zero cópias de $E$ é um conjunto unitário e o conjunto das funções a valores em $F$ definidas nesse conjunto unitário identifica-se
com $F$.} com o espaço $F$ e o espaço $\Lin_1(E,F)$ é igual ao espaço $\Lin(E,F)$ das aplicações lineares de $E$ para $F$.

\begin{exercise}[aplicações multilineares ficam determinadas por seus valores em bases]\label{exe:multlinbase}
Sejam $E_1$, \dots, $E_k$, $F$ espaços vetoriais e $(e^i_\lambda)_{\lambda\in\Lambda^i}$ uma base para o espaço vetorial $E_i$, $i=1,\ldots,k$.
Dada uma função:
\[f:\Lambda^1\times\cdots\times\Lambda^k\longrightarrow F,\]
mostre que existe uma única aplicação multilinear $\mathfrak t:E_1\times\cdots\times E_k\to F$
tal que:
\[\mathfrak t(e^1_{\lambda_1},\ldots,e^k_{\lambda_k})=f(\lambda_1,\ldots,\lambda_k),\]
para todos $\lambda_1\in\Lambda^1$, \dots, $\lambda_k\in\Lambda^k$. Mais especificamente, mostre que $\mathfrak t$ é dada por:
\begin{equation}\label{eq:fraktbases}
\mathfrak t(x_1,\ldots,x_k)=\sum_{\lambda_1\in\Lambda^1}\cdots\sum_{\lambda_k\in\Lambda^k}\pi^1_{\lambda_1}(x_1)\cdots\pi^k_{\lambda_k}(x_k)f(\lambda_1,\ldots,\lambda_k),
\end{equation}
para $x_1\in E_1$, \dots, $x_k\in E_k$,
onde, para cada $i=1,\ldots,k$ e cada $\lambda\in\Lambda^i$, $\pi^i_\lambda:E_i\to\R$ denota o funcional linear que associa a cada vetor de $E_i$ sua $\lambda$-ésima
coordenada na base dada. Note que nos somatórios possivelmente infinitos que aparecem em \eqref{eq:fraktbases} há, na verdade, no máximo um número finito de termos
não nulos, já que o número de coordenadas não nulas de um vetor numa dada base é sempre finito.
\end{exercise}

Dadas aplicações multilineares $\mathfrak t:E_1\times\cdots\times E_k\to\R$, $\mathfrak t':E'_1\times\cdots\times E'_l\to\R$, então seu {\em produto tensorial\/}
é a aplicação multilinear:
\[\mathfrak t\otimes\mathfrak t':E_1\times\cdots\times E_k\times E'_1\times\cdots\times E'_l\longrightarrow\R\]
definida por:
\[(\mathfrak t\otimes\mathfrak t')(x_1,\ldots,x_k,x'_1,\ldots,x'_l)=\mathfrak t(x_1,\ldots,x_k)\mathfrak t'(x'_1,\ldots,x'_l),\]
para todos $x_1\in E_1$, \dots, $x_k\in E_k$, $x'_1\in E'_1$, \dots, $x'_l\in E'_l$. A operação $\otimes$ é associativa, i.e.,
dadas aplicações multilineares $\mathfrak t$, $\mathfrak t'$, $\mathfrak t''$ a valores reais então:
\[(\mathfrak t\otimes\mathfrak t')\otimes\mathfrak t''=\mathfrak t\otimes(\mathfrak t'\otimes\mathfrak t''),\]
de modo que podemos escrever, sem ambigüidade, produtos tensoriais de várias aplicações multilineares a valores reais sem usar parênteses.
A aplicação:
\[(\mathfrak t_1,\ldots,\mathfrak t_k)\longmapsto\mathfrak t_1\otimes\cdots\otimes\mathfrak t_k\]
(definida num certo produto cartesiano de $k$ espaços de aplicações multilineares a valores reais e tomando valores no espaço apropriado
de aplicações multilineares) é multilinear. Observamos também que o produto tensorial de aplicações multilineares inclui como caso particular
o produto de escalares por aplicações multilineares, se pensamos nos escalares como aplicações zero-lineares.

Sejam $E$, $E'$ e $F$ espaços vetoriais e $T:E'\to E$ uma aplicação linear. Para cada número natural $k$, o {\em pull-back\/} por $T$ é a aplicação linear:
\[T^*:\Lin_k(E,F)\longrightarrow\Lin_k(E',F)\]
definida por:
\[(T^*\mathfrak t)(x_1,\ldots,x_k)=\mathfrak t\big(T(x_1),\ldots,T(x_k)\big),\]
para quaisquer $\mathfrak t\in\Lin_k(E,F)$, $x_1,\ldots,x_k\in E'$. Note que quando $F=\R$ e $k=1$, o pull-back $T^*:\Lin(E,\R)\to\Lin(E',\R)$ não é nada
além da usual {\em aplicação transposta\/} $T^*:E^*\to{E'}^*$ definida entre os espaços duais.
Quando $T:E'\to E$ é um isomorfismo linear, podemos também definir o {\em push-forward\/}
por $T$:
\[T_*:\Lin_k(E',F)\longrightarrow\Lin_k(E,F)\]
que é o pull-back pela aplicação inversa $T^{-1}$, i.e., $T_*=(T^{-1})^*$.

\begin{exercise}\label{exe:pullbackcomposta}
Sejam $T:E'\to E$, $S:E''\to E'$ aplicações lineares. Mostre que:
\[(T\circ S)^*=S^*\circ T^*\]
e que, quando $T$, $S$ são isomorfismos, então:
\[(T\circ S)_*=T_*\circ S_*.\]
\end{exercise}

\begin{exercise}\label{exe:pullbackhomo}
Dadas $\mathfrak t\in\Lin_k(E,\R)$, $\mathfrak t'\in\Lin_l(E,\R)$ e uma aplicação linear $T:E'\to E$, mostre que:
\[T^*(\mathfrak t\otimes\mathfrak t')=(T^*\mathfrak t)\otimes(T^*\mathfrak t').\]
Enuncie e prove o resultado análogo para o push-forward.
\end{exercise}

\subsection{Propriedade universal do produto tensorial}\label{sub:univtensorial}
Sejam $E_1$, \dots, $E_k$ espaços vetoriais. Um {\em produto tensorial\/} para esses espaços consiste de um espaço vetorial $Z$
e de uma aplicação multilinear $\tau:E_1\times\cdots\times E_k\to Z$ tais que, dado um espaço vetorial qualquer $F$ e uma aplicação multilinear
qualquer $\mathfrak t:E_1\times\cdots\times E_k\to F$, existe uma {\em única\/} transformação linear $T:Z\to F$ tal que $T\circ\tau=\mathfrak t$, i.e., tal que
o diagrama:
\[\xymatrix{%
E_1\times\cdots\times E_k\ar[dr]^{\mathfrak t}\ar[d]_\tau\\
Z\ar[r]_T&F}\]
comuta.

\begin{exercise}[unicidade do produto tensorial]\label{exe:uniquetensorprod}
Sejam $(Z,\tau)$, $(Z',\tau')$ dois produtos tensoriais para os espaços vetoriais $E_1$, \dots, $E_k$. Tomando $\mathfrak t=\tau'$ na definição de produto
tensorial, vemos que existe uma única aplicação linear $T:Z\to Z'$ tal que $T\circ\tau=\tau'$, i.e., tal que o diagrama:
\[\xymatrix{%
&E_1\times\cdots\times E_k\ar[dl]_\tau\ar[dr]^{\tau'}\\
Z\ar[rr]_T&&Z'}\]
comuta. Mostre que $T$ é um isomorfismo (sugestão: usando-se novamente a definição de produto tensorial, obtêm-se uma aplicação linear $T':Z'\to Z$ tal que
$T'\circ\tau'=\tau$. Note que $(T'\circ T)\circ\tau=\tau$ e conclua, usando o requerimento sobre unicidade que aparece na definição de produto tensorial,
que $T'\circ T$ é a aplicação identidade de $Z$. De modo similar, mostre que $T\circ T'$ é a aplicação identidade de $Z'$).
\end{exercise}

\begin{rem}
É possível mostrar que qualquer $k$-upla de espaços vetoriais $E_1$, \dots, $E_k$ admite um produto tensorial. Ele pode ser construído, por exemplo,
como um certo quociente do espaço das combinações lineares formais de elementos do produto cartesiano $E_1\times\cdots\times E_k$ (i.e., o espaço
das funções quase nulas de $E_1\times\cdots\times E_k$ em $\R$). Essa construção é simples, mas está fora do escopo do nosso curso. Para espaços de dimensão
finita veremos logo abaixo que o produto tensorial pode ser obtido como um espaço de aplicações multilineares.
\end{rem}

Se $(Z,\tau)$ é um produto tensorial dos espaços $E_1$, \dots, $E_k$, escreve-se normalmente $Z=E_1\otimes\cdots\otimes E_k$ ou $Z=\bigotimes_{i=1}^kE_i$ e:
\[\tau(x_1,\ldots,x_k)=x_1\otimes\cdots\otimes x_k,\quad x_1\in E_1,\ldots,x_k\in E_k.\]
A definição de produto tensorial nos diz então que para qualquer espaço vetorial $F$ e qualquer aplicação multilinear $\mathfrak t:E_1\times\cdots\times E_k\to F$
existe uma única transformação linear $T:E_1\otimes\cdots\otimes E_k\to F$ tal que:
\[T(x_1\otimes\cdots\otimes x_k)=\mathfrak t(x_1,\ldots,x_k),\]
para quaisquer $x_1\in E_1$, \dots, $x_k\in E_k$, i.e., tal que o diagrama:
\[\xymatrix@C+20pt{%
E_1\times\cdots\times E_k\ar[d]_\otimes\ar[dr]^-{\mathfrak t}\\
E_1\otimes\cdots\otimes E_k\ar[r]_-T&F}\]
comuta.

\begin{exercise}[caracterização da definição de produto tensorial em termos de bases]\label{exe:tensorbases}
Sejam $E_1$, \dots, $E_k$ espaços vetoriais e $(e^i_\lambda)_{\lambda\in\Lambda^i}$ uma base para o espaço vetorial $E_i$, $i=1,\ldots,k$.
Dados um espaço vetorial $Z$ e uma aplicação multilinear $\tau:E_1\times\cdots\times E_k\to Z$, mostre que as seguintes condições são equivalentes:
\begin{itemize}
\item[(a)] a família:
\begin{equation}\label{eq:basetaue}
\tau(e^1_{\lambda_1},\ldots,e^k_{\lambda_k}),\quad\lambda_1\in\Lambda^1,\ldots,\lambda_k\in\Lambda^k
\end{equation}
é uma base de $Z$;
\item[(b)] $(Z,\tau)$ é um produto tensorial para $E_1$, \dots, $E_k$
\end{itemize}
(sugestão: para provar (a)$\Rightarrow$(b), note que a condição almejada $T\circ\tau=\mathfrak t$ nos diz como definir $T$ sobre a base \eqref{eq:basetaue}
e que, em vista do resultado do Exercício~\ref{exe:multlinbase}, a condição $T\circ\tau=\mathfrak t$ será satisfeita se tal igualdade valer sobre $k$-uplas de vetores
das bases dadas. Para provar (b)$\Rightarrow$(a), use o resultado do Exercício~\ref{exe:multlinbase} para mostrar a existência de uma aplicação multilinear
$\mathfrak t$ que leva as $k$-uplas $(e^1_{\lambda_1},\ldots,e^k_{\lambda_k})$ sobre uma família linearmente independente de vetores de um espaço vetorial $F$
e conclua, usando a aplicação
linear $T$ tal que $T\circ\tau=\mathfrak t$, que a família \eqref{eq:basetaue} é linearmente independente. Finalmente, para mostrar que a família
\eqref{eq:basetaue} gera $Z$, suponha por absurdo que isso não seja verdade e obtenha uma aplicação linear não nula $T$ tal que $T\circ\tau=0$. Obtenha
uma contradição com o requerimento sobre unicidade que aparece na definição de produto tensorial).
\end{exercise}

No próximo exercício estabelecemos uma relação entre o produto tensorial definido acima e o produto tensorial de aplicações multilineares definido na
Subseção~\ref{sub:multilinear}.
\begin{exercise}\label{exe:Lintensor}
Sejam $E_1$, \dots, $E_k$ espaços vetoriais de dimensão finita e seja $(e^i_r)_{r=1}^{n_i}$ uma base de $E_i$, $i=1,\ldots,k$.
Denote por $(\pi^i_r)_{r=1}^{n_i}$ a base dual de $(e^i_r)_{r=1}^{n_i}$, i.e., $(\pi^i_r)_{r=1}^{n_i}$ é a base do espaço dual $E_i^*$ tal que:
\[\pi^i_r(e^i_s)=\delta_{rs},\quad r,s=1,\ldots,n_i,\]
onde $\delta_{rs}=1$ para $r=s$ e $\delta_{rs}=0$ para $r\ne s$.
\begin{itemize}
\item[(a)] Mostre que:
\[(\pi^1_{r_1}\otimes\cdots\otimes\pi^k_{r_k})(e^1_{s_1},\ldots,e^k_{s_k})=1,\]
se $r_1=s_1$, \dots, $r_k=s_k$ e:
\[(\pi^1_{r_1}\otimes\cdots\otimes\pi^k_{r_k})(e^1_{s_1},\ldots,e^k_{s_k})=0,\]
caso contrário.
\item[(b)] Use o resultado do item~(a) para mostrar que a família:
\begin{equation}\label{eq:basetensor}
\pi^1_{r_1}\otimes\cdots\otimes\pi^k_{r_k}\in\Lin(E_1,\ldots,E_k;\R),\quad r_i=1,\ldots,n_i,\ i=1,\ldots,k
\end{equation}
é linearmente independente (sugestão: avalie uma combinação linear de \eqref{eq:basetensor} em $(e^1_{s_1},\ldots,e^k_{s_k})$).
\item[(c)] Use o resultado do Exercício~\ref{exe:multlinbase} para mostrar que:
\[\mathfrak t=\sum_{r_1=1}^{n_1}\cdots\sum_{r_k=1}^{n_k}\mathfrak t(e^1_{r_1},\ldots,e^k_{r_k})(\pi^1_{r_1}\otimes\cdots\otimes\pi^k_{r_k}),\]
para qualquer $\mathfrak t\in\Lin(E_1,\ldots,E_k;\R)$.
\item[(d)] Use o resultado dos itens~(b) e (c) para concluir que a família \eqref{eq:basetensor} é uma base de $\Lin(E_1,\ldots,E_k;\R)$. Por fim, use o resultado
do Exercício~\ref{exe:tensorbases} e conclua também que:
\[E_1^*\times\cdots\times E_k^*\ni(\alpha_1,\ldots,\alpha_k)\longmapsto\alpha_1\otimes\cdots\otimes\alpha_k\in\Lin(E_1,\ldots,E_k;\R)\]
nos dá um produto tensorial dos espaços duais $E_1^*$, \dots, $E_k^*$.
\end{itemize}
\end{exercise}

Em vista do resultado do item~(d) do Exercício~\ref{exe:Lintensor}, escreveremos:
\begin{equation}\label{eq:LinEitensor}
\Lin(E_1,\ldots,E_k;\R)=E_1^*\otimes\cdots\otimes E_k^*=\bigotimes_{i=1}^kE_i^*
\end{equation}
{\em quando os espaços $E_1$, \dots, $E_k$ tiverem dimensão finita}. Note que o resultado do item~(c) do Exercício~\ref{exe:Lintensor}
nos diz que as coordenadas de um elemento $\mathfrak t$ de $\Lin(E_1,\ldots,E_k;\R)=E_1^*\otimes\cdots\otimes E_k^*$ na base \eqref{eq:basetensor} são:
\[\mathfrak t(e^1_{r_1},\ldots,e^k_{r_k}),\quad r_i=1,\ldots,n_i,\ i=1,\ldots,k.\]
Note também que:
\[\Dim\!\big(\!\Lin(E_1,\ldots,E_k;\R)\big)=\Dim(E_1^*\otimes\cdots\otimes E_k^*)=\prod_{i=1}^k\Dim(E_i).\]
Levando-se em conta a identificação natural entre um espaço (de dimensão finita)
e seu bidual\footnote{%
Identifica-se um vetor $x\in E$ com o funcional linear $E^*\ni\alpha\mapsto\alpha(x)\in\R$ de avaliação em $x$.}, podemos escrever também:
\[\Lin(E_1^*,\ldots,E_k^*;\R)=E_1\otimes\cdots\otimes E_k=\bigotimes_{i=1}^kE_i,\]
quando os espaços $E_1$, \dots, $E_k$ tiverem dimensão finita. Uma base para $\bigotimes_{i=1}^kE_i$ é formada pelos vetores:
\[e^1_{r_1}\otimes\cdots\otimes e^k_{r_k},\quad r_i=1,\ldots,n_i,\ i=1,\ldots,k,\]
onde $(e^i_r)_{r=1}^{n_i}$ é uma base de $E_i$, $i=1,\ldots,k$. A aplicação multilinear:
\[e^1_{r_1}\otimes\cdots\otimes e^k_{r_k}\in\Lin(E_1^*,\ldots,E_k^*;\R)\]
é dada por:
\[(e^1_{r_1}\otimes\cdots\otimes e^k_{r_k})(\alpha_1,\ldots,\alpha_k)=\alpha_1(e^1_{r_1})\cdots\alpha_k(e^k_{r_k}),\quad\alpha_1\in E_1^*,\ldots,\alpha_k\in E_k^*.\]

\begin{exercise}
Sejam $E_1$, \dots, $E_k$, $F$ espaços vetoriais e assuma que os espaços $E_i$ tenham dimensão finita. Dados $\alpha_1\in E_1^*$, \dots, $\alpha_k\in E_k^*$
e $y\in F$, denote por $\alpha_1\otimes\cdots\otimes\alpha_k\otimes y\in\Lin(E_1,\ldots,E_k;F)$ a aplicação multilinear definida por:
\[(\alpha_1\otimes\cdots\otimes\alpha_k\otimes y)(x_1,\ldots,x_k)=\alpha_1(x_1)\cdots\alpha_k(x_k)y\in F,\]
para todos $x_1\in E_1$, \dots, $x_k\in E_k$. Mostre que:
\[E_1^*\times\cdots\times E_k^*\times F\ni(\alpha_1,\ldots,\alpha_k,y)\longmapsto\alpha_1\otimes\cdots\otimes\alpha_k\otimes y\in\Lin(E_1,\ldots,E_k;F)\]
define um produto tensorial dos espaços $E_1^*$, \dots, $E_k^*$, $F$, i.e., podemos escrever:
\[\Lin(E_1,\ldots,E_k;F)=E_1^*\otimes\cdots\otimes E_k^*\otimes F\]
(sugestão: sejam $(e^i_r)_{r=1}^{n_i}$ uma base de $E_i$, $(\pi^i_r)_{r=1}^{n_i}$ sua base dual e $(f_\lambda)_{\lambda\in\Lambda}$ uma base de $F$.
Siga um roteiro análogo ao apresentado no enunciado do Exercício~\ref{exe:Lintensor} para mostrar que a família:
\[\pi^1_{r_1}\otimes\cdots\otimes\pi^k_{r_k}\otimes f_\lambda,\quad r_i=1,\ldots,n_i,\ i=1,\ldots,k,\ \lambda\in\Lambda,\]
é uma base de $\Lin(E_1,\ldots,E_k;F)$. Obtenha sua conclusão usando o resultado do Exercício~\ref{exe:tensorbases}).
\end{exercise}

\subsection{$\mathbf{(p,q)}$-tensores}\label{sub:pqtensores}
Nesta subseção, assumimos sempre que os espaços vetoriais envolvidos tenham {\em dimensão finita}.
Dados um espaço vetorial $E$ e números naturais $p$, $q$ então um {\em $(p,q)$-tensor\/} em $E$ é uma aplicação multilinear:
\[\mathfrak t:E\times\cdots\times E\times E^*\times\cdots\times E^*\longrightarrow\R\]
sendo $p$ cópias de $E$ e $q$ cópias do espaço dual $E^*$ no produto cartesiano $E\times\cdots\times E\times E^*\times\cdots\times E^*$. Denotamos
por:
\[\Lin_{(p,q)}(E,\R)\]
o espaço dos $(p,q)$-tensores em $E$. Um $(p,q)$-tensor é também chamado de um {\em tensor $p$ vezes covariante e $q$ vezes contravariante}.
Se $(e_i)_{i=1}^n$ é uma base de $E$ e $(e_i^*)_{i=1}^n$ denota a sua base dual, então escrevemos:
\begin{equation}\label{eq:hipermatriz}
\mathfrak t_{i_1\ldots i_p}^{j_1\ldots j_q}=\mathfrak t(e_{i_1},\ldots,e_{i_p},e_{j_1}^*,\ldots,e_{j_q}^*),\quad
i_1,\ldots,i_p,j_1,\ldots,j_q=1,\ldots,n.
\end{equation}
Em vista do resultado do Exercício~\ref{exe:multlinbase}, um $(p,q)$-tensor $\mathfrak t$ fica unicamente determinado pela hipermatriz
$\mathfrak t_{i_1\ldots i_p}^{j_1\ldots j_q}$ (com $p+q$ índices $i_1$, \dots, $i_p$, $j_1$, \dots, $j_q$ variando de $1$ a $n$). Os $p$ índices
de baixo nessa hipermatriz são normalmente chamados ``índices covariantes'' e os $q$ índices de cima são normalmente chamados ``índices contravariantes''.

\begin{example}\label{exa:guardachuva}
O conceito de $(p,q)$-tensor é um guarda-chuva bem largo que cobre (usando identificações adequadas) uma grande quantidade de objetos relevantes
da álgebra multilinear.
Elementos do corpo de escalares $\R$ podem ser pensados como $(0,0)$-tensores. Elementos do espaço dual $E^*$ são $(1,0)$-tensores e,
mais geralmente, aplicações multilineares $\mathfrak t\in\Lin_k(E,\R)$ são $(k,0)$-tensores. Em vista da identificação natural de $E$ com seu
bidual $E^{**}$, podemos pensar nos próprios vetores de $E$ como $(0,1)$-tensores. Operadores lineares $L:E\to E$ podem ser identificados com aplicações
bilineares:
\[E\times E^*\ni(x,\alpha)\longmapsto\alpha\big(L(x)\big)\in\R\]
e portanto podem ser pensados como $(1,1)$-tensores. Mais geralmente, uma aplicação multilinear:
\[\mathfrak t:E\times\cdots\times E\times E^*\times\cdots\times E^*\longrightarrow E\]
sendo $p$ cópias de $E$ e $q$ cópias de $E^*$ no produto $E\times\cdots\times E\times E^*\times\cdots\times E^*$ pode ser identificada com o seguinte
$(p,q+1)$-tensor:
\[(x_1,\ldots,x_p,\alpha,\alpha_1,\ldots,\alpha_q)\longmapsto\alpha\big(\mathfrak t(x_1,\ldots,x_p,\alpha_1,\ldots,\alpha_q)\big),\]
onde $x_1,\ldots,x_p\in E$, $\alpha,\alpha_1,\ldots,\alpha_q\in E^*$.
\end{example}

Note que, em vista da igualdade \eqref{eq:LinEitensor} (e da identificação $E\cong E^{**}$), podemos escrever:
\begin{equation}\label{eq:Linpqtensores}
\Lin_{(p,q)}(E,\R)=\Big(\bigotimes_pE^*\Big)\otimes\Big(\bigotimes_qE\Big).
\end{equation}
Usando o resultado do Exercício~\ref{exe:Lintensor} vemos que uma base para \eqref{eq:Linpqtensores} é formada pelos vetores:
\begin{equation}\label{eq:basepqtensores}
e_{i_1}^*\otimes\cdots\otimes e_{i_p}^*\otimes e_{j_1}\otimes\cdots\otimes e_{j_q},\quad i_1,\ldots,i_p,j_1,\ldots,j_q=1,\ldots,n,
\end{equation}
onde $(e_i)_{i=1}^n$ denota uma base de $E$ e $(e_i^*)_{i=1}^n$ denota sua base dual. As entradas $\mathfrak t_{i_1\ldots i_p}^{j_1\ldots j_q}$ da hipermatriz
\eqref{eq:hipermatriz} associada a $\mathfrak t$ nessas bases contém justamente as coordenadas de $\mathfrak t$ na base \eqref{eq:basepqtensores}.

As noções de pull-back e push-forward podem ser estendidas para $(p,q)$-tensores sob as seguintes condições: se $T:E'\to E$ é uma aplicação linear qualquer
então o pull-back:
\[T^*:\Lin_{(p,q)}(E,\R)\longrightarrow\Lin_{(p,q)}(E',\R)\]
pode ser definido (como já o fizemos na Subseção~\ref{sub:multilinear})
quando $q=0$ e pode ser definido para quaisquer $p$, $q$ {\em desde que $T$ seja um isomorfismo\/} fazendo:
\[(T^*\mathfrak t)(x_1,\ldots,x_p,\alpha_1,\ldots,\alpha_q)=\mathfrak t\big(T(x_1),\ldots,T(x_p),{T^*}^{-1}(\alpha_1),\ldots,{T^*}^{-1}(\alpha_q)\big),\]
para todos $\mathfrak t\in\Lin_{(p,q)}(E,\R)$,
$x_1,\ldots,x_p\in E'$, $\alpha_1,\ldots,\alpha_q\in{E'}^*$, onde usamos também $T^*$ para denotar a usual aplicação transposta $T^*:E^*\to{E'}^*$.
Ainda quando $T$ é um isomorfismo, o {\em push-forward}:
\[T_*:\Lin_{(p,q)}(E',\R)\longrightarrow\Lin_{(p,q)}(E,\R)\]
pode ser definido para $p$, $q$ arbitrários, fazendo:
\[(T_*\mathfrak t)(x_1,\ldots,x_p,\alpha_1,\ldots,\alpha_q)=\mathfrak t\big(T^{-1}(x_1),\ldots,T^{-1}(x_p),T^*(\alpha_1),\ldots,T^*(\alpha_q)\big),\]
para todos $\mathfrak t\in\Lin_{(p,q)}(E',\R)$,
$x_1,\ldots,x_p\in E$, $\alpha_1,\ldots,\alpha_q\in E^*$. Note que o push-forward $T_*$ pode ser definido para uma transformação linear $T$ arbitrária
quando $p=0$.

Quando $T$ é um isomorfismo então o push-forward $T_*$ é exatamente o pull-back $(T^{-1})^*$ pela aplicação inversa de $T$. Resultados
similares aos enunciados no Exercício~\ref{exe:pullbackcomposta} valem para pull-back e push-forward de $(p,q)$-tensores. Além do mais, pode-se
definir uma operação bilinear associativa $(\mathfrak t,\mathfrak t')\mapsto\mathfrak t\otimes\mathfrak t'$
que leva um $(p,q)$-tensor $\mathfrak t$ e um $(p',q')$-tensor $\mathfrak t'$ num $(p+p',q+q')$-tensor $\mathfrak t\otimes\mathfrak t'$. Essa operação
coincide com a operação de produto tensorial de aplicações multilineares introduzida na Subseção~\ref{sub:multilinear}, a menos de um ajuste na ordem das variáveis, já que
no $(p+p',q+q')$-tensor $\mathfrak t\otimes\mathfrak t'$ as $p+p'$ variáveis que tomam valores em $E$ devem ficar à esquerda das $q+q'$ variáveis que tomam
valores em $E^*$. Um resultado análogo àquele que aparece no Exercício~\ref{exe:pullbackhomo} vale então para $(p,q)$-tensores.

\begin{exercise}\label{exe:pullbackpqpartic}
Identifique explicitamente o significado das operações de pull-back e push-forward por uma aplicação linear $T:E'\to E$ no caso de $(p,q)$-tensores
com $0\le p,q\le1$. No caso $p=0$, $q=1$, utilize a identificação entre $\Lin_{(0,1)}(E,\R)=E^{**}$ e $E$. No caso $p=q=1$, utilize a
identificação entre $\Lin_{(1,1)}(E,\R)$ e $\Lin(E,E)$ explicada no Exemplo~\ref{exa:guardachuva}.
\end{exercise}

\subsection{Simetria e anti-simetria}\label{sub:tensorsimetria}
Seja $S_k$ o grupo das bijeções do conjunto $\{1,\ldots,k\}$. Dada $\sigma\in S_k$ e uma aplicação $k$-linear $\mathfrak t\in\Lin_k(E,F)$ (onde $E$, $F$
são espaços vetoriais fixados), definimos $\sigma\cdot\mathfrak t\in\Lin_k(E,F)$ fazendo:
\[(\sigma\cdot\mathfrak t)(x_1,\ldots,x_k)=\mathfrak t(x_{\sigma(1)},\ldots,x_{\sigma(k)}),\]
para todos $x_1,\ldots,x_k\in E$. Em outras palavras, se pensamos nos elementos $x=(x_1,\ldots,x_k)$ do domínio $E^k$ de $\mathfrak t$ como
funções de $\{1,\ldots,k\}$ em $E$, então $\sigma\cdot\mathfrak t$ é dada por:
\[(\sigma\cdot\mathfrak t)(x)=\mathfrak t(x\circ\sigma),\quad x:\{1,\ldots,k\}\to E.\]
Para cada $\sigma\in S_k$, a aplicação $\mathfrak t\mapsto\sigma\cdot\mathfrak t$ é um operador linear em $\Lin_k(E,F)$ e, além do mais, temos que:
\[(\sigma_1\circ\sigma_2)\cdot\mathfrak t=\sigma_1\cdot(\sigma_2\cdot\mathfrak t),\quad\sigma_1,\sigma_2\in S_k,\ \mathfrak t\in\Lin_k(E,F),\]
e $\sigma\cdot\mathfrak t=\mathfrak t$ se $\sigma$ é a identidade de $\{1,\ldots,k\}$. Em outras palavras, a aplicação
$(\sigma,\mathfrak t)\mapsto\sigma\cdot\mathfrak t$ define uma {\em representação\/} do grupo $S_k$ no espaço vetorial $\Lin_k(E,F)$.
Seja:
\[\sgn:S_k\longrightarrow\{-1,1\}\]
o homomorfismo de grupos dado por $\sgn(\sigma)=1$ se $\sigma$ é uma permutação par e $\sgn(\sigma)=-1$ se $\sigma$ é uma permutação ímpar.
Uma aplicação multilinear\footnote{%
Obviamente, os conceitos de simetria e anti-simetria são independentes da multilinearidade, mas só estamos interessados em aplicações multilineares.}
$\mathfrak t\in\Lin_k(E,F)$ é dita {\em simétrica\/} (resp., {\em anti-simétrica}) se $\sigma\cdot\mathfrak t=\mathfrak t$
(resp., se $\sigma\cdot\mathfrak t=\sgn(\sigma)\mathfrak t$), para todo $\sigma\in S_k$. Denotamos por:
\[\Lin^{\mathrm s}_k(E,F),\quad\Lin^{\mathrm a}_k(E,F),\]
respectivamente, os subespaços de $\Lin_k(E,F)$ formados pelas aplicações si\-mé\-tri\-cas e pelas aplicações anti-simétricas.

\begin{exercise}\label{exe:bastatrocas}
Dada uma aplicação multilinear $\mathfrak t\in\Lin_k(E,F)$, mostre que:
\[\big\{\sigma\in S_k:\sigma\cdot\mathfrak t=\mathfrak t\big\},\quad\big\{\sigma\in S_k:\sigma\cdot\mathfrak t=\sgn(\sigma)\mathfrak t\big\}\]
são subgrupos de $S_k$. Conclua que $\mathfrak t$ é simétrica (resp., anti-simétrica) se e somente se a igualdade $\sigma\cdot\mathfrak t=\mathfrak t$
(resp., a igualdade $\sigma\cdot\mathfrak t=\sgn(\sigma)\mathfrak t$) vale para qualquer $\sigma$ pertencente a algum conjunto de geradores do grupo
$S_k$ (por exemplo, o conjunto das transposições de elementos consecutivos $(i\ i+1)$, $i=1,\ldots,k-1$ é um tal conjunto de geradores).
\end{exercise}

\begin{exercise}\label{exe:antisimetricaalternada}
Seja $\mathfrak t\in\Lin_k(E,F)$ uma aplicação multilinear. Mostre que as seguintes condições são equivalentes:
\begin{itemize}
\item[(a)] dados $x_1,\ldots,x_k\in E$, se existem $i,j\in\{1,\ldots,k\}$ com $i\ne j$ e $x_i=x_j$ então $\mathfrak t(x_1,\ldots,x_k)=0$;
\item[(b)] dados $x_1,\ldots,x_k\in E$, se a família $(x_i)_{i=1}^k$ é linearmente dependente então $\mathfrak t(x_1,\ldots,x_k)=0$;
\item[(c)] $\mathfrak t$ é anti-simétrica
\end{itemize}
(sugestão: para mostrar (a)$\Rightarrow$(b) observe que se $(x_i)_{i=1}^k$ é linearmente dependente então algum $x_i$ se escreve como combinação
linear dos outros $x_j$ e para mostrar (b)$\Rightarrow$(a) observe que se $x_1$, \dots, $x_k$ não são dois a dois distintos então a família
$(x_i)_{i=1}^k$ é linearmente dependente. Para mostrar (c)$\Rightarrow$(a) suponha $x_i=x_j$ com $i\ne j$ e considere a permutação que troca as posições $i$
e $j$. Finalmente, para mostrar (a)$\Rightarrow$(c) considere a igualdade $\mathfrak t(\ldots,x+y,x+y,\ldots)=0$ e use o resultado do Exercício~\ref{exe:bastatrocas}).
Uma aplicação multilinear satisfazendo a condição (a) é normalmente chamada de {\em alternada}. O resultado deste exercício mostra que uma aplicação multilinear
é alternada se e somente se for anti-simétrica\footnote{%
Mas esse resultado depende do fato que a característica do corpo de escalares é diferente de $2$! Para corpos
de característica $2$, anti-simetria é {\em equivalente\/} a simetria e a condição de ser alternada é mais forte do que a anti-simetria.}.
\end{exercise}

\begin{exercise}\label{exe:multlinebasessimetria}
Sejam $E$, $F$ espaços vetoriais e $(e_\lambda)_{\lambda\in\Lambda}$ uma base de $E$. Suponha que o conjunto $\Lambda$ esteja munido de uma ordem total\footnote{%
Se preferir considerar apenas o caso em que $E$ tem dimensão finita, então tome $\Lambda=\{1,\ldots,n\}$ e a ordem total como sendo a ordem
usual desse conjunto.} $\le$.
\begin{itemize}
\item[(a)] Dada uma aplicação:
\[f:\big\{(\lambda_1,\ldots,\lambda_k)\in\Lambda^k:\lambda_1\le\cdots\le\lambda_k\big\}\longrightarrow F,\]
mostre que existe uma única aplicação multilinear simétrica $\mathfrak t$ em $\Lin_k^{\mathrm s}(E,F)$ tal que:
\[\mathfrak t(e_{\lambda_1},\ldots,e_{\lambda_k})=f(\lambda_1,\ldots,\lambda_k),\]
para todos $\lambda_1,\ldots,\lambda_k\in\Lambda$ tais que $\lambda_1\le\cdots\le\lambda_k$ (sugestão: a simetria nos permite dizer, em termos de $f$,
o valor de $\mathfrak t$ em $k$-uplas arbitrárias de elementos da base; use o resultado do Exercício~\ref{exe:multlinbase}).
\item[(b)] Dada uma aplicação:
\[f:\big\{(\lambda_1,\ldots,\lambda_k)\in\Lambda^k:\lambda_1<\cdots<\lambda_k\big\}\longrightarrow F,\]
mostre que existe uma única aplicação multilinear anti-simétrica $\mathfrak t$ em $\Lin_k^{\mathrm a}(E,F)$ tal que:
\[\mathfrak t(e_{\lambda_1},\ldots,e_{\lambda_k})=f(\lambda_1,\ldots,\lambda_k),\]
para todos $\lambda_1,\ldots,\lambda_k\in\Lambda$ tais que $\lambda_1<\cdots<\lambda_k$ (sugestão: tenha em mente que aplicações anti-simétricas são alternadas;
veja Exercício~\ref{exe:antisimetricaalternada}).
\end{itemize}
\end{exercise}

\begin{exercise}
Dados $\sigma\in S_k$ e funcionais lineares $\alpha_1,\ldots,\alpha_k\in E^*$, mostre que:
\[\sigma\cdot(\alpha_1\otimes\cdots\otimes\alpha_k)=\alpha_{\sigma^{-1}(1)}\otimes\cdots\otimes\alpha_{\sigma^{-1}(k)}.\]
\end{exercise}

\begin{exercise}\label{exe:trocattlinha}
Dados $\mathfrak t\in\Lin_k(E,\R)$, $\mathfrak t'\in\Lin_l(E,\R)$, mostre que:
\[\sigma\cdot(\mathfrak t\otimes\mathfrak t')=\mathfrak t'\otimes\mathfrak t,\]
onde $\sigma\in S_{k+l}$ é definida por:
\[\sigma(i)=i+l,\ i=1,\ldots,k,\quad\sigma(i)=i-k,\ i=k+1,\ldots,k+l.\]
Mostre também que $\sigma$ é um produto de $kl$ transposições, de modo que $\sgn(\sigma)=(-1)^{kl}$.
\end{exercise}

Para cada $\sigma\in S_k$, vamos denotar também por $\sigma$ o operador linear $\mathfrak t\mapsto\sigma\cdot\mathfrak t$ em $\Lin_k(E,F)$.
O {\em simetrizador\/} é o operador linear em $\Lin_k(E,F)$ definido por:
\[\Sym=\frac1{k!}\sum_{\sigma\in S_k}\sigma,\]
e o {\em alternador\/} é o operador linear em $\Lin_k(E,F)$ definido por:
\[\Alt=\frac1{k!}\sum_{\sigma\in S_k}\sgn(\sigma)\sigma.\]

\begin{exercise}\label{exe:sigmaSymAlt}\
\begin{itemize}
\item[(a)] Dada $\sigma\in S_k$, mostre que:
\[\sigma\circ\Sym=\Sym\circ\sigma=\Sym,\quad\sigma\circ\Alt=\Alt\circ\sigma=\sgn(\sigma)\Alt.\]
\item[(b)] Dado $\mathfrak t\in\Lin_k(E,F)$, mostre que $\Sym(\mathfrak t)$ é simétrica e $\Alt(\mathfrak t)$ é anti-simétrica.
\item[(c)] Se $\mathfrak t\in\Lin_k(E,F)$ é simétrica (resp., anti-simétrica) mostre que $\Sym(\mathfrak t)=\mathfrak t$ (resp.,
$\Alt(\mathfrak t)=\mathfrak t$).
\end{itemize}
\end{exercise}

\begin{exercise}\label{exe:SymAltG}
Sejam $G$ um subgrupo de $S_k$, $S_k/G=\big\{\sigma G:\sigma\in S_k\big\}$ o conjunto das coclasses à esquerda de $G$ em $S_k$ e $X$ um conjunto escolha para
$S_k/G$, i.e., $X$ contém precisamente um elemento de cada coclasse à esquerda de $G$ em $S_k$.
\begin{itemize}
\item[(a)] Se $\mathfrak t\in\Lin_k(E,F)$ é tal que $\sigma\cdot\mathfrak t=\mathfrak t$, para todo $\sigma\in G$, mostre que:
\[\Sym(\mathfrak t)=\frac{\vert G\vert}{k!}\sum_{\sigma\in X}\sigma\cdot\mathfrak t,\]
onde $\vert G\vert$ denota o número de elementos de $G$.
\item[(b)] Se $\mathfrak t\in\Lin_k(E,F)$ é tal que $\sigma\cdot\mathfrak t=\sgn(\sigma)\mathfrak t$, para todo $\sigma\in G$, mostre que:
\[\Alt(\mathfrak t)=\frac{\vert G\vert}{k!}\sum_{\sigma\in X}\sgn(\sigma)\,\sigma\cdot\mathfrak t\]
\end{itemize}
(sugestão: $S_k$ é união disjunta das coclasses à esquerda $\sigma G$, com $\sigma$ percorrendo $X$; assim, em uma fórmula onde ocorre
um somatório da forma $\sum_{\sigma\in S_k}$, pode-se trocar $\sigma$ por $\sigma\circ\tau$ e o somatório em questão por $\sum_{\sigma\in X}\sum_{\tau\in G}$).
\end{exercise}

Seja $\Z_+$ o conjunto dos inteiros positivos. No que segue, vamos identificar $S_k$ com o subgrupo do grupo de todas as bijeções $\sigma:\Z_+\to\Z_+$
tais que $\sigma(i)=i$, para todo $i>k$. Assim, para $k\le l$, $S_k$ é um subgrupo de $S_l$. Dada $\sigma\in S_k$ e um número natural $l$, denotamos por
$\sigma^{(l)}\in S^{k+l}$ o {\em deslocamento\/} de $\sigma$ definido por:
\begin{equation}\label{eq:defdeslocamento}
\sigma^{(l)}(i)=i,\ i=1,\ldots,l,\quad\sigma^{(l)}(i)=\sigma(i-l)+l,\ i>l.
\end{equation}
É fácil ver que:
\[(\sigma\circ\tau)^{(l)}=\sigma^{(l)}\circ\tau^{(l)},\quad\sgn(\sigma^{(l)})=\sgn(\sigma),\]
para quaisquer $\sigma,\tau\in S_k$.

\begin{exercise}\label{exe:sigmatensor}
Dados $\mathfrak t\in\Lin_k(E,\R)$, $\mathfrak t'\in\Lin_l(E,\R)$, mostre que:
\[(\sigma\cdot\mathfrak t)\otimes\mathfrak t'=\sigma\cdot(\mathfrak t\otimes\mathfrak t'),\]
para todo $\sigma\in S_k$ e:
\[\mathfrak t\otimes(\sigma\cdot\mathfrak t')=\sigma^{(k)}\cdot(\mathfrak t\otimes\mathfrak t'),\]
para todo $\sigma\in S_l$.
\end{exercise}

\begin{exercise}\label{exe:SymSymAltAlt}
Dados $\mathfrak t\in\Lin_k(E,\R)$, $\mathfrak t'\in\Lin_l(E,\R)$, mostre que:
\[\Sym\!\big(\!\Sym(\mathfrak t)\otimes\mathfrak t'\big)=\Sym(\mathfrak t\otimes\mathfrak t')=\Sym\!\big(\mathfrak t\otimes\Sym(\mathfrak t')\big)\]
e que:
\[\Alt\!\big(\!\Alt(\mathfrak t)\otimes\mathfrak t'\big)=\Alt(\mathfrak t\otimes\mathfrak t')=\Alt\!\big(\mathfrak t\otimes\Alt(\mathfrak t')\big)\]
(sugestão: expanda o $\Sym$ e o $\Alt$ que aparecem dentro do parênteses usando a definição, mas mantenha o $\Sym$ e o $\Alt$ que estão fora do parênteses
do jeito que estão. Use o resultado do Exercício~\ref{exe:sigmatensor} e o resultado do item~(a) do Exercício~\ref{exe:sigmaSymAlt}).
\end{exercise}

Sejam $\mathfrak t\in\Lin_k(E,\R)$, $\mathfrak t'\in\Lin_l(E,\R)$ aplicações multilineares. Evidentemente, se $\mathfrak t$ e $\mathfrak t'$ são
simétricas (resp., anti-simétricas), não é verdade em geral que o produto tensorial $\mathfrak t\otimes\mathfrak t'$ seja simétrico (resp., anti-simétrico).
Se $\mathfrak t$, $\mathfrak t'$ são simétricas, nós definimos o seu {\em produto simétrico\/} $\mathfrak t\vee\mathfrak t'$ fazendo:
\[\mathfrak t\vee\mathfrak t'=\frac{(k+l)!}{k!l!}\Sym(\mathfrak t\otimes\mathfrak t')\in\Lin_{k+l}^{\mathrm s}(E,\R),\]
e, se $\mathfrak t$, $\mathfrak t'$ são anti-simétricas, nós definimos o seu {\em produto exterior\/} $\mathfrak t\wedge\mathfrak t'$ fazendo:
\[\mathfrak t\wedge\mathfrak t'=\frac{(k+l)!}{k!l!}\Alt(\mathfrak t\otimes\mathfrak t')\in\Lin_{k+l}^{\mathrm a}(E,\R).\]
Obviamente, as operações $(\mathfrak t,\mathfrak t')\mapsto\mathfrak t\vee\mathfrak t'$ e $(\mathfrak t,\mathfrak t')\mapsto\mathfrak t\wedge\mathfrak t'$
são bilineares.

\begin{exercise}\label{exe:wedgeveezero}
Mostre que o produto de escalares por aplicações multilineares simétricas (resp., por aplicações multilineares anti-simétricas) é um caso particular
do produto simétrico (resp., do produto exterior), se pensamos nos escalares como aplicações zero-lineares simétricas (resp., anti-simétricas).
\end{exercise}

\begin{exercise}\label{exe:veewedgeassoc}
Sejam $\mathfrak t\in\Lin_k(E,\R)$, $\mathfrak t'\in\Lin_l(E,\R)$, $\mathfrak t''\in\Lin_m(E,\R)$. Se $\mathfrak t$, $\mathfrak t'$ e $\mathfrak t''$ são
simétricas, mostre que:
\[(\mathfrak t\vee\mathfrak t')\vee\mathfrak t''=\frac{(k+l+m)!}{k!l!m!}\Sym(\mathfrak t\otimes\mathfrak t'\otimes\mathfrak t'')=
\mathfrak t\vee(\mathfrak t'\vee\mathfrak t''),\]
e, se $\mathfrak t$, $\mathfrak t'$ e $\mathfrak t''$ são anti-simétricas, mostre que:
\[(\mathfrak t\wedge\mathfrak t')\wedge\mathfrak t''=\frac{(k+l+m)!}{k!l!m!}\Alt(\mathfrak t\otimes\mathfrak t'\otimes\mathfrak t'')=
\mathfrak t\wedge(\mathfrak t'\wedge\mathfrak t'')\]
(sugestão: use o resultado do Exercício~\ref{exe:SymSymAltAlt}).
\end{exercise}

Em vista do resultado do Exercício~\ref{exe:veewedgeassoc}, as operações $\vee$ e $\wedge$ são associativas, de modo que podemos escrever, sem ambigüidade,
produtos simétricos ou exteriores de várias aplicações multilineares sem usar parênteses.
\begin{exercise}
Sejam $\mathfrak t_i\in\Lin_{k_i}(E,\R)$, $i=1,\ldots,n$. Se cada $\mathfrak t_i$ é simétrica, mostre que:
\[\mathfrak t_1\vee\cdots\vee\mathfrak t_n=\frac{(k_1+\cdots+k_n)!}{k_1!\cdots k_n!}\Sym(\mathfrak t_1\otimes\cdots\otimes\mathfrak t_n),\]
e, se cada $\mathfrak t_i$ é anti-simétrica, mostre que:
\[\mathfrak t_1\wedge\cdots\wedge\mathfrak t_n=\frac{(k_1+\cdots+k_n)!}{k_1!\cdots k_n!}\Alt(\mathfrak t_1\otimes\cdots\otimes\mathfrak t_n)\]
(sugestão: use indução em $n$ e o resultado do Exercício~\ref{exe:SymSymAltAlt}).
Conclua que se $\alpha_1,\ldots,\alpha_n\in E^*$ são funcionais lineares sobre $E$ então:
\begin{equation}\label{eq:prodveefunc}
(\alpha_1\vee\cdots\vee\alpha_n)(x_1,\ldots,x_n)=\sum_{\sigma\in S_n}\alpha_1(x_{\sigma(1)})\cdots\alpha_n(x_{\sigma(n)})
\end{equation}
e:
\begin{multline}\label{eq:prodwedgefunc}
(\alpha_1\wedge\cdots\wedge\alpha_n)(x_1,\ldots,x_n)=\sum_{\sigma\in S_n}\sgn(\sigma)\alpha_1(x_{\sigma(1)})\cdots\alpha_n(x_{\sigma(n)})\\
=\det\!\big[\big(\alpha_i(x_j)\big)_{n\times n}\big],
\end{multline}
para todos $x_1,\ldots,x_n\in E$.
\end{exercise}

\begin{exercise}
Sejam $\mathfrak t\in\Lin_k(E,\R)$, $\mathfrak t'\in\Lin_l(E,\R)$. Se $\mathfrak t$ e $\mathfrak t'$ são simétricas, mostre que:
\[\mathfrak t\vee\mathfrak t'=\mathfrak t'\vee\mathfrak t,\]
e se $\mathfrak t$ e $\mathfrak t'$ são anti-simétricas, mostre que:
\[\mathfrak t\wedge\mathfrak t'=(-1)^{kl}\mathfrak t'\wedge\mathfrak t\]
(sugestão: use o resultado do Exercício~\ref{exe:trocattlinha} e o resultado do item~(a) do Exercício~\ref{exe:sigmaSymAlt}).
\end{exercise}

\begin{exercise}\label{exe:XSkSl}
Sejam $k$, $l$ números naturais e $G$ o subgrupo de $S_{k+l}$ definido de uma das formas equivalentes abaixo:
\begin{align*}
G&=\big\{\sigma\in S_{k+l}:\sigma\big(\{1,\ldots,k\}\big)=\{1,\ldots,k\}\big\}\\
&=\big\{\sigma\in S_{k+l}:\sigma\big(\{k+1,\ldots,k+l\}\big)=\{k+1,\ldots,k+l\}\big\}\\
&=\big\{\sigma\circ\tau^{(k)}:\sigma\in S_k,\ \tau\in S_l\big\}\cong S_k\times S_l.
\end{align*}
Mostre que $X$ definido abaixo é um conjunto escolha para o conjunto das coclasses à esquerda $S_{k+l}/G$:
\[X=\big\{\sigma\in S_{k+l}:\sigma(1)<\cdots<\sigma(k),\ \sigma(k+1)<\cdots<\sigma(k+l)\big\}\]
(sugestão: considere a ação de $S_{k+l}$ no conjunto $\wp_k\big(\{1,\ldots,k+l\}\big)$ de todas as partes de tamanho $k$ de $\{1,\ldots,k+l\}$.
Note que a ação é transitiva e que a isotropia do ponto $\{1,\ldots,k\}$ é $G$. Observe que para mostrar que $X$ é um conjunto escolha para $S_{k+l}/G$
você precisa mostrar que $\sigma\mapsto\sigma\big(\{1,\ldots,k\}\big)$ define uma bijeção de $X$ sobre $\wp_k\big(\{1,\ldots,k+l\}\big)$).
\end{exercise}

\begin{exercise}\label{exe:veewedgeeconomico}
Seja $X$ definido como no Exercício~\ref{exe:XSkSl}.
Use o resultado do Exercício~\ref{exe:SymAltG} para concluir que:
\[\mathfrak t\vee\mathfrak t'=\sum_{\sigma\in X}\sigma\cdot(\mathfrak t\otimes\mathfrak t'),\quad
\mathfrak t\in\Lin_k^{\mathrm s}(E,\R),\ \mathfrak t'\in\Lin_l^{\mathrm s}(E,\R),\]
e que:
\[\mathfrak t\wedge\mathfrak t'=\sum_{\sigma\in X}\sgn(\sigma)\,\sigma\cdot(\mathfrak t\otimes\mathfrak t'),\quad
\mathfrak t\in\Lin_k^{\mathrm a}(E,\R),\ \mathfrak t'\in\Lin_l^{\mathrm a}(E,\R).\]
\end{exercise}

\begin{exercise}\label{exe:sigmapullback}
Seja $T:E'\to E$ uma aplicação linear. Mostre que:
\begin{itemize}
\item[(a)] se $\mathfrak t\in\Lin_k(E,F)$ e $\sigma\in S_k$ então:
\[T^*(\sigma\cdot\mathfrak t)=\sigma\cdot(T^*\mathfrak t);\]
\item[(b)] se $\mathfrak t\in\Lin_k(E,F)$ é simétrica (resp., anti-simétrica) então $T^*\mathfrak t$ é simétrica (resp., anti-simétrica);
\item[(c)] se $\mathfrak t\in\Lin_k(E,F)$ então:
\[T^*\big(\!\Sym(\mathfrak t)\big)=\Sym(T^*\mathfrak t),\quad T^*\big(\!\Alt(\mathfrak t)\big)=\Alt(T^*\mathfrak t);\]
\item[(d)] se $\mathfrak t\in\Lin_k^{\mathrm s}(E,\R)$, $\mathfrak t'\in\Lin_l^{\mathrm s}(E,\R)$ então:
\[T^*(\mathfrak t\vee\mathfrak t')=(T^*\mathfrak t)\vee(T^*\mathfrak t')\]
e se $\mathfrak t\in\Lin_k^{\mathrm a}(E,\R)$, $\mathfrak t'\in\Lin_l^{\mathrm a}(E,\R)$ então:
\[T^*(\mathfrak t\wedge\mathfrak t')=(T^*\mathfrak t)\wedge(T^*\mathfrak t')\]
(sugestão para o item~(d): use o resultado do Exercício~\ref{exe:pullbackhomo}).
\end{itemize}
Enuncie e prove os resultados análogos para o push-forward.
\end{exercise}

\begin{defin}\label{thm:defprodint}
Dado $k\ge1$, uma aplicação multilinear $\mathfrak t\in\Lin_k(E,F)$ e um vetor $x\in E$, definimos o {\em produto interior\/} $i_x\mathfrak t\in\Lin_{k-1}(E,F)$
fazendo:
\[(i_x\mathfrak t)(x_1,\ldots,x_{k-1})=\mathfrak t(x,x_1,\ldots,x_{k-1}),\quad x_1,\ldots,x_{k-1}\in E.\]
Se $k=0$, convencionamos $i_x\mathfrak t=0$.
\end{defin}

\begin{lem}\label{thm:lemiveewedge}
Sejam $\mathfrak t\in\Lin_k(E,\R)$, $\mathfrak t'\in\Lin_l(E,\R)$. Se $\mathfrak t$ e $\mathfrak t'$ são simétricos então:
\[i_x(\mathfrak t\vee\mathfrak t')=(i_x\mathfrak t)\vee\mathfrak t'+\mathfrak t\vee(i_x\mathfrak t'),\]
para todo $x\in E$ e se $\mathfrak t$, $\mathfrak t'$ são anti-simétricos então:
\[i_x(\mathfrak t\wedge\mathfrak t')=(i_x\mathfrak t)\wedge\mathfrak t'+(-1)^k\mathfrak t\wedge(i_x\mathfrak t'),\]
para todo $x\in E$.
\end{lem}
\begin{proof}
Podemos supor $k,l\ge1$, caso contrário o resultado é trivial (tenha em mente o resultado do Exercício~\ref{exe:wedgeveezero}).
Vamos demonstrar apenas a fórmula que vale no caso em que $\mathfrak t$, $\mathfrak t'$ são anti-simétricos; o caso simétrico é mais simples
e será deixado a cargo do leitor. Escreva $x_1=x$ e sejam
$x_2,\ldots,x_{k+l}\in E$. Seja $X$ definido como no enunciado do Exercício~\ref{exe:XSkSl}. Em vista do resultado do Exercício~\ref{exe:veewedgeeconomico} temos:
\begin{multline*}
[i_x(\mathfrak t\wedge\mathfrak t')](x_2,\ldots,x_{k+l})=(\mathfrak t\wedge\mathfrak t')(x_1,x_2,\ldots,x_{k+l})\\
=\sum_{\sigma\in X}\sgn(\sigma)\mathfrak t(x_{\sigma(1)},\ldots,x_{\sigma(k)})
\mathfrak t'(x_{\sigma(k+1)},\ldots,x_{\sigma(k+l)}).
\end{multline*}
Evidentemente, para $\sigma\in X$ temos $\sigma(1)=1$ ou $\sigma(k+1)=1$. Escrevemos então $X$ como união disjunta dos conjuntos $X_1$, $X_2$ definidos por:
\[X_1=\big\{\sigma\in X:\sigma(1)=1\big\},\quad X_2=\big\{\sigma\in X:\sigma(k+1)=1\big\}.\]
Escreva $y_i=x_{i+1}$, $i=1,\ldots,k+l-1$. Usando novamente o resultado do Exercício~\ref{exe:veewedgeeconomico}, obtemos:
\begin{align}
\notag[(i_x\mathfrak t)\wedge\mathfrak t'\,]&(x_2,\ldots,x_{k+l})=[(i_x\mathfrak t)\wedge\mathfrak t'\,](y_1,\ldots,y_{k+l-1})\\
&=\sum_{\tau\in Y_1}\sgn(\tau)\mathfrak t(x_1,y_{\tau(1)},\ldots,y_{\tau(k-1)})
\mathfrak t'(y_{\tau(k)},\ldots,y_{\tau(k+l-1)}),\label{eq:Y1}
\end{align}
onde:
\[Y_1=\big\{\tau\in S_{k+l-1}:\tau(1)<\cdots<\tau(k-1),\ \tau(k)<\cdots<\tau(k+l-1)\big\}\]
e:
\begin{align}
\notag[\mathfrak t\wedge(i_x\mathfrak t')]&(x_2,\ldots,x_{k+l})=[\mathfrak t\wedge(i_x\mathfrak t')](y_1,\ldots,y_{k+l-1})\\
&=\sum_{\tau\in Y_2}\sgn(\tau)\mathfrak t(y_{\tau(1)},\ldots,y_{\tau(k)})
\mathfrak t'(x_1,y_{\tau(k+1)},\ldots,y_{\tau(k+l-1)}),\label{eq:Y2}
\end{align}
onde:
\[Y_2=\big\{\tau\in S_{k+l-1}:\tau(1)<\cdots<\tau(k),\ \tau(k+1)<\cdots<\tau(k+l-1)\big\}.\]
A demonstração estará completa se verificarmos que o somatório em \eqref{eq:Y1} é igual a:
\begin{equation}\label{eq:X1}
\sum_{\sigma\in X_1}\sgn(\sigma)\mathfrak t(x_{\sigma(1)},\ldots,x_{\sigma(k)})\mathfrak t'(x_{\sigma(k+1)},\ldots,x_{\sigma(k+l)})
\end{equation}
e que o somatório em \eqref{eq:Y2} é igual a:
\begin{equation}\label{eq:X2}
(-1)^k\sum_{\sigma\in X_2}\sgn(\sigma)\mathfrak t(x_{\sigma(1)},\ldots,x_{\sigma(k)})\mathfrak t'(x_{\sigma(k+1)},\ldots,x_{\sigma(k+l)}).
\end{equation}
É fácil verificar que a aplicação $Y_1\ni\tau\mapsto\sigma=\tau^{(1)}\in X_1$ (recorde \eqref{eq:defdeslocamento})
é uma bijeção e que o $\tau$-ésimo termo do somatório \eqref{eq:Y1} é igual ao $\sigma$-ésimo termo do somatório \eqref{eq:X1}. É também fácil
verificar que a aplicação:
\[Y_2\ni\tau\longmapsto\sigma=\tau^{(1)}\circ(1\ 2\ \cdots\ k+1)\in X_2\]
é uma bijeção e que o $\tau$-ésimo termo do somatório \eqref{eq:Y2} é igual a $(-1)^k$ vezes o $\sigma$-ésimo termo do somatório \eqref{eq:X2},
onde $(1\ 2\ \cdots\ k+1)$ denota o ciclo:
\[(1\ 2\ \cdots\ k+1):1\mapsto2\mapsto3\mapsto\cdots\mapsto k+1\mapsto1,\quad i\mapsto i,\ i\ge k+2.\]
Isso completa a demonstração.
\end{proof}

\subsection{Propriedades universais dos produtos exterior e simétrico}\label{sub:univextsim}
Seja $E$ um espaço vetorial. Dado um número natural $k$, então uma {\em $k$-ésima potência simétrica\/} para $E$ consiste
de um espaço vetorial $Z$ e de uma aplicação multilinear simétrica $\tau\in\Lin_k^{\mathrm s}(E,Z)$ tais que
dado um espaço vetorial qualquer $F$ e uma aplicação multilinear
simétrica $\mathfrak t\in\Lin_k^{\mathrm s}(E,F)$, existe uma única transformação linear $T:Z\to F$ tal que $T\circ\tau=\mathfrak t$, i.e., tal que
o diagrama:
\[\xymatrix{%
E^k\ar[dr]^{\mathfrak t}\ar[d]_\tau\\
Z\ar[r]_T&F}\]
comuta. Similarmente, uma {\em $k$-ésima potência exterior\/} para $E$ consiste de um espaço vetorial $Z$ e de uma aplicação multilinear anti-simétrica
$\tau$ em $\Lin_k^{\mathrm a}(E,Z)$ tais que dado um espaço vetorial qualquer $F$ e uma aplicação multilinear
anti-simétrica $\mathfrak t\in\Lin_k^{\mathrm a}(E,F)$, existe uma única transformação linear $T:Z\to F$ tal que $T\circ\tau=\mathfrak t$.

\begin{exercise}[unicidade das $k$-ésimas potências simétrica e exterior]
Sejam $(Z,\tau)$, $(Z',\tau')$ duas $k$-ésimas potências simétricas (resp., duas $k$-ésimas potências exteriores) de um espaço vetorial $E$.
Seja $T:Z\to Z'$ a única aplicação linear tal que $T\circ\tau=\tau'$, i.e., tal que o diagrama:
\[\xymatrix{%
&E^k\ar[dl]_\tau\ar[dr]^{\tau'}\\
Z\ar[rr]_T&&Z'}\]
comuta. Mostre que $T$ é um isomorfismo (sugestão: faça o mesmo que você fez para resolver o Exercício~\ref{exe:uniquetensorprod}).
\end{exercise}

\begin{rem}
É possível mostrar que qualquer espaço vetorial $E$ admite uma $k$-ésima potência simétrica e uma $k$-ésima potência exterior. Elas podem ser construídas, por exemplo,
considerando-se um certo quociente da $k$-ésima potência tensorial $\bigotimes_kE$. Essa construção é simples, mas está fora do escopo do nosso curso.
Para espaços de dimensão finita veremos logo abaixo que a $k$-ésima potência simétrica pode ser obtida como um espaço de aplicações multilineares simétricas
e que a $k$-ésima potência exterior pode ser obtida como um espaço de aplicações multilineares anti-simétricas.
\end{rem}

Se $(Z,\tau)$ é uma $k$-ésima potência simétrica do espaço $E$, escreve-se normalmente $Z=\bigvee_kE$ e:
\[\tau(x_1,\ldots,x_k)=x_1\vee\cdots\vee x_k,\quad x_1,\ldots,x_k\in E.\]
A definição da $k$-ésima potência simétrica nos diz então que para qualquer espaço vetorial $F$ e qualquer aplicação multilinear simétrica
$\mathfrak t\in\Lin_k^{\mathrm s}(E,F)$ existe uma única transformação linear $T:\bigvee_kE\to F$ tal que:
\[T(x_1\vee\cdots\vee x_k)=\mathfrak t(x_1,\ldots,x_k),\]
para quaisquer $x_1,\ldots,x_k\in E$, i.e., tal que o diagrama:
\[\xymatrix@C+10pt{%
E^k\ar[dr]^{\mathfrak t}\ar[d]_\vee\\
\bigvee_kE\ar[r]_-T&F}\]
comuta. Similarmente, se $(Z,\tau)$ é uma $k$-ésima potência exterior do espaço $E$, escreve-se normalmente $Z=\bigwedge_kE$ e:
\[\tau(x_1,\ldots,x_k)=x_1\wedge\cdots\wedge x_k,\quad x_1,\ldots,x_k\in E.\]
A definição da $k$-ésima potência exterior nos diz que para qualquer espaço vetorial $F$ e qualquer aplicação multilinear anti-simétrica
$\mathfrak t\in\Lin_k^{\mathrm a}(E,F)$ existe uma única transformação linear $T:\bigwedge_kE\to F$ tal que:
\[T(x_1\wedge\cdots\wedge x_k)=\mathfrak t(x_1,\ldots,x_k),\]
para quaisquer $x_1,\ldots,x_k\in E$, i.e., tal que o diagrama:
\[\xymatrix@C+10pt{%
E^k\ar[dr]^{\mathfrak t}\ar[d]_\wedge\\
\bigwedge_kE\ar[r]_-T&F}\]
comuta.

\begin{exercise}[caracterização da $k$-ésima potência simétrica em termos de bases]\label{exe:simetricobases}
Sejam $E$ um espaço vetorial e $(e_\lambda)_{\lambda\in\Lambda}$ uma base de $E$, sendo $\Lambda$ munido de uma ordem total $\le$.
Dados um espaço vetorial $Z$ e uma aplicação multilinear simétrica $\tau\in\Lin_k^{\mathrm s}(E,Z)$, mostre que as seguintes condições são equivalentes:
\begin{itemize}
\item[(a)] a família:
\[\tau(e_{\lambda_1},\ldots,e_{\lambda_k}),\quad\lambda_1,\ldots,\lambda_k\in\Lambda,\ \lambda_1\le\cdots\le\lambda_k,\]
é uma base de $Z$;
\item[(b)] $(Z,\tau)$ é uma $k$-ésima potência simétrica para $E$
\end{itemize}
(sugestão: considere a mesma sugestão que foi dada para o Exercício~\ref{exe:tensorbases}, mas use o resultado do item~(a) do
Exercício~\ref{exe:multlinebasessimetria} em vez do resultado do Exercício~\ref{exe:multlinbase}).
\end{exercise}

\begin{exercise}[caracterização da $k$-ésima potência exterior em termos de bases]\label{exe:asimetricobases}
Sejam $E$ um espaço vetorial e $(e_\lambda)_{\lambda\in\Lambda}$ uma base de $E$, sendo $\Lambda$ munido de uma ordem total $\le$.
Dados um espaço vetorial $Z$ e uma aplicação multilinear anti-simétrica $\tau\in\Lin_k^{\mathrm a}(E,Z)$, mostre que as seguintes condições são equivalentes:
\begin{itemize}
\item[(a)] a família:
\[\tau(e_{\lambda_1},\ldots,e_{\lambda_k}),\quad\lambda_1,\ldots,\lambda_k\in\Lambda,\ \lambda_1<\cdots<\lambda_k,\]
é uma base de $Z$;
\item[(b)] $(Z,\tau)$ é uma $k$-ésima potência exterior para $E$
\end{itemize}
(sugestão: considere a mesma sugestão que foi dada para o Exercício~\ref{exe:tensorbases}, mas use o resultado do item~(b) do
Exercício~\ref{exe:multlinebasessimetria} em vez do resultado do Exercício~\ref{exe:multlinbase}).
\end{exercise}

\begin{exercise}\label{exe:Linsbigvee}
Sejam $E$ um espaço vetorial de dimensão finita e $(e_r)_{r=1}^n$ uma base de $E$. Denote por $(e_r^*)_{r=1}^n$ a base dual de $(e_r)_{r=1}^n$.
Dada uma $k$-upla $(r_1,\ldots,r_k)$ de elementos do conjunto $\{1,\ldots,n\}$, i.e., uma função $r:\{1,\ldots,k\}\to\{1,\ldots,n\}$, escrevemos:
\[\langle r\rangle=\prod_{i=1}^n\vert r^{-1}(i)\vert!,\]
onde $\vert C\vert$ denota o número de elementos de um conjunto $C$. Note que $\langle r\rangle$ é precisamente o número de elementos do conjunto:
\[\big\{\sigma\in S_k:r\circ\sigma=r\big\}.\]
\begin{itemize}
\item[(a)] Dados $r_1,\ldots,r_k,s_1,\ldots,s_k\in\{1,\ldots,n\}$ tais que:
\[r_1\le\cdots\le r_k,\quad s_1\le\cdots\le s_k,\]
mostre que:
\[(e_{r_1}^*\vee\cdots\vee e_{r_k}^*)(e_{s_1},\ldots,e_{s_k})=\langle r\rangle\ne0,\]
se $r_1=s_1$, \dots, $r_k=s_k$ e:
\[(e_{r_1}^*\vee\cdots\vee e_{r_k}^*)(e_{s_1},\ldots,e_{s_k})=0,\]
caso contrário (sugestão: use a fórmula \eqref{eq:prodveefunc}).
\item[(b)] Use o resultado do item~(a) para mostrar que a família:
\begin{equation}\label{eq:basetensorsim}
e_{r_1}^*\vee\cdots\vee e_{r_k}^*\in\Lin_k^{\mathrm s}(E,\R),\quad1\le r_1\le\cdots\le r_k\le n,
\end{equation}
é linearmente independente (sugestão: avalie uma combinação linear de \eqref{eq:basetensorsim} em $(e_{s_1},\ldots,e_{s_k})$, com $1\le s_1\le\cdots\le s_k\le n$).
\item[(c)] Mostre que, para qualquer $\mathfrak t\in\Lin_k^{\mathrm s}(E,\R)$, temos:
\begin{equation}\label{eq:fraktbasesim}
\mathfrak t=\sum_{1\le r_1\le\cdots\le r_k\le n}\frac1{\langle r\rangle}\,\mathfrak t(e_{r_1},\ldots,e_{r_k})(e_{r_1}^*\vee\cdots\vee e_{r_k}^*)
\end{equation}
(sugestão: mostre que ambos os lados da igualdade \eqref{eq:fraktbasesim} são aplicações multilineares si\-mé\-tri\-cas que coincidem
sobre $k$-uplas da forma $(e_{s_1},\ldots,e_{s_k})$, com $1\le s_1\le\cdots\le s_k\le n$. Obtenha sua conclusão usando o resultado do item~(a) do
Exercício~\ref{exe:multlinebasessimetria}).
\item[(d)] Use o resultado do item~(c) para concluir que a família \eqref{eq:basetensorsim} é uma base de $\Lin_k^{\mathrm s}(E,\R)$. Use então o resultado
do Exercício~\ref{exe:simetricobases} e conclua também que:
\[(E^*)^k\ni(\alpha_1,\ldots,\alpha_k)\longmapsto\alpha_1\vee\cdots\vee\alpha_k\in\Lin_k^{\mathrm s}(E,\R)\]
nos dá uma $k$-ésima potência simétrica de $E$.
\end{itemize}
\end{exercise}

Em vista do resultado do item~(d) do Exercício~\ref{exe:Linsbigvee}, escreveremos:
\[\Lin_k^{\mathrm s}(E,\R)=\bigvee_kE^*\]
{\em quando o espaço $E$ tiver dimensão finita}. Note que o resultado do item~(c) do Exercício~\ref{exe:Linsbigvee}
nos diz que as coordenadas de $\mathfrak t\in\Lin_k^{\mathrm s}(E,\R)=\bigvee_kE^*$ na base \eqref{eq:basetensorsim} são:
\[\frac1{\langle r\rangle}\,\mathfrak t(e_{r_1},\ldots,e_{r_k}),\quad1\le r_1\le\cdots\le r_k\le n.\]
Note também que\footnote{%
A dimensão do espaço vetorial $\bigvee_kE^*$ é igual ao número de funções crescentes $r:\{1,\ldots,k\}\to\{1,\ldots,n\}$. Uma tal função fica caracterizada
pelas cardinalidades dos conjuntos $r^{-1}(i)$, $i=1,\ldots,n$, que constituem uma solução da equação $u_1+\cdots+u_n=k$, com $u_1$, \dots, $u_n$
números naturais. O número de soluções dessa equação é igual ao número de permutações da string de caracteres $\mathrm o\mathrm o\cdots\mathrm o||\cdots|$,
sendo $k$ ocorrências do símbolo `$\mathrm o$' e $n-1$ ocorrências do símbolo `$|$'. Os símbolos `$|$' devem ser entendidos como delimitadores que separam
$k$ ocorrências do símbolo `$\mathrm o$' em $n$ blocos, sendo os números naturais $u_i$, $i=1,\ldots,n$ entendidos como o número de ocorrências do símbolo
`$\mathrm o$' em cada bloco.}:
\[\Dim\!\big(\!\Lin_k^{\mathrm s}(E,\R)\big)=\Dim\!\Big(\bigvee_kE^*\Big)=\binom{n+k-1}k,\]
onde $n=\Dim(E)$ (usamos a convenção de que o número combinatório $\binom ab$ é zero se $b>a\ge0$, o que faz com que a fórmula acima seja correta mesmo
para $n=0$; se $n=k=0$, no entanto, o valor correto para a dimensão é $1$).
Levando em conta a identificação natural entre um espaço (de dimensão finita) e seu bidual, podemos escrever também:
\[\Lin_k^{\mathrm s}(E^*,\R)=\bigvee_kE,\]
quando $E$ tiver dimensão finita.

\begin{exercise}\label{exe:Linabigwedge}
Sejam $E$ um espaço vetorial de dimensão finita e $(e_r)_{r=1}^n$ uma base de $E$. Denote por $(e_r^*)_{r=1}^n$ a base dual de $(e_r)_{r=1}^n$.
\begin{itemize}
\item[(a)] Dados $r_1,\ldots,r_k,s_1,\ldots,s_k\in\{1,\ldots,n\}$ tais que:
\[r_1<\cdots<r_k,\quad s_1<\cdots<s_k,\]
mostre que:
\[(e_{r_1}^*\wedge\cdots\wedge e_{r_k}^*)(e_{s_1},\ldots,e_{s_k})=1,\]
se $r_1=s_1$, \dots, $r_k=s_k$ e:
\[(e_{r_1}^*\wedge\cdots\wedge e_{r_k}^*)(e_{s_1},\ldots,e_{s_k})=0,\]
caso contrário (sugestão: use a fórmula \eqref{eq:prodwedgefunc}).
\item[(b)] Use o resultado do item~(a) para mostrar que a família:
\begin{equation}\label{eq:basetensorasim}
e_{r_1}^*\wedge\cdots\wedge e_{r_k}^*\in\Lin_k^{\mathrm a}(E,\R),\quad1\le r_1<\cdots<r_k\le n,
\end{equation}
é linearmente independente (sugestão: avalie uma combinação linear de \eqref{eq:basetensorasim} em $(e_{s_1},\ldots,e_{s_k})$, com $1\le s_1<\cdots<s_k\le n$).
\item[(c)] Mostre que, para qualquer $\mathfrak t\in\Lin_k^{\mathrm a}(E,\R)$, temos:
\begin{equation}\label{eq:fraktbaseasim}
\mathfrak t=\sum_{1\le r_1<\cdots<r_k\le n}\mathfrak t(e_{r_1},\ldots,e_{r_k})(e_{r_1}^*\wedge\cdots\wedge e_{r_k}^*)
\end{equation}
(sugestão: mostre que ambos os lados da igualdade \eqref{eq:fraktbaseasim} são aplicações multilineares anti-si\-mé\-tri\-cas que coincidem
sobre $k$-uplas da forma $(e_{s_1},\ldots,e_{s_k})$, com $1\le s_1<\cdots<s_k\le n$. Obtenha sua conclusão usando o resultado do item~(b) do
Exercício~\ref{exe:multlinebasessimetria}).
\item[(d)] Use o resultado do item~(c) para concluir que a família \eqref{eq:basetensorasim} é uma base de $\Lin_k^{\mathrm a}(E,\R)$. Use então o resultado
do Exercício~\ref{exe:asimetricobases} e conclua também que:
\[(E^*)^k\ni(\alpha_1,\ldots,\alpha_k)\longmapsto\alpha_1\wedge\cdots\wedge\alpha_k\in\Lin_k^{\mathrm a}(E,\R)\]
nos dá uma $k$-ésima potência exterior de $E$.
\end{itemize}
\end{exercise}

Em vista do resultado do item~(d) do Exercício~\ref{exe:Linabigwedge}, escreveremos:
\[\Lin_k^{\mathrm a}(E,\R)=\bigwedge_kE^*\]
{\em quando o espaço $E$ tiver dimensão finita}. Note que o resultado do item~(c) do Exercício~\ref{exe:Linabigwedge}
nos diz que as coordenadas de $\mathfrak t\in\Lin_k^{\mathrm a}(E,\R)=\bigwedge_kE^*$ na base \eqref{eq:basetensorasim} são:
\[\mathfrak t(e_{r_1},\ldots,e_{r_k}),\quad1\le r_1<\cdots<r_k\le n.\]
Note também que:
\[\Dim\!\big(\!\Lin_k^{\mathrm a}(E,\R)\big)=\Dim\!\Big(\bigwedge_kE^*\Big)=\binom nk,\]
onde $n=\Dim(E)$ (em particular, note que $\Lin_k^{\mathrm a}(E,\R)$ é o espaço nulo para $k>n$).
Levando em conta a identificação natural entre um espaço (de dimensão finita) e seu bidual, podemos escrever também:
\[\Lin_k^{\mathrm a}(E^*,\R)=\bigwedge_kE,\]
quando $E$ tiver dimensão finita.

\begin{exercise}
Seja $E$ um espaço vetorial de dimensão finita $n$. Como o espaço $\Lin_n^{\mathrm a}(E,\R)$ tem dimensão $\binom nn=1$, temos que para qualquer aplicação
linear $T:E\to E$, o pull-back:
\[T^*:\Lin_n^{\mathrm a}(E,\R)\longrightarrow\Lin_n^{\mathrm a}(E,\R)\]
é dado pela multiplicação por um escalar. Mostre que esse escalar é precisamente o determinante de $T$ (i.e., o determinante da matriz que representa
$T$ com respeito a qualquer base de $E$); em outras palavras, mostre que:
\[T^*\mathfrak t=\det(T)\mathfrak t,\]
para todo $\mathfrak t\in\Lin_n^{\mathrm a}(E,\R)$ (sugestão: se $(e_r)_{r=1}^n$ denota uma base de $E$ e $(e_r^*)_{r=1}^n$ denota sua base dual
então $\mathfrak t=e_1^*\wedge\cdots\wedge e_n^*$ é uma base do espaço unidimensional $\Lin_n^{\mathrm a}(E,\R)$. Note que se $c\in\R$ é o escalar
tal que $T^*\mathfrak t$ é igual a $c\mathfrak t$ então $c=(T^*\mathfrak t)(e_1,\ldots,e_n)$. Obtenha sua conclusão usando a fórmula \eqref{eq:prodwedgefunc}).
\end{exercise}

\end{section}

\begin{section}{Diferencial exterior}
\label{sec:difext}

Neste apêndice, denotaremos a diferencial comum de uma aplicação usando o símbolo
``$\Dd$'' em vez de ``$\dd$'', já que o símbolo ``$\dd$'' é tipicamente usado para a
diferencial exterior. Em outras partes do texto, continuaremos usando ``$\dd$'' para a diferencial comum, a menos que haja risco de confusão com a diferencial exterior.
Dados espaços vetoriais reais de dimensão finita $E$, $F$, um subconjunto
aberto $U$ de $E$ e uma aplicação $\phi:U\to F$ de classe $C^\infty$ então a diferencial comum\footnote{%
Como sempre, trabalharemos sempre com aplicações de classe $C^\infty$, embora essa hipótese seja exagerada.}:
\[\Dd\phi:U\ni x\longmapsto\Dd\phi(x)=\Dd\phi_x\in\Lin(E,F)\]
satisfaz:
\[\Dd\phi_x(v)=\frac{\partial\phi}{\partial v}(x),\quad x\in U,\ v\in E,\]
onde $\frac{\partial\phi}{\partial v}(x)$ denota a derivada direcional de $\phi$ no ponto $x\in U$, na direção do vetor $v\in E$, i.e.:
\[\frac{\partial\phi}{\partial v}(x)=\lim_{s\to0}\frac{\phi(x+sv)-\phi(x)}s.\]

\begin{exercise}[derivada direcional comuta com operações lineares]\label{exe:oplinderdir}
Sejam $E$, $F$, $F'$ espaços vetoriais reais de dimensão finita, $U$ um aberto de $E$, $\phi:U\to F$ uma aplicação de classe $C^\infty$ e $L:F\to F'$ uma
aplicação linear. Mostre que:
\[\frac{\partial(L\circ\phi)}{\partial v}(x)=L\Big(\frac{\partial\phi}{\partial v}(x)\Big),\]
para todos $x\in U$, $v\in E$ (sugestão: use a regra da cadeia, tendo em mente que $\Dd L\big(\phi(x)\big)=L$).
\end{exercise}

\begin{exercise}[regra do produto]\label{exe:regraproduto}
Sejam $E$, $F_1$, \dots, $F_k$, $F'$ espaços vetoriais reais de dimensão finita, $U$ um subconjunto
aberto de $E$, $\phi_i:U\to F_i$, $i=1,\ldots,k$ aplicações de classe $C^\infty$
e $\mathfrak t:F_1\times\cdots\times F_k\to F'$ uma aplicação multilinear. Mostre que:
\begin{multline*}
\Big(\frac{\partial}{\partial v}\big[\mathfrak t\circ(\phi_1,\ldots,\phi_k)\big]\Big)(x)\\
=\sum_{i=1}^k\mathfrak t\Big(\phi_1(x),\ldots,\phi_{i-1}(x),\frac{\partial\phi_i}{\partial v}(x),\phi_{i+1}(x),\ldots,\phi_k(x)\Big),
\end{multline*}
para todos $x\in U$, $v\in E$ (sugestão: use a regra da cadeia, tendo em mente que:
\[\Dd\mathfrak t(y_1,\ldots,y_k)(u_1,\ldots,u_k)=\sum_{i=1}^k\mathfrak t(y_1,\ldots,y_{i-1},u_i,y_{i+1},\ldots,y_k),\]
para todos $y_i,u_i\in F_i$, $i=1,\ldots,k$).
\end{exercise}

Sejam $E$, $E_1$, \dots, $E_k$, $F$ espaços vetoriais reais de dimensão finita, $U$ um subconjunto aberto de $E$ e:
\[\mathfrak t:U\subset E\longrightarrow\Lin(E_1,\ldots,E_k;F)\]
uma aplicação de classe $C^\infty$. A derivada comum $\Dd\mathfrak t$ é uma aplicação:
\[\Dd\mathfrak t:U\subset E\longrightarrow\Lin\!\big(E,\Lin(E_1,\ldots,E_k;F)\big).\]
Nós adotaremos a identificação:
\[\Lin\!\big(E,\Lin(E_1,\ldots,E_k;F)\big)\cong\Lin(E,E_1,\ldots,E_k;F)\]
de modo que $\Dd\mathfrak t$ identifica-se com uma aplicação:
\[\Dd\mathfrak t:U\subset E\longrightarrow\Lin(E,E_1,\ldots,E_k;F)\]
dada por:
\[\Dd\mathfrak t_x(v,v_1,\ldots,v_k)=\frac{\partial\mathfrak t}{\partial v}(x)(v_1,\ldots,v_k),\]
para todos $x\in U$, $v\in E$, $v_i\in E_i$, $i=1,\ldots,k$. Note que, já que a avaliação em $(v_1,\ldots,v_k)$ é uma operação linear, temos
(em vista do resultado do Exercício~\ref{exe:oplinderdir}) que $\Dd\mathfrak t_x(v,v_1,\ldots,v_k)$ coincide com a derivada direcional no ponto $x\in U$,
na direção de $v\in E$, da função $U\ni x\mapsto\mathfrak t(x)(v_1,\ldots,v_k)$, i.e.:
\[\Dd\mathfrak t_x(v,v_1,\ldots,v_k)=\lim_{s\to0}\frac{\mathfrak t(x+sv)(v_1,\ldots,v_k)-\mathfrak t(x)(v_1,\ldots,v_k)}s,\]
para todos $x\in U$, $v\in E$, $v_i\in E_i$, $i=1,\ldots,k$.

\medskip

No que segue, se $\mathfrak t$ é uma função tomando valores em $\Lin_k(E,F)$ e $\sigma\in S_k$ é uma permutação, escrevemos $\sigma\cdot\mathfrak t$
para denotar a aplicação com o mesmo domínio que $\mathfrak t$ tal que:
\[(\sigma\cdot\mathfrak t)(x)=\sigma\cdot\mathfrak t(x),\]
para todo $x$ no domínio
de $\mathfrak t$. Similarmente, denotamos por $\Alt(\mathfrak t)$ (resp., $\Sym(\mathfrak t)$) a aplicação $x\mapsto\Alt\!\big(\mathfrak t(x)\big)$
(resp., a aplicação $x\mapsto\Sym\!\big(\mathfrak t(x)\big)$).
Se $\mathfrak t$ toma valores em $\Lin_k(E,\R)$, $\mathfrak t'$ toma valores em $\Lin_l(E,\R)$ e $\mathfrak t$, $\mathfrak t'$
têm o mesmo domínio então denotamos por $\mathfrak t\otimes\mathfrak t'$ a aplicação a valores em $\Lin_{k+l}(E,\R)$, com o mesmo domínio que $\mathfrak t$
e $\mathfrak t'$, definida por:
\[(\mathfrak t\otimes\mathfrak t')(x)=\mathfrak t(x)\otimes\mathfrak t'(x),\]
para todo $x$ no domínio de $\mathfrak t$ e
$\mathfrak t'$. Convenções análogas são adotadas para os produtos exterior $\wedge$ e simétrico $\vee$.

\begin{exercise}\label{exe:sigmaDd}
Sejam $E$, $F$ espaços vetoriais reais de dimensão finita, $U$ um subconjunto aberto de $E$ e $\mathfrak t:U\to\Lin_k(E,F)$ uma aplicação de classe $C^\infty$.
Dada uma permutação $\sigma\in S_k$, mostre que (recorde \eqref{eq:defdeslocamento}):
\[\Dd(\sigma\cdot\mathfrak t)=\sigma^{(1)}\cdot\Dd\mathfrak t\]
(sugestão: note que $\sigma$ é uma operação linear e, em vista do resultado do Exercício~\ref{exe:oplinderdir},
comuta com a derivada direcional $\frac{\partial}{\partial v}$, i.e.:
\[\frac{\partial(\sigma\cdot\mathfrak t)}{\partial v}(x)=\sigma\cdot\frac{\partial\mathfrak t}{\partial v}(x),\]
para quaisquer $x\in U$, $v\in E$).
\end{exercise}

\begin{exercise}\label{exe:Schwarz}
Sejam $E$, $E_1$, \dots, $E_k$, $F$ espaços vetoriais reais de dimensão finita, $U$ um subconjunto aberto de $E$ e
$\mathfrak t:U\to\Lin(E_1,\ldots,E_k;F)$ uma aplicação de classe $C^\infty$. Mostre que $\Dd^2\mathfrak t=\Dd\Dd\mathfrak t$ é simétrica nas suas duas primeiras
variáveis, i.e.:
\[\Dd^2\mathfrak t_x(u,v,w_1,\ldots,w_k)=\Dd^2\mathfrak t_x(v,u,w_1,\ldots,w_k),\]
para todos $x\in U$, $u,v\in E$, $w_i\in E_i$, $i=1,\ldots,k$ (sugestão: pelo Teorema de Schwarz, as derivadas direcionais $\frac{\partial}{\partial u}$
e $\frac{\partial}{\partial v}$ comutam).
\end{exercise}

\begin{exercise}\label{exe:Dotimes}
Sejam $E$ um espaço vetorial real de dimensão finita, $U$ um subconjunto aberto de $E$ e $\mathfrak t:U\to\Lin_k(E,\R)$,
$\mathfrak t':U\to\Lin_l(E,\R)$ aplicações de classe $C^\infty$. Mostre que:
\[\Dd(\mathfrak t\otimes\mathfrak t')=(\Dd\mathfrak t)\otimes\mathfrak t'+\sigma\cdot\big(\mathfrak t\otimes(\Dd\mathfrak t')\big),\]
onde $\sigma\in S_{k+l+1}$ denota o ciclo:
\[(1\ 2\ \cdots\ k+1):1\mapsto2\mapsto3\mapsto\cdots\mapsto k+1\mapsto1,\quad i\mapsto i,\ i\ge k+2\]
(sugestão: comece usando o resultado do Exercício~\ref{exe:regraproduto} para mostrar que:
\[\frac{\partial(\mathfrak t\otimes\mathfrak t')}{\partial v}(x)=\frac{\partial\mathfrak t}{\partial v}(x)\otimes\mathfrak t'(x)+
\mathfrak t(x)\otimes\frac{\partial\mathfrak t'}{\partial v}(x),\]
para quaisquer $x\in U$, $v\in E$).
\end{exercise}

\begin{defin}
Sejam $E$, $E'$, $F$ espaços vetoriais reais de dimensão finita, $U$ um subconjunto aberto de $E$, $U'$ um subconjunto aberto de $E'$ e $\phi:U'\to U$
uma aplicação de classe $C^\infty$. Dada uma aplicação $\mathfrak t:U\to\Lin_k(E,F)$, então o {\em pull-back\/} de $\mathfrak t$ por $\phi$ é a aplicação
$\phi^*\mathfrak t:U'\to\Lin_k(E',F)$ definida por:
\[(\phi^*\mathfrak t)(x)=\Dd\phi_x^*\big[\mathfrak t\big(\phi(x)\big)\big],\quad x\in U',\]
ou seja:
\begin{multline*}
(\phi^*\mathfrak t)(x)(v_1,\ldots,v_k)=\mathfrak t\big(\phi(x)\big)\big(\Dd\phi_x(v_1),\ldots,\Dd\phi_x(v_k)\big),\\
x\in U',\ v_1,\ldots,v_k\in E'.
\end{multline*}
\end{defin}

\begin{exercise}\label{exe:sigmapullback2}
Sejam $E$, $E'$ espaços vetoriais reais de dimensão finita, $U$ um subconjunto aberto de $E$, $U'$ um subconjunto aberto de $E'$
e $\phi:U'\to U$ uma aplicação de classe $C^\infty$. Dada uma aplicação $\mathfrak t:U\to\Lin_k(E,F)$, mostre que:
\begin{itemize}
\item[(a)] se $\sigma\in S_k$ então $\sigma\cdot(\phi^*\mathfrak t)=\phi^*(\sigma\cdot\mathfrak t)$;
\item[(b)] se $\mathfrak t$ toma valores em $\Lin_k^{\mathrm a}(E,F)$ (resp., em $\Lin_k^{\mathrm s}(E,F)$) então
$\phi^*\mathfrak t$ toma valores em $\Lin_k^{\mathrm a}(E',F)$ (resp., em $\Lin_k^{\mathrm s}(E',F)$);
\item[(c)] $\Alt(\phi^*\mathfrak t)=\phi^*\Alt(\mathfrak t)$ e $\Sym(\phi^*\mathfrak t)=\phi^*\Sym(\mathfrak t)$
\end{itemize}
(sugestão: use o resultado do Exercício~\ref{exe:sigmapullback}).
\end{exercise}

\begin{exercise}\label{exe:pullbacksmoothvecspa}
Sejam $E$, $E'$, $F$ espaços vetoriais reais de dimensão finita, $U$ um subconjunto aberto de $E$ e $U'$ um subconjunto aberto de $E'$.
Mostre que se:
\[\phi:U'\longrightarrow U,\quad\mathfrak t:U\longrightarrow\Lin_k(E,F)\]
são aplicações de classe $C^\infty$ então também $\phi^*\mathfrak t:U'\to\Lin_k(E',F)$ é de classe $C^\infty$ (sugestão:
use o fato que a aplicação:
\[\rho:\Lin_k(E,F)\times\Lin(E',E)^k\longrightarrow\Lin_k(E',F)\]
definida por:
\begin{multline*}
\rho(\lambda,T_1,\ldots,T_k)(u_1,\ldots,u_k)=\lambda\big(T_1(u_1),\ldots,T_k(u_k)\big),\\
T_1,\ldots,T_k\in\Lin(E',E),\ \lambda\in\Lin_k(E,F),\ u_1,\ldots,u_k\in E',
\end{multline*}
é multilinear e portanto de classe $C^\infty$).
\end{exercise}

\begin{exercise}\label{exe:Dpullback}
Sejam $E$, $E'$, $F$ espaços vetoriais reais de dimensão finita, $U$ um subconjunto aberto de $E$ e $U'$ um subconjunto aberto de $E'$.
Mostre que se:
\[\phi:U'\longrightarrow U,\quad\mathfrak t:U\longrightarrow\Lin_k(E,F)\]
são aplicações de classe $C^\infty$ então a diferencial de $\phi^*\mathfrak t$ é dada por:
\begin{multline*}
\Dd(\phi^*\mathfrak t)_x(v_1,v_2,\ldots,v_{k+1})=(\phi^*\Dd\mathfrak t)(x)(v_1,v_2,\ldots,v_{k+1})\\
+\sum_{i=2}^{k+1}\mathfrak t\big(\phi(x)\big)\big(\Dd\phi_x(v_2),\ldots,\Dd\phi_x(v_{i-1}),\Dd^2\phi_x(v_1,v_i),\\
\Dd\phi_x(v_{i+1}),\ldots,\Dd\phi_x(v_{k+1})\big),
\end{multline*}
para quaisquer $x\in U'$, $v_1,\ldots,v_{k+1}\in E'$ (sugestão: aplique $\frac{\partial}{\partial v_1}$ aos dois lados da igualdade:
\[(\phi^*\mathfrak t)(x)(v_2,\ldots,v_{k+1})=\mathfrak t\big(\phi(x)\big)\big(\Dd\phi_x(v_2),\ldots,\Dd\phi_x(v_{k+1})\big).\]
Use a regra da cadeia e o resultado do Exercício~\ref{exe:regraproduto}, tendo em mente que
a aplicação:
\[\Lin_k(E,F)\times E^k\ni(\lambda,u_1,\ldots,u_k)\longmapsto\lambda(u_1,\ldots,u_k)\in F\]
é multilinear).
\end{exercise}

\begin{defin}
Sejam $E$ um espaço vetorial real de dimensão finita e $U$ um subconjunto aberto de $E$. Uma {\em $k$-forma diferencial\/} em $U$ é uma aplicação:
\[\omega:U\longrightarrow\Lin_k^{\mathrm a}(E,\R).\]
\end{defin}
Em vista do resultado do item~(b) do Exercício~\ref{exe:sigmapullback2},
o pull-back de uma $k$-forma diferencial por uma aplicação de classe $C^\infty$ é uma $k$-forma diferencial.

\begin{defin}
Sejam $E$ um espaço vetorial real de dimensão finita, $U$ um subconjunto aberto de $E$ e $\omega$ uma $k$-forma diferencial em $U$
de classe $C^\infty$. A {\em diferencial exterior\/}
de $\omega$, denotada por $\dd\omega$, é a $(k+1)$-forma diferencial em $U$ definida por:
\[\dd\omega=(k+1)\Alt(\Dd\omega).\]
\end{defin}

\begin{exercise}
Sejam $E$ um espaço vetorial real de dimensão finita, $U$ um subconjunto aberto de $E$ e $\omega$ uma $k$-forma diferencial em $U$
de classe $C^\infty$. Mostre que:
\[\dd\omega_x(v_1,v_2,\ldots,v_{k+1})=\sum_{i=1}^{k+1}(-1)^{i+1}\Dd\omega_x(v_i,v_1,\ldots,v_{i-1},v_{i+1},\ldots,v_{k+1}),\]
para quaisquer $x\in U$, $v_1,\ldots,v_{k+1}\in E$ (sugestão: use o resultado do Exercício~\ref{exe:sigmaDd} para mostrar que
$\sigma\cdot\Dd\omega=\sgn(\sigma)\Dd\omega$ para qualquer permutação $\sigma$ pertencente ao grupo:
\[G=\big\{\sigma\in S_{k+1}:\sigma(1)=1\big\}=\big\{\sigma^{(1)}:\sigma\in S_k\big\}.\]
Em seguida, use o resultado
do item~(b) do Exercício~\ref{exe:SymAltG} escolhendo o conjunto $X$ usando o resultado do Exercício~\ref{exe:XSkSl} com $k=1$).
\end{exercise}

\begin{exercise}\label{exe:KerAlt}
Sejam $E$, $F$ espaços vetoriais. Suponha que $\mathfrak t\in\Lin_k(E,F)$ seja simétrica com respeito a algum par de suas variáveis, i.e., suponha que
exista uma transposição $\sigma\in S_k$ tal que $\sigma\cdot\mathfrak t=\mathfrak t$. Mostre que $\Alt(\mathfrak t)=0$ (sugestão: use o resultado do
item~(a) do Exercício~\ref{exe:sigmaSymAlt}).
\end{exercise}

\begin{exercise}\label{exe:AltDAlt}
Sejam $E$, $F$ espaços vetoriais reais de dimensão finita, $U$ um subconjunto aberto de $E$ e $\mathfrak t:U\to\Lin_k(E,F)$ uma aplicação de classe $C^\infty$.
Mostre que:
\begin{equation}\label{eq:AltDAlt}
\Alt\!\big(\Dd(\Alt\mathfrak t)\big)=\Alt(\Dd\mathfrak t)
\end{equation}
(sugestão: expanda o alternador que aparece à direita de $\Dd$ no lado esquerdo de \eqref{eq:AltDAlt} e use o resultado do Exercício~\ref{exe:sigmaDd}
e o resultado do item~(a) do Exercício~\ref{exe:sigmaSymAlt}).
\end{exercise}

\begin{exercise}[propriedades da diferencial exterior]\label{exe:propdifext}
Sejam $E$ um espaço vetorial real de dimensão finita e $U$ um subconjunto aberto de $E$.
\begin{itemize}
\item[(a)] Se $f:U\to\R$ é uma $0$-forma (i.e., uma função a valores reais) de classe $C^\infty$, mostre que a diferencial exterior $\dd f$ coincide
com a diferencial comum $\Dd f$.
\item[(b)] Mostre que a diferenciação exterior define uma aplicação linear do espaço vetorial real das $k$-formas de classe $C^\infty$ em $U$ no espaço
vetorial real das $(k+1)$-formas de classe $C^\infty$ em $U$.
\item[(c)] Se $\omega$ é uma $k$-forma de classe $C^\infty$ em $U$, mostre que $\dd(\dd\omega)=0$ (sugestão: use os resultados dos Exercícios~\ref{exe:AltDAlt},
\ref{exe:Schwarz} e \ref{exe:KerAlt}).
\item[(d)] Se $\omega$ é uma $k$-forma de classe $C^\infty$ em $U$ e $\lambda$ é uma $l$-forma de classe $C^\infty$ em $U$, mostre que:
\[\dd(\omega\wedge\lambda)=(\dd\omega)\wedge\lambda+(-1)^k\omega\wedge(\dd\lambda)\]
(sugestão: use os resultados dos Exercícios~\ref{exe:AltDAlt}, \ref{exe:Dotimes}, \ref{exe:SymSymAltAlt} e o resultado do item~(a) do Exercício~\ref{exe:sigmaSymAlt}).
\item[(e)] Se $E'$ é um espaço vetorial real de dimensão finita, $U'$ é um subconjunto aberto de $E'$, $\phi:U'\to U$ é uma aplicação de classe $C^\infty$
e $\omega$ é uma $k$-forma de classe $C^\infty$ em $U$, mostre que:
\[\dd(\phi^*\omega)=\phi^*\dd\omega\]
(sugestão: use o resultado do Exercício~\ref{exe:Dpullback}, o resultado do item~(c) do Exercício~\ref{exe:sigmapullback2}
e o resultado do Exercício~\ref{exe:KerAlt}).
\end{itemize}
\end{exercise}

Quando se trabalha com formas diferenciais em $\R^n$, é costumeiro usar as seguintes notações: denota-se por:
\[\frac{\partial}{\partial x_1},\ldots,\frac{\partial}{\partial x_n}\]
a base canônica de $\R^n$ (ou os correspondentes campos vetoriais constantes em $\R^n$), por:
\[\dd x_1,\ldots,\dd x_n\]
a base dual da base canônica de $\R^n$ (ou as correspondentes $1$-formas constantes em $\R^n$). Note que se denotamos por $x_i:\R^n\to\R$ a projeção
sobre a $i$-ésima coordenada (que é uma $0$-forma em $\R^n$) então a diferencial exterior $\dd x_i$ de $x_i$ coincide com a $1$-forma constante e igual
ao $i$-ésimo vetor da base dual da base canônica de $\R^n$. Em vista do resultado do Exercício~\ref{exe:Linabigwedge}, temos que:
\[\dd x_{r_1}\wedge\cdots\wedge\dd x_{r_k},\quad1\le r_1<\cdots<r_k\le n\]
é uma base de $\Lin_k^{\mathrm a}(\R^n,\R)=\bigwedge_k{\R^n}^*$ e, dada uma $k$-forma $\omega$ num aberto $U$ de $\R^n$ então:
\begin{equation}\label{eq:omegabase}
\omega=\sum_{1\le r_1<\cdots<r_k\le n}\omega_{r_1\ldots r_k}\;\dd x_{r_1}\wedge\cdots\wedge\dd x_{r_k},
\end{equation}
onde a função $\omega_{r_1\ldots r_k}:U\to\R$ é dada por:
\[\omega_{r_1\ldots r_k}(x)=\omega(x)\Big(\frac{\partial}{\partial x_{r_1}},\ldots,\frac{\partial}{\partial x_{r_k}}\Big),\]
para todo $x\in U$. Evidentemente, a $k$-forma $\omega$ é de classe $C^\infty$ se e somente se as funções $\omega_{r_1\ldots r_k}$ são todas de classe
$C^\infty$.

\begin{exercise}
Se uma $k$-forma $\omega$ de classe $C^\infty$ é dada como em \eqref{eq:omegabase}, use os resultados dos itens~(a)---(d) do Exercício~\ref{exe:propdifext}
para mostrar que:
\begin{multline*}
\dd\omega=\sum_{1\le r_1<\cdots<r_k\le n}\dd\omega_{r_1\ldots r_k}\wedge\dd x_{r_1}\wedge\cdots\wedge\dd x_{r_k}\\
=\sum_{1\le r_1<\cdots<r_k\le n}\;\sum_{j=1}^n\frac{\partial\omega_{r_1\ldots r_k}}{\partial x_j}\,\dd x_j\wedge\dd x_{r_1}\wedge\cdots\wedge\dd x_{r_k}
\end{multline*}
(sugestão: note que o produto de uma função escalar por uma $k$-forma pode ser pensado como o produto exterior de uma $0$-forma por uma $k$-forma).
\end{exercise}

\end{section}

\end{document}
